\section[DB Architectures]{Database Architectures} \label{sec:architectures}

\begin{frame}
	\vbox to .6\textheight{%
		\parbox[c][.225\textheight][c]{\linewidth}{%
			\sectionpage%
		}
		\centering
		\resizebox{!}{.975\architecturesheight}{%
			\begin{tabular}{>{\raggedleft}m{.15\linewidth}cc>{\raggedright}m{.15\linewidth}}
				Shared Everything/ Non-Partitioned	&	\seDiagram		&	\doraDiagram	&	Data-Oriented Transaction Execution (DORA)	\\
				Delegation					&	\delegationDiagram	&	\pseDiagram	&	Partitioned Serial Execution (PSE)
			\end{tabular}%
		}%
	}
\end{frame}

\subsection[Shared Everything/\-Non-Partitioned]{Shared Everything/\-Non-Partitioned (SE/NP)}

\begin{frame}
	\vbox to .6\textheight{%
		\parbox[c][.225\textheight][c]{\linewidth}{%
			\subsectionpage%
		}
		\centering
		\only<1>{%
			\resizebox{!}{.975\architecturesheight}{%
				\begin{tabular}{>{\raggedleft}m{.15\linewidth}cc>{\raggedright}m{.15\linewidth}}
					Shared Everything/ Non-Partitioned	&	\seDiagram			&	\doraDiagram[0.5]	&	\textcolor{black!50}{Data-Oriented Transaction Execution (DORA)}	\\
					\textcolor{black!50}{Delegation}		&	\delegationDiagram[0.5]	&	\pseDiagram[0.5]	&	\textcolor{black!50}{Partitioned Serial Execution (PSE)}
				\end{tabular}%
			}%
		}%
		\only<2>{%
			\resizebox{!}{.91\architecturesheight}{%
				\seDiagram%
			}%
		}%
	}
\end{frame}

\begin{frame}
	\frametitle{Properties of SE/NP}
	
	\begin{itemize}
		\item	metadata (incl. locks) are not partitioned
		\item[$\rightarrow$]	physical synchronization (latches, atomics) required
		\item	data and indices are not partitioned
		\item[$\rightarrow$]	logical synchronization using a concurrency control protocol also required
		\item	transactions completely executed by one thread
		\item	thread-assignment depends only on load
	\end{itemize}
\end{frame}

\begin{frame}[fragile]
	\frametitle{Pros \& Cons of SE/NP}
	
	\begin{itemize}
		\visible<1->{\item[$+$]	no partitioning required (e.g. manual selection of a strategy)}
		\visible<1->{\item[$+$]	partitioning would be sensitive to the workload}
		\visible<1->{\item[$+$]	changed workloads would require repartitioning to benefit from partitioning}
		\visible<2->{\item[$-$]	each thread might access every record at arbitrary times}
			\begin{itemize}
				\visible<2->{\item[$-$]	each CPU cache may contain any part of the data \\ \bm{$\rightarrow$} cache pollution}
				\visible<2->{\item[$-$]	each CPU may access any part of the data \\ \bm{$\rightarrow$} data movement between NUMA regions}
				\visible<2->{\item[$-$]	each CPU may acquire any latch \\ \bm{$\rightarrow$} data movement between NUMA regions}
				\visible<2->{\item[$-$]	each CPU may atomically write to any semaphore \\ \bm{$\rightarrow$} hardware cache coherence overhead}
			\end{itemize}
	\end{itemize}
    
	\tikzset{%
		changes/.style = { -> , very thick, > = stealth},
		volatile/.style = {fill = red!50},
		nonvolatile/.style = {fill = green!50},
		changes/.append style = {color = red}
	}

	\begin{tikzpicture}[remember picture, overlay]
		\node[visible on = <3-13>, shape = rectangle, draw = black, fill = black!0] at (current page.center) {%
			\begin{adjustbox}{width = .85\enhancedtextwidth}%
				\begin{tikzpicture}
					\begin{axis}[xlabel = {TPC-C runtime $\left[\si{\second}\right]$},
							  xlabel near ticks,
							  xmin = 0,
							  xmax = 0.9,
							  xtick distance = {0.1},
							  scaled x ticks = false,
							  minor x tick num = 9,
							  ylabel = {TPC-C district},
							  ylabel near ticks,
							  ymin = 0,
							  ymax = 100,
							  ymode = normal,
							  scaled y ticks = false,
							  legend entries = {{Thread 0}, {Thread 1}, {Thread 2}, {Thread 3}, {Thread 4}, {Thread 5}, {Thread 6}, {Thread 7}, {Thread 8}, {Thread 9}},
							  legend style = {font = \footnotesize},
							  legend pos = outer north east,
							  width = \linewidth,
							  height = .95\textheight,
							  axis on top = true,
							  title = {Record Accesses of Conventional DB Threads (\cite{Pandis:2010})}]		
						\addplot[visible on = <4->, only marks, mark = x, draw = black] table {data/record_accesses_per_thread_conventional_0.csv};
						\addplot[visible on = <5->, only marks, mark = x, draw = cyan] table {data/record_accesses_per_thread_conventional_1.csv};
						\addplot[visible on = <6->, only marks, mark = x, draw = red] table {data/record_accesses_per_thread_conventional_2.csv};
						\addplot[visible on = <7->, only marks, mark = x, draw = green] table {data/record_accesses_per_thread_conventional_3.csv};
						\addplot[visible on = <8->, only marks, mark = x, draw = blue] table {data/record_accesses_per_thread_conventional_4.csv};
						\addplot[visible on = <9->, only marks, mark = x, draw = black!50] table {data/record_accesses_per_thread_conventional_5.csv};
						\addplot[visible on = <10->, only marks, mark = x, draw = cyan!50] table {data/record_accesses_per_thread_conventional_6.csv};
						\addplot[visible on = <11->, only marks, mark = x, draw = red!50] table {data/record_accesses_per_thread_conventional_7.csv};
						\addplot[visible on = <12->, only marks, mark = x, draw = green!50] table {data/record_accesses_per_thread_conventional_8.csv};
						\addplot[visible on = <13->, only marks, mark = x, draw = blue!50] table {data/record_accesses_per_thread_conventional_9.csv};
					\end{axis}
				\end{tikzpicture}%
			\end{adjustbox}%
    												   };
		\node[visible on = <15>, shape = rectangle, draw = black, fill = black!0] at (current page.center) {%
			\begin{adjustbox}{width = .85\enhancedtextwidth}
				\begin{tikzpicture}[]
							% Center points of the baseline of each pyramid layer (0: Peak of the pyramid; 1: Center of the baseline of the top layer; ...)
							\node[]			(0)															{};
							\node[]			(1)		[below = 1.2*1.5cm of 0.center, anchor = center]					{};
							\node[]			(2a)		[below = 1.2*(1/3)*1.25cm of 1.center, anchor = center]				{};
							\node[]			(2b)		[below = 1.2*(1/3)*1.25cm of 2a.center, anchor = center]				{};
							\node[]			(2c)		[below = 1.2*(1/3)*1.25cm of 2b.center, anchor = center]				{};
							\node[]			(3)		[below = 1.2*0.75cm of 2c.center, anchor = center]					{};
							\node[]			(4)		[below = 1.2*0.75cm of 3.center, anchor = center]					{};
							\node[]			(5)		[below = 1.2*0.75cm of 4.center, anchor = center]					{};
							\node[]			(6)		[below = 1.2*0.75cm of 5.center, anchor = center]					{};
	
							% Anchor points of the pyramid (n0: Left anchor point; n1: Right anchor point):
							\node[]			(10)		[left = (2/3)*1.5cm of 1.center, anchor = center]					{};
							\node[]			(11)		[right = (2/3)*1.5cm of 1.center, anchor = center]				{};
							\node[]			(2a0)		[left = (2/3)*(1.5cm+(1/3)*1.25cm) of 2a.center, anchor = center]	{};
							\node[]			(2a1)		[right = (2/3)*(1.5cm+(1/3)*1.25cm) of 2a.center, anchor = center]	{};
							\node[]			(2b0)		[left = (2/3)*(1.5cm+(2/3)*1.25cm) of 2b.center, anchor = center]	{};
							\node[]			(2b1)		[right = (2/3)*(1.5cm+(2/3)*1.25cm) of 2b.center, anchor = center]	{};
							\node[]			(2c0)		[left = (2/3)*(1.5cm+(3/3)*1.25cm) of 2c.center, anchor = center]	{};
							\node[]			(2c1)		[right = (2/3)*(1.5cm+(3/3)*1.25cm) of 2c.center, anchor = center]	{};
							\node[]			(30)		[left = (2/3)*3.5cm of 3.center, anchor = center]					{};
							\node[]			(31)		[right = (2/3)*3.5cm of 3.center, anchor = center]				{};
							\node[]			(40)		[left = (2/3)*4.25cm of 4.center, anchor = center]				{};
							\node[]			(41)		[right = (2/3)*4.25cm of 4.center, anchor = center]				{};
							\node[]			(50)		[left = (2/3)*5cm of 5.center, anchor = center]					{};
							\node[]			(51)		[right = (2/3)*5cm of 5.center, anchor = center]					{};
							\node[]			(60)		[left = (2/3)*5.75cm of 6.center, anchor = center]				{};
							\node[]			(61)		[right = (2/3)*5.75cm of 6.center, anchor = center]				{};
			
							% "Center" of the pyramid (shifted towards top) used for the designation of performance measures:
							\node[]			(cen)		[below = 2.5cm of 0.center]								{};
			
							% Designation of performance measures:
							\node[rotate = 61]	(cpu)		[left = 3.25cm of cen.center, anchor = center]				{\footnotesize \textit{\textbf{Capacity per CPU}}};
							\node[rotate = -61]	(acc)		[right = 4.25cm of cen.center, anchor = center]				{\footnotesize \textit{\textbf{Access Time}}};
			
							% Fills the pyramid to mark volatile and non-volatile layers:
							\fill[volatile]
								(0.center) -- (30.center) -- (31.center) -- cycle;
							\fill[nonvolatile]
								(30.center) -- (31.center) -- (61.center) -- (60.center) -- cycle;
			
							% Key the markings for volatile and non-volatile layers
							\node[draw, volatile]		(red)		[left = 4cm of 0.center, anchor = north]		{};
							\node[]					(legred)	[right = 0cm of red.east, anchor = west]		{\footnotesize Volatile};
							\node[draw, nonvolatile]		(green)	[below = 0.2cm of red.south, anchor = north]	{};
							\node[]					(leggreen)	[right = 0cm of green.east, anchor = west]		{\footnotesize Non-volatile};
	
							% Storage layer names:
							\node[]	(reg)		[above = 0cm of 1.center, anchor = south]		{\footnotesize Registers};
							\node[]	(l1)		[above = 0cm of 2a.center, anchor = south]	{\footnotesize L1 Cache};
							\node[]	(l2)		[above = 0cm of 2b.center, anchor = south]	{\footnotesize L2 Cache};
							\node[]	(l3)		[above = 0cm of 2c.center, anchor = south]	{\footnotesize L3 Cache};
							\node[]	(mem)	[above = 0.05cm of 3.center, anchor = south]	{\small Main Memory};
							\node[]	(ssd)		[above = 0.05cm of 4.center, anchor = south]	{\small Online Storage \scriptsize \textit{(Secondary S.) - SDD}};
							\node[]	(mas)	[above = 0.05cm of 5.center, anchor = south]	{\small Online Storage \scriptsize \textit{(Secondary S.) - HDD}};
							\node[]	(mem)	[above = 0.05cm of 6.center, anchor = south]	{\small Offline Storage \scriptsize \textit{(Tertiary Storage)}};
	
							% Anchor points for the brace used to mark the example data from Intel Xeon:
							\node[]	(0h)		[right = 2.75cm of 0]		{};
							\node[]	(31h)		[right = 2.75cm of 31]	{};
			
							% Draw the pyramid:
							\path[ - , thick]
								(0.center)			edge	[]	(10.center)
								(0.center)			edge[]	(11.center)
								(10.center)		edge[]	(11.center)
								(10.center)		edge	[]	(2a0.center)
								(11.center)		edge	[]	(2a1.center)
								(2a0.center)		edge	[]	(2a1.center)
								(2a0.center)		edge	[]	(2b0.center)
								(2a1.center)		edge[]	(2b1.center)
								(2b0.center)		edge[]	(2b1.center)
								(2b0.center)		edge[]	(2c0.center)
								(2b1.center)		edge[]	(2c1.center)
								(2c0.center)		edge[]	(2c1.center)
								(2c0.center)		edge[]	(30.center)
								(2c1.center)		edge[]	(31.center)
								(30.center)		edge[]	(31.center)
								(30.center)		edge[]	(40.center)
								(31.center)		edge	[]	(41.center)
								(40.center)		edge	[]	(41.center)
								(40.center)		edge[]	(50.center)
								(41.center)		edge	[]	(51.center)
								(50.center)		edge	[]	(51.center)
								(50.center)		edge	[]	(60.center)
								(51.center)		edge[]	(61.center)
								(60.center)		edge	[]	(61.center);
			
							% Draw the connections between the brace used to mark the example data from Intel Xeon and the appropriate pyramid layers:
							\path[dotted]
								(0.center)				edge		(0h.center)
								(31.center)			edge		(31h.center);
				
							% Add the capacities:
							\draw [decorate, decoration = {brace, mirror, amplitude = 3.75pt}]
								(0.center) -- (10.center) node[black, midway, xshift = -0.1cm, yshift = 0.1cm, anchor = east] {\SI{128}{\byte}};
							\draw [decorate, decoration = {brace, mirror, amplitude = 3.75pt}]
								(10.center) -- (2a0.center) node[black, midway, xshift = -0.1cm, yshift = 0.1cm, anchor = east] {\small \SI{256}{\kibi\byte}};
							\draw [decorate, decoration = {brace, mirror, amplitude = 3.75pt}]
								(2a0.center) -- (2b0.center) node[black, midway, xshift = -0.1cm, yshift = 0.1cm, anchor = east] {\small \SI{1}{\mebi\byte}};
							\draw [decorate, decoration = {brace, mirror, amplitude = 3.75pt}]
								(2b0.center) -- (2c0.center) node[black, midway, xshift = -0.1cm, yshift = 0.1cm, anchor = east] {\small \SI{8}{\mebi\byte}};
							\draw [decorate, decoration = {brace, mirror, amplitude = 3.75pt}]
								(2c0.center) -- (30.center) node[black, midway, xshift = -0.1cm, yshift = 0.1cm, anchor = east] {\SI{64}{\gibi\byte}};
							\draw [decorate, decoration = {brace, mirror, amplitude = 3.75pt}]
								(30.center) -- (40.center) node[black, midway, xshift = -0.1cm, yshift = 0.1cm, anchor = east] {\si{\tebi\byte}};
							\draw [decorate, decoration = {brace, mirror, amplitude = 3.75pt}]
								(40.center) -- (50.center) node[black, midway, xshift = -0.1cm, yshift = 0.1cm, anchor = east] {\si{\tebi\byte}--\si{\pebi\byte}};
							\draw [decorate, decoration = {brace, mirror, amplitude = 3.75pt}]
								(50.center) -- (60.center) node[black, midway, xshift = -0.1cm, yshift = 0.1cm, anchor = east] {\si{\pebi\byte}};
	
							% Add the access times:
							\draw [decorate, decoration = {brace, amplitude = 3.75pt}]
								(0.center) -- (11.center) node[black, midway, xshift = 0.1cm, yshift = 0.1cm, anchor = west] {\SI{1}{\cy} $\approx$ \SI{0.25}{\nano\second}};
							\draw [decorate, decoration = {brace, amplitude = 3.75pt}]
								(11.center) -- (2a1.center) node[black, midway, xshift = 0.1cm, yshift = 0.1cm, anchor = west] {\small \SI{4}{\cy} $\approx$ \SI{1}{\nano\second}};
							\draw [decorate, decoration = {brace, amplitude = 3.75pt}]
								(2a1.center) -- (2b1.center) node[black, midway, xshift = 0.1cm, yshift = 0.1cm, anchor = west] {\small \SI{12}{\cy} $\approx$ \SI{3}{\nano\second}};
							\draw [decorate, decoration = {brace, amplitude = 3.75pt}]
								(2b1.center) -- (2c1.center) node[black, midway, xshift = 0.1cm, yshift = 0.1cm, anchor = west] {\small \SI{42}{\cy} $\approx$ \SI{11}{\nano\second}};
							\draw [decorate, decoration = {brace, amplitude = 3.75pt}]
								(2c1.center) -- (31.center) node[black, midway, xshift = 0.1cm, yshift = 0.1cm, anchor = west] {\SI{51}{\nano\second}};
							\draw [decorate, decoration = {brace, amplitude = 3.75pt}]
								(31.center) -- (41.center) node[black, midway, xshift = 0.1cm, yshift = 0.1cm, anchor = west] {\SI{\approx 0.1}{\milli\second} \tiny \cite{samsung:ssdprimer}};
							\draw [decorate, decoration = {brace, amplitude = 3.75pt}]
								(41.center) -- (51.center) node[black, midway, xshift = 0.1cm, yshift = 0.1cm, anchor = west] {\numrange{5}{20}\si{\milli\second} \tiny \cite{EntHDD} \cite{HDD2016}};
							\draw [decorate, decoration = {brace, amplitude = 3.75pt}]
								(51.center) -- (61.center) node[black, midway, xshift = 0.1cm, yshift = 0.1cm, anchor = west] {$s - min$};
	
							% Mark the source of the performance data of the volatile layers:
							\draw [decorate, decoration = {brace, amplitude = 3.75pt}]
								(0h.center) -- (31h.center) node[black, midway, xshift = -0.625cm, yshift = 0.1cm, anchor = west] {\footnotesize \begin{tabular}{r}\textit{\textbf{Example System:}}\\\textit{Intel\textsuperscript{\tiny\textregistered} Xeon\textsuperscript{\tiny\textregistered}}\\\textit{E3-1280 v5}\\\textit{(Skylake)}\\\cite{7cpu:skylake}\end{tabular}};
				
							% Orientation of changes of performance data:
							\path[changes]
								([xshift = (-1/3)*4cm+0.25cm]60.center)			edge		node[above, rotate = 90]		{\footnotesize Price per Capacity}	([xshift = (-1/3)*14cm-0.25cm]0.center)
								([xshift = (1/3)*18cm+0.5cm]0.center)			edge		node[above, rotate = -90]		{\footnotesize Access Time}		([xshift = (1/3)*8cm]61.center);
				\end{tikzpicture}
			\end{adjustbox}
    												   };
		\node[visible on = <17>, shape = rectangle, draw = black, fill = black!0] at (current page.center) {%
			\begin{adjustbox}{width = .85\enhancedtextwidth}%
				\begin{tikzpicture}
					\begin{axis}[major x tick style = transparent,
							   ybar = 2*\pgflinewidth,
							   bar width = 14pt,
							   ymajorgrids = true,
							   ylabel = {Message Throughput $\left[\si{\kilo\messages\per\second}\right]$},
							   symbolic x coords = {{FIFO}, {POSIX Message Queues}, {Pipes}, {TCP Sockets}, {UNIX Sockets}},
							   xtick = data,
							   x tick label style = {text width = width("Message"),
											align = center},
							   scaled y ticks = false,
							   enlarge x limits = 0.1,
							   ymin = 0,
							   legend style = {at = {(.5, 1.09)},
										  anchor = north},
							   width = \linewidth,
							   height = .7\textheight,
							   title= {Throughput of Inter-Process Communication Mechanisms (\cite{Porobic:2012})}]
						\addplot[style = {blue, fill = blue, mark = none}]	coordinates {({FIFO}, 58.9078) ({POSIX Message Queues}, 61.5127) ({Pipes}, 59.832) ({TCP Sockets}, 21.2451) ({UNIX Sockets}, 66.8908)};
						\addplot[style = {red, fill = red, mark = none}]	coordinates {({FIFO}, 41.5969) ({POSIX Message Queues}, 42.6893) ({Pipes}, 40.0843) ({TCP Sockets}, 16.6391) ({UNIX Sockets}, 44.454)};

						\legend{{Local NUMA region}, {Remote NUMA region}}
					\end{axis}
				\end{tikzpicture}%
			\end{adjustbox}%
    												   };
	\end{tikzpicture}
\end{frame}

\subsection[DORA]{Data-Oriented Transaction Execution (DORA)}

\begin{frame}
	\vbox to .6\textheight{%
		\parbox[c][.225\textheight][c]{\linewidth}{%
			\subsectionpage%
		}
		\centering
		\only<1>{%
			\resizebox{!}{.975\architecturesheight}{%
				\begin{tabular}{>{\raggedleft}m{.15\linewidth}cc>{\raggedright}m{.15\linewidth}}
					\textcolor{black!50}{Shared Everything/ Non-Partitioned}	&	\seDiagram[0.5]		&	\doraDiagram		&	Data-Oriented Transaction Execution (DORA)			\\
					\textcolor{black!50}{Delegation}						&	\delegationDiagram[0.5]	&	\pseDiagram[0.5]	&	\textcolor{black!50}{Partitioned Serial Execution (PSE)}
				\end{tabular}%
			}%
		}%
		\only<2>{%
			\resizebox{!}{.91\architecturesheight}{%
				\doraDiagram%
			}%
		}%
	}
\end{frame}

\begin{frame}
	\frametitle{Properties of DORA}
	
	\begin{itemize}
		\item	metadata (incl. locks) are physically partitioned
		\item[$\rightarrow$]	no physical synchronization (latches, atomics) required
		\item	data and indices are logically partitioned
		\item[$\rightarrow$]	logical synchronization using a concurrency control protocol only locally required
		\item	threads are assigned to data
		\item	transactions migrate to threads owning the accessed data
	\end{itemize}
\end{frame}

\begin{frame}[fragile]
	\frametitle{Interactive Example}
	
	\tikzset{
		fiber/.style = {shape = rectangle},
		thread/.style = {shape = rectangle,
					anchor = north west},
		record/.style = {shape = rectangle,
					draw = black,
					anchor = west,
					scale = 2}
	}
	
	\begin{adjustbox}{width = \linewidth}
		\begin{tikzpicture}
			\node[fiber, visible on = <2-26>]							(f0)	{%
				\begin{tabular}{| r | l |}
					\multicolumn{2}{c}{\textbf{Fiber 0}}															\\	\hline
					\only<-3>{\textbf{Locks:}		&															\\	\hline}%
					\only<4-7>{\textbf{Locks:}		&	$\left(5, S\right)$											\\	\hline}%
					\only<8>{\textbf{Locks:}		&	$\left(5, S\right), \left(3, S\right)$								\\	\hline}%
					\only<9-14>{\textbf{Locks:}	&	$\left(5, S\right), \left(3, S\right), \left(2, X\right)$					\\	\hline}%
					\only<15>{\textbf{Locks:}		&	$\left(5, S\right), \left(3, S\right), \left(2, X\right),$					\\%
											&	$\left(0, S\right)$											\\	\hline}%
					\only<16>{\textbf{Locks:}		&	$\left(5, S\right), \left(3, S\right), \left(2, X\right),$					\\%
											&	$\left(0, S\right), \left(1, S\right)$								\\	\hline}%
					\only<17-20>{\textbf{Locks:}	&	$\left(5, S\right), \left(3, S\right), \left(2, X\right)$					\\	\hline}%
					\only<21-24>{\textbf{Locks:}	&	$\left(5, S\right)$											\\	\hline}%
					\only<25->{\textbf{Locks:}		&															\\	\hline}%
					\textbf{Code:}				&	\only<4->{\color{green}}\ttfamily BOT								\\
											&	\only<5->{\color{green}}\ttfamily \quad read(5);\visible<4>{ <-}			\\
											&	\only<9->{\color{green}}\ttfamily \quad read(3);\visible<8>{ <-}			\\
											&	\only<10->{\color{green}}\ttfamily \quad write(2, x);\visible<9>{ <-}		\\
											&	\only<16->{\color{green}}\ttfamily \quad read(1);\visible<15>{ <-}		\\
											&	\only<26->{\color{green}}\ttfamily \quad commit();\visible<16-25>{ <-}	\\
											&	\only<26->{\color{green}}\ttfamily EOT							\\	\hline
					\textbf{Locks:}				&															\\	\hline
				\end{tabular}%
																};
			\node[fiber, below = 17.5em of f0.north, visible on = <9-14>]	(f1)	{%
				\begin{tabular}{| r | l |}
					\multicolumn{2}{c}{\textbf{Fiber 1}}													\\	\hline
					\only<-10>{\textbf{Locks:}	&														\\	\hline}%
					\only<11>{\textbf{Locks:}	&	$\left(0, S\right)$										\\	\hline}%
					\only<12>{\textbf{Locks:}	&	$\left(0, S\right), \left(1, S\right)$							\\	\hline}%
					\only<13->{\textbf{Locks:}	&														\\	\hline}%
					\textbf{Code:}			&	\only<11->{\color{green}}\ttfamily BOT						\\
										&	\only<12->{\color{green}}\ttfamily \quad read(0);\visible<11>{ <-}	\\
										&	\only<13->{\color{green}}\ttfamily \quad read(1);\visible<12>{ <-}	\\
										&	\only<14->{\color{green}}\ttfamily \quad commit();\visible<13>{ <-}	\\
										&	\only<14->{\color{green}}\ttfamily EOT						\\	\hline
				\end{tabular}%
																};

			\node[above right = 8em and 4em of f0.north east, thread]	(t0)	{%
				\begin{tabular}{| r | p{\fiberwidth} |}
					\multicolumn{2}{c}{\textbf{Thread 0}}									\\	\hline
					\only<-8>{\textbf{Fibers:}		&	\textit{idle}							\\	\hline}%
					\only<9>{\textbf{Fibers:}		&	Fiber 1 \textit{created}				\\	\hline}%
					\only<10>{\textbf{Fibers:}		&	Fiber 1 \textit{waiting}				\\	\hline}%
					\only<11>{\textbf{Fibers:}		&	Fiber 1 \textit{running}				\\%
											&	Fiber 0 \textit{suspended}				\\	\hline}%
					\only<12>{\textbf{Fibers:}		&	Fiber 1 \textit{running}				\\%
											&	Fiber 0 \textit{waiting}				\\	\hline}%
					\only<13>{\textbf{Fibers:}		&	Fiber 1 \textit{committing}				\\%
											&	Fiber 0 \textit{waiting}				\\	\hline}%
					\only<14>{\textbf{Fibers:}		&	Fiber 1 \textit{terminated}				\\%
											&	Fiber 0 \textit{waiting}				\\	\hline}%
					\only<15-16>{\textbf{Fibers:}	&	Fiber 0 \textit{running}				\\	\hline}%
					\only<17>{\textbf{Fibers:}		&	Fiber 0 \textit{committing}				\\	\hline}%
					\only<18>{\textbf{Fibers:}		&	Fiber 0 \textit{suspended}				\\	\hline}%
					\only<19->{\textbf{Fibers:}		&	\textit{idle}							\\	\hline}%
					\only<-10>{\textbf{Locks:}		&									\\	\hline}%
					\only<11>{\textbf{Locks:}		&	$\left(0, S\right)_1$					\\	\hline}%
					\only<12>{\textbf{Locks:}		&	$\left(0, S\right)_1, \left(1, S\right)_1$	\\	\hline}%
					\only<13-14>{\textbf{Locks:}	&									\\	\hline}%
					\only<15>{\textbf{Locks:}		&	$\left(0, S\right)_0$					\\	\hline}%
					\only<16>{\textbf{Locks:}		&	$\left(0, S\right)_0, \left(1, S\right)_0$	\\	\hline}%
					\only<17->{\textbf{Locks:}		&									\\	\hline}%
					\textbf{Partition:}			&	$0-1, ...$							\\	\hline
					
				\end{tabular}%
										};
			\node[right = 1em of t0.north east, thread]	(t1)	{%
				\begin{tabular}{| r | p{\fiberwidth} |}
					\multicolumn{2}{c}{\textbf{Thread 1}}									\\	\hline
					\only<-5>{\textbf{Fibers:}		&	\textit{idle}							\\	\hline}%
					\only<6>{\textbf{Fibers:}		&	Fiber 0 \textit{suspended}				\\	\hline}%
					\only<7>{\textbf{Fibers:}		&	Fiber 0 \textit{waiting}				\\	\hline}%
					\only<8-9>{\textbf{Fibers:}		&	Fiber 0 \textit{running}				\\	\hline}%
					\only<10>{\textbf{Fibers:}		&	Fiber 0 \textit{suspended}				\\	\hline}%
					\only<11-18>{\textbf{Fibers:}	&	\textit{idle}							\\	\hline}%
					\only<19>{\textbf{Fibers:}		&	Fiber 0 \textit{suspended}				\\	\hline}%
					\only<20>{\textbf{Fibers:}		&	Fiber 0 \textit{waiting}				\\	\hline}%
					\only<21>{\textbf{Fibers:}		&	Fiber 0 \textit{committing}				\\	\hline}%
					\only<22>{\textbf{Fibers:}		&	Fiber 0 \textit{suspended}				\\	\hline}%
					\only<23->{\textbf{Fibers:}		&	\textit{idle}							\\	\hline}%
					\only<-7>{\textbf{Locks:}		&									\\	\hline}%
					\only<8>{\textbf{Locks:}		&	$\left(3, S\right)_0$					\\	\hline}%
					\only<9-20>{\textbf{Locks:}	&	$\left(3, S\right)_0, \left(2, X\right)_0$	\\	\hline}%
					\only<21->{\textbf{Locks:}		&									\\	\hline}%
					\textbf{Partition:}			&	$2-3, ...$							\\	\hline
					
				\end{tabular}%
										};
			\node[right = 1em of t1.north east, thread]	(t2)	{%
				\begin{tabular}{| r | p{\fiberwidth} |}
					\multicolumn{2}{c}{\textbf{Thread 2}}						\\	\hline
					\only<-1>{\textbf{Fibers:}		&	\textit{idle}				\\	\hline}%
					\only<2>{\textbf{Fibers:}		&	Fiber 0 \textit{created}	\\	\hline}%
					\only<3>{\textbf{Fibers:}		&	Fiber 0 \textit{waiting}	\\	\hline}%
					\only<4>{\textbf{Fibers:}		&	Fiber 0 \textit{running}	\\	\hline}%
					\only<5>{\textbf{Fibers:}		&	Fiber 0 \textit{suspended}	\\	\hline}%
					\only<6-22>{\textbf{Fibers:}	&	\textit{idle}				\\	\hline}%
					\only<23>{\textbf{Fibers:}		&	Fiber 0 \textit{suspended}	\\	\hline}%
					\only<24>{\textbf{Fibers:}		&	Fiber 0 \textit{waiting}	\\	\hline}%
					\only<25>{\textbf{Fibers:}		&	Fiber 0 \textit{committing}	\\	\hline}%
					\only<26>{\textbf{Fibers:}		&	Fiber 0 \textit{terminated}	\\	\hline}%
					\only<27->{\textbf{Fibers:}		&	\textit{idle}				\\	\hline}%
					\only<-3>{\textbf{Locks:}		&						\\	\hline}%
					\only<4-24>{\textbf{Locks:}	&	$\left(5, S\right)_0$		\\	\hline}%
					\only<25->{\textbf{Locks:}		&						\\	\hline}%
					\textbf{Partition:}			&	$4-5, ...$				\\	\hline
					
				\end{tabular}%
										};
			
			\node[right = 20em of f1, record]	(r0)		{0};
			\node[right = 0em of r0.east, record]	(r1)		{1};
			\node[right = 0em of r1.east, record]	(r2)		{2};
			\node[right = 0em of r2.east, record]	(r3)		{3};
			\node[right = 0em of r3.east, record]	(r4)		{4};
			\node[right = 0em of r4.east, record]	(r5)		{5};
			\node[below = 1em of r2.south east, scale = 2]	{...};
			
			\draw[ -> , visible on = <2>]		(f0)		edge[bend right = 15]	node[below]		{create \textbf{Fiber 0}}	(t2);
			\draw[ -> , visible on = <4>]		(t2)		edge[bend right = 15]	node[right]		{\ttfamily read}			(r5);
			\draw[ -> , visible on = <6>]		(t2)		edge[bend left = 45]		node[below]		{migrate \textbf{Fiber 0}}	(t1);
			\draw[ -> , visible on = <8>]		(t1)		edge[]				node[right]		{\ttfamily read}			(r3);
			\draw[ -> , visible on = <9>]		(t1)		edge[]				node[right]		{\ttfamily write(x)}		(r2);
			\draw[ -> , visible on = <9>]		(f1)		edge[bend right = 15]	node[below right]	{create \textbf{Fiber 1}}	(t0);
			\draw[ -> , visible on = <11>]		(t1)		edge[bend left = 45]		node[below]		{migrate \textbf{Fiber 0}}	(t0);
			\draw[ -> , visible on = <11>]		(t0)		edge[bend left = 15]		node[left]			{\ttfamily read}			(r0);
			\draw[ -> , visible on = <12>]		(t0)		edge[bend left = 15]		node[left]			{\ttfamily read}			(r1);
			\draw[ -> , visible on = <15>]		(t0)		edge[bend left = 15]		node[left]			{\ttfamily read}			(r0);
			\draw[ -> , visible on = <16>]		(t0)		edge[bend left = 15]		node[left]			{\ttfamily read}			(r1);
			\draw[ -> , visible on = <19>]		(t0)		edge[bend right = 45]	node[below]		{migrate \textbf{Fiber 0}}	(t1);
			\draw[ -> , visible on = <23>]		(t1)		edge[bend right = 45]	node[below]		{migrate \textbf{Fiber 0}}	(t2);
		\end{tikzpicture}%
	\end{adjustbox}
\end{frame}

\begin{frame}
	\frametitle{Pros of DORA}
	
	\begin{itemize}
		\item[$+$]	each thread accesses only the records of its partition
			\begin{itemize}
				\item[$+$]	each CPU cache may contain only data of its partition \\ \bm{$\rightarrow$} lower cache pollution
				\item[$+$]	each CPU may access only data of its partitions \\ \bm{$\rightarrow$} no data movement between NUMA regions (for single-CPU transactions)
				\item[$\rightarrow$]	No physical synchronization required!
			\end{itemize}
		\item[$+$]	logical partitioning allows fast repartitioning when the workload changes
		\item[$+$]	intra-transaction parallelism could be exploited for multi-site transactions
	\end{itemize}
	
	\begin{tikzpicture}[remember picture, overlay]
		\node[visible on = <2>, shape = rectangle, draw = black, fill = black!0] at (current page.center) {%
			\begin{adjustbox}{width = .85\enhancedtextwidth}%
				\begin{tikzpicture}
					\begin{axis}[xlabel = {TPC-C runtime $\left[\si{\second}\right]$},
							  xlabel near ticks,
							  xmin = 0,
							  xmax = 0.9,
							  xtick distance = {0.1},
							  scaled x ticks = false,
							  minor x tick num = 9,
							  ylabel = {TPC-C district},
							  ylabel near ticks,
							  ymin = 0,
							  ymax = 100,
							  ymode = normal,
							  scaled y ticks = false,
							  legend entries = {{Thread 0}, {Thread 1}, {Thread 2}, {Thread 3}, {Thread 4}, {Thread 5}, {Thread 6}, {Thread 7}, {Thread 8}, {Thread 9}},
							  legend style = {font = \footnotesize},
							  legend pos = outer north east,
							  width = \linewidth,
							  height = .95\textheight,
							  axis on top = true,
							  title = {Record Accesses of DORA DB Threads (\cite{Pandis:2010})}]		
						\addplot[only marks, mark = x, draw = black] table {data/record_accesses_per_thread_dora_0.csv};
						\addplot[only marks, mark = x, draw = cyan] table {data/record_accesses_per_thread_dora_1.csv};
						\addplot[only marks, mark = x, draw = red] table {data/record_accesses_per_thread_dora_2.csv};
						\addplot[only marks, mark = x, draw = green] table {data/record_accesses_per_thread_dora_3.csv};
						\addplot[only marks, mark = x, draw = blue] table {data/record_accesses_per_thread_dora_4.csv};
						\addplot[only marks, mark = x, draw = black!50] table {data/record_accesses_per_thread_dora_5.csv};
						\addplot[only marks, mark = x, draw = cyan!50] table {data/record_accesses_per_thread_dora_6.csv};
						\addplot[only marks, mark = x, draw = red!50] table {data/record_accesses_per_thread_dora_7.csv};
						\addplot[only marks, mark = x, draw = green!50] table {data/record_accesses_per_thread_dora_8.csv};
						\addplot[only marks, mark = x, draw = blue!50] table {data/record_accesses_per_thread_dora_9.csv};
					\end{axis}
				\end{tikzpicture}%
			\end{adjustbox}%
		};
	\end{tikzpicture}
\end{frame}

\begin{frame}
	\frametitle{Cons of DORA}
	
	\begin{itemize}
		\item[$-$]	partitioning required (e.g. manual selection of a partitioning strategy---\textit{called routing rule})
		\item[$-$]	partitioning is sensitive to the workload
		\item[$-$]	multi-site transactions require expensive fiber-migration (probably between NUMA regions)
		\item[$-$]	accessed partitions need to be calculated during query analysis for optimal performance \\ \bm{$\rightarrow$} slower accesses with secondary index
		\item[$-$]	primary index is shared \\ \bm{$\rightarrow$} centralized latching for inserts/\-deletes still required \\ \bm{$\rightarrow$} some contention on the shared latch
		\item[$-$]	centralized deadlock detection still required (for \lstinline{DL\_DETECT})
	\end{itemize}
\end{frame}

\subsection[Delegation]{Delegation}

\begin{frame}
	\vbox to .6\textheight{%
		\parbox[c][.225\textheight][c]{\linewidth}{%
			\subsectionpage%
		}
		\centering
		\only<1>{%
			\resizebox{!}{.975\architecturesheight}{%
				\begin{tabular}{>{\raggedleft}m{.15\linewidth}cc>{\raggedright}m{.15\linewidth}}
					\textcolor{black!50}{Shared Everything/ Non-Partitioned}	&	\seDiagram[0.5]	&	\doraDiagram[0.5]	&	\textcolor{black!50}{Data-Oriented Transaction Execution (DORA)}	\\
					Delegation									&	\delegationDiagram	&	\pseDiagram[0.5]	&	\textcolor{black!50}{Partitioned Serial Execution (PSE)}
				\end{tabular}%
			}%
		}%
		\only<2>{%
			\resizebox{!}{.91\architecturesheight}{%
				\delegationDiagram
			}%
		}%
	}
\end{frame}

\subsection[Partitioned Serial Execution]{Partitioned Serial Execution (PSE)}

\begin{frame}
	\vbox to .6\textheight{%
		\parbox[c][.225\textheight][c]{\linewidth}{%
			\subsectionpage%
		}
		\centering
		\only<1>{%
			\resizebox{!}{.975\architecturesheight}{%
				\begin{tabular}{>{\raggedleft}m{.15\linewidth}cc>{\raggedright}m{.15\linewidth}}
					\textcolor{black!50}{Shared Everything/ Non-Partitioned}	&	\seDiagram[0.5]		&	\doraDiagram[0.5]	&	\textcolor{black!50}{Data-Oriented Transaction Execution (DORA)}	\\
					\textcolor{black!50}{Delegation}						&	\delegationDiagram[0.5]	&	\pseDiagram		&	Partitioned Serial Execution (PSE)
				\end{tabular}%
			}%
		}%
		\only<2>{%
			\resizebox{!}{.91\architecturesheight}{%
				\pseDiagram
			}%
		}%
	}
\end{frame}

\subsection[Summary]{Summary}

\begin{frame}
	\frametitle{Summary}

	\centering
	\begin{tabular}{|>{\raggedright}m{\widthof{DORA}}|>{\raggedright}m{\widthof{Message}}|>{\raggedright}m{\widthof{thread-to-data}}|>{\raggedright}m{\widthof{\textbf{Synchro-}}}|>{\raggedright}m{\widthof{Transaction}}|}
																																																																												\hline
		\multirow{2}{*}{\centering\parbox[c]{\widthof{DORA}}{\centering\vspace{.75\baselineskip}\textbf{Ar\-chi\-tec\-ture}}}	&	\multicolumn{2}{c|}{\hspace{0pt}\visible<2->{\textbf{Process Management}}}											&	\multicolumn{2}{c|}{\parbox[c]{\widthof{\textbf{Transactional Storage}}}{\centering\vspace{.1\baselineskip}\hspace{0pt}\visible<4->{\textbf{Transactional Storage Management}}\vspace{.1\baselineskip}}} 	\\	\cline{2-5}
																								&	\hspace{0pt}\visible<2->{\centering\textbf{Pa\-ral\-lel\-ism}}	&	\hspace{0pt}\visible<3->{\centering\textbf{Thread Assignment}}	&	\hspace{0pt}\visible<4->{\centering\textbf{Logical Synchronization}}	&	\hspace{0pt}\visible<5->{\centering\textbf{Physical Synchronization}}													\\	\hline
		SE/NP																					&	\hspace{0pt}\visible<2->{Shared Memory}	 				& 	\hspace{0pt}\visible<3->{thread-to-txn}	 				&	\hspace{0pt}\visible<4->{CC Protocols}	 					& 	\hspace{0pt}\visible<5->{latch/\-atomics}																		\\	\hline
		PSE																						&	\hspace{0pt}\visible<2->{Shared Nothing}	 				& 	\hspace{0pt}\visible<3->{thread-to-txn}	 				& 	\hspace{0pt}\visible<4->{Partition Lock	}						& 	\hspace{0pt}\visible<5->{partition lock}																		\\	\hline
		De\-le\-ga\-tion																				&	\hspace{0pt}\visible<2->{Message Passing}	 			& 	\hspace{0pt}\visible<3->{thread-to-txn}	 				& 	\hspace{0pt}\visible<4->{CC Protocols}	 					& 	\hspace{0pt}\visible<5->{Message Passing}																	\\	\hline
		DORA																					&	\hspace{0pt}\visible<2->{Shared Memory}					& 	\hspace{0pt}\visible<3->{thread-to-data}					& 	\hspace{0pt}\visible<4->{CC Protocols}	 					& 	\hspace{0pt}\visible<5->{Transaction Migration}																	\\	\hline
	\end{tabular}
\end{frame}
