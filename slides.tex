%%%%%%%%%%%%%%%%%%%%%%%%%%%%%%%%%%%%%%%%%
% Beamer Presentation
% LaTeX Template
% Version 1.0 (10/11/12)
%
% This template has been downloaded from:
% http://www.LaTeXTemplates.com
%
% License:
% CC BY-NC-SA 3.0 (http://creativecommons.org/licenses/by-nc-sa/3.0/)
%
%%%%%%%%%%%%%%%%%%%%%%%%%%%%%%%%%%%%%%%%%

%----------------------------------------------------------------------------------------
%	PACKAGES AND THEMES
%----------------------------------------------------------------------------------------

\documentclass
		[
			xcolor = table,			% allows the coloring of whole tabular by changing the color before the definition of the tabular
			usenames,			% allows the usage of the names of the 16 default colors
			dvipsnames,			% allows the usage of the names of 64 additional colors
			svgnames,			% allows the usage of the names of ca. 150 additional colors
			x11names				% allows the usage of the names of ca. 300 additional colors
			final
		]{beamer}					% setting of the document class

%%%%%%%%%%%%%%%%%%%%%%%%%%%%%%%%%%%%%%%%%%%%%%%
%%%%%%%%%%%%%%%%%%%%%%%%%%%%%%%%%%%%%%%%%%%%%%%
%%%%%%%%%%%%%%%%%%%%%%%%%%%%%%%%%%%%%%%%%%%%%%%
% Defines a toggle to switch between grayscale and colored output:
%%%%%%%%%%%%%%%%%%%%%%%%%%%%%%%%%%%%%%%%%%%%%%% begindefinition
\usepackage{etoolbox}			% allows the usage of different 

\newtoggle{bwmode}
%%%%%%%%%%%%%%%%%%%%%%%%%%%%%%%%%%%%%%%%%%%%%%% enddefinition

%%%%%%%%%%%%%%%%%%%%%%%%%%%%%%%%%%%%%%%%%%%%%%%
%%%%%%%%%%%%%%%%%%%%%%%%%%%%%%%%%%%%%%%%%%%%%%%
%%%%%%%%%%%%%%%%%%%%%%%%%%%%%%%%%%%%%%%%%%%%%%%
% Redefines \thefootnote to use numbering starting from 0:
%%%%%%%%%%%%%%%%%%%%%%%%%%%%%%%%%%%%%%%%%%%%%%% begindefinition
\newcounter{indianfoot}
\newcommand{\useindianfootnotes}{
	\renewcommand{\thefootnote}{%
		\setcounter{indianfoot}{0}%
		\addtocounter{indianfoot}{\value{footnote}}%
		\arabic{indianfoot}}%
}
%%%%%%%%%%%%%%%%%%%%%%%%%%%%%%%%%%%%%%%%%%%%%%% enddefinition

%%%%%%%%%%%%%%%%%%%%%%%%%%%%%%%%%%%%%%%%%%%%%%%
%%%%%%%%%%%%%%%%%%%%%%%%%%%%%%%%%%%%%%%%%%%%%%%
%%%%%%%%%%%%%%%%%%%%%%%%%%%%%%%%%%%%%%%%%%%%%%%
% Allows the definition of multiple authors using:
% \addauthor{<name0>}{<email0>}
% \addauthor{<name1>}{<email1>}
% \addauthor{<name2>}{<email2>}
% \setauthors
%%%%%%%%%%%%%%%%%%%%%%%%%%%%%%%%%%%%%%%%%%%%%%% begindefinition
\usepackage{etoolbox}

\newcounter{columns}
\def \finishedauthors {}

\newcommand{\addauthor}[2]{%
	\ifundef{\shortauthors}{%
		\appto\shortauthors{#1}%
	}{%
		\appto\shortauthors{ \and #1}%
	}%
	\ifundef{\authors}{%
		\appto\authors{\usebeamerfont*{author} #1}%
		\appto\emails{\usebeamerfont*{institute}\itshape\href{mailto:#2}{#2}}%
	}{%
		\appto\authors{&\usebeamerfont*{author} #1}%
		\appto\emails{&\usebeamerfont*{institute}\itshape\href{mailto:#2}{#2}}%
	}%
	\addtocounter{columns}{1}%
}

%\newcommand{\newlineauthors}{%
%	\ifnum\thecolumns=1%
%		\appto\finishedauthors{\begin{tabular}{cc}\authors \\ \emails\end{tabular}}%
%	\fi%
%	\ifnum\thecolumns=2%
%		\appto\finishedauthors{\begin{tabular}{cc}\authors \\ \emails\end{tabular}}%
%	\fi%
%	\ifnum\thecolumns=3%
%		\appto\finishedauthors{\begin{tabular}{ccc}\authors \\ \emails\end{tabular}}%
%	\fi%
%	\ifnum\thecolumns=4%
%		\appto\finishedauthors{\begin{tabular}{cccc}\authors \\ \emails\end{tabular}}%
%	\fi%
%	\ifnum\thecolumns=5%
%		\appto\finishedauthors{\begin{tabular}{ccccc}\authors \\ \emails\end{tabular}}%
%	\fi%
%	\setcounter{columns}{0}%
%	\let\authors=\relax%
%	\let\emails=\relax%
%}

\newcommand{\setauthors}{%
	\ifundef{\finishedauthors}{%
		\ifnum\thecolumns=0%
			\author[]{}%
		\fi%
		\ifnum\thecolumns=1%
			\author[\shortauthors]{\begin{tabular}{c}\authors \\ \emails\end{tabular}}%
		\fi%
		\ifnum\thecolumns=2%
			\author[\shortauthors]{\begin{tabular}{cc}\authors \\ \emails\end{tabular}}%
		\fi%
		\ifnum\thecolumns=3%
			\author[\shortauthors]{\begin{tabular}{ccc}\authors \\ \emails\end{tabular}}%
		\fi%
		\ifnum\thecolumns=4%
			\author[\shortauthors]{\begin{tabular}{cccc}\authors \\ \emails\end{tabular}}%
		\fi%
		\ifnum\thecolumns=5%
			\author[\shortauthors]{\begin{tabular}{ccccc}\authors \\ \emails\end{tabular}}%
		\fi%
	}{%
		\ifnum\thecolumns=0%
			\author[\shortauthors]{\finishedauthors}%
		\fi%
		\ifnum\thecolumns=1%
			\author[\shortauthors]{\finishedauthors\begin{tabular}{c}\authors \\ \emails\end{tabular}}%
		\fi%
		\ifnum\thecolumns=2%
			\author[\shortauthors]{\finishedauthors\begin{tabular}{cc}\authors \\ \emails\end{tabular}}%
		\fi%
		\ifnum\thecolumns=3%
			\author[\shortauthors]{\finishedauthors\begin{tabular}{ccc}\authors \\ \emails\end{tabular}}%
		\fi%
		\ifnum\thecolumns=4%
			\author[\shortauthors]{\finishedauthors\begin{tabular}{cccc}\authors \\ \emails\end{tabular}}%
		\fi%
		\ifnum\thecolumns=5%
			\author[\shortauthors]{\finishedauthors\begin{tabular}{ccccc}\authors \\ \emails\end{tabular}}%
		\fi%
	}%
}
%%%%%%%%%%%%%%%%%%%%%%%%%%%%%%%%%%%%%%%%%%%%%%% enddefinition

%%%%%%%%%%%%%%%%%%%%%%%%%%%%%%%%%%%%%%%%%%%%%%%
%%%%%%%%%%%%%%%%%%%%%%%%%%%%%%%%%%%%%%%%%%%%%%%
%%%%%%%%%%%%%%%%%%%%%%%%%%%%%%%%%%%%%%%%%%%%%%%
% Redefines \thempfootnote to use numbering starting from 1 (should be 0 but buggy):
%%%%%%%%%%%%%%%%%%%%%%%%%%%%%%%%%%%%%%%%%%%%%%% begindefinition
\newcounter{indianmpfoot}
\newcommand{\useindianmpfootnotes}{
	\renewcommand\thempfootnote{\arabic{mpfootnote}}
%	\renewcommand{\thempfootnote}{%
%		\setcounter{indianmpfoot}{0}%
%		\addtocounter{indianmpfoot}{\value{mpfootnote}}%
%		\arabic{indianmpfoot}}%
}
%%%%%%%%%%%%%%%%%%%%%%%%%%%%%%%%%%%%%%%%%%%%%%% enddefinition

%%%%%%%%%%%%%%%%%%%%%%%%%%%%%%%%%%%%%%%%%%%%%%%
%%%%%%%%%%%%%%%%%%%%%%%%%%%%%%%%%%%%%%%%%%%%%%%
%%%%%%%%%%%%%%%%%%%%%%%%%%%%%%%%%%%%%%%%%%%%%%%
% Allows the highlighting of lines in lstlisting environments using the key: 
%     linebackgroundcolor = {\btLstHL{line ranges}}
%%%%%%%%%%%%%%%%%%%%%%%%%%%%%%%%%%%%%%%%%%%%%%% begindefinition
\usepackage{listings}

% Define backgroundcolor
    \usepackage[
%        style=1,   % deprecated option
        skipbelow=\topskip,
        skipabove=\topskip
    ]{mdframed}

    \definecolor{bggray}{rgb}{0.85, 0.85, 0.85}
    \mdfsetup{
        leftmargin = 20pt,
        rightmargin = 20pt,
        backgroundcolor = bggray,
        middlelinecolor = black,
        roundcorner = 15
    }
    \BeforeBeginEnvironment{lstlisting}{\begin{mdframed}\vskip-.5\baselineskip}
    \AfterEndEnvironment{lstlisting}{\end{mdframed}}

\makeatletter
%
% \btIfInRange{number}{range list}{TRUE}{FALSE}
%
% Test if int number <number> is element of a (comma separated) list of ranges
% (such as: {1,3-5,7,10-12,14}) and processes <TRUE> or <FALSE> respectively
%
        \newcount\bt@rangea
        \newcount\bt@rangeb

        \newcommand\btIfInRange[2]{%
            \global\let\bt@inrange\@secondoftwo%
            \edef\bt@rangelist{#2}%
            \foreach \range in \bt@rangelist {%
                \afterassignment\bt@getrangeb%
                \bt@rangea=0\range\relax%
                \pgfmathtruncatemacro\result{ ( #1 >= \bt@rangea) && (#1 <= \bt@rangeb) }%
                \ifnum\result=1\relax%
                    \breakforeach%
                    \global\let\bt@inrange\@firstoftwo%
                \fi%
            }%
            \bt@inrange%
        }

        \newcommand\bt@getrangeb{%
            \@ifnextchar\relax%
            {\bt@rangeb=\bt@rangea}%
            {\@getrangeb}%
        }

        \def\@getrangeb-#1\relax{%
            \ifx\relax#1\relax%
                \bt@rangeb=100000%   \maxdimen is too large for pgfmath
            \else%
                \bt@rangeb=#1\relax%
            \fi%
        }

%
% \btLstHL{range list}
%
    \definecolor{lsthighlight}{RGB}{217, 216, 255}
        \newcommand{\btLstHL}[1]{%
            \btIfInRange{\value{lstnumber}}{#1}%
            {\color{lsthighlight}}%
            {\def\lst@linebgrd}%
        }%

%
% \btInputEmph[listing options]{range list}{file name}
%
        \newcommand{\btLstInputEmph}[3][\empty]{%
            \lstset{%
                linebackgroundcolor=\btLstHL{#2}%
                \lstinputlisting{#3}%
            }% \only
        }

% Patch line number key to call line background macro
        \lst@Key{numbers}{none}{%
            \def\lst@PlaceNumber{\lst@linebgrd}%
            \lstKV@SwitchCases{#1}{%
                none&\\%
                left&\def\lst@PlaceNumber{\llap{\normalfont
                \lst@numberstyle{\thelstnumber}\kern\lst@numbersep}\lst@linebgrd}\\%
                right&\def\lst@PlaceNumber{\rlap{\normalfont
                \kern\linewidth \kern\lst@numbersep
                \lst@numberstyle{\thelstnumber}}\lst@linebgrd}%
            }{%
                \PackageError{Listings}{Numbers #1 unknown}\@ehc%
            }%
        }

% New keys
        \lst@Key{linebackgroundcolor}{}{%
            \def\lst@linebgrdcolor{#1}%
        }
        \lst@Key{linebackgroundsep}{0pt}{%
            \def\lst@linebgrdsep{#1}%
        }
        \lst@Key{linebackgroundwidth}{\linewidth}{%
            \def\lst@linebgrdwidth{#1}%
        }
        \lst@Key{linebackgroundheight}{\ht\strutbox}{%
            \def\lst@linebgrdheight{#1}%
        }
        \lst@Key{linebackgrounddepth}{\dp\strutbox}{%
            \def\lst@linebgrddepth{#1}%
        }
        \lst@Key{linebackgroundcmd}{\color@block}{%
            \def\lst@linebgrdcmd{#1}%
        }

% Line Background macro
        \newcommand{\lst@linebgrd}{%
            \ifx\lst@linebgrdcolor\empty\else
                \rlap{%
                    \lst@basicstyle
                    \color{-.}% By default use the opposite (`-`) of the current color (`.`) as background
                    \lst@linebgrdcolor{%
                        \kern-\dimexpr\lst@linebgrdsep\relax%
                        \lst@linebgrdcmd{\lst@linebgrdwidth}{\lst@linebgrdheight}{\lst@linebgrddepth}%
                    }%
                }%
            \fi
        }

\makeatother
%%%%%%%%%%%%%%%%%%%%%%%%%%%%%%%%%%%%%%%%%%%%%%% enddefinition

%%%%%%%%%%%%%%%%%%%%%%%%%%%%%%%%%%%%%%%%%%%%%%%
%%%%%%%%%%%%%%%%%%%%%%%%%%%%%%%%%%%%%%%%%%%%%%%
%%%%%%%%%%%%%%%%%%%%%%%%%%%%%%%%%%%%%%%%%%%%%%%
% Defines a key=value switch for lstlistings with matchrangestart=<true/false> that allows
% the numbering following the linerange key=value settings:
%%%%%%%%%%%%%%%%%%%%%%%%%%%%%%%%%%%%%%%%%%%%%%% begindefinition
\usepackage{listings}

\makeatletter
\lst@Key{matchrangestart}{false}{\lstKV@SetIf{#1}\lst@ifmatchrangestart}
\def\lst@SkipToFirst{%
    \lst@ifmatchrangestart\c@lstnumber=\numexpr-1+\lst@firstline\fi
    \ifnum \lst@lineno<\lst@firstline
        \def\lst@next{\lst@BeginDropInput\lst@Pmode
        \lst@Let{13}\lst@MSkipToFirst
        \lst@Let{10}\lst@MSkipToFirst}%
        \expandafter\lst@next
    \else
        \expandafter\lst@BOLGobble
    \fi}
\makeatother
%%%%%%%%%%%%%%%%%%%%%%%%%%%%%%%%%%%%%%%%%%%%%%% enddefinition

%%%%%%%%%%%%%%%%%%%%%%%%%%%%%%%%%%%%%%%%%%%%%%%
%%%%%%%%%%%%%%%%%%%%%%%%%%%%%%%%%%%%%%%%%%%%%%%
%%%%%%%%%%%%%%%%%%%%%%%%%%%%%%%%%%%%%%%%%%%%%%%
% Allows the writing of dates specified as \specificdate{YYYY}{MM}{DD}:
%%%%%%%%%%%%%%%%%%%%%%%%%%%%%%%%%%%%%%%%%%%%%%% begindefinition
\usepackage{datenumber}
\newcommand{\specificdate}[3]{%
    \setdatenumber{#1}{#2}{#3}%
    \datedate%
}
%%%%%%%%%%%%%%%%%%%%%%%%%%%%%%%%%%%%%%%%%%%%%%% enddefinition

%%%%%%%%%%%%%%%%%%%%%%%%%%%%%%%%%%%%%%%%%%%%%%%
%%%%%%%%%%%%%%%%%%%%%%%%%%%%%%%%%%%%%%%%%%%%%%%
%%%%%%%%%%%%%%%%%%%%%%%%%%%%%%%%%%%%%%%%%%%%%%%
% Allows inline comments using \ignore{comment text}:
%%%%%%%%%%%%%%%%%%%%%%%%%%%%%%%%%%%%%%%%%%%%%%% begindefinition
\newcommand{\ignore}[1]{}
%%%%%%%%%%%%%%%%%%%%%%%%%%%%%%%%%%%%%%%%%%%%%%% enddefinition

%%%%%%%%%%%%%%%%%%%%%%%%%%%%%%%%%%%%%%%%%%%%%%%
%%%%%%%%%%%%%%%%%%%%%%%%%%%%%%%%%%%%%%%%%%%%%%%
%%%%%%%%%%%%%%%%%%%%%%%%%%%%%%%%%%%%%%%%%%%%%%%
% 
%%%%%%%%%%%%%%%%%%%%%%%%%%%%%%%%%%%%%%%%%%%%%%% begindefinition
\usepackage{tikz}
\usetikzlibrary{matrix}

\makeatletter
\newdimen\multi@col@width
\newdimen\multi@col@margin
\newcount\multi@col@count
\multi@col@width=0pt

\tikzset{
  multicol/.code={%
    \global\multi@col@count=#1\relax
    \global\let\orig@pgfmatrixendcode=\pgfmatrixendcode
    \global\let\orig@pgfmatrixemptycode=\pgfmatrixemptycode
    \def\pgfmatrixendcode##1{\orig@pgfmatrixendcode%
      ##1%
      \pgfutil@tempdima=\pgf@picmaxx
      \global\multi@col@margin=\pgf@picminx
      \advance\pgfutil@tempdima by -\pgf@picminx
      \divide\pgfutil@tempdima by #1\relax
      \global\multi@col@width=\pgfutil@tempdima
      \pgf@picmaxx=.5\multi@col@width
      \pgf@picminx=-.5\multi@col@width
      \global\pgf@picmaxx=\pgf@picmaxx
      \global\pgf@picminx=\pgf@picminx
      \gdef\multi@adjust@position{%
        \setbox\pgf@matrix@cell=\hbox\bgroup
        \hfil\hskip-\multi@col@margin
        \hfil\hskip-.5\multi@col@width
        \box\pgf@matrix@cell
        \egroup
      }%
      \gdef\multi@temp{\aftergroup\multi@adjust@position}%
      \aftergroup\multi@temp
    }
    \gdef\pgfmatrixemptycode{%
      \orig@pgfmatrixemptycode
      \global\advance\multi@col@count by -1\relax
      \global\pgf@picmaxx=.5\multi@col@width
      \global\pgf@picminx=-.5\multi@col@width
      \ifnum\multi@col@count=1\relax
       \global\let\pgfmatrixemptycode=\orig@pgfmatrixemptycode
      \fi
    }
  }
}
\makeatother
%%%%%%%%%%%%%%%%%%%%%%%%%%%%%%%%%%%%%%%%%%%%%%% enddefinition

\settoggle{bwmode}{false}				% enable/disable grayscale mode for the output (see ./tex/command_definitions.tex)

% Settings regarding the used fonts and typographical details:
\usepackage[utf8]{inputenc}					% character encoding of this .tex-file
\usepackage[T1]{fontenc}						% character encoding within the compiled document
\usepackage{libertine\ignore{, libertinust1math}}	% used font/fonts
\usepackage{microtype}						% enables microtypography (can be further configured but the default mode is well)

% Settings regarding the text alignment:
\usepackage
		[
			newcommands,				% the commands \centering, \raggedleft, and \raggedright are redefined to work like \Centering, \RaggedLeft, and \RaggedRight
			newparameters				% the commands \Centering, \RaggedLeft, and \RaggedRight don't behave like vanilla \centering, \raggedleft, and \raggedright 
		]{ragged2e}					% offers new environments for ragged ( and justified) text alignment with many parameters that can be changed using \setlength 
\usepackage{multicol}

% Settings regarding the used languages:
\usepackage[main = english, ngerman]{babel}	% define the used languages (has influence on different things like hyphenation, date format and figure labels)
\babeltags{eng = english, de = ngerman}		% allows to switch between the loaded languages by using the tags like \begin{<tag>} ... \end{<tag>} or \text<tag>{ ... }
\usepackage[
			detect-all]					% use the font settings of the surrounding text for numbers and units set with siunitx
		{siunitx}						% allows the easy typesetting of numbers, units and combinations including lists and ranges
\sisetup{
	range-phrase = \text{--},				% select the phrase between upper and lower bounds of ranges (here: -)
	binary-units = true,					% enable/disable loading of binary prefixes
	per-mode = fraction,					% select how to display \per (symbol: uses exponents; fraction: uses fractions)
	fraction-function = \sfrac}				% select the kind of fraction used in siunitx (e.g. frac, cfrac, rfrac, sfrac, ...)
\DeclareSIUnit{\euro}{\text{\texteuro}}		% define the unit \euro using the €-symbol
\DeclareSIUnit{\usdollar}{\text{US-\textdollar}}	% define the unit \usdollar using the US-$-symbol
\DeclareSIUnit{\cy}{\text{Cyc.}}				% define the unit \cy used for Cycles using the abbreviation Cyc.
\DeclareSIUnit\century{\text{century}}		% define the unit \century
\DeclareSIUnit\year{\text{year}}				% define the unit \year
\DeclareSIUnit\queries{\text{queries}}		% define the unit \queries
\DeclareSIUnit\transactions{\text{txn}}		% define the unit \transactions
\DeclareSIUnit\messages{\text{msg}}		% define the unit \messages

\usepackage{graphicx}				% allows including of graphics and the scaling and rotating of elements
\usepackage{etoolbox}				% toolbox used by packages and classes
\usepackage{xpatch}					% extends the patching facility of etoolbox
%\usepackage{csquotes}				% context sensitive quotation environment e.g. used by biblatex
%\usepackage{datenumber}			% allows to create a number from a date and especially it allows to create a specific date
%\usepackage{tabularx}				% a tabular* environment that can control the width of columns
\usepackage{multirow}				% allows tabular cells spanning multiple rows
\usepackage{prelim2e}				% marks every page as being a preliminary version when this document is compiled as a draft

\usepackage
		[
			style = alphabetic, 		% select the style of the citation and of the bibliography
			backend = biber		% select the backend that processes the .bib file (run the selected backend instead of BibTeX)
		]{biblatex}					% used to create the bibliography, more modern alternative to the standard BibTeX
\addbibresource[datatype = bibtex]{./tex/references.bib}		% loads the specified bibliography file (in BibTeX format) into BibLaTeX
\usepackage{breakcites}				% multiple citations within one \cite break at the end of the line
\usepackage{hyperref}				% allows hyperlinks within the output document (hyperfootnotes = false to make it compatible with package footmisc)
\usepackage{nameref}				% allows the usage of the command \nameref which prints the title of the referenced label instead of its number

\usepackage{tikz}					% extremely powerful facility to create diagrams
\usetikzlibrary{shapes.geometric, shapes.misc, shapes.callouts, shapes.multipart, shapes.symbols}			% provide several shapes besides the standard ones
\usetikzlibrary{arrows.meta}			% 
\usetikzlibrary{decorations.pathreplacing}	% allows decorated paths without having the original (undecorated) line
\usetikzlibrary{decorations.pathmorphing}	% 
\usetikzlibrary{patterns}				% allows the usage of several patterns to fill shapes
\usetikzlibrary{positioning}				% defines additional options for placing nodes conventionally
\usetikzlibrary{calc}					% allows extended coordinate calculation
\usetikzlibrary{spy}					% allows the magnification of parts of a tikz diagram
\usetikzlibrary{chains}				% allows the creation of chains of nodes
\usepackage{pgfplots}				% allows the creation of plots to visualize data
\usepackage{pgfplotstable}			% allows the loading of .csv-files for pgfplots
\usepgflibrary{plotmarks}				% extends the available plot marks used e.g. for pgfplots
\usepgfplotslibrary{fillbetween}			% allows the filling of areas between curves of pgfplots using colors or patterns
\usetikzlibrary{fit}					% 
\usepackage{./tex/tikz-uml}			% 
\usepackage{ifthen}					% allows the usage of the \ifthenelse control structure and some boolean operations with it

\pgfplotsset{compat = 1.14}

\newcommand{\tikzmark}[2]{\tikz[overlay, remember picture, baseline = (#1.base)] \node (#1) {#2};}

\usepackage{xstring}				% 
\usepackage{xfrac}					% adds the \sfrac fraction mode
\usepackage{rotating}				% allows different kinds of rotations for many kinds of elements
\usepackage{stackengine}			% allows the stacking of elements like symbols
\usepackage{bm}						% adds one way of bold math
\usepackage{ulem}					% allows many kinds of text decorations like underlines or strikes
\robustify\uline					% allows the usage of \uline together with typewriter fonts
\normalem							% \emph uses italic fonts to emphasize text

%%%%%%%%%%%%%%%%%%%%%%%%%%%%%%%%%%%%%%%%%%%%%%%%%%%%%%%%%%%%%%%%%%%%%%

\usepackage{url, enumerate, relsize, color, ulem}
\usepackage{anyfontsize}
\usepackage{adjustbox}
\usepackage{listings}

\usepackage{geometry}
\usepackage{setspace}
\usepackage{wrapfig}
\usepackage[para]{footmisc}
\usepackage{calc}

\usepackage{scalerel}
\newcommand\dangersign[1][2ex]{%
  \renewcommand\stacktype{L}%
  \scaleto{\stackon[1.3pt]{\color{red}$\triangle$}{\tiny\textrm{!}}}{#1}%
}

\newcommand{\tabitem}{~~\llap{\textbullet}~~}

\defbibenvironment{bibliography}
  {\list{}
     {\settowidth{\labelwidth}{\usebeamertemplate{bibliography item}}%
      \setlength{\leftmargin}{\labelwidth}%
      \setlength{\labelsep}{\biblabelsep}%
      \addtolength{\leftmargin}{\labelsep}%
      \setlength{\itemsep}{\bibitemsep}%
      \setlength{\parsep}{\bibparsep}}}
  {\endlist}
  {\item}
  
\let\thempfootnote\thefootnote

\mode<presentation> {
\usetheme{Dresden}

\newcommand{\frameofframes}{/}
\newcommand{\setframeofframes}[1]{\renewcommand{\frameofframes}{#1}}
\setframeofframes{of}

\setbeamertemplate{headline}
{%
  \begin{beamercolorbox}[colsep=1.5pt]{upper separation line head}
  \end{beamercolorbox}
  \begin{beamercolorbox}{section in head/foot}
    \vskip2pt\insertsectionnavigationhorizontal{\textwidth}{}{}\vskip2pt
  \end{beamercolorbox}%
    \begin{beamercolorbox}[colsep=1.5pt]{middle separation line head}
    \end{beamercolorbox}
    \begin{beamercolorbox}[ht=2.5ex,dp=1.125ex,%
      leftskip=.3cm,rightskip=.3cm plus1fil]{subsection in head/foot}
      \usebeamerfont{subsection in head/foot}\insertsubsectionhead
      \hfill%
      {\usebeamerfont{frame number}\usebeamercolor[fg]{frame number}\insertframenumber~\frameofframes~\inserttotalframenumber}    \end{beamercolorbox}%
  \begin{beamercolorbox}[colsep=1.5pt]{lower separation line head}
  \end{beamercolorbox}
}

%\setbeamertemplate{navigation symbols}{} % To remove the navigation symbols from the bottom of all slides uncomment this line
}

\usepackage{booktabs} % Allows the use of \toprule, \midrule and \bottomrule in tables

\newcommand*{\ptsans}{\fontfamily{PTSans-TLF}\selectfont}	% toggle to change to the standard font of the University of Kaiserslautern: PT Sans
\DeclareTextFontCommand{\textptsans}{\ptsans}			% environment to change to the standard font of the University of Kaiserslautern: PT Sans

\iftoggle{bwmode}{
	\definecolor{TUblue}{RGB}{0,0,0}					% The blue color used in the logo of the University of Kaiserslautern is printed black when in grayscale mode
	\definecolor{TUred}{RGB}{127,127,127}					% The red color used in the logo of the University of Kaiserslautern is printed black when in grayscale mode
}{
	\definecolor{TUblue}{RGB}{0,96,142}				% The blue color used in the logo of the University of Kaiserslautern
	\definecolor{TUred}{RGB}{188,38,26}				% The red color used in the logo of the University of Kaiserslautern
}

% The following new commands are tikzpicture-environments containing different logos of the University of Kaiserslautern. They are taken from logos published on their website.
% It defines the following commands: \TULogo, \TULogoWithText, \CSLogo

% The logo of the University of Kaiserslautern ():
% Taken from: http://www.uni-kl.de/fileadmin/prum/tupublic/TU_Logo_ohne_Feld/TUKL_LOGO_4C.svg on the 2016-12-14
% Manipulated using: Inkscape (https://inkscape.org/)
% Converted to TikZ using: svg2tikz (https://github.com/kjellmf/svg2tikz) as an Inkscape extension
\newcommand{\TULogo}[1][1]{
	\begin{tikzpicture}[
		y = 5pt,
		x = 5pt,
		opacity = #1
	]
		% Top part:
		\path[fill = TUblue] (2.898, 23.855) -- (2.898, 20.561) -- (24.299, 20.561)  -- (24.299, 24.034) -- (13.541, 27.197) -- cycle;
		% Left part:
		\path[fill = TUblue] (5.679, 19.307) -- (9.993, 19.307) -- (4.3, 0)  -- (0, 0) -- cycle;
		% Top rectangle:
		\path[fill = TUblue] (17.481, 19.316) rectangle (22.247, 14.727);
		% Middle rectangle:
		\path[fill = TUred] (17.481, 11.953) rectangle (22.247, 7.363);
		% Bottom rectangle:
		\path[fill = TUblue] (17.481, 4.59) rectangle (22.247, 0);
	\end{tikzpicture}
}

% The logo with text of the University of Kaiserslautern:
% Taken from: http://www.uni-kl.de/fileadmin/prum/tupublic/TU_Logo_ohne_Feld/TUKL_LOGO_4C.svg on the 2016-12-14
% Converted to TikZ using: svg2tikz (https://github.com/kjellmf/svg2tikz) as an Inkscape (https://inkscape.org/) extension
\newcommand{\TULogoWithText}{
	\begin{tikzpicture}[
		y = 0.8pt,
		x = 0.8pt,
		yscale = -1,
		xscale = 1
	]
		% KAISERSLAUTERN:
		\path[fill = TUblue] (164.0310,41.3410) -- (164.0310,36.5480) -- (163.8200,35.1050) -- (163.8890,35.1050) -- (164.6070,36.5480) -- (168.0610,41.4050) -- (169.3700,41.4050) -- (169.3700,32.1510) -- (167.6650,32.1510) -- (167.6650,36.9830) -- (167.8760,38.3870) -- (167.8120,38.3870) -- (167.1170,36.9810) -- (163.6440,32.0860) -- (162.3270,32.0860) -- (162.3270,41.3400) -- (164.0310,41.3400) -- cycle(155.4790,33.6880) .. controls (155.5790,33.6630) and (155.7170,33.6450) .. (155.8970,33.6370) .. controls (156.0780,33.6290) and (156.2590,33.6250) .. (156.4440,33.6250) .. controls (156.9200,33.6250) and (157.2800,33.7360) .. (157.5230,33.9570) .. controls (157.7670,34.1800) and (157.8890,34.4880) .. (157.8890,34.8810) .. controls (157.8890,35.4060) and (157.7410,35.7830) .. (157.4420,36.0120) .. controls (157.1450,36.2410) and (156.7460,36.3540) .. (156.2450,36.3540) -- (155.4790,36.3540) -- (155.4790,33.6880) -- cycle(155.4790,41.3410) -- (155.4790,37.5730) -- (156.4560,37.7820) -- (158.5240,41.3410) -- (160.5940,41.3410) -- (158.5110,37.8720) -- (157.8510,37.4330) .. controls (158.3790,37.2490) and (158.8020,36.9190) .. (159.1200,36.4470) .. controls (159.4350,35.9720) and (159.5950,35.3570) .. (159.5950,34.6050) .. controls (159.5950,34.1010) and (159.5020,33.6810) .. (159.3150,33.3430) .. controls (159.1290,33.0070) and (158.8800,32.7410) .. (158.5680,32.5480) .. controls (158.2570,32.3570) and (157.9040,32.2200) .. (157.5090,32.1420) .. controls (157.1140,32.0620) and (156.7140,32.0230) .. (156.3090,32.0230) .. controls (156.1320,32.0230) and (155.9380,32.0290) .. (155.7270,32.0370) .. controls (155.5160,32.0450) and (155.3000,32.0580) .. (155.0780,32.0760) .. controls (154.8540,32.0920) and (154.6310,32.1170) .. (154.4050,32.1480) .. controls (154.1790,32.1810) and (153.9700,32.2140) .. (153.7770,32.2480) -- (153.7770,41.3420) -- (155.4790,41.3420) -- cycle(151.9100,41.3410) -- (151.9100,39.7390) -- (148.1030,39.7390) -- (148.1030,37.4970) -- (151.5180,37.4970) -- (151.5180,35.8930) -- (148.1030,35.8930) -- (148.1030,33.7520) -- (151.8450,33.7520) -- (151.8450,32.1500) -- (146.3980,32.1500) -- (146.3980,41.3390) -- (151.9100,41.3390) -- cycle(140.2020,33.7530) -- (140.2020,41.3410) -- (141.9070,41.3410) -- (141.9070,33.7530) -- (144.6810,33.7530) -- (144.6810,32.1510) -- (137.4200,32.1510) -- (137.4200,33.7530) -- (140.2020,33.7530) -- cycle(132.7710,41.5010) .. controls (133.2500,41.5010) and (133.6930,41.4350) .. (134.0960,41.3040) .. controls (134.5000,41.1710) and (134.8420,40.9640) .. (135.1220,40.6850) .. controls (135.4020,40.4060) and (135.6200,40.0500) .. (135.7770,39.6210) .. controls (135.9320,39.1910) and (136.0100,38.6800) .. (136.0100,38.0860) -- (136.0100,32.1520) -- (134.3050,32.1520) -- (134.3050,37.9590) .. controls (134.3050,38.6390) and (134.1820,39.1310) .. (133.9360,39.4390) .. controls (133.6890,39.7460) and (133.2900,39.9000) .. (132.7390,39.9000) .. controls (132.4610,39.9000) and (132.2170,39.8670) .. (132.0070,39.8000) .. controls (131.7970,39.7330) and (131.6220,39.6240) .. (131.4800,39.4700) .. controls (131.3390,39.3160) and (131.2360,39.1160) .. (131.1690,38.8680) .. controls (131.1020,38.6200) and (131.0700,38.3150) .. (131.0700,37.9580) -- (131.0700,32.1510) -- (129.3650,32.1510) -- (129.3650,38.3090) .. controls (129.3660,40.4370) and (130.5010,41.5010) .. (132.7710,41.5010)(123.9030,35.8310) -- (124.1770,34.3760) -- (124.2410,34.3760) -- (124.5220,35.8170) -- (125.2030,37.8620) -- (123.2280,37.8620) -- (123.9030,35.8310) -- cycle(122.0790,41.3410) -- (122.7000,39.4640) -- (125.7220,39.4640) -- (126.3310,41.3410) -- (128.0350,41.3410) -- (124.7280,32.0870) -- (123.4420,32.0870) -- (120.2810,41.3410) -- (122.0790,41.3410) -- cycle(119.2220,41.3410) -- (119.2220,39.7390) -- (115.1150,39.7390) -- (115.1150,32.1510) -- (113.4100,32.1510) -- (113.4100,41.3400) -- (119.2220,41.3400) -- cycle(106.5570,41.3410) .. controls (107.0430,41.4640) and (107.6040,41.5290) .. (108.2410,41.5290) .. controls (108.7220,41.5290) and (109.1670,41.4700) .. (109.5720,41.3530) .. controls (109.9760,41.2380) and (110.3220,41.0600) .. (110.6060,40.8240) .. controls (110.8910,40.5880) and (111.1130,40.2910) .. (111.2700,39.9330) .. controls (111.4280,39.5760) and (111.5070,39.1480) .. (111.5070,38.6540) .. controls (111.5070,38.1770) and (111.4160,37.7790) .. (111.2330,37.4570) .. controls (111.0510,37.1350) and (110.8230,36.8650) .. (110.5450,36.6460) .. controls (110.2680,36.4290) and (109.9500,36.2380) .. (109.5900,36.0740) .. controls (109.2310,35.9080) and (108.8940,35.7540) .. (108.5810,35.6110) .. controls (108.2680,35.4680) and (107.9950,35.3080) .. (107.7620,35.1360) .. controls (107.5300,34.9620) and (107.4120,34.7470) .. (107.4120,34.4880) .. controls (107.4120,34.2090) and (107.5230,33.9880) .. (107.7470,33.8220) .. controls (107.9690,33.6580) and (108.2910,33.5760) .. (108.7110,33.5760) .. controls (109.1480,33.5760) and (109.5540,33.6250) .. (109.9300,33.7220) .. controls (110.3050,33.8240) and (110.5860,33.9330) .. (110.7740,34.0540) -- (111.3070,32.5250) .. controls (111.0130,32.3450) and (110.6360,32.2090) .. (110.1770,32.1150) .. controls (109.7170,32.0190) and (109.2280,31.9720) .. (108.7110,31.9720) .. controls (108.2630,31.9720) and (107.8570,32.0250) .. (107.4920,32.1300) .. controls (107.1270,32.2370) and (106.8100,32.4000) .. (106.5440,32.6200) .. controls (106.2780,32.8400) and (106.0700,33.1180) .. (105.9250,33.4500) .. controls (105.7800,33.7840) and (105.7080,34.1750) .. (105.7080,34.6240) .. controls (105.7080,35.1340) and (105.8100,35.5540) .. (106.0150,35.8840) .. controls (106.2200,36.2120) and (106.4790,36.4880) .. (106.7910,36.7060) .. controls (107.1030,36.9250) and (107.4380,37.1140) .. (107.7990,37.2740) .. controls (108.1600,37.4340) and (108.4950,37.5830) .. (108.8080,37.7210) .. controls (109.1210,37.8620) and (109.3660,38.0140) .. (109.5400,38.1840) .. controls (109.7140,38.3540) and (109.8030,38.5790) .. (109.8030,38.8540) .. controls (109.8030,39.2150) and (109.6600,39.4830) .. (109.3760,39.6610) .. controls (109.0910,39.8370) and (108.6810,39.9250) .. (108.1450,39.9250) .. controls (107.9260,39.9250) and (107.7140,39.9090) .. (107.5080,39.8740) .. controls (107.3010,39.8410) and (107.1050,39.7980) .. (106.9200,39.7450) .. controls (106.7330,39.6920) and (106.5660,39.6360) .. (106.4190,39.5750) .. controls (106.2700,39.5130) and (106.1480,39.4580) .. (106.0540,39.4070) -- (105.4770,40.9620) .. controls (105.7110,41.0890) and (106.0710,41.2160) .. (106.5570,41.3410)(99.0840,33.6880) .. controls (99.1830,33.6630) and (99.3220,33.6450) .. (99.5020,33.6370) .. controls (99.6820,33.6290) and (99.8640,33.6250) .. (100.0480,33.6250) .. controls (100.5230,33.6250) and (100.8830,33.7360) .. (101.1270,33.9570) .. controls (101.3720,34.1800) and (101.4940,34.4880) .. (101.4940,34.8810) .. controls (101.4940,35.4060) and (101.3450,35.7830) .. (101.0470,36.0120) .. controls (100.7490,36.2410) and (100.3500,36.3540) .. (99.8490,36.3540) -- (99.0840,36.3540) -- (99.0840,33.6880) -- cycle(99.0840,41.3410) -- (99.0840,37.5730) -- (100.0600,37.7820) -- (102.1280,41.3410) -- (104.1980,41.3410) -- (102.1160,37.8720) -- (101.4550,37.4330) .. controls (101.9840,37.2490) and (102.4070,36.9190) .. (102.7230,36.4470) .. controls (103.0400,35.9720) and (103.1990,35.3570) .. (103.1990,34.6050) .. controls (103.1990,34.1010) and (103.1050,33.6810) .. (102.9200,33.3430) .. controls (102.7330,33.0070) and (102.4850,32.7410) .. (102.1730,32.5480) .. controls (101.8610,32.3570) and (101.5080,32.2200) .. (101.1130,32.1420) .. controls (100.7180,32.0620) and (100.3190,32.0230) .. (99.9140,32.0230) .. controls (99.7360,32.0230) and (99.5420,32.0290) .. (99.3320,32.0370) .. controls (99.1210,32.0450) and (98.9040,32.0580) .. (98.6810,32.0760) .. controls (98.4580,32.0920) and (98.2340,32.1170) .. (98.0090,32.1480) .. controls (97.7840,32.1810) and (97.5740,32.2140) .. (97.3800,32.2480) -- (97.3800,41.3420) -- (99.0840,41.3420) -- cycle(95.2520,41.3410) -- (95.2520,39.7390) -- (91.4460,39.7390) -- (91.4460,37.4970) -- (94.8610,37.4970) -- (94.8610,35.8930) -- (91.4460,35.8930) -- (91.4460,33.7520) -- (95.1880,33.7520) -- (95.1880,32.1500) -- (89.7410,32.1500) -- (89.7410,41.3390) -- (95.2520,41.3390) -- cycle(82.6260,41.3410) .. controls (83.1120,41.4640) and (83.6730,41.5290) .. (84.3090,41.5290) .. controls (84.7910,41.5290) and (85.2350,41.4700) .. (85.6400,41.3530) .. controls (86.0450,41.2380) and (86.3900,41.0600) .. (86.6750,40.8240) .. controls (86.9590,40.5880) and (87.1810,40.2910) .. (87.3390,39.9330) .. controls (87.4970,39.5750) and (87.5760,39.1480) .. (87.5760,38.6540) .. controls (87.5760,38.1770) and (87.4840,37.7790) .. (87.3020,37.4570) .. controls (87.1200,37.1350) and (86.8900,36.8650) .. (86.6130,36.6460) .. controls (86.3360,36.4290) and (86.0180,36.2380) .. (85.6580,36.0740) .. controls (85.2990,35.9080) and (84.9620,35.7540) .. (84.6490,35.6110) .. controls (84.3360,35.4680) and (84.0630,35.3080) .. (83.8300,35.1360) .. controls (83.5970,34.9620) and (83.4800,34.7470) .. (83.4800,34.4880) .. controls (83.4800,34.2090) and (83.5920,33.9880) .. (83.8150,33.8220) .. controls (84.0370,33.6580) and (84.3590,33.5760) .. (84.7780,33.5760) .. controls (85.2160,33.5760) and (85.6220,33.6250) .. (85.9980,33.7220) .. controls (86.3730,33.8240) and (86.6540,33.9330) .. (86.8410,34.0540) -- (87.3750,32.5250) .. controls (87.0810,32.3450) and (86.7040,32.2090) .. (86.2450,32.1150) .. controls (85.7850,32.0190) and (85.2960,31.9720) .. (84.7780,31.9720) .. controls (84.3310,31.9720) and (83.9250,32.0250) .. (83.5600,32.1300) .. controls (83.1950,32.2370) and (82.8790,32.4000) .. (82.6120,32.6200) .. controls (82.3440,32.8410) and (82.1380,33.1180) .. (81.9930,33.4500) .. controls (81.8480,33.7840) and (81.7760,34.1750) .. (81.7760,34.6240) .. controls (81.7760,35.1340) and (81.8780,35.5540) .. (82.0830,35.8840) .. controls (82.2880,36.2120) and (82.5470,36.4880) .. (82.8590,36.7060) .. controls (83.1710,36.9240) and (83.5070,37.1140) .. (83.8670,37.2740) .. controls (84.2270,37.4340) and (84.5630,37.5830) .. (84.8760,37.7210) .. controls (85.1890,37.8620) and (85.4330,38.0140) .. (85.6080,38.1840) .. controls (85.7830,38.3540) and (85.8710,38.5790) .. (85.8710,38.8540) .. controls (85.8710,39.2150) and (85.7280,39.4830) .. (85.4430,39.6610) .. controls (85.1590,39.8370) and (84.7480,39.9250) .. (84.2130,39.9250) .. controls (83.9940,39.9250) and (83.7820,39.9090) .. (83.5760,39.8740) .. controls (83.3690,39.8410) and (83.1730,39.7980) .. (82.9880,39.7450) .. controls (82.8020,39.6920) and (82.6350,39.6360) .. (82.4870,39.5750) .. controls (82.3380,39.5130) and (82.2160,39.4580) .. (82.1210,39.4070) -- (81.5450,40.9620) .. controls (81.7800,41.0890) and (82.1400,41.2160) .. (82.6260,41.3410)(79.7610,32.1510) -- (78.0560,32.1510) -- (78.0560,41.3400) -- (79.7610,41.3400) -- (79.7610,32.1510) -- cycle(72.1060,35.8310) -- (72.3820,34.3760) -- (72.4460,34.3760) -- (72.7280,35.8170) -- (73.4070,37.8620) -- (71.4340,37.8620) -- (72.1060,35.8310) -- cycle(70.2820,41.3410) -- (70.9030,39.4640) -- (73.9260,39.4640) -- (74.5350,41.3410) -- (76.2400,41.3410) -- (72.9330,32.0870) -- (71.6470,32.0870) -- (68.4850,41.3410) -- (70.2820,41.3410) -- cycle(62.0580,41.3410) -- (62.0580,37.4560) -- (62.5690,37.4560) -- (65.2560,41.3410) -- (67.4730,41.3410) -- (64.4440,37.0620) -- (63.6760,36.5440) -- (64.3890,36.0480) -- (67.0700,32.1520) -- (65.0190,32.1520) -- (62.4820,36.0430) -- (62.0580,36.2210) -- (62.0580,32.1530) -- (60.3530,32.1530) -- (60.3530,41.3420) -- (62.0580,41.3420) -- cycle;
		% TECHNISCHE UNIVERSITÄT:
		\path[fill=TUred] (166.8350,23.1740) -- (166.8350,28.6690) -- (168.0690,28.6690) -- (168.0690,23.1740) -- (170.0790,23.1740) -- (170.0790,22.0140) -- (164.8210,22.0140) -- (164.8210,23.1740) -- (166.8350,23.1740) -- cycle(162.7020,21.4690) .. controls (162.8160,21.5720) and (162.9970,21.6240) .. (163.2450,21.6240) .. controls (163.4980,21.6240) and (163.6830,21.5720) .. (163.7950,21.4690) .. controls (163.9080,21.3650) and (163.9640,21.2260) .. (163.9640,21.0510) .. controls (163.9640,20.8740) and (163.9080,20.7310) .. (163.7950,20.6240) .. controls (163.6830,20.5170) and (163.4980,20.4640) .. (163.2450,20.4640) .. controls (162.9970,20.4640) and (162.8160,20.5180) .. (162.7020,20.6270) .. controls (162.5870,20.7350) and (162.5300,20.8770) .. (162.5300,21.0510) .. controls (162.5300,21.2260) and (162.5870,21.3650) .. (162.7020,21.4690)(160.6690,21.4690) .. controls (160.7840,21.5720) and (160.9680,21.6240) .. (161.2210,21.6240) .. controls (161.4660,21.6240) and (161.6460,21.5720) .. (161.7600,21.4690) .. controls (161.8740,21.3650) and (161.9320,21.2260) .. (161.9320,21.0510) .. controls (161.9320,20.8740) and (161.8750,20.7310) .. (161.7620,20.6240) .. controls (161.6500,20.5170) and (161.4690,20.4640) .. (161.2210,20.4640) .. controls (160.9680,20.4640) and (160.7840,20.5180) .. (160.6690,20.6270) .. controls (160.5560,20.7350) and (160.4970,20.8770) .. (160.4970,21.0510) .. controls (160.4970,21.2260) and (160.5560,21.3650) .. (160.6690,21.4690)(161.9410,24.6780) -- (162.1400,23.6240) -- (162.1870,23.6240) -- (162.3910,24.6680) -- (162.8820,26.1490) -- (161.4530,26.1490) -- (161.9410,24.6780) -- cycle(160.6200,28.6690) -- (161.0700,27.3090) -- (163.2580,27.3090) -- (163.6990,28.6690) -- (164.9330,28.6690) -- (162.5380,21.9670) -- (161.6060,21.9670) -- (159.3170,28.6690) -- (160.6200,28.6690) -- cycle(156.2580,23.1740) -- (156.2580,28.6690) -- (157.4920,28.6690) -- (157.4920,23.1740) -- (159.5020,23.1740) -- (159.5020,22.0140) -- (154.2440,22.0140) -- (154.2440,23.1740) -- (156.2580,23.1740) -- cycle(153.3940,22.0140) -- (152.1600,22.0140) -- (152.1600,28.6690) -- (153.3940,28.6690) -- (153.3940,22.0140) -- cycle(147.3960,28.6680) .. controls (147.7480,28.7580) and (148.1540,28.8030) .. (148.6150,28.8030) .. controls (148.9640,28.8030) and (149.2850,28.7610) .. (149.5790,28.6770) .. controls (149.8720,28.5930) and (150.1220,28.4650) .. (150.3280,28.2940) .. controls (150.5340,28.1220) and (150.6940,27.9070) .. (150.8080,27.6480) .. controls (150.9220,27.3900) and (150.9800,27.0810) .. (150.9800,26.7220) .. controls (150.9800,26.3770) and (150.9140,26.0880) .. (150.7820,25.8540) .. controls (150.6500,25.6210) and (150.4840,25.4260) .. (150.2830,25.2680) .. controls (150.0830,25.1110) and (149.8520,24.9720) .. (149.5920,24.8530) .. controls (149.3310,24.7340) and (149.0880,24.6220) .. (148.8620,24.5190) .. controls (148.6340,24.4150) and (148.4370,24.3000) .. (148.2690,24.1740) .. controls (148.1000,24.0490) and (148.0160,23.8930) .. (148.0160,23.7060) .. controls (148.0160,23.5040) and (148.0950,23.3430) .. (148.2570,23.2230) .. controls (148.4180,23.1040) and (148.6520,23.0440) .. (148.9550,23.0440) .. controls (149.2710,23.0440) and (149.5660,23.0800) .. (149.8370,23.1520) .. controls (150.1090,23.2240) and (150.3140,23.3040) .. (150.4480,23.3920) -- (150.8350,22.2850) .. controls (150.6220,22.1540) and (150.3500,22.0550) .. (150.0170,21.9870) .. controls (149.6840,21.9180) and (149.3290,21.8840) .. (148.9550,21.8840) .. controls (148.6300,21.8840) and (148.3360,21.9220) .. (148.0720,21.9990) .. controls (147.8080,22.0760) and (147.5790,22.1940) .. (147.3850,22.3530) .. controls (147.1920,22.5130) and (147.0420,22.7130) .. (146.9380,22.9550) .. controls (146.8330,23.1960) and (146.7800,23.4790) .. (146.7800,23.8040) .. controls (146.7800,24.1740) and (146.8540,24.4780) .. (147.0030,24.7160) .. controls (147.1510,24.9550) and (147.3390,25.1530) .. (147.5650,25.3120) .. controls (147.7910,25.4710) and (148.0350,25.6080) .. (148.2940,25.7240) .. controls (148.5560,25.8390) and (148.7990,25.9480) .. (149.0250,26.0480) .. controls (149.2520,26.1490) and (149.4280,26.2610) .. (149.5550,26.3840) .. controls (149.6820,26.5070) and (149.7450,26.6680) .. (149.7450,26.8690) .. controls (149.7450,27.1300) and (149.6410,27.3240) .. (149.4350,27.4520) .. controls (149.2290,27.5790) and (148.9320,27.6430) .. (148.5430,27.6430) .. controls (148.3850,27.6430) and (148.2310,27.6310) .. (148.0820,27.6060) .. controls (147.9330,27.5820) and (147.7900,27.5510) .. (147.6560,27.5130) .. controls (147.5210,27.4740) and (147.4000,27.4330) .. (147.2930,27.3890) .. controls (147.1860,27.3450) and (147.0980,27.3040) .. (147.0290,27.2670) -- (146.6110,28.3940) .. controls (146.7820,28.4860) and (147.0430,28.5770) .. (147.3960,28.6680)(142.4110,23.1280) .. controls (142.4820,23.1090) and (142.5830,23.0970) .. (142.7140,23.0900) .. controls (142.8440,23.0840) and (142.9770,23.0810) .. (143.1100,23.0810) .. controls (143.4540,23.0810) and (143.7140,23.1620) .. (143.8920,23.3220) .. controls (144.0690,23.4830) and (144.1570,23.7060) .. (144.1570,23.9910) .. controls (144.1570,24.3710) and (144.0490,24.6440) .. (143.8340,24.8100) .. controls (143.6180,24.9750) and (143.3290,25.0580) .. (142.9660,25.0580) -- (142.4120,25.0580) -- (142.4120,23.1280) -- cycle(142.4110,28.6690) -- (142.4110,25.9400) -- (143.1180,26.0910) -- (144.6150,28.6690) -- (146.1140,28.6690) -- (144.6060,26.1570) -- (144.1270,25.8380) .. controls (144.5100,25.7050) and (144.8160,25.4670) .. (145.0450,25.1230) .. controls (145.2750,24.7800) and (145.3900,24.3360) .. (145.3900,23.7910) .. controls (145.3900,23.4260) and (145.3230,23.1210) .. (145.1880,22.8770) .. controls (145.0520,22.6330) and (144.8730,22.4410) .. (144.6470,22.3020) .. controls (144.4200,22.1620) and (144.1660,22.0640) .. (143.8790,22.0070) .. controls (143.5920,21.9500) and (143.3040,21.9210) .. (143.0100,21.9210) .. controls (142.8820,21.9210) and (142.7410,21.9240) .. (142.5890,21.9300) .. controls (142.4370,21.9370) and (142.2790,21.9460) .. (142.1180,21.9580) .. controls (141.9560,21.9710) and (141.7940,21.9880) .. (141.6310,22.0110) .. controls (141.4680,22.0350) and (141.3170,22.0590) .. (141.1760,22.0830) -- (141.1760,28.6690) -- (142.4110,28.6690) -- cycle(140.1590,28.6690) -- (140.1590,27.5080) -- (137.4030,27.5080) -- (137.4030,25.8840) -- (139.8760,25.8840) -- (139.8760,24.7240) -- (137.4030,24.7240) -- (137.4030,23.1740) -- (140.1130,23.1740) -- (140.1130,22.0140) -- (136.1680,22.0140) -- (136.1680,28.6690) -- (140.1590,28.6690) -- cycle(132.1210,28.7150) -- (133.0490,28.7150) -- (135.5030,22.0140) -- (134.2700,22.0140) -- (132.9980,25.9120) -- (132.8090,27.0540) -- (132.7620,27.0540) -- (132.5900,25.9210) -- (131.2550,22.0140) -- (129.7500,22.0140) -- (132.1210,28.7150) -- cycle(129.0200,22.0140) -- (127.7860,22.0140) -- (127.7860,28.6690) -- (129.0200,28.6690) -- (129.0200,22.0140) -- cycle(122.5950,28.6690) -- (122.5950,25.1970) -- (122.4430,24.1530) -- (122.4940,24.1530) -- (123.0150,25.1970) -- (125.5160,28.7150) -- (126.4620,28.7150) -- (126.4620,22.0140) -- (125.2280,22.0140) -- (125.2280,25.5130) -- (125.3800,26.5290) -- (125.3340,26.5290) -- (124.8300,25.5120) -- (122.3160,21.9670) -- (121.3620,21.9670) -- (121.3620,28.6690) -- (122.5950,28.6690) -- cycle(117.7930,28.7860) .. controls (118.1410,28.7860) and (118.4600,28.7370) .. (118.7530,28.6410) .. controls (119.0450,28.5450) and (119.2930,28.3960) .. (119.4960,28.1930) .. controls (119.6990,27.9900) and (119.8570,27.7340) .. (119.9700,27.4230) .. controls (120.0830,27.1120) and (120.1400,26.7410) .. (120.1400,26.3110) -- (120.1400,22.0140) -- (118.9060,22.0140) -- (118.9060,26.2180) .. controls (118.9060,26.7100) and (118.8160,27.0680) .. (118.6370,27.2900) .. controls (118.4590,27.5130) and (118.1700,27.6250) .. (117.7710,27.6250) .. controls (117.5700,27.6250) and (117.3930,27.6010) .. (117.2410,27.5530) .. controls (117.0890,27.5050) and (116.9620,27.4250) .. (116.8600,27.3140) .. controls (116.7560,27.2020) and (116.6820,27.0570) .. (116.6330,26.8770) .. controls (116.5850,26.6980) and (116.5620,26.4780) .. (116.5620,26.2180) -- (116.5620,22.0140) -- (115.3280,22.0140) -- (115.3280,26.4740) .. controls (115.3270,28.0140) and (116.1490,28.7860) .. (117.7930,28.7860)(111.8860,28.6690) -- (111.8860,27.5080) -- (109.1290,27.5080) -- (109.1290,25.8840) -- (111.6030,25.8840) -- (111.6030,24.7240) -- (109.1290,24.7240) -- (109.1290,23.1740) -- (111.8390,23.1740) -- (111.8390,22.0140) -- (107.8960,22.0140) -- (107.8960,28.6690) -- (111.8860,28.6690) -- cycle(102.8960,28.6690) -- (102.8960,25.8840) -- (105.3360,25.8840) -- (105.3360,28.6690) -- (106.5710,28.6690) -- (106.5710,22.0140) -- (105.3360,22.0140) -- (105.3360,24.7240) -- (102.8960,24.7240) -- (102.8960,22.0140) -- (101.6620,22.0140) -- (101.6620,28.6690) -- (102.8960,28.6690) -- cycle(96.3130,26.9470) .. controls (96.4580,27.3870) and (96.6530,27.7450) .. (96.8990,28.0200) .. controls (97.1440,28.2950) and (97.4450,28.4940) .. (97.8020,28.6180) .. controls (98.1590,28.7420) and (98.5360,28.8030) .. (98.9360,28.8030) .. controls (99.2630,28.8030) and (99.5850,28.7720) .. (99.8990,28.7090) .. controls (100.2120,28.6460) and (100.4720,28.5420) .. (100.6750,28.3980) -- (100.4060,27.3360) .. controls (100.2600,27.4260) and (100.0890,27.5000) .. (99.8930,27.5570) .. controls (99.6960,27.6140) and (99.4560,27.6430) .. (99.1710,27.6430) .. controls (98.8680,27.6430) and (98.6000,27.5870) .. (98.3680,27.4760) .. controls (98.1360,27.3640) and (97.9430,27.2080) .. (97.7880,27.0080) .. controls (97.6340,26.8080) and (97.5180,26.5670) .. (97.4430,26.2850) .. controls (97.3670,26.0030) and (97.3290,25.6900) .. (97.3290,25.3460) .. controls (97.3290,24.5580) and (97.4890,23.9780) .. (97.8090,23.6040) .. controls (98.1290,23.2310) and (98.5520,23.0440) .. (99.0780,23.0440) .. controls (99.3630,23.0440) and (99.6050,23.0600) .. (99.8050,23.0910) .. controls (100.0040,23.1220) and (100.1770,23.1730) .. (100.3220,23.2440) -- (100.5770,22.1390) .. controls (100.4070,22.0680) and (100.1910,22.0080) .. (99.9280,21.9580) .. controls (99.6650,21.9090) and (99.3430,21.8840) .. (98.9620,21.8840) .. controls (98.6070,21.8840) and (98.2520,21.9430) .. (97.8970,22.0610) .. controls (97.5430,22.1780) and (97.2360,22.3720) .. (96.9750,22.6410) .. controls (96.7140,22.9110) and (96.5020,23.2660) .. (96.3390,23.7060) .. controls (96.1760,24.1450) and (96.0940,24.6920) .. (96.0940,25.3460) .. controls (96.0950,25.9730) and (96.1680,26.5070) .. (96.3130,26.9470)(91.6750,28.6680) .. controls (92.0270,28.7580) and (92.4330,28.8030) .. (92.8940,28.8030) .. controls (93.2430,28.8030) and (93.5640,28.7610) .. (93.8570,28.6770) .. controls (94.1510,28.5930) and (94.4010,28.4650) .. (94.6070,28.2940) .. controls (94.8130,28.1220) and (94.9730,27.9070) .. (95.0870,27.6480) .. controls (95.2020,27.3900) and (95.2590,27.0810) .. (95.2590,26.7220) .. controls (95.2590,26.3770) and (95.1930,26.0880) .. (95.0610,25.8540) .. controls (94.9290,25.6210) and (94.7630,25.4260) .. (94.5620,25.2680) .. controls (94.3620,25.1110) and (94.1310,24.9720) .. (93.8710,24.8530) .. controls (93.6100,24.7340) and (93.3660,24.6220) .. (93.1400,24.5190) .. controls (92.9130,24.4150) and (92.7150,24.3000) .. (92.5470,24.1740) .. controls (92.3780,24.0490) and (92.2940,23.8930) .. (92.2940,23.7060) .. controls (92.2940,23.5040) and (92.3740,23.3430) .. (92.5360,23.2230) .. controls (92.6970,23.1040) and (92.9290,23.0440) .. (93.2340,23.0440) .. controls (93.5500,23.0440) and (93.8450,23.0800) .. (94.1160,23.1520) .. controls (94.3880,23.2240) and (94.5920,23.3040) .. (94.7270,23.3920) -- (95.1140,22.2850) .. controls (94.9010,22.1540) and (94.6280,22.0550) .. (94.2950,21.9870) .. controls (93.9620,21.9180) and (93.6080,21.8840) .. (93.2340,21.8840) .. controls (92.9090,21.8840) and (92.6150,21.9220) .. (92.3510,21.9990) .. controls (92.0870,22.0760) and (91.8580,22.1940) .. (91.6650,22.3530) .. controls (91.4710,22.5130) and (91.3220,22.7130) .. (91.2170,22.9550) .. controls (91.1120,23.1960) and (91.0590,23.4790) .. (91.0590,23.8040) .. controls (91.0590,24.1740) and (91.1330,24.4780) .. (91.2820,24.7160) .. controls (91.4300,24.9550) and (91.6180,25.1530) .. (91.8430,25.3120) .. controls (92.0690,25.4710) and (92.3120,25.6080) .. (92.5730,25.7240) .. controls (92.8340,25.8390) and (93.0780,25.9480) .. (93.3040,26.0480) .. controls (93.5310,26.1490) and (93.7080,26.2610) .. (93.8340,26.3840) .. controls (93.9610,26.5070) and (94.0250,26.6680) .. (94.0250,26.8690) .. controls (94.0250,27.1300) and (93.9210,27.3240) .. (93.7150,27.4520) .. controls (93.5090,27.5790) and (93.2120,27.6430) .. (92.8240,27.6430) .. controls (92.6660,27.6430) and (92.5120,27.6310) .. (92.3620,27.6060) .. controls (92.2130,27.5820) and (92.0710,27.5510) .. (91.9370,27.5130) .. controls (91.8020,27.4740) and (91.6810,27.4330) .. (91.5740,27.3890) .. controls (91.4660,27.3450) and (91.3780,27.3040) .. (91.3090,27.2670) -- (90.8920,28.3940) .. controls (91.0620,28.4860) and (91.3230,28.5770) .. (91.6750,28.6680)(89.8370,22.0140) -- (88.6030,22.0140) -- (88.6030,28.6690) -- (89.8370,28.6690) -- (89.8370,22.0140) -- cycle(83.4140,28.6690) -- (83.4140,25.1970) -- (83.2600,24.1530) -- (83.3110,24.1530) -- (83.8310,25.1970) -- (86.3330,28.7150) -- (87.2790,28.7150) -- (87.2790,22.0140) -- (86.0450,22.0140) -- (86.0450,25.5130) -- (86.1970,26.5290) -- (86.1510,26.5290) -- (85.6470,25.5120) -- (83.1330,21.9670) -- (82.1790,21.9670) -- (82.1790,28.6690) -- (83.4140,28.6690) -- cycle(77.1800,28.6690) -- (77.1800,25.8840) -- (79.6210,25.8840) -- (79.6210,28.6690) -- (80.8550,28.6690) -- (80.8550,22.0140) -- (79.6210,22.0140) -- (79.6210,24.7240) -- (77.1800,24.7240) -- (77.1800,22.0140) -- (75.9450,22.0140) -- (75.9450,28.6690) -- (77.1800,28.6690) -- cycle(70.5980,26.9470) .. controls (70.7430,27.3870) and (70.9380,27.7450) .. (71.1830,28.0200) .. controls (71.4290,28.2950) and (71.7300,28.4940) .. (72.0870,28.6180) .. controls (72.4430,28.7420) and (72.8210,28.8030) .. (73.2200,28.8030) .. controls (73.5480,28.8030) and (73.8690,28.7720) .. (74.1830,28.7090) .. controls (74.4970,28.6460) and (74.7560,28.5420) .. (74.9600,28.3980) -- (74.6910,27.3360) .. controls (74.5460,27.4260) and (74.3750,27.5000) .. (74.1780,27.5570) .. controls (73.9820,27.6140) and (73.7410,27.6430) .. (73.4570,27.6430) .. controls (73.1540,27.6430) and (72.8860,27.5870) .. (72.6540,27.4760) .. controls (72.4220,27.3640) and (72.2290,27.2080) .. (72.0740,27.0080) .. controls (71.9190,26.8080) and (71.8040,26.5670) .. (71.7280,26.2850) .. controls (71.6520,26.0030) and (71.6140,25.6900) .. (71.6140,25.3460) .. controls (71.6140,24.5580) and (71.7750,23.9780) .. (72.0950,23.6040) .. controls (72.4150,23.2310) and (72.8380,23.0440) .. (73.3640,23.0440) .. controls (73.6490,23.0440) and (73.8910,23.0600) .. (74.0900,23.0910) .. controls (74.2900,23.1220) and (74.4620,23.1730) .. (74.6080,23.2440) -- (74.8630,22.1390) .. controls (74.6930,22.0680) and (74.4760,22.0080) .. (74.2130,21.9580) .. controls (73.9500,21.9090) and (73.6290,21.8840) .. (73.2480,21.8840) .. controls (72.8920,21.8840) and (72.5370,21.9430) .. (72.1830,22.0610) .. controls (71.8290,22.1780) and (71.5210,22.3720) .. (71.2610,22.6410) .. controls (71.0000,22.9110) and (70.7880,23.2660) .. (70.6250,23.7060) .. controls (70.4620,24.1450) and (70.3800,24.6920) .. (70.3800,25.3460) .. controls (70.3800,25.9730) and (70.4530,26.5070) .. (70.5980,26.9470)(69.5710,28.6690) -- (69.5710,27.5080) -- (66.8150,27.5080) -- (66.8150,25.8840) -- (69.2880,25.8840) -- (69.2880,24.7240) -- (66.8150,24.7240) -- (66.8150,23.1740) -- (69.5250,23.1740) -- (69.5250,22.0140) -- (65.5800,22.0140) -- (65.5800,28.6690) -- (69.5710,28.6690) -- cycle(61.6160,23.1740) -- (61.6160,28.6690) -- (62.8500,28.6690) -- (62.8500,23.1740) -- (64.8600,23.1740) -- (64.8600,22.0140) -- (59.6020,22.0140) -- (59.6020,23.1740) -- (61.6160,23.1740) -- cycle;

		% Top part:
		\path[fill = TUblue] (31.2450,17.5150) -- (31.2450,20.8090) -- (52.6440,20.8090) -- (52.6460,17.3360) -- (41.8880,14.1730) -- cycle;
		% Left part:
		\path[fill = TUblue] (34.0260,22.0630) -- (38.3400,22.0630) -- (32.6470,41.3700) -- (28.3470,41.3700) -- cycle;
		% Top rectangle:
		\path[fill = TUblue,rounded corners=0.0000cm] (45.8280,22.0540) rectangle (50.5940,26.6430);
		% Middle rectangle:
		\path[fill = TUred,rounded corners=0.0000cm] (45.8280,29.4170) rectangle (50.5940,34.0070);
		% Bottom rectangle:
		\path[fill = TUblue,rounded corners=0.0000cm] (45.8280,36.7800) rectangle (50.5940,41.3700);
	\end{tikzpicture}
}

% Colors needed for the logo (sketchy) of the Department of Computer Science of the University of Kaiserslautern (black when in grayscale mode)
\iftoggle{bwmode}{
	\definecolor{ce5e8f5}{RGB}{0,0,0}
	\definecolor{cdfdbe2}{RGB}{0,0,0}
	\definecolor{cafcde9}{RGB}{0,0,0}
	\definecolor{c9c9afc}{RGB}{0,0,0}
	\definecolor{cb9babc}{RGB}{0,0,0}
	\definecolor{cddae9b}{RGB}{0,0,0}
	\definecolor{cf8776f}{RGB}{0,0,0}
	\definecolor{cac9b8b}{RGB}{0,0,0}
	\definecolor{c878cb9}{RGB}{0,0,0}
	\definecolor{c848387}{RGB}{0,0,0}
	\definecolor{c6c6898}{RGB}{0,0,0}
	\definecolor{cf52d21}{RGB}{0,0,0}
	\definecolor{c0503fc}{RGB}{0,0,0}
	\definecolor{cfa0305}{RGB}{0,0,0}
}{
	\definecolor{ce5e8f5}{RGB}{229,232,245}
	\definecolor{cdfdbe2}{RGB}{223,219,226}
	\definecolor{cafcde9}{RGB}{175,205,233}
	\definecolor{c9c9afc}{RGB}{156,154,252}
	\definecolor{cb9babc}{RGB}{185,186,188}
	\definecolor{cddae9b}{RGB}{221,174,155}
	\definecolor{cf8776f}{RGB}{248,119,111}
	\definecolor{cac9b8b}{RGB}{172,155,139}
	\definecolor{c878cb9}{RGB}{135,140,185}
	\definecolor{c848387}{RGB}{132,131,135}
	\definecolor{c6c6898}{RGB}{108,104,152}
	\definecolor{cf52d21}{RGB}{245,45,33}
	\definecolor{c0503fc}{RGB}{5,3,252}
	\definecolor{cfa0305}{RGB}{250,3,5}
}

% The logo (sketchy) of the Department of Computer Science of the University of Kaiserslautern:
% Taken from: http://dekanat.informatik.uni-kl.de/logo_dekanat_400x145.png on the 2016-12-14
% Converted to SVG using: vectorizer (https://www.vectorizer.io/)
% Manipulated using: Inkscape (https://inkscape.org/)
% Converted to TikZ using: svg2tikz (https://github.com/kjellmf/svg2tikz) as an Inkscape extension
\newcommand{\CSLogoSketchy}{
	\begin{tikzpicture}[
		y = 0.1pt,
		x = 0.1pt,
		yscale = -1,
		xscale = 1,
	]
		\begin{scope}[fill = ce5e8f5]
			\path[fill] (1043,1330) .. controls (1043,1305) and   (1045,1295) .. (1047,1308) .. controls   (1049,1320) and (1049,1340) .. (1047,1353) ..   controls (1045,1365) and (1043,1355) ..   (1043,1330) -- cycle;
			\path[fill] (1043,1050) .. controls (1043,1020) and   (1045,1007) .. (1047,1023) .. controls   (1049,1038) and (1049,1062) .. (1047,1078) ..   controls (1045,1093) and (1043,1080) ..   (1043,1050) -- cycle;
			\path[fill] (780,890) .. controls (767,882) and   (768,880) .. (783,880) .. controls (792,880) and   (800,885) .. (800,890) .. controls (800,902) and   (799,902) .. (780,890) -- cycle;
			\path[fill] (774,640) .. controls (774,516) and   (776,466) .. (777,528) .. controls (779,589) and   (779,691) .. (777,753) .. controls (776,814) and   (774,764) .. (774,640) -- cycle;
			\path[fill] (838,853) .. controls (844,851) and   (856,851) .. (863,853) .. controls (869,856) and   (864,858) .. (850,858) .. controls (836,858) and   (831,856) .. (838,853) -- cycle;
			\path[fill] (1043,715) .. controls (1043,682) and   (1045,670) .. (1047,688) .. controls (1049,706)   and (1049,733) .. (1047,748) .. controls   (1045,763) and (1043,748) .. (1043,715) --   cycle;
			\path[fill] (1043,420) .. controls (1043,384) and   (1045,370) .. (1047,388) .. controls (1049,405)   and (1049,435) .. (1047,453) .. controls   (1045,470) and (1043,456) .. (1043,420) --   cycle;
			\path[fill] (133,265) .. controls (133,221) and   (135,204) .. (137,228) .. controls (139,251) and   (139,287) .. (137,308) .. controls (135,328) and   (133,309) .. (133,265) -- cycle;
		\end{scope}
		\begin{scope}[fill = cdfdbe2]
			\path[fill] (933,1283) .. controls (942,1281) and (958,1281) .. (968,1283) .. controls (977,1286) and (969,1288) .. (950,1288) .. controls (931,1288) and (923,1286) .. (933,1283) -- cycle;
			\path[fill] (1090,1283) -- (1129,1279) -- (1133,1202) -- (1136,1125) -- (1136,1205) -- (1135,1285) -- (1092,1286) -- (1050,1287) -- (1090,1283) -- cycle;
			\path[fill] (909,923) .. controls (896,907) and (897,906) .. (913,919) .. controls (922,926) and (930,934) .. (930,936) .. controls (930,944) and (922,939) .. (909,923) -- cycle;
			\path[fill] (823,903) .. controls (838,901) and (860,901) .. (873,903) .. controls (885,905) and (873,907) .. (845,907) .. controls (818,907) and (807,905) .. (823,903) -- cycle;
			\path[fill] (880,846) .. controls (880,844) and (888,836) .. (898,829) .. controls (913,816) and (914,817) .. (901,833) .. controls (888,849) and (880,854) .. (880,846) -- cycle;
			\path[fill] (1133,580) .. controls (1133,533) and (1135,514) .. (1137,538) .. controls (1139,561) and (1139,599) .. (1137,623) .. controls (1135,646) and (1133,627) .. (1133,580) -- cycle;
			\path[fill=cdfdbe2] (828,393) .. controls (856,391) and (904,391) .. (933,393) .. controls (961,395) and (938,396) .. (880,396) .. controls (822,396) and (799,395) .. (828,393) -- cycle;
			\end{scope}
		\begin{scope}[fill = cafcde9]
			\path[fill] (328,383) .. controls (356,381) and (404,381) .. (433,383) .. controls (461,385) and (438,386) .. (380,386) .. controls (322,386) and (299,385) .. (328,383) -- cycle;
			\path[fill] (678,13) .. controls (685,10) and (694,11) .. (697,14) .. controls (701,17) and (695,20) .. (684,19) .. controls (673,19) and (670,16) .. (678,13) -- cycle;
		\end{scope}
		\begin{scope}[fill = c9c9afc]
			\path[fill] (993,1343) .. controls (985,1341) and (980,1321) .. (980,1299) .. controls (980,1267) and (983,1260) .. (1000,1260) .. controls (1017,1260) and (1020,1267) .. (1020,1305) .. controls (1020,1349) and (1017,1353) .. (993,1343) -- cycle;
			\path[fill] (56,1188) .. controls (79,1104) and (103,1019) .. (109,1000) .. controls (115,981) and (148,866) .. (181,745) .. controls (214,624) and (250,495) .. (261,458) -- (281,390) -- (381,390) .. controls (458,390) and (481,393) .. (477,403) .. controls (475,409) and (436,543) .. (390,700) .. controls (345,857) and (285,1064) .. (256,1160) -- (204,1335) -- (108,1338) -- (13,1341) -- (56,1188) -- cycle;
			\path[fill] (980,1046) .. controls (980,997) and (982,992) .. (1000,997) .. controls (1017,1001) and (1020,1011) .. (1020,1051) .. controls (1020,1093) and (1017,1100) .. (1000,1100) .. controls (982,1100) and (980,1093) .. (980,1046) -- cycle;
			\path[fill] (780,640) -- (780,400) -- (880,400) -- (980,400) -- (980,365) -- (980,330) -- (560,330) -- (140,330) -- (140,257) -- (140,183) -- (363,115) .. controls (485,78) and (608,41) .. (636,32) -- (688,17) -- (891,79) .. controls (1003,113) and (1127,151) .. (1165,163) -- (1235,185) -- (1238,258) -- (1241,330) -- (1130,330) -- (1020,330) -- (1020,405) .. controls (1020,473) and (1018,480) .. (1000,480) .. controls (984,480) and (980,473) .. (980,445) -- (980,410) -- (890,410) -- (800,410) -- (800,635) -- (800,860) -- (830,860) .. controls (847,860) and (860,865) .. (860,870) .. controls (860,876) and (842,880) .. (820,880) -- (780,880) -- (780,640) -- cycle;
			\path[fill] (980,688) .. controls (980,647) and (983,640) .. (1000,640) .. controls (1017,640) and (1020,647) .. (1020,688) .. controls (1020,729) and (1017,736) .. (1000,736) .. controls (983,736) and (980,729) .. (980,688) -- cycle;
		\end{scope}
		\begin{scope}[fill = cb9babc]
			\path[fill] (868,433) .. controls (891,431) and (927,431) .. (948,433) .. controls (968,435) and (949,437) .. (905,437) .. controls (861,437) and (844,435) .. (868,433) -- cycle;
			\path[fill] (363,353) .. controls (477,351) and (663,351) .. (778,353) .. controls (892,354) and (798,355) .. (570,355) .. controls (342,355) and (248,354) .. (363,353) -- cycle;
			\path[fill] (1093,353) .. controls (1124,351) and (1176,351) .. (1208,353) .. controls (1239,355) and (1213,356) .. (1150,356) .. controls (1087,356) and (1061,355) .. (1093,353) -- cycle;
		\end{scope}
		\begin{scope}[fill = cddae9b]
			\path[fill] (1080,815) .. controls (1056,790) and (1038,770) .. (1041,770) .. controls (1044,770) and (1066,790) .. (1090,815) .. controls (1114,840) and (1132,860) .. (1129,860) .. controls (1126,860) and (1104,840) .. (1080,815) -- cycle;
		\end{scope}
		\begin{scope}[fill = cf8776f]
			\path[fill] (959,963) -- (935,935) -- (963,959) .. controls (988,982) and (995,990) .. (987,990) .. controls (985,990) and (973,978) .. (959,963) -- cycle;
		\end{scope}
		\begin{scope}[fill = cac9b8b]
			\path[fill] (1090,945) .. controls (1120,915) and (1147,890) .. (1149,890) .. controls (1152,890) and (1130,915) .. (1100,945) .. controls (1070,975) and (1043,1000) .. (1041,1000) .. controls (1038,1000) and (1060,975) .. (1090,945) -- cycle;
		\end{scope}
		\begin{scope}[fill = c878cb9]
			\path[fill] (77,1343) .. controls (101,1341) and (139,1341) .. (162,1343) .. controls (186,1345) and (167,1347) .. (120,1347) .. controls (73,1347) and (54,1345) .. (77,1343) -- cycle;
			\path[fill] (805,635) -- (805,415) -- (890,414) -- (975,414) -- (893,417) -- (810,421) -- (807,638) -- (804,855) -- (805,635) -- cycle;
			\path[fill] (363,333) .. controls (477,331) and (663,331) .. (778,333) .. controls (892,334) and (798,335) .. (570,335) .. controls (342,335) and (248,334) .. (363,333) -- cycle;
			\path[fill] (1073,333) .. controls (1104,331) and (1156,331) .. (1188,333) .. controls (1219,335) and (1193,336) .. (1130,336) .. controls (1067,336) and (1041,335) .. (1073,333) -- cycle;
		\end{scope}
		\begin{scope}[fill = c848387]
			\path[fill] (1014,1354) .. controls (1017,1345) and (1020,1323) .. (1020,1304) .. controls (1020,1270) and (1020,1270) .. (1065,1270) -- (1110,1270) -- (1110,1195) .. controls (1110,1152) and (1114,1120) .. (1120,1120) .. controls (1126,1120) and (1130,1153) .. (1130,1200) -- (1130,1280) -- (1085,1280) -- (1040,1280) -- (1040,1325) .. controls (1040,1360) and (1036,1370) .. (1024,1370) .. controls (1013,1370) and (1010,1365) .. (1014,1354) -- cycle;
			\path[fill] (113,1353) .. controls (193,1350) and (202,1347) .. (210,1327) .. controls (219,1304) and (247,1211) .. (393,705) .. controls (439,546) and (482,414) .. (489,412) .. controls (495,410) and (500,412) .. (500,417) .. controls (500,425) and (422,694) .. (274,1198) -- (227,1360) -- (126,1358) -- (25,1356) -- (113,1353) -- cycle;
			\path[fill] (933,1273) .. controls (942,1271) and (958,1271) .. (968,1273) .. controls (977,1276) and (969,1278) .. (950,1278) .. controls (931,1278) and (923,1276) .. (933,1273) -- cycle;
			\path[fill] (1020,1042) .. controls (1020,985) and (1021,984) .. (1079,926) .. controls (1111,894) and (1140,872) .. (1143,877) .. controls (1146,882) and (1125,911) .. (1095,940) .. controls (1042,992) and (1040,996) .. (1040,1047) .. controls (1040,1076) and (1036,1100) .. (1030,1100) .. controls (1024,1100) and (1020,1074) .. (1020,1042) -- cycle;
			\path[fill] (800,890) .. controls (800,885) and (815,880) .. (834,880) .. controls (853,880) and (872,885) .. (875,890) .. controls (879,896) and (865,900) .. (841,900) .. controls (818,900) and (800,896) .. (800,890) -- cycle;
			\path[fill] (812,643) -- (810,420) -- (898,422) -- (985,424) -- (903,427) -- (820,431) -- (817,648) -- (815,865) -- (812,643) -- cycle;
			\path[fill] (1028,754) .. controls (1023,750) and (1020,723) .. (1020,693) -- (1020,640) -- (1065,640) -- (1110,640) -- (1110,575) .. controls (1110,538) and (1114,510) .. (1120,510) .. controls (1126,510) and (1130,542) .. (1130,585) -- (1130,660) -- (1086,660) -- (1041,660) -- (1038,711) .. controls (1036,739) and (1032,758) .. (1028,754) -- cycle;
			\path[fill] (920,650) .. controls (920,645) and (934,640) .. (950,640) .. controls (967,640) and (980,645) .. (980,650) .. controls (980,656) and (967,660) .. (950,660) .. controls (934,660) and (920,656) .. (920,650) -- cycle;
			\path[fill] (1020,410) -- (1020,340) -- (1130,340) -- (1240,340) -- (1240,270) .. controls (1240,230) and (1244,200) .. (1250,200) .. controls (1256,200) and (1260,232) .. (1260,275) -- (1260,350) -- (1150,350) -- (1040,350) -- (1040,415) .. controls (1040,452) and (1036,480) .. (1030,480) .. controls (1024,480) and (1020,450) .. (1020,410) -- cycle;
			\path[fill] (363,343) .. controls (477,341) and (663,341) .. (778,343) .. controls (892,344) and (798,345) .. (570,345) .. controls (342,345) and (248,344) .. (363,343) -- cycle;
		\end{scope}
		\begin{scope}[fill = c6c6898]
			\path[fill] (933,1263) .. controls (942,1261) and (958,1261) .. (968,1263) .. controls (977,1266) and (969,1268) .. (950,1268) .. controls (931,1268) and (923,1266) .. (933,1263) -- cycle;
			\path[fill] (1043,1263) .. controls (1058,1261) and (1080,1261) .. (1093,1263) .. controls (1105,1265) and (1093,1267) .. (1065,1267) .. controls (1038,1267) and (1027,1265) .. (1043,1263) -- cycle;
		\end{scope}
		\begin{scope}[fill = cf52d21]
			\path[fill] (1065,930) .. controls (1098,897) and (1127,870) .. (1129,870) .. controls (1132,870) and (1108,897) .. (1075,930) .. controls (1042,963) and (1013,990) .. (1011,990) .. controls (1008,990) and (1032,963) .. (1065,930) -- cycle;
			\path[fill] (914,918) -- (895,895) -- (918,914) .. controls (939,932) and (945,940) .. (937,940) .. controls (935,940) and (925,930) .. (914,918) -- cycle;
		\end{scope}
		\begin{scope}[fill = c0503fc]
			\path[fill] (890,1180) -- (890,1100) -- (994,1100) .. controls (1114,1100) and (1110,1097) .. (1110,1196) -- (1110,1260) -- (1000,1260) -- (890,1260) -- (890,1180) -- cycle;
			\path[fill] (890,560) -- (890,480) -- (994,480) .. controls (1114,480) and (1110,477) .. (1110,576) -- (1110,640) -- (1000,640) -- (890,640) -- (890,560) -- cycle;
		\end{scope}
		\begin{scope}[fill = cfa0305]
			\path[fill] (932,933) -- (869,868) -- (934,802) -- (1000,735) -- (1064,800) -- (1129,865) -- (1064,933) .. controls (1027,970) and (997,999) .. (996,998) .. controls (996,998) and (967,968) .. (932,933) -- cycle;
		\end{scope}
	
	\end{tikzpicture}
}

% Colors needed for the logo (sketchy) of the Department of Computer Science of the University of Kaiserslautern (black when in grayscale mode)
\iftoggle{bwmode}{
	\definecolor{c808080}{RGB}{191,191,191}
	\definecolor{cff0000}{RGB}{127,127,127}
	\definecolor{c9999ff}{RGB}{0,0,0}
	\definecolor{c0000ff}{RGB}{63,63,63}
}{
	\definecolor{c808080}{RGB}{128,128,128}
	\definecolor{cff0000}{RGB}{255,0,0}
	\definecolor{c9999ff}{RGB}{153,153,255}
	\definecolor{c0000ff}{RGB}{0,0,255}
}

% The logo of the Department of Computer Science of the University of Kaiserslautern:
% Taken from: http://sci.informatik.uni-kl.de/rechnerzugang/terminals/lageplan_sci/Lageplan_SCI.pdf on the 2017-03-16
% Manipulated using: Inkscape (https://inkscape.org/)
% Converted to TikZ using: svg2tikz (https://github.com/kjellmf/svg2tikz) as an Inkscape extension
\newcommand{\CSLogo}{
	\begin{tikzpicture}[
		y = 1.65pt,
		x = 1.65pt,
		yscale = -1,
		xscale = 1
	]
		\begin{scope}[cm={{0.0, 1.25, 1.25, 0.0, (-153.75, -108.75)}}]
			\path[cm = {{0.0, 0.82808, 1.0, 0.0, (125.0216, 604.3906)}}, fill = c808080, nonzero rule] (0.0000, 0.0000) node[above right] (text2307) {};
			\path[cm = {{0.0, 0.82379, 1.0, 0.0, (125.0216, 609.1211)}}, fill = c808080, nonzero rule] (0.0000, 0.0000) node[above right] (text2311) {};
			\path[cm = {{0.0, 0.82808, 1.0, 0.0, (145.8526, 604.3906)}}, fill = c808080, nonzero rule] (0.0000, 0.0000) node[above right] (text2315) {};
			\path[cm = {{0.0, 0.82379, 1.0, 0.0, (145.8526, 609.1211)}}, fill = c808080, nonzero rule] (0.0000, 0.0000) node[above right] (text2319) {};
			\path[cm = {{0.0, 0.75, 1.0, 0.0, (168.3434, 604.3906)}}, fill = c808080, nonzero rule] (0.0000, 0.0000) node[above right] (text2323) {};
			\path[cm = {{0.0, 1.0, 0.97692, 0.0, (125.1046, 587.6261)}}, fill = c808080, nonzero rule] (0.0000, 0.0000) node[above right] (text2327) {};
			\path[cm = {{0.0, 1.0, 0.95143, 0.0, (147.0145, 587.3771)}}, fill = c808080, nonzero rule] (0.0000, 0.0000) node[above right] (text2331) {};
			\path[cm = {{0.0, 1.0, 0.62474, 0.0, (169.6713, 591.1118)}}, fill = c808080, nonzero rule] (0.0000, 0.0000) node[above right] (text2355) {};
			\path[cm = {{0.0, 1.0, 0.97692, 0.0, (123.8597, 585.9662)}}, fill = cff0000, nonzero rule] (0.0000, 0.0000) node[above right] (text2379) {};
			\path[cm = {{0.0, 1.0, 0.95143, 0.0, (145.8526, 585.7173)}}, fill = cff0000, nonzero rule] (0.0000, 0.0000) node[above right] (text2383) {};
			\path[cm = {{0.0, 1.0, 0.62474, 0.0, (168.5094, 589.452)}}, fill = cff0000, nonzero rule] (0.0000, 0.0000) node[above right] (text2407) {};
			\path[fill = c808080, even odd rule] (123.5280, 563.8070) -- (123.5280, 558.9940) -- (122.4490, 558.9940) -- (122.4490, 568.6210) -- (123.5280, 568.6210) -- (123.5280, 563.8070);
			\path[fill = c808080, even odd rule] (144.9400, 563.8070) -- (144.9400, 558.9940) -- (143.7780, 558.9940) -- (143.7780, 568.6210) -- (144.9400, 568.6210) -- (144.9400, 563.8070);
			\path[fill = c808080, even odd rule] (133.7360, 559.0770) -- (122.4490, 559.0770) -- (122.4490, 560.2390) -- (145.0230, 560.2390) -- (145.0230, 559.0770) -- (133.7360, 559.0770);
			\path[fill = c808080, even odd rule] (143.1970, 568.6210) -- (119.7100, 568.6210) -- (119.7100, 570.3640) -- (166.6010, 570.3640) -- (166.6010, 568.6210) -- (143.1970, 568.6210);
			\path[fill = c808080, even odd rule] (112.4070, 529.1170) -- (104.6890, 554.7610) -- (112.4070, 580.5720) -- (119.7100, 580.5720) -- (119.7100, 529.2830) -- (112.4070, 529.2830) -- (112.5730, 529.1170) -- (112.4070, 529.1170);
			\path[fill = c808080, even odd rule] (122.1170, 535.7560) -- (122.1170, 545.3830) -- (166.4350, 532.4360) -- (166.4350, 523.2240) -- (122.1170, 535.7560);
			\path[fill = c808080, even odd rule] (147.2630, 566.2970) -- (144.2760, 563.2260) -- (138.1340, 569.4510) -- (144.1930, 575.5920) -- (150.3340, 569.3680) -- (147.2630, 566.2970);
			\path[fill = c808080, even odd rule] (133.8190, 569.4510) -- (133.8190, 564.3880) -- (126.4320, 564.3880) -- (126.4320, 574.4300) -- (133.8190, 574.4300) -- (133.8190, 569.4510);
			\path[fill = c808080, even odd rule] (162.6170, 569.4510) -- (162.6170, 564.3880) -- (155.1480, 564.3880) -- (155.1480, 574.4300) -- (162.6170, 574.4300) -- (162.6170, 569.4510);
			\path[fill = c9999ff, even odd rule] (122.6150, 562.8110) -- (122.6150, 557.9150) -- (121.5360, 557.9150) -- (121.5360, 567.6250) -- (122.6150, 567.6250) -- (122.6150, 562.8110);
			\path[fill = c9999ff, even odd rule] (144.0270, 562.8110) -- (144.0270, 557.9150) -- (142.8650, 557.9150) -- (142.8650, 567.6250) -- (144.0270, 567.6250) -- (144.0270, 562.8110);
			\path[fill = c9999ff, even odd rule] (132.8230, 557.9980) -- (121.5360, 557.9980) -- (121.5360, 559.1600) -- (144.1100, 559.1600) -- (144.1100, 557.9980) -- (132.8230, 557.9980);
			\path[fill = c9999ff, even odd rule] (142.2010, 567.6250) -- (118.7140, 567.6250) -- (118.7140, 569.2850) -- (165.6880, 569.2850) -- (165.6880, 567.6250) -- (142.2010, 567.6250);
			\path[fill = c9999ff, even odd rule] (111.4940, 528.0380) -- (103.6930, 553.6820) -- (111.4940, 579.4930) -- (118.7140, 579.4930) -- (118.7140, 528.2040) -- (111.4940, 528.2040) -- (111.6600, 528.0380) -- (111.4940, 528.0380);
			\path[fill = c9999ff, even odd rule] (121.1210, 534.6770) -- (121.1210, 544.3040) -- (165.5220, 531.3570) -- (165.5220, 522.1450) -- (121.1210, 534.6770);
			\path[fill = cff0000, even odd rule] (146.3510, 565.2180) -- (143.2800, 562.1480) -- (137.1380, 568.3720) -- (143.2800, 574.5130) -- (149.4210, 568.2890) -- (146.3510, 565.2180);
			\path[fill = c0000ff, even odd rule] (132.9060, 568.3720) -- (132.9060, 563.3090) -- (125.4370, 563.3090) -- (125.4370, 573.4340) -- (132.9060, 573.4340) -- (132.9060, 568.3720);
			\path[fill = c0000ff, even odd rule] (161.7040, 568.3720) -- (161.7040, 563.3090) -- (154.2350, 563.3090) -- (154.2350, 573.4340) -- (161.7040, 573.4340) -- (161.7040, 568.3720);
			\path[cm = {{0.0, 0.82808, 1.0, 0.0, (124.0257, 603.3947)}}, fill = c9999ff, nonzero rule] (0.0000, 0.0000) node[above right] (text2467) {};
			\path[cm = {{0.0, 0.82379, 1.0, 0.0, (124.0257, 608.1252)}}, fill = c9999ff, nonzero rule] (0.0000, 0.0000) node[above right] (text2471) {};
			\path[cm = {{0.0, 0.82808, 1.0, 0.0, (144.8567, 603.3947)}}, fill = c9999ff, nonzero rule] (0.0000, 0.0000) node[above right] (text2475) {};
			\path[cm = {{0.0, 0.82379, 1.0, 0.0, (144.8567, 608.1252)}}, fill = c9999ff, nonzero rule] (0.0000, 0.0000) node[above right] (text2479) {};
			\path[cm = {{0.0, 0.75, 1.0, 0.0, (167.4305, 603.3947)}}, fill = c0000ff, nonzero rule] (0.0000, 0.0000) node[above right] (text2483) {};
		\end{scope}
		
	\end{tikzpicture}
}


\newcommand{\seDiagram}[1][1.0]{%
	\begin{tikzpicture}[y = 0.80pt,
	                             x = 0.80pt,
	                             yscale = -1.000000,
	                             xscale = 1.000000,
	                             inner sep = 0pt,
	                             outer sep = 0pt,
	                             opacity = #1,
	                             baseline = (current bounding box.center)]
		\begin{scope}[cm = {{1.33333, 0.0, 0.0, -1.33333, (0.0, 1056.0)}}]% g823
			% path941
			\path[fill = gray!50, even odd rule, line width = 0.480pt] (91.4850, 677.3840) -- (94.2930, 677.3840) -- (94.2930, 685.3040) -- (91.4850, 685.3040) -- cycle;
		
			% path943
			\path[draw = black, line join = miter, line cap = butt, miter limit = 10.00, line width = 0.461pt] (91.4850, 677.3840) -- (94.2930, 677.3840) -- (94.2930, 685.3040) -- (91.4850, 685.3040) -- cycle;
		
			% path945
			\path[fill = gray!50, even odd rule, line width = 0.480pt] (95.2290, 677.3840) -- (98.0370, 677.3840) -- (98.0370, 685.3040) -- (95.2290, 685.3040) -- cycle;
		
			% path947
			\path[draw = black, line join = miter, line cap = butt, miter limit = 10.00, line width = 0.461pt] (95.2290, 677.3840) -- (98.0370, 677.3840) -- (98.0370, 685.3040) -- (95.2290, 685.3040) -- cycle;
		
			% path949
			\path[fill = gray!50, even odd rule, line width = 0.480pt] (98.9730, 677.3840) -- (101.7810, 677.3840) -- (101.7810, 685.3040) -- (98.9730, 685.3040) -- cycle;
		
			% path951
			\path[draw = black, line join = miter, line cap = butt, miter limit = 10.00, line width = 0.461pt] (98.9730, 677.3840) -- (101.7810, 677.3840) -- (101.7810, 685.3040) -- (98.9730, 685.3040) -- cycle;
		
			% path953
			\path[fill = gray!50, even odd rule, line width = 0.480pt] (102.7890, 677.3840) -- (105.5970, 677.3840) -- (105.5970, 685.3040) -- (102.7890, 685.3040) -- cycle;
		
			% path955
			\path[draw = black, line join = miter, line cap = butt, miter limit = 10.00, line width = 0.461pt] (102.7890, 677.3840) -- (105.5970, 677.3840) -- (105.5970, 685.3040) -- (102.7890, 685.3040) -- cycle;
		
			% path957
			\path[fill = gray!50, even odd rule, line width = 0.480pt] (106.5330, 677.3840) -- (109.3410, 677.3840) -- (109.3410, 685.3040) -- (106.5330, 685.3040) -- cycle;
		
			% path959
			\path[draw = black, line join = miter, line cap = butt, miter limit = 10.00, line width = 0.461pt] (106.5330, 677.3840) -- (109.3410, 677.3840) -- (109.3410, 685.3040) -- (106.5330, 685.3040) -- cycle;
		
			% path961
			\path[draw = black, line join = miter, line cap = butt, miter limit = 10.00, line width = 0.749pt] (93.3930, 689.8760) -- (96.6372, 685.2620);
		
			% path963
			\path[draw = black, line join = miter, line cap = butt, miter limit = 10.00, line width = 0.749pt] (96.1290, 689.8760) -- (104.2098, 685.2620);
		
			% path965
			\path[draw = black, line join = miter, line cap = butt, miter limit = 10.00, line width = 0.749pt] (100.4190, 689.8760) -- (92.8890, 685.2620);
		
			% path967
			\path[draw = black, line join = miter, line cap = butt, miter limit = 10.00, line width = 0.749pt] (103.0410, 689.8760) -- (119.4966, 685.2620);
		
			% path969
			\path[draw = black, line join = miter, line cap = butt, miter limit = 10.00, line width = 0.749pt] (106.5834, 689.8760) -- (100.4490, 685.3040);
		
			% path971
			\path[fill = gray!50, even odd rule, line width = 0.480pt] (110.4930, 677.3840) -- (113.3010, 677.3840) -- (113.3010, 685.3040) -- (110.4930, 685.3040) -- cycle;
		
			% path973
			\path[draw = black, line join = miter, line cap = butt, miter limit = 10.00, line width = 0.461pt] (110.4930, 677.3840) -- (113.3010, 677.3840) -- (113.3010, 685.3040) -- (110.4930, 685.3040) -- cycle;
		
			% path975
			\path[fill = gray!50, even odd rule, line width = 0.480pt] (114.2370, 677.3840) -- (117.0450, 677.3840) -- (117.0450, 685.3040) -- (114.2370, 685.3040) -- cycle;
		
			% path977
			\path[draw = black, line join = miter, line cap = butt, miter limit = 10.00, line width = 0.461pt] (114.2370, 677.3840) -- (117.0450, 677.3840) -- (117.0450, 685.3040) -- (114.2370, 685.3040) -- cycle;
		
			% path979
			\path[fill = gray!50, even odd rule, line width = 0.480pt] (118.0530, 677.3840) -- (120.7890, 677.3840) -- (120.7890, 685.3040) -- (118.0530, 685.3040) -- cycle;
		
			% path981
			\path[draw = black, line join = miter, line cap = butt, miter limit = 10.00, line width = 0.461pt] (118.0530, 677.3840) -- (120.7890, 677.3840) -- (120.7890, 685.3040) -- (118.0530, 685.3040) -- cycle;
		
			% path983
			\path[fill = gray!50, even odd rule, line width = 0.480pt] (121.7970, 677.3840) -- (124.6050, 677.3840) -- (124.6050, 685.3040) -- (121.7970, 685.3040) -- cycle;
		
			% path985
			\path[draw = black, line join = miter, line cap = butt, miter limit = 10.00, line width = 0.461pt] (121.7970, 677.3840) -- (124.6050, 677.3840) -- (124.6050, 685.3040) -- (121.7970, 685.3040) -- cycle;
		
			% path987
			\path[fill = gray!50, even odd rule, line width = 0.480pt] (125.5410, 677.3840) -- (128.3490, 677.3840) -- (128.3490, 685.3040) -- (125.5410, 685.3040) -- cycle;
		
			% path989
			\path[draw = black, line join = miter, line cap = butt, miter limit = 10.00, line width = 0.461pt] (125.5410, 677.3840) -- (128.3490, 677.3840) -- (128.3490, 685.3040) -- (125.5410, 685.3040) -- cycle;
		
			% path991
			\path[fill = gray!50, even odd rule, line width = 0.480pt] (91.4850, 689.9120) -- (129.7890, 712.6640) -- (168.0930, 689.9120) -- cycle;
		
			% path993
			\path[draw = black, line join = miter, line cap = butt, miter limit = 10.00, line width = 0.461pt] (91.4850, 689.9120) -- (129.7890, 712.6640) -- (168.0930, 689.9120) -- cycle;
		
			% path995
			\path[draw = black, line join = miter, line cap = butt, miter limit = 10.00, line width = 0.749pt] (112.4238, 689.8760) -- (107.9370, 685.2620);
		
			% path997
			\path[draw = black, line join = miter, line cap = butt, miter limit = 10.00, line width = 0.749pt] (115.1370, 689.8760) -- (123.2178, 685.2620);
		
			% path999
			\path[draw = black, line join = miter, line cap = butt, miter limit = 10.00, line width = 0.749pt] (129.7752, 689.8760) -- (111.8970, 685.2620);
		
			% path1001
			\path[draw = black, line join = miter, line cap = butt, miter limit = 10.00, line width = 0.749pt] (122.0490, 689.8760) -- (127.0086, 685.2620);
		
			% path1003
			\path[draw = black, line join = miter, line cap = butt, miter limit = 10.00, line width = 0.749pt] (125.5914, 689.8760) -- (119.4570, 685.3040);
		
			% path1005
			\path[fill = gray!50, even odd rule, line width = 0.480pt] (130.1490, 677.3840) -- (132.9570, 677.3840) -- (132.9570, 685.3040) -- (130.1490, 685.3040) -- cycle;
		
			% path1007
			\path[draw = black, line join = miter, line cap = butt, miter limit = 10.00, line width = 0.461pt] (130.1490, 677.3840) -- (132.9570, 677.3840) -- (132.9570, 685.3040) -- (130.1490, 685.3040) -- cycle;
		
			% path1009
			\path[fill = gray!50, even odd rule, line width = 0.480pt] (133.8930, 677.3840) -- (136.7010, 677.3840) -- (136.7010, 685.3040) -- (133.8930, 685.3040) -- cycle;
		
			% path1011
			\path[draw = black, line join = miter, line cap = butt, miter limit = 10.00, line width = 0.461pt] (133.8930, 677.3840) -- (136.7010, 677.3840) -- (136.7010, 685.3040) -- (133.8930, 685.3040) -- cycle;
		
			% path1013
			\path[fill = gray!50, even odd rule, line width = 0.480pt] (137.7090, 677.3840) -- (140.4450, 677.3840) -- (140.4450, 685.3040) -- (137.7090, 685.3040) -- cycle;
		
			% path1015
			\path[draw = black, line join = miter, line cap = butt, miter limit = 10.00, line width = 0.461pt] (137.7090, 677.3840) -- (140.4450, 677.3840) -- (140.4450, 685.3040) -- (137.7090, 685.3040) -- cycle;
		
			% path1017
			\path[fill = gray!50, even odd rule, line width = 0.480pt] (141.4530, 677.3840) -- (144.2610, 677.3840) -- (144.2610, 685.3040) -- (141.4530, 685.3040) -- cycle;
		
			% path1019
			\path[draw = black, line join = miter, line cap = butt, miter limit = 10.00, line width = 0.461pt] (141.4530, 677.3840) -- (144.2610, 677.3840) -- (144.2610, 685.3040) -- (141.4530, 685.3040) -- cycle;
		
			% path1021
			\path[fill = gray!50, even odd rule, line width = 0.480pt] (145.1970, 677.3840) -- (148.0050, 677.3840) -- (148.0050, 685.3040) -- (145.1970, 685.3040) -- cycle;
		
			% path1023
			\path[draw = black, line join = miter, line cap = butt, miter limit = 10.00, line width = 0.461pt] (145.1970, 677.3840) -- (148.0050, 677.3840) -- (148.0050, 685.3040) -- (145.1970, 685.3040) -- cycle;
		
			% path1025
			\path[draw = black, line join = miter, line cap = butt, miter limit = 10.00, line width = 0.749pt] (132.1290, 689.8760) -- (135.3732, 685.2620);
		
			% path1027
			\path[draw = black, line join = miter, line cap = butt, miter limit = 10.00, line width = 0.749pt] (134.7930, 689.8760) -- (142.8738, 685.2620);
		
			% path1029
			\path[draw = black, line join = miter, line cap = butt, miter limit = 10.00, line width = 0.749pt] (139.0830, 689.8760) -- (131.5530, 685.2620);
		
			% path1031
			\path[draw = black, line join = miter, line cap = butt, miter limit = 10.00, line width = 0.749pt] (141.7050, 689.8760) -- (146.6670, 685.2620);
		
			% path1033
			\path[draw = black, line join = miter, line cap = butt, miter limit = 10.00, line width = 0.749pt] (145.2330, 689.8760) -- (151.1850, 685.3040);
		
			% path1035
			\path[fill = gray!50, even odd rule, line width = 0.480pt] (149.7330, 677.3840) -- (152.5410, 677.3840) -- (152.5410, 685.3040) -- (149.7330, 685.3040) -- cycle;
		
			% path1037
			\path[draw = black, line join = miter, line cap = butt, miter limit = 10.00, line width = 0.461pt] (149.7330, 677.3840) -- (152.5410, 677.3840) -- (152.5410, 685.3040) -- (149.7330, 685.3040) -- cycle;
		
			% path1039
			\path[fill = gray!50, even odd rule, line width = 0.480pt] (153.5490, 677.3840) -- (156.2850, 677.3840) -- (156.2850, 685.3040) -- (153.5490, 685.3040) -- cycle;
		
			% path1041
			\path[draw = black, line join = miter, line cap = butt, miter limit = 10.00, line width = 0.461pt] (153.5490, 677.3840) -- (156.2850, 677.3840) -- (156.2850, 685.3040) -- (153.5490, 685.3040) -- cycle;
		
			% path1043
			\path[fill = gray!50, even odd rule, line width = 0.480pt] (157.2930, 677.3840) -- (160.1010, 677.3840) -- (160.1010, 685.3040) -- (157.2930, 685.3040) -- cycle;
		
			% path1045
			\path[draw = black, line join = miter, line cap = butt, miter limit = 10.00, line width = 0.461pt] (157.2930, 677.3840) -- (160.1010, 677.3840) -- (160.1010, 685.3040) -- (157.2930, 685.3040) -- cycle;
		
			% path1047
			\path[fill = gray!50, even odd rule, line width = 0.480pt] (161.0370, 677.3840) -- (163.8450, 677.3840) -- (163.8450, 685.3040) -- (161.0370, 685.3040) -- cycle;
		
			% path1049
			\path[draw = black, line join = miter, line cap = butt, miter limit = 10.00, line width = 0.461pt] (161.0370, 677.3840) -- (163.8450, 677.3840) -- (163.8450, 685.3040) -- (161.0370, 685.3040) -- cycle;
		
			% path1051
			\path[fill = gray!50, even odd rule, line width = 0.480pt] (164.8530, 677.3840) -- (167.5890, 677.3840) -- (167.5890, 685.3040) -- (164.8530, 685.3040) -- cycle;
		
			% path1053
			\path[draw = black, line join = miter, line cap = butt, miter limit = 10.00, line width = 0.461pt] (164.8530, 677.3840) -- (167.5890, 677.3840) -- (167.5890, 685.3040) -- (164.8530, 685.3040) -- cycle;
		
			% path1055
			\path[draw = black, line join = miter, line cap = butt, miter limit = 10.00, line width = 0.749pt] (151.7130, 689.8760) -- (154.9590, 685.2620);
		
			% path1057
			\path[draw = black, line join = miter, line cap = butt, miter limit = 10.00, line width = 0.749pt] (154.3770, 689.8760) -- (162.4590, 685.2620);
		
			% path1059
			\path[draw = black, line join = miter, line cap = butt, miter limit = 10.00, line width = 0.749pt] (158.7630, 689.8760) -- (146.6730, 685.2620);
		
			% path1061
			\path[draw = black, line join = miter, line cap = butt, miter limit = 10.00, line width = 0.749pt] (161.2890, 689.8760) -- (166.2510, 685.2620);
		
			% path1063
			\path[draw = black, line join = miter, line cap = butt, miter limit = 10.00, line width = 0.749pt] (164.8290, 689.8760) -- (158.6970, 685.3040);
		
			% path1583
			\path[fill = red!50, even odd rule, line width = 0.480pt] (116.6358, 672.9260) -- (117.1116, 672.9260) -- (117.1116, 674.4260) -- (117.2016, 674.7020) -- (117.3378, 674.9420) -- (117.4734, 675.1220) -- (117.5754, 675.2600) -- (117.7062, 675.3860) -- (117.8646, 675.4940) -- (118.0518, 675.6020) -- (118.2156, 675.6740) -- (118.4424, 675.7580) -- (118.6632, 675.8060) -- (118.9236, 675.8540) -- (119.2014, 675.8660) -- (119.4390, 675.8540) -- (119.6262, 675.8360) -- (119.7846, 675.7940) -- (119.9658, 675.7460) -- (120.1416, 675.6860) -- (120.1980, 675.6620) -- (120.2604, 675.6320) -- (120.3054, 675.6080) -- (120.3564, 675.5780) -- (120.3678, 675.5720) -- (120.4074, 675.5480) -- (120.4470, 675.5180) -- (120.4866, 675.4820) -- (120.5322, 675.4520) -- (120.6684, 675.3440) -- (120.7530, 675.2600) -- (120.8664, 675.1520) -- (120.9570, 675.0140) -- (121.0362, 674.8940) -- (121.1214, 674.7560) -- (121.1892, 674.6060) -- (121.2570, 674.4260) -- (121.3026, 674.2820) -- (121.3076, 672.9260) -- (121.8686, 672.9260) -- (121.8686, 670.8740) -- (121.8230, 670.6040) -- (121.7324, 670.3820) -- (121.6418, 670.2380) -- (121.5626, 670.0880) -- (121.4606, 669.9440) -- (121.3304, 669.8000) -- (121.1888, 669.6620) -- (121.0076, 669.5600) -- (120.8150, 669.4700) -- (120.6338, 669.3740) -- (120.3962, 669.3020) -- (120.2144, 669.2660) -- (119.9372, 669.2240) -- (118.6514, 669.2300) -- (118.3568, 669.2660) -- (118.0964, 669.3320) -- (117.8414, 669.4040) -- (117.5864, 669.5060) -- (117.4052, 669.6200) -- (117.2864, 669.6980) -- (117.1676, 669.8240) -- (117.0656, 669.9200) -- (116.9750, 670.0220) -- (116.8730, 670.1420) -- (116.7938, 670.2740) -- (116.6918, 670.4540) -- (116.6522, 670.6460) -- (116.6354, 670.8200) -- cycle(118.0968, 672.9320) -- (118.0968, 674.2700) -- (118.1196, 674.3660) -- (118.1478, 674.4500) -- (118.1988, 674.5340) -- (118.2384, 674.6180) -- (118.2894, 674.6840) -- (118.3914, 674.7920) -- (118.5216, 674.8640) -- (118.6572, 674.9480) -- (118.8108, 675.0020) -- (118.9632, 675.0440) -- (119.1672, 675.0680) -- (119.3202, 675.0560) -- (119.5296, 675.0140) -- (119.6598, 674.9840) -- (119.7792, 674.9420) -- (119.8470, 674.8940) -- (119.9490, 674.8400) -- (119.9886, 674.7980) -- (120.0336, 674.7560) -- (120.0906, 674.7080) -- (120.1248, 674.6600) -- (120.1698, 674.6120) -- (120.2208, 674.5520) -- (120.2430, 674.4800) -- (120.2718, 674.4080) -- (120.3054, 674.3300) -- (120.3228, 674.2700) -- (120.3228, 672.9320) -- cycle;
		
			% path1585
			\path[fill = red!50, even odd rule, line width = 0.480pt] (119.4564, 670.6280) -- (119.5980, 669.9500) -- (118.9632, 669.9500) -- (119.1162, 670.6220) -- (119.0484, 670.6640) -- (118.9974, 670.7420) -- (118.9860, 670.8380) -- (119.0028, 670.9160) -- (119.0484, 670.9640) -- (119.0826, 671.0060) -- (119.1390, 671.0300) -- (119.2068, 671.0600) -- (119.2920, 671.0660) -- (119.3598, 671.0480) -- (119.4390, 671.0300) -- (119.5068, 670.9820) -- (119.5410, 670.9400) -- (119.5752, 670.8980) -- (119.5802, 670.8380) -- (119.5574, 670.7420) -- (119.5292, 670.7000) -- (119.4956, 670.6640) -- (119.4560, 670.6280);
		
			% path1587
			\path[draw = black, line join = miter, line cap = butt, miter limit = 10.00, line width = 0.461pt] (116.6358, 672.9260) -- (117.1116, 672.9260) -- (117.1116, 674.4260) -- (117.2016, 674.7020) -- (117.3378, 674.9420) -- (117.4734, 675.1220) -- (117.5754, 675.2600) -- (117.7062, 675.3860) -- (117.8646, 675.4940) -- (118.0518, 675.6020) -- (118.2156, 675.6740) -- (118.4424, 675.7580) -- (118.6632, 675.8060) -- (118.9236, 675.8540) -- (119.2014, 675.8660) -- (119.4390, 675.8540) -- (119.6262, 675.8360) -- (119.7846, 675.7940) -- (119.9658, 675.7460) -- (120.1416, 675.6860) -- (120.1980, 675.6620) -- (120.2604, 675.6320) -- (120.3054, 675.6080) -- (120.3564, 675.5780) -- (120.3678, 675.5720) -- (120.4074, 675.5480) -- (120.4470, 675.5180) -- (120.4866, 675.4820) -- (120.5322, 675.4520) -- (120.6684, 675.3440) -- (120.7530, 675.2600) -- (120.8664, 675.1520) -- (120.9570, 675.0140) -- (121.0362, 674.8940) -- (121.1214, 674.7560) -- (121.1892, 674.6060) -- (121.2570, 674.4260) -- (121.3026, 674.2820) -- (121.3076, 672.9260) -- (121.8686, 672.9260) -- (121.8686, 670.8740) -- (121.8230, 670.6040) -- (121.7324, 670.3820) -- (121.6418, 670.2380) -- (121.5626, 670.0880) -- (121.4606, 669.9440) -- (121.3304, 669.8000) -- (121.1888, 669.6620) -- (121.0076, 669.5600) -- (120.8150, 669.4700) -- (120.6338, 669.3740) -- (120.3962, 669.3020) -- (120.2144, 669.2660) -- (119.9372, 669.2240) -- (118.6514, 669.2300) -- (118.3568, 669.2660) -- (118.0964, 669.3320) -- (117.8414, 669.4040) -- (117.5864, 669.5060) -- (117.4052, 669.6200) -- (117.2864, 669.6980) -- (117.1676, 669.8240) -- (117.0656, 669.9200) -- (116.9750, 670.0220) -- (116.8730, 670.1420) -- (116.7938, 670.2740) -- (116.6918, 670.4540) -- (116.6522, 670.6460) -- (116.6354, 670.8200) -- cycle(118.0968, 672.9320) -- (118.0968, 674.2700) -- (118.1196, 674.3660) -- (118.1478, 674.4500) -- (118.1988, 674.5340) -- (118.2384, 674.6180) -- (118.2894, 674.6840) -- (118.3914, 674.7920) -- (118.5216, 674.8640) -- (118.6572, 674.9480) -- (118.8108, 675.0020) -- (118.9632, 675.0440) -- (119.1672, 675.0680) -- (119.3202, 675.0560) -- (119.5296, 675.0140) -- (119.6598, 674.9840) -- (119.7792, 674.9420) -- (119.8470, 674.8940) -- (119.9490, 674.8400) -- (119.9886, 674.7980) -- (120.0336, 674.7560) -- (120.0906, 674.7080) -- (120.1248, 674.6600) -- (120.1698, 674.6120) -- (120.2208, 674.5520) -- (120.2430, 674.4800) -- (120.2718, 674.4080) -- (120.3054, 674.3300) -- (120.3228, 674.2700) -- (120.3228, 672.9320) -- cycle;
		
			% path1589
			\path[draw = black, line join = miter, line cap = butt, miter limit = 10.00, line width = 0.461pt] (116.6358, 672.9260) -- (121.8690, 672.9260) -- cycle;
		
			% path1591
			\path[draw = black, line join = miter, line cap = butt, miter limit = 10.00, line width = 0.461pt] (119.4564, 670.6280) -- (119.5980, 669.9500) -- (118.9632, 669.9500) -- (119.1162, 670.6220) -- (119.0484, 670.6640) -- (118.9974, 670.7420) -- (118.9860, 670.8380) -- (119.0028, 670.9160) -- (119.0484, 670.9640) -- (119.0826, 671.0060) -- (119.1390, 671.0300) -- (119.2068, 671.0600) -- (119.2920, 671.0660) -- (119.3598, 671.0480) -- (119.4390, 671.0300) -- (119.5068, 670.9820) -- (119.5410, 670.9400) -- (119.5752, 670.8980) -- (119.5802, 670.8380) -- (119.5574, 670.7420) -- (119.5292, 670.7000) -- (119.4956, 670.6640) -- (119.4560, 670.6280);
		
			% path1773
			\path[draw = red, line join = miter, line cap = butt, miter limit = 10.00, line width = 1.094pt] (119.3130, 737.3240) .. controls (120.4842, 736.8440) and (121.6560, 736.3640) .. (121.7082, 735.5420) .. controls (121.7610, 734.7200) and (119.6598, 733.1420) .. (119.6280, 732.3860) .. controls (119.5968, 731.6360) and (121.4880, 731.8640) .. (121.5192, 731.0180) .. controls (121.5510, 730.1720) and (120.1428, 727.9580) .. (119.8176, 727.3160);
		
			% path1775
			\path[draw = red, line join = miter, line cap = butt, miter limit = 10.00, line width = 1.094pt] (99.2970, 737.3240) .. controls (100.4682, 736.8380) and (101.6400, 736.3580) .. (101.6922, 735.5300) .. controls (101.7450, 734.7020) and (99.6438, 733.1120) .. (99.6120, 732.3560) .. controls (99.5808, 731.5940) and (101.4720, 731.8220) .. (101.5032, 730.9700) .. controls (101.5350, 730.1180) and (100.1268, 727.8860) .. (99.8016, 727.2440);
		
			% path1777
			\path[draw = red, line join = miter, line cap = butt, miter limit = 10.00, line width = 1.094pt] (157.9050, 737.4680) .. controls (159.0450, 736.9820) and (160.1790, 736.5020) .. (160.2330, 735.6740) .. controls (160.2810, 734.8460) and (158.2410, 733.2560) .. (158.2110, 732.5000) .. controls (158.1810, 731.7380) and (160.0170, 731.9660) .. (160.0470, 731.1140) .. controls (160.0770, 730.2620) and (158.7090, 728.0300) .. (158.3970, 727.3880);
		
			% path1779
			\path[draw = red, line join = miter, line cap = butt, miter limit = 10.00, line width = 1.094pt] (137.3850, 737.4680) .. controls (138.5220, 736.9820) and (139.6590, 736.5020) .. (139.7100, 735.6740) .. controls (139.7610, 734.8460) and (137.7216, 733.2560) .. (137.6910, 732.5000) .. controls (137.6604, 731.7380) and (139.4958, 731.9660) .. (139.5264, 731.1140) .. controls (139.5570, 730.2620) and (138.1908, 728.0300) .. (137.8746, 727.3880);
		
			% path1781
			\path[fill = gray!50, even odd rule, line width = 0.480pt] (97.0290, 714.8960) -- (104.2290, 714.8960) -- (104.2290, 722.1680) -- (97.0290, 722.1680) -- cycle;
		
			% path1783
			\path[draw = black, line join = miter, line cap = butt, miter limit = 10.00, line width = 0.461pt] (97.0290, 714.8960) -- (104.2290, 714.8960) -- (104.2290, 722.1680) -- (97.0290, 722.1680) -- cycle;
		
			\begin{scope}[cm = {{0.6, 0.0, 0.0, 0.6, (85.149, 560.6)}}]% g1791
				\begin{scope}% g1789
					% text1787
					\path[cm = {{1.0, 0.0, 0.0, -1.0, (22.056, 258.43)}}, fill = black, nonzero rule] (0.0000, 0.0000) node[above right, font = \scriptsize, inner sep = 0em] (text1787) {C};
				\end{scope}
			\end{scope}
			% path1793
			\path[fill = gray!50, even odd rule, line width = 0.480pt] (116.7570, 714.9680) -- (123.9570, 714.9680) -- (123.9570, 722.2400) -- (116.7570, 722.2400) -- cycle;
		
			% path1795
			\path[draw = black, line join = miter, line cap = butt, miter limit = 10.00, line width = 0.461pt] (116.7570, 714.9680) -- (123.9570, 714.9680) -- (123.9570, 722.2400) -- (116.7570, 722.2400) -- cycle;
		
			\begin{scope}[cm = {{0.6, 0.0, 0.0, 0.6, (85.149, 560.6)}}]% g1803
				\begin{scope}% g1801
					% text1799
					\path[cm = {{1.0, 0.0, 0.0, -1.0, (54.936, 258.58)}}, fill = black, nonzero rule] (0.0000, 0.0000) node[above right, font = \scriptsize, inner sep = 0em] (text1799) {C};
				\end{scope}
			\end{scope}
			% path1805
			\path[fill = gray!50, even odd rule, line width = 0.480pt] (135.3330, 714.6800) -- (142.6050, 714.6800) -- (142.6050, 721.9520) -- (135.3330, 721.9520) -- cycle;
		
			% path1807
			\path[draw = black, line join = miter, line cap = butt, miter limit = 10.00, line width = 0.461pt] (135.3330, 714.6800) -- (142.6050, 714.6800) -- (142.6050, 721.9520) -- (135.3330, 721.9520) -- cycle;
		
			\begin{scope}[cm = {{0.6, 0.0, 0.0, 0.6, (85.149, 560.6)}}]% g1815
				\begin{scope}% g1813
					% text1811
					\path[cm = {{1.0, 0.0, 0.0, -1.0, (85.944, 258.17)}}, fill = black, nonzero rule] (0.0000, 0.0000) node[above right, font = \scriptsize, inner sep = 0em] (text1811) {C};
				\end{scope}
			\end{scope}
			% path1817
			\path[fill = gray!50, even odd rule, line width = 0.480pt] (155.7090, 714.6080) -- (162.9810, 714.6080) -- (162.9810, 721.8800) -- (155.7090, 721.8800) -- cycle;
		
			% path1819
			\path[draw = black, line join = miter, line cap = butt, miter limit = 10.00, line width = 0.461pt] (155.7090, 714.6080) -- (162.9810, 714.6080) -- (162.9810, 721.8800) -- (155.7090, 721.8800) -- cycle;
		
			\begin{scope}[cm = {{0.6, 0.0, 0.0, 0.6, (85.149, 560.6)}}]% g1827
				\begin{scope}% g1825
					% text1823
					\path[cm = {{1.0, 0.0, 0.0, -1.0, (119.98, 258.0)}}, fill = black, nonzero rule] (0.0000, 0.0000) node[above right, font = \scriptsize, inner sep = 0em] (text1823) {C};
				\end{scope}
			\end{scope}
		\end{scope}
	\end{tikzpicture}%
}


\newcommand{\delegationDiagram}[1][1.0]{%
	\begin{tikzpicture}[y = 0.80pt,
	                             x = 0.80pt,
	                             yscale = -1.000000,
	                             xscale = 1.000000,
	                             inner sep = 0pt,
	                             outer sep = 0pt,
	                             opacity = #1,
	                             baseline = (current bounding box.center)]
		\begin{scope}[cm = {{1.33333, 0.0, 0.0, -1.33333, (0.0, 1056.0)}}]% g823
			% path1209
			\path[fill = gray!50, even odd rule, line width = 0.480pt] (90.2610, 582.9920) -- (92.9970, 582.9920) -- (92.9970, 590.9120) -- (90.2610, 590.9120) -- cycle;
		
			% path1211
			\path[draw = black, line join = miter, line cap = butt, miter limit = 10.00, line width = 0.461pt] (90.2610, 582.9920) -- (92.9970, 582.9920) -- (92.9970, 590.9120) -- (90.2610, 590.9120) -- cycle;
		
			% path1213
			\path[fill = gray!50, even odd rule, line width = 0.480pt] (94.0050, 582.9920) -- (96.8130, 582.9920) -- (96.8130, 590.9120) -- (94.0050, 590.9120) -- cycle;
		
			% path1215
			\path[draw = black, line join = miter, line cap = butt, miter limit = 10.00, line width = 0.461pt] (94.0050, 582.9920) -- (96.8130, 582.9920) -- (96.8130, 590.9120) -- (94.0050, 590.9120) -- cycle;
		
			% path1217
			\path[fill = gray!50, even odd rule, line width = 0.480pt] (97.7490, 582.9920) -- (100.5570, 582.9920) -- (100.5570, 590.9120) -- (97.7490, 590.9120) -- cycle;
		
			% path1219
			\path[draw = black, line join = miter, line cap = butt, miter limit = 10.00, line width = 0.461pt] (97.7490, 582.9920) -- (100.5570, 582.9920) -- (100.5570, 590.9120) -- (97.7490, 590.9120) -- cycle;
		
			% path1221
			\path[fill = gray!50, even odd rule, line width = 0.480pt] (101.5650, 582.9920) -- (104.3010, 582.9920) -- (104.3010, 590.9120) -- (101.5650, 590.9120) -- cycle;
		
			% path1223
			\path[draw = black, line join = miter, line cap = butt, miter limit = 10.00, line width = 0.461pt] (101.5650, 582.9920) -- (104.3010, 582.9920) -- (104.3010, 590.9120) -- (101.5650, 590.9120) -- cycle;
		
			% path1225
			\path[fill = gray!50, even odd rule, line width = 0.480pt] (105.3090, 582.9920) -- (108.1170, 582.9920) -- (108.1170, 590.9120) -- (105.3090, 590.9120) -- cycle;
		
			% path1227
			\path[draw = black, line join = miter, line cap = butt, miter limit = 10.00, line width = 0.461pt] (105.3090, 582.9920) -- (108.1170, 582.9920) -- (108.1170, 590.9120) -- (105.3090, 590.9120) -- cycle;
		
			% path1229
			\path[fill = gray!50, even odd rule, line width = 0.480pt] (90.2610, 595.5200) -- (99.1890, 608.7680) -- (108.1170, 595.5200) -- cycle;
		
			% path1231
			\path[draw = black, line join = miter, line cap = butt, miter limit = 10.00, line width = 0.461pt] (90.2610, 595.5200) -- (99.1890, 608.7680) -- (108.1170, 595.5200) -- cycle;
		
			% path1233
			\path[draw = black, line join = miter, line cap = butt, miter limit = 10.00, line width = 0.749pt] (92.1690, 595.4840) -- (95.4132, 590.8682);
		
			% path1235
			\path[draw = black, line join = miter, line cap = butt, miter limit = 10.00, line width = 0.749pt] (94.9050, 595.4840) -- (102.9858, 590.8682);
		
			% path1237
			\path[draw = black, line join = miter, line cap = butt, miter limit = 10.00, line width = 0.749pt] (99.1950, 595.4840) -- (91.6650, 590.8682);
		
			% path1239
			\path[draw = black, line join = miter, line cap = butt, miter limit = 10.00, line width = 0.749pt] (101.7450, 595.4840) -- (106.7046, 590.8682);
		
			% path1241
			\path[draw = black, line join = miter, line cap = butt, miter limit = 10.00, line width = 0.749pt] (105.3594, 595.4840) -- (99.2250, 590.9096);
		
			% path1243
			\path[fill = gray!50, even odd rule, line width = 0.480pt] (109.2690, 582.9920) -- (112.0770, 582.9920) -- (112.0770, 590.9120) -- (109.2690, 590.9120) -- cycle;
		
			% path1245
			\path[draw = black, line join = miter, line cap = butt, miter limit = 10.00, line width = 0.461pt] (109.2690, 582.9920) -- (112.0770, 582.9920) -- (112.0770, 590.9120) -- (109.2690, 590.9120) -- cycle;
		
			% path1247
			\path[fill = gray!50, even odd rule, line width = 0.480pt] (113.0130, 582.9920) -- (115.8210, 582.9920) -- (115.8210, 590.9120) -- (113.0130, 590.9120) -- cycle;
		
			% path1249
			\path[draw = black, line join = miter, line cap = butt, miter limit = 10.00, line width = 0.461pt] (113.0130, 582.9920) -- (115.8210, 582.9920) -- (115.8210, 590.9120) -- (113.0130, 590.9120) -- cycle;
		
			% path1251
			\path[fill = gray!50, even odd rule, line width = 0.480pt] (116.7570, 582.9920) -- (119.5650, 582.9920) -- (119.5650, 590.9120) -- (116.7570, 590.9120) -- cycle;
		
			% path1253
			\path[draw = black, line join = miter, line cap = butt, miter limit = 10.00, line width = 0.461pt] (116.7570, 582.9920) -- (119.5650, 582.9920) -- (119.5650, 590.9120) -- (116.7570, 590.9120) -- cycle;
		
			% path1255
			\path[fill = gray!50, even odd rule, line width = 0.480pt] (120.5730, 582.9920) -- (123.3810, 582.9920) -- (123.3810, 590.9120) -- (120.5730, 590.9120) -- cycle;
		
			% path1257
			\path[draw = black, line join = miter, line cap = butt, miter limit = 10.00, line width = 0.461pt] (120.5730, 582.9920) -- (123.3810, 582.9920) -- (123.3810, 590.9120) -- (120.5730, 590.9120) -- cycle;
		
			% path1259
			\path[fill = gray!50, even odd rule, line width = 0.480pt] (124.3170, 582.9920) -- (127.1250, 582.9920) -- (127.1250, 590.9120) -- (124.3170, 590.9120) -- cycle;
		
			% path1261
			\path[draw = black, line join = miter, line cap = butt, miter limit = 10.00, line width = 0.461pt] (124.3170, 582.9920) -- (127.1250, 582.9920) -- (127.1250, 590.9120) -- (124.3170, 590.9120) -- cycle;
		
			% path1263
			\path[fill = gray!50, even odd rule, line width = 0.480pt] (109.2690, 595.5200) -- (118.1970, 608.7680) -- (127.1250, 595.5200) -- cycle;
		
			% path1265
			\path[draw = black, line join = miter, line cap = butt, miter limit = 10.00, line width = 0.461pt] (109.2690, 595.5200) -- (118.1970, 608.7680) -- (127.1250, 595.5200) -- cycle;
		
			% path1267
			\path[draw = black, line join = miter, line cap = butt, miter limit = 10.00, line width = 0.749pt] (111.2490, 595.4840) -- (114.4932, 590.8682);
		
			% path1269
			\path[draw = black, line join = miter, line cap = butt, miter limit = 10.00, line width = 0.749pt] (113.9130, 595.4840) -- (121.9938, 590.8682);
		
			% path1271
			\path[draw = black, line join = miter, line cap = butt, miter limit = 10.00, line width = 0.749pt] (118.2030, 595.4840) -- (110.6730, 590.8682);
		
			% path1273
			\path[draw = black, line join = miter, line cap = butt, miter limit = 10.00, line width = 0.749pt] (120.8250, 595.4840) -- (125.7846, 590.8682);
		
			% path1275
			\path[draw = black, line join = miter, line cap = butt, miter limit = 10.00, line width = 0.749pt] (124.3674, 595.4840) -- (118.2330, 590.9096);
		
			% path1277
			\path[fill = gray!50, even odd rule, line width = 0.480pt] (128.9250, 582.9920) -- (131.6610, 582.9920) -- (131.6610, 590.9120) -- (128.9250, 590.9120) -- cycle;
		
			% path1279
			\path[draw = black, line join = miter, line cap = butt, miter limit = 10.00, line width = 0.461pt] (128.9250, 582.9920) -- (131.6610, 582.9920) -- (131.6610, 590.9120) -- (128.9250, 590.9120) -- cycle;
		
			% path1281
			\path[fill = gray!50, even odd rule, line width = 0.480pt] (132.6690, 582.9920) -- (135.4770, 582.9920) -- (135.4770, 590.9120) -- (132.6690, 590.9120) -- cycle;
		
			% path1283
			\path[draw = black, line join = miter, line cap = butt, miter limit = 10.00, line width = 0.461pt] (132.6690, 582.9920) -- (135.4770, 582.9920) -- (135.4770, 590.9120) -- (132.6690, 590.9120) -- cycle;
		
			% path1285
			\path[fill = gray!50, even odd rule, line width = 0.480pt] (136.4130, 582.9920) -- (139.2210, 582.9920) -- (139.2210, 590.9120) -- (136.4130, 590.9120) -- cycle;
		
			% path1287
			\path[draw = black, line join = miter, line cap = butt, miter limit = 10.00, line width = 0.461pt] (136.4130, 582.9920) -- (139.2210, 582.9920) -- (139.2210, 590.9120) -- (136.4130, 590.9120) -- cycle;
		
			% path1289
			\path[fill = gray!50, even odd rule, line width = 0.480pt] (140.2290, 582.9920) -- (142.9650, 582.9920) -- (142.9650, 590.9120) -- (140.2290, 590.9120) -- cycle;
		
			% path1291
			\path[draw = black, line join = miter, line cap = butt, miter limit = 10.00, line width = 0.461pt] (140.2290, 582.9920) -- (142.9650, 582.9920) -- (142.9650, 590.9120) -- (140.2290, 590.9120) -- cycle;
		
			% path1293
			\path[fill = gray!50, even odd rule, line width = 0.480pt] (143.9730, 582.9920) -- (146.7810, 582.9920) -- (146.7810, 590.9120) -- (143.9730, 590.9120) -- cycle;
		
			% path1295
			\path[draw = black, line join = miter, line cap = butt, miter limit = 10.00, line width = 0.461pt] (143.9730, 582.9920) -- (146.7810, 582.9920) -- (146.7810, 590.9120) -- (143.9730, 590.9120) -- cycle;
		
			% path1297
			\path[fill = gray!50, even odd rule, line width = 0.480pt] (128.9250, 595.5200) -- (137.8530, 608.7680) -- (146.7810, 595.5200) -- cycle;
		
			% path1299
			\path[draw = black, line join = miter, line cap = butt, miter limit = 10.00, line width = 0.461pt] (128.9250, 595.5200) -- (137.8530, 608.7680) -- (146.7810, 595.5200) -- cycle;
		
			% path1301
			\path[draw = black, line join = miter, line cap = butt, miter limit = 10.00, line width = 0.749pt] (130.8330, 595.4840) -- (134.0772, 590.8682);
		
			% path1303
			\path[draw = black, line join = miter, line cap = butt, miter limit = 10.00, line width = 0.749pt] (133.5690, 595.4840) -- (141.6498, 590.8682);
		
			% path1305
			\path[draw = black, line join = miter, line cap = butt, miter limit = 10.00, line width = 0.749pt] (137.8590, 595.4840) -- (130.3290, 590.8682);
		
			% path1307
			\path[draw = black, line join = miter, line cap = butt, miter limit = 10.00, line width = 0.749pt] (140.4090, 595.4840) -- (145.3710, 590.8682);
		
			% path1309
			\path[draw = black, line join = miter, line cap = butt, miter limit = 10.00, line width = 0.749pt] (144.0234, 595.4840) -- (137.8890, 590.9096);
		
			% path1311
			\path[fill = gray!50, even odd rule, line width = 0.480pt] (148.5090, 582.9920) -- (151.3170, 582.9920) -- (151.3170, 590.9120) -- (148.5090, 590.9120) -- cycle;
		
			% path1313
			\path[draw = black, line join = miter, line cap = butt, miter limit = 10.00, line width = 0.461pt] (148.5090, 582.9920) -- (151.3170, 582.9920) -- (151.3170, 590.9120) -- (148.5090, 590.9120) -- cycle;
		
			% path1315
			\path[fill = gray!50, even odd rule, line width = 0.480pt] (152.3250, 582.9920) -- (155.0610, 582.9920) -- (155.0610, 590.9120) -- (152.3250, 590.9120) -- cycle;
		
			% path1317
			\path[draw = black, line join = miter, line cap = butt, miter limit = 10.00, line width = 0.461pt] (152.3250, 582.9920) -- (155.0610, 582.9920) -- (155.0610, 590.9120) -- (152.3250, 590.9120) -- cycle;
		
			% path1319
			\path[fill = gray!50, even odd rule, line width = 0.480pt] (156.0690, 582.9920) -- (158.8770, 582.9920) -- (158.8770, 590.9120) -- (156.0690, 590.9120) -- cycle;
		
			% path1321
			\path[draw = black, line join = miter, line cap = butt, miter limit = 10.00, line width = 0.461pt] (156.0690, 582.9920) -- (158.8770, 582.9920) -- (158.8770, 590.9120) -- (156.0690, 590.9120) -- cycle;
		
			% path1323
			\path[fill = gray!50, even odd rule, line width = 0.480pt] (159.8130, 582.9920) -- (162.6210, 582.9920) -- (162.6210, 590.9120) -- (159.8130, 590.9120) -- cycle;
		
			% path1325
			\path[draw = black, line join = miter, line cap = butt, miter limit = 10.00, line width = 0.461pt] (159.8130, 582.9920) -- (162.6210, 582.9920) -- (162.6210, 590.9120) -- (159.8130, 590.9120) -- cycle;
		
			% path1327
			\path[fill = gray!50, even odd rule, line width = 0.480pt] (163.5570, 582.9920) -- (166.3650, 582.9920) -- (166.3650, 590.9120) -- (163.5570, 590.9120) -- cycle;
		
			% path1329
			\path[draw = black, line join = miter, line cap = butt, miter limit = 10.00, line width = 0.461pt] (163.5570, 582.9920) -- (166.3650, 582.9920) -- (166.3650, 590.9120) -- (163.5570, 590.9120) -- cycle;
		
			% path1331
			\path[fill = gray!50, even odd rule, line width = 0.480pt] (148.5090, 595.5200) -- (157.4370, 608.7680) -- (166.3650, 595.5200) -- cycle;
		
			% path1333
			\path[draw = black, line join = miter, line cap = butt, miter limit = 10.00, line width = 0.461pt] (148.5090, 595.5200) -- (157.4370, 608.7680) -- (166.3650, 595.5200) -- cycle;
		
			% path1335
			\path[draw = black, line join = miter, line cap = butt, miter limit = 10.00, line width = 0.749pt] (150.4890, 595.4840) -- (153.7350, 590.8682);
		
			% path1337
			\path[draw = black, line join = miter, line cap = butt, miter limit = 10.00, line width = 0.749pt] (153.1530, 595.4840) -- (161.2350, 590.8682);
		
			% path1339
			\path[draw = black, line join = miter, line cap = butt, miter limit = 10.00, line width = 0.749pt] (157.5150, 595.4840) -- (149.9850, 590.8682);
		
			% path1341
			\path[draw = black, line join = miter, line cap = butt, miter limit = 10.00, line width = 0.749pt] (160.0650, 595.4840) -- (165.0270, 590.8682);
		
			% path1343
			\path[draw = black, line join = miter, line cap = butt, miter limit = 10.00, line width = 0.749pt] (163.6050, 595.4840) -- (157.4730, 590.9096);
		
			% path1505
			\path[fill = black, nonzero rule, line width = 0.480pt] (97.9188, 644.8520) -- (99.1098, 645.5120) -- (100.3110, 646.1540) -- (101.5272, 646.7600) -- (102.7602, 647.3060) -- (104.0142, 647.7740) -- (104.6574, 647.9780) -- (105.3084, 648.1460) -- (105.9594, 648.2900) -- (106.6254, 648.3980) -- (107.2968, 648.4760) -- (107.9814, 648.5060) -- (108.6726, 648.5000) -- (109.3698, 648.4580) -- (110.0664, 648.3800) -- (110.7738, 648.2660) -- (111.4788, 648.1220) -- (112.1946, 647.9600) -- (113.6250, 647.5460) -- (115.0698, 647.0600) -- (116.5176, 646.5080) -- (117.9690, 645.9260) -- (118.5654, 645.6800) -- (118.2054, 644.8160) -- (117.6120, 645.0620) -- (117.6170, 645.0560) -- (116.1728, 645.6380) -- (116.1808, 645.6380) -- (114.7450, 646.1780) -- (114.7618, 646.1720) -- (113.3350, 646.6580) -- (113.3554, 646.6520) -- (111.9466, 647.0540) -- (111.9706, 647.0480) -- (111.2728, 647.2100) -- (111.2860, 647.2100) -- (110.5972, 647.3480) -- (110.6176, 647.3420) -- (109.9294, 647.4500) -- (109.9462, 647.4500) -- (109.2712, 647.5280) -- (109.2982, 647.5280) -- (108.6280, 647.5700) -- (108.6532, 647.5700) -- (107.9872, 647.5700) -- (108.0124, 647.5700) -- (107.3554, 647.5400) -- (107.3854, 647.5400) -- (106.7416, 647.4680) -- (106.7674, 647.4740) -- (106.1236, 647.3660) -- (106.1440, 647.3720) -- (105.5140, 647.2340) -- (105.5356, 647.2400) -- (104.9056, 647.0720) -- (104.9254, 647.0780) -- (104.3044, 646.8860) -- (104.3290, 646.8920) -- (103.1002, 646.4360) -- (103.1272, 646.4480) -- (101.9170, 645.9080) -- (101.9350, 645.9140) -- (100.7332, 645.3140) -- (100.7446, 645.3200) -- (99.5518, 644.6900) -- (99.5584, 644.6900) -- (98.3704, 644.0360) -- cycle(116.2416, 648.8480) -- (119.2428, 644.8880) -- (114.3174, 644.2280) .. 	controls (114.0612, 644.1920) and (113.8254, 644.3720) .. (113.7912, 644.6300) .. 	controls (113.7564, 644.8820) and (113.9364, 645.1220) .. (114.1926, 645.1520) -- (118.3230, 645.7100) -- (118.0122, 644.9600) -- (115.4958, 648.2840) .. controls (115.3398, 648.4880) and (115.3800, 648.7820) .. (115.5858, 648.9380) .. controls (115.7922, 649.0940) and (116.0856, 649.0580) .. (116.2416, 648.8480) -- cycle;
		
			% path1507
			\path[fill = black, nonzero rule, line width = 0.480pt] (99.1968, 630.6500) -- (99.5592, 630.4700) -- (100.7604, 629.8940) -- (101.9760, 629.3480) -- (103.2072, 628.8560) -- (104.4642, 628.4300) -- (105.7458, 628.1000) -- (106.4004, 627.9740) -- (107.0658, 627.8720) -- (107.7318, 627.8060) -- (108.4134, 627.7760) -- (109.1046, 627.7820) -- (109.7970, 627.8240) -- (110.4912, 627.8900) -- (111.2016, 627.9920) -- (112.6122, 628.2740) -- (114.0438, 628.6400) -- (115.4874, 629.0780) -- (116.9346, 629.5700) -- (118.3860, 630.0980) -- (119.8362, 630.6380) -- (119.5098, 631.5200) -- (118.0608, 630.9800) -- (118.0638, 630.9800) -- (116.6190, 630.4520) -- (116.6292, 630.4580) -- (115.1940, 629.9720) -- (115.2072, 629.9720) -- (113.7804, 629.5400) -- (113.8014, 629.5460) -- (112.3932, 629.1860) -- (112.4178, 629.1920) -- (111.0318, 628.9160) -- (111.0564, 628.9160) -- (110.3676, 628.8200) -- (110.3880, 628.8200) -- (109.7130, 628.7540) -- (109.7310, 628.7540) -- (109.0608, 628.7180) -- (109.0860, 628.7180) -- (108.4200, 628.7120) -- (108.4452, 628.7120) -- (107.7882, 628.7420) -- (107.8116, 628.7420) -- (107.1678, 628.8020) -- (107.1906, 628.8020) -- (106.5468, 628.8980) -- (106.5678, 628.8920) -- (105.9378, 629.0120) -- (105.9666, 629.0060) -- (104.7156, 629.3300) -- (104.7474, 629.3240) -- (103.5192, 629.7380) -- (103.5438, 629.7260) -- (102.3336, 630.2120) -- (102.3510, 630.2060) -- (101.1492, 630.7460) -- (101.1600, 630.7400) -- (99.9672, 631.3100) -- (99.9744, 631.3100) -- (99.6150, 631.4900) -- cycle(103.5348, 631.8020) -- (98.5752, 631.4840) -- (101.2950, 627.3260) .. 	controls (101.4366, 627.1100) and (101.7264, 627.0500) .. (101.9430, 627.1880) .. 	controls (102.1590, 627.3320) and (102.2196, 627.6200) .. (102.0780, 627.8360) -- (102.0780, 627.8360) -- (99.7974, 631.3280) -- (99.4362, 630.6020) -- (103.5948, 630.8720) .. controls (103.8528, 630.8900) and (104.0484, 631.1120) .. 	(104.0322, 631.3700) .. controls (104.0154, 631.6280) and (103.7928, 631.8200) .. 	(103.5348, 631.8020) -- cycle;
		
			% path1533
			\path[fill = green!50, even odd rule, line width = 0.480pt] (119.0838, 579.8300) -- (119.5662, 579.8300) -- (119.5662, 581.3306) -- (119.6580, 581.6054) -- (119.7960, 581.8472) -- (119.9334, 582.0260) -- (120.0366, 582.1634) -- (120.1692, 582.2912) -- (120.3294, 582.3962) -- (120.5190, 582.5060) -- (120.6858, 582.5792) -- (120.9156, 582.6620) -- (121.1394, 582.7124) -- (121.4034, 582.7580) -- (121.6848, 582.7718) -- (121.9260, 582.7580) -- (122.1150, 582.7394) -- (122.2758, 582.6986) -- (122.4600, 582.6482) -- (122.6376, 582.5888) -- (122.6952, 582.5654) -- (122.7582, 582.5336) -- (122.8044, 582.5108) -- (122.8560, 582.4832) -- (122.8674, 582.4792) -- (122.9076, 582.4516) -- (122.9478, 582.4240) -- (122.9880, 582.3874) -- (123.0336, 582.3556) -- (123.1716, 582.2458) -- (123.2580, 582.1636) -- (123.3726, 582.0538) -- (123.4644, 581.9164) -- (123.5448, 581.7976) -- (123.6312, 581.6602) -- (123.6996, 581.5090) -- (123.7686, 581.3308) -- (123.8148, 581.1844) -- (123.8198, 579.8302) -- (124.3886, 579.8302) -- (124.3886, 577.7812) -- (124.3424, 577.5064) -- (124.2506, 577.2868) -- (124.1588, 577.1404) -- (124.0784, 576.9940) -- (123.9752, 576.8476) -- (123.8432, 576.7012) -- (123.6992, 576.5686) -- (123.5156, 576.4636) -- (123.3206, 576.3718) -- (123.1370, 576.2806) -- (122.8958, 576.2074) -- (122.7122, 576.1708) -- (122.4308, 576.1294) -- (121.1276, 576.1344) -- (120.8288, 576.1710) -- (120.5648, 576.2346) -- (120.3062, 576.3084) -- (120.0482, 576.4086) -- (119.8640, 576.5232) -- (119.7434, 576.6054) -- (119.6234, 576.7290) -- (119.5196, 576.8250) -- (119.4278, 576.9258) -- (119.3246, 577.0488) -- (119.2442, 577.1772) -- (119.1410, 577.3602) -- (119.1008, 577.5522) -- (119.0834, 577.7262) -- cycle(120.5652, 579.8350) -- (120.5652, 581.1754) -- (120.5880, 581.2714) -- (120.6168, 581.3536) -- (120.6684, 581.4358) -- (120.7086, 581.5228) -- (120.7602, 581.5870) -- (120.8634, 581.6968) -- (120.9954, 581.7700) -- (121.1334, 581.8522) -- (121.2888, 581.9074) -- (121.4436, 581.9482) -- (121.6500, 581.9710) -- (121.8054, 581.9620) -- (122.0178, 581.9212) -- (122.1498, 581.8888) -- (122.2704, 581.8474) -- (122.3394, 581.7976) -- (122.4426, 581.7424) -- (122.4828, 581.7016) -- (122.5284, 581.6602) -- (122.5860, 581.6146) -- (122.6208, 581.5642) -- (122.6664, 581.5180) -- (122.7180, 581.4544) -- (122.7408, 581.3860) -- (122.7696, 581.3122) -- (122.8044, 581.2348) -- (122.8212, 581.1754) -- (122.8212, 579.8350) -- cycle;
		
			% path1535
			\path[fill = green!50, even odd rule, line width = 0.480pt] (121.9428, 577.5290) -- (122.0868, 576.8522) -- (121.4436, 576.8522) -- (121.5984, 577.5248) -- (121.5294, 577.5704) -- (121.4778, 577.6478) -- (121.4664, 577.7444) -- (121.4838, 577.8170) -- (121.5294, 577.8674) -- (121.5642, 577.9088) -- (121.6218, 577.9364) -- (121.6902, 577.9640) -- (121.7766, 577.9680) -- (121.8456, 577.9542) -- (121.9260, 577.9362) -- (121.9944, 577.8858) -- (122.0292, 577.8444) -- (122.0634, 577.7988) -- (122.0694, 577.7442) -- (122.0466, 577.6476) -- (122.0178, 577.6068) -- (121.9830, 577.5702) -- (121.9428, 577.5288);
		
			% path1537
			\path[draw = black, line join = miter, line cap = butt, miter limit = 10.00, line width = 0.461pt] (119.0838, 579.8300) -- (119.5662, 579.8300) -- (119.5662, 581.3306) -- (119.6580, 581.6054) -- (119.7960, 581.8472) -- (119.9334, 582.0260) -- (120.0366, 582.1634) -- (120.1692, 582.2912) -- (120.3294, 582.3962) -- (120.5190, 582.5060) -- (120.6858, 582.5792) -- (120.9156, 582.6620) -- (121.1394, 582.7124) -- (121.4034, 582.7580) -- (121.6848, 582.7718) -- (121.9260, 582.7580) -- (122.1150, 582.7394) -- (122.2758, 582.6986) -- (122.4600, 582.6482) -- (122.6376, 582.5888) -- (122.6952, 582.5654) -- (122.7582, 582.5336) -- (122.8044, 582.5108) -- (122.8560, 582.4832) -- (122.8674, 582.4792) -- (122.9076, 582.4516) -- (122.9478, 582.4240) -- (122.9880, 582.3874) -- (123.0336, 582.3556) -- (123.1716, 582.2458) -- (123.2580, 582.1636) -- (123.3726, 582.0538) -- (123.4644, 581.9164) -- (123.5448, 581.7976) -- (123.6312, 581.6602) -- (123.6996, 581.5090) -- (123.7686, 581.3308) -- (123.8148, 581.1844) -- (123.8198, 579.8302) -- (124.3886, 579.8302) -- (124.3886, 577.7812) -- (124.3424, 577.5064) -- (124.2506, 577.2868) -- (124.1588, 577.1404) -- (124.0784, 576.9940) -- (123.9752, 576.8476) -- (123.8432, 576.7012) -- (123.6992, 576.5686) -- (123.5156, 576.4636) -- (123.3206, 576.3718) -- (123.1370, 576.2806) -- (122.8958, 576.2074) -- (122.7122, 576.1708) -- (122.4308, 576.1294) -- (121.1276, 576.1344) -- (120.8288, 576.1710) -- (120.5648, 576.2346) -- (120.3062, 576.3084) -- (120.0482, 576.4086) -- (119.8640, 576.5232) -- (119.7434, 576.6054) -- (119.6234, 576.7290) -- (119.5196, 576.8250) -- (119.4278, 576.9258) -- (119.3246, 577.0488) -- (119.2442, 577.1772) -- (119.1410, 577.3602) -- (119.1008, 577.5522) -- (119.0834, 577.7262) -- cycle(120.5652, 579.8350) -- (120.5652, 581.1754) -- (120.5880, 581.2714) -- (120.6168, 581.3536) -- (120.6684, 581.4358) -- (120.7086, 581.5228) -- (120.7602, 581.5870) -- (120.8634, 581.6968) -- (120.9954, 581.7700) -- (121.1334, 581.8522) -- (121.2888, 581.9074) -- (121.4436, 581.9482) -- (121.6500, 581.9710) -- (121.8054, 581.9620) -- (122.0178, 581.9212) -- (122.1498, 581.8888) -- (122.2704, 581.8474) -- (122.3394, 581.7976) -- (122.4426, 581.7424) -- (122.4828, 581.7016) -- (122.5284, 581.6602) -- (122.5860, 581.6146) -- (122.6208, 581.5642) -- (122.6664, 581.5180) -- (122.7180, 581.4544) -- (122.7408, 581.3860) -- (122.7696, 581.3122) -- (122.8044, 581.2348) -- (122.8212, 581.1754) -- (122.8212, 579.8350) -- cycle;
		
			% path1539
			\path[draw = black, line join = miter, line cap = butt, miter limit = 10.00, line width = 0.461pt] (119.0838, 579.8300) -- (124.3890, 579.8300) -- cycle;
		
			% path1541
			\path[draw = black, line join = miter, line cap = butt, miter limit = 10.00, line width = 0.461pt] (121.9428, 577.5290) -- (122.0868, 576.8522) -- (121.4436, 576.8522) -- (121.5984, 577.5248) -- (121.5294, 577.5704) -- (121.4778, 577.6478) -- (121.4664, 577.7444) -- (121.4838, 577.8170) -- (121.5294, 577.8674) -- (121.5642, 577.9088) -- (121.6218, 577.9364) -- (121.6902, 577.9640) -- (121.7766, 577.9680) -- (121.8456, 577.9542) -- (121.9260, 577.9362) -- (121.9944, 577.8858) -- (122.0292, 577.8444) -- (122.0634, 577.7988) -- (122.0694, 577.7442) -- (122.0466, 577.6476) -- (122.0178, 577.6068) -- (121.9830, 577.5702) -- (121.9428, 577.5288);
		
			% path1661
			\path[draw = red, line join = miter, line cap = butt, miter limit = 10.00, line width = 1.094pt] (117.8730, 642.8600) .. controls (119.0100, 642.3800) and (120.1470, 641.9000) .. (120.1980, 641.0780) .. controls (120.2490, 640.2560) and (118.2096, 638.6780) .. (118.1790, 637.9220) .. controls (118.1484, 637.1720) and (119.9838, 637.4000) .. (120.0144, 636.5540) .. controls (120.0450, 635.7080) and (118.6788, 633.4940) .. (118.3626, 632.8520);
		
			% path1663
			\path[draw = green, line join = miter, line cap = butt, miter limit = 10.00, line width = 1.094pt] (97.6410, 642.9320) .. controls (98.8122, 642.4460) and (99.9840, 641.9660) .. (100.0362, 641.1380) .. controls (100.0890, 640.3100) and (97.9878, 638.7200) .. (97.9560, 637.9640) .. controls (97.9248, 637.2020) and (99.8160, 637.4300) .. (99.8472, 636.5780) .. controls (99.8790, 635.7260) and (98.4708, 633.4940) .. (98.1456, 632.8520);
		
			% path1665
			\path[draw = orange, line join = miter, line cap = butt, miter limit = 10.00, line width = 1.094pt] (156.4650, 643.0040) .. controls (157.6050, 642.5180) and (158.7390, 642.0380) .. (158.7930, 641.2100) .. controls (158.8410, 640.3820) and (156.8010, 638.7920) .. (156.7710, 638.0360) .. controls (156.7410, 637.2740) and (158.5770, 637.5020) .. (158.6070, 636.6500) .. controls (158.6370, 635.7980) and (157.2690, 633.5660) .. (156.9570, 632.9240);
		
			% path1667
			\path[draw = blue, line join = miter, line cap = butt, miter limit = 10.00, line width = 1.094pt] (135.9450, 643.0040) .. controls (137.0820, 642.5180) and (138.2190, 642.0380) .. (138.2700, 641.2100) .. controls (138.3210, 640.3820) and (136.2816, 638.7920) .. (136.2510, 638.0360) .. controls (136.2204, 637.2740) and (138.0558, 637.5020) .. (138.0864, 636.6500) .. controls (138.1170, 635.7980) and (136.7508, 633.5660) .. (136.4346, 632.9240);
		
			% path1669
			\path[fill = gray!50, even odd rule, line width = 0.480pt] (95.5890, 620.4320) -- (102.7890, 620.4320) -- (102.7890, 627.7040) -- (95.5890, 627.7040) -- cycle;
		
			% path1671
			\path[draw = black, line join = miter, line cap = butt, miter limit = 10.00, line width = 0.461pt] (95.5890, 620.4320) -- (102.7890, 620.4320) -- (102.7890, 627.7040) -- (95.5890, 627.7040) -- cycle;
		
			\begin{scope}[cm = {{0.6, 0.0, 0.0, 0.6, (85.149, 560.6)}}]% g1679
				\begin{scope}% g1677
					% text1675
					\path[cm = {{1.0, 0.0, 0.0, -1.0, (19.656, 100.99)}}, fill = black, nonzero rule] (0.0000, 0.0000) node[above right, font = \scriptsize, inner sep = 0em] (text1675) {C};
				\end{scope}
			\end{scope}
			% path1681
			\path[fill = gray!50, even odd rule, line width = 0.480pt] (115.2450, 620.5040) -- (122.5170, 620.5040) -- (122.5170, 627.7760) -- (115.2450, 627.7760) -- cycle;
		
			% path1683
			\path[draw = black, line join = miter, line cap = butt, miter limit = 10.00, line width = 0.461pt] (115.2450, 620.5040) -- (122.5170, 620.5040) -- (122.5170, 627.7760) -- (115.2450, 627.7760) -- cycle;
		
			\begin{scope}[cm = {{0.6, 0.0, 0.0, 0.6, (85.149, 560.6)}}]% g1691
				\begin{scope}% g1689
					% text1687
					\path[cm = {{1.0, 0.0, 0.0, -1.0, (52.536, 101.14)}}, fill = black, nonzero rule] (0.0000, 0.0000) node[above right, font = \scriptsize, inner sep = 0em] (text1687) {C};
				\end{scope}
			\end{scope}
			% path1693
			\path[fill = gray!50, even odd rule, line width = 0.480pt] (133.8930, 620.2880) -- (141.1650, 620.2880) -- (141.1650, 627.4880) -- (133.8930, 627.4880) -- cycle;
		
			% path1695
			\path[draw = black, line join = miter, line cap = butt, miter limit = 10.00, line width = 0.461pt] (133.8930, 620.2880) -- (141.1650, 620.2880) -- (141.1650, 627.4880) -- (133.8930, 627.4880) -- cycle;
		
			\begin{scope}[cm = {{0.6, 0.0, 0.0, 0.6, (85.149, 560.6)}}]% g1703
				\begin{scope}% g1701
					% text1699
					\path[cm = {{1.0, 0.0, 0.0, -1.0, (83.544, 100.7)}}, fill = black, nonzero rule] (0.0000, 0.0000) node[above right, font = \scriptsize, inner sep = 0em] (text1699) {C};
				\end{scope}
			\end{scope}
			% path1705
			\path[fill = gray!50, even odd rule, line width = 0.480pt] (154.2690, 620.1440) -- (161.5410, 620.1440) -- (161.5410, 627.4160) -- (154.2690, 627.4160) -- cycle;
		
			% path1707
			\path[draw = black, line join = miter, line cap = butt, miter limit = 10.00, line width = 0.461pt] (154.2690, 620.1440) -- (161.5410, 620.1440) -- (161.5410, 627.4160) -- (154.2690, 627.4160) -- cycle;
		
			\begin{scope}[cm = {{0.6, 0.0, 0.0, 0.6, (85.149, 560.6)}}]% g1715
				\begin{scope}% g1713
					% text1711
					\path[cm = {{1.0, 0.0, 0.0, -1.0, (117.58, 100.54)}}, fill = black, nonzero rule] (0.0000, 0.0000) node[above right, font = \scriptsize, inner sep = 0em] (text1711) {C};
				\end{scope}
			\end{scope}
			% path1837
			\path[draw = black, line join = miter, line cap = butt, miter limit = 10.00, line width = 0.461pt] (89.6130, 615.1160) .. controls (89.6130, 616.8986) and (91.0583, 618.3440) .. (92.8410, 618.3440) -- (105.7530, 618.3440) .. controls (107.5356, 618.3440) and (108.9810, 616.8986) .. (108.9810, 615.1160) -- (108.9810, 578.6600) .. controls (108.9810, 576.8774) and (107.5356, 575.4320) .. 	(105.7530, 575.4320) -- (92.8410, 575.4320) .. controls (91.0583, 575.4320) and (89.6130, 576.8774) .. (89.6130, 578.6600) -- cycle;
		
			% path1839
			\path[draw = black, line join = miter, line cap = butt, miter limit = 10.00, line width = 0.461pt] (108.9810, 615.1640) .. controls (108.9810, 616.9202) and (110.4048, 618.3440) .. (112.1610, 618.3440) -- (124.8810, 618.3440) .. controls (126.6372, 618.3440) and (128.0610, 616.9202) .. (128.0610, 615.1640) -- (128.0610, 578.6120) .. controls (128.0610, 576.8558) and (126.6372, 575.4320) .. 	(124.8810, 575.4320) -- (112.1610, 575.4320) .. controls (110.4048, 575.4320) and (108.9810, 576.8558) .. (108.9810, 578.6120) -- cycle;
		
			% path1841
			\path[draw = black, line join = miter, line cap = butt, miter limit = 10.00, line width = 0.461pt] (127.9170, 614.9960) .. controls (127.9170, 616.8050) and (129.3840, 618.2720) .. (131.1930, 618.2720) -- (144.2970, 618.2720) .. controls (146.1090, 618.2720) and (147.5730, 616.8050) .. (147.5730, 614.9960) -- (147.5730, 578.6360) .. controls (147.5730, 576.8270) and (146.1090, 575.3600) .. 	(144.2970, 575.3600) -- (131.1930, 575.3600) .. controls (129.3840, 575.3600) and (127.9170, 576.8270) .. (127.9170, 578.6360) -- cycle;
		
			% path1843
			\path[draw = black, line join = miter, line cap = butt, miter limit = 10.00, line width = 0.461pt] (147.6450, 614.9960) .. controls (147.6450, 616.8050) and (149.1090, 618.2720) .. (150.9210, 618.2720) -- (164.0250, 618.2720) .. controls (165.8370, 618.2720) and (167.3010, 616.8050) .. (167.3010, 614.9960) -- (167.3010, 578.6360) .. controls (167.3010, 576.8270) and (165.8370, 575.3600) .. 	(164.0250, 575.3600) -- (150.9210, 575.3600) .. controls (149.1090, 575.3600) and (147.6450, 576.8270) .. (147.6450, 578.6360) -- cycle;
		\end{scope}
	\end{tikzpicture}%
}


\newcommand{\doraDiagram}[1][1.0]{%
	\begin{tikzpicture}[y = 0.80pt,
	                             x = 0.80pt,
	                             yscale = -1.000000,
	                             xscale = 1.000000,
	                             inner sep = 0pt,
	                             outer sep = 0pt,
	                             opacity = #1,
	                             baseline = (current bounding box.center)]
		\begin{scope}[cm = {{1.33333, 0.0, 0.0, -1.33333, (0.0, 1056.0)}}]% g823
			% path1345
			\path[fill = gray!50, even odd rule, line width = 0.480pt] (178.8930, 582.9920) -- (181.7010, 582.9920) -- (181.7010, 590.9120) -- (178.8930, 590.9120) -- cycle;
		
			% path1347
			\path[draw = black, line join = miter, line cap = butt, miter limit = 10.00, line width = 0.461pt] (178.8930, 582.9920) -- (181.7010, 582.9920) -- (181.7010, 590.9120) -- (178.8930, 590.9120) -- cycle;
		
			% path1349
			\path[fill = gray!50, even odd rule, line width = 0.480pt] (182.6370, 582.9920) -- (185.4450, 582.9920) -- (185.4450, 590.9120) -- (182.6370, 590.9120) -- cycle;
		
			% path1351
			\path[draw = black, line join = miter, line cap = butt, miter limit = 10.00, line width = 0.461pt] (182.6370, 582.9920) -- (185.4450, 582.9920) -- (185.4450, 590.9120) -- (182.6370, 590.9120) -- cycle;
		
			% path1353
			\path[fill = gray!50, even odd rule, line width = 0.480pt] (186.3810, 582.9920) -- (189.1890, 582.9920) -- (189.1890, 590.9120) -- (186.3810, 590.9120) -- cycle;
		
			% path1355
			\path[draw = black, line join = miter, line cap = butt, miter limit = 10.00, line width = 0.461pt] (186.3810, 582.9920) -- (189.1890, 582.9920) -- (189.1890, 590.9120) -- (186.3810, 590.9120) -- cycle;
		
			% path1357
			\path[fill = gray!50, even odd rule, line width = 0.480pt] (190.1970, 582.9920) -- (193.0050, 582.9920) -- (193.0050, 590.9120) -- (190.1970, 590.9120) -- cycle;
		
			% path1359
			\path[draw = black, line join = miter, line cap = butt, miter limit = 10.00, line width = 0.461pt] (190.1970, 582.9920) -- (193.0050, 582.9920) -- (193.0050, 590.9120) -- (190.1970, 590.9120) -- cycle;
		
			% path1361
			\path[fill = gray!50, even odd rule, line width = 0.480pt] (193.9410, 582.9920) -- (196.7490, 582.9920) -- (196.7490, 590.9120) -- (193.9410, 590.9120) -- cycle;
		
			% path1363
			\path[draw = black, line join = miter, line cap = butt, miter limit = 10.00, line width = 0.461pt] (193.9410, 582.9920) -- (196.7490, 582.9920) -- (196.7490, 590.9120) -- (193.9410, 590.9120) -- cycle;
		
			% path1365
			\path[fill = gray!50, even odd rule, line width = 0.480pt] (197.9010, 582.9920) -- (200.7090, 582.9920) -- (200.7090, 590.9120) -- (197.9010, 590.9120) -- cycle;
		
			% path1367
			\path[draw = black, line join = miter, line cap = butt, miter limit = 10.00, line width = 0.461pt] (197.9010, 582.9920) -- (200.7090, 582.9920) -- (200.7090, 590.9120) -- (197.9010, 590.9120) -- cycle;
		
			% path1369
			\path[fill = gray!50, even odd rule, line width = 0.480pt] (201.6450, 582.9920) -- (204.4530, 582.9920) -- (204.4530, 590.9120) -- (201.6450, 590.9120) -- cycle;
		
			% path1371
			\path[draw = black, line join = miter, line cap = butt, miter limit = 10.00, line width = 0.461pt] (201.6450, 582.9920) -- (204.4530, 582.9920) -- (204.4530, 590.9120) -- (201.6450, 590.9120) -- cycle;
		
			% path1373
			\path[fill = gray!50, even odd rule, line width = 0.480pt] (205.4610, 582.9920) -- (208.1970, 582.9920) -- (208.1970, 590.9120) -- (205.4610, 590.9120) -- cycle;
		
			% path1375
			\path[draw = black, line join = miter, line cap = butt, miter limit = 10.00, line width = 0.461pt] (205.4610, 582.9920) -- (208.1970, 582.9920) -- (208.1970, 590.9120) -- (205.4610, 590.9120) -- cycle;
		
			% path1377
			\path[fill = gray!50, even odd rule, line width = 0.480pt] (209.2050, 582.9920) -- (212.0130, 582.9920) -- (212.0130, 590.9120) -- (209.2050, 590.9120) -- cycle;
		
			% path1379
			\path[draw = black, line join = miter, line cap = butt, miter limit = 10.00, line width = 0.461pt] (209.2050, 582.9920) -- (212.0130, 582.9920) -- (212.0130, 590.9120) -- (209.2050, 590.9120) -- cycle;
		
			% path1381
			\path[fill = gray!50, even odd rule, line width = 0.480pt] (212.9490, 582.9920) -- (215.7570, 582.9920) -- (215.7570, 590.9120) -- (212.9490, 590.9120) -- cycle;
		
			% path1383
			\path[draw = black, line join = miter, line cap = butt, miter limit = 10.00, line width = 0.461pt] (212.9490, 582.9920) -- (215.7570, 582.9920) -- (215.7570, 590.9120) -- (212.9490, 590.9120) -- cycle;
		
			% path1385
			\path[fill = gray!50, even odd rule, line width = 0.480pt] (217.5570, 582.9920) -- (220.3650, 582.9920) -- (220.3650, 590.9120) -- (217.5570, 590.9120) -- cycle;
		
			% path1387
			\path[draw = black, line join = miter, line cap = butt, miter limit = 10.00, line width = 0.461pt] (217.5570, 582.9920) -- (220.3650, 582.9920) -- (220.3650, 590.9120) -- (217.5570, 590.9120) -- cycle;
		
			% path1389
			\path[fill = gray!50, even odd rule, line width = 0.480pt] (221.3010, 582.9920) -- (224.1090, 582.9920) -- (224.1090, 590.9120) -- (221.3010, 590.9120) -- cycle;
		
			% path1391
			\path[draw = black, line join = miter, line cap = butt, miter limit = 10.00, line width = 0.461pt] (221.3010, 582.9920) -- (224.1090, 582.9920) -- (224.1090, 590.9120) -- (221.3010, 590.9120) -- cycle;
		
			% path1393
			\path[fill = gray!50, even odd rule, line width = 0.480pt] (225.1170, 582.9920) -- (227.8530, 582.9920) -- (227.8530, 590.9120) -- (225.1170, 590.9120) -- cycle;
		
			% path1395
			\path[draw = black, line join = miter, line cap = butt, miter limit = 10.00, line width = 0.461pt] (225.1170, 582.9920) -- (227.8530, 582.9920) -- (227.8530, 590.9120) -- (225.1170, 590.9120) -- cycle;
		
			% path1397
			\path[fill = gray!50, even odd rule, line width = 0.480pt] (228.8610, 582.9920) -- (231.6690, 582.9920) -- (231.6690, 590.9120) -- (228.8610, 590.9120) -- cycle;
		
			% path1399
			\path[draw = black, line join = miter, line cap = butt, miter limit = 10.00, line width = 0.461pt] (228.8610, 582.9920) -- (231.6690, 582.9920) -- (231.6690, 590.9120) -- (228.8610, 590.9120) -- cycle;
		
			% path1401
			\path[fill = gray!50, even odd rule, line width = 0.480pt] (232.6050, 582.9920) -- (235.4130, 582.9920) -- (235.4130, 590.9120) -- (232.6050, 590.9120) -- cycle;
		
			% path1403
			\path[draw = black, line join = miter, line cap = butt, miter limit = 10.00, line width = 0.461pt] (232.6050, 582.9920) -- (235.4130, 582.9920) -- (235.4130, 590.9120) -- (232.6050, 590.9120) -- cycle;
		
			% path1405
			\path[fill = gray!50, even odd rule, line width = 0.480pt] (237.1410, 582.9920) -- (239.9490, 582.9920) -- (239.9490, 590.9120) -- (237.1410, 590.9120) -- cycle;
		
			% path1407
			\path[draw = black, line join = miter, line cap = butt, miter limit = 10.00, line width = 0.461pt] (237.1410, 582.9920) -- (239.9490, 582.9920) -- (239.9490, 590.9120) -- (237.1410, 590.9120) -- cycle;
		
			% path1409
			\path[fill = gray!50, even odd rule, line width = 0.480pt] (240.9570, 582.9920) -- (243.7650, 582.9920) -- (243.7650, 590.9120) -- (240.9570, 590.9120) -- cycle;
		
			% path1411
			\path[draw = black, line join = miter, line cap = butt, miter limit = 10.00, line width = 0.461pt] (240.9570, 582.9920) -- (243.7650, 582.9920) -- (243.7650, 590.9120) -- (240.9570, 590.9120) -- cycle;
		
			% path1413
			\path[fill = gray!50, even odd rule, line width = 0.480pt] (244.7010, 582.9920) -- (247.5090, 582.9920) -- (247.5090, 590.9120) -- (244.7010, 590.9120) -- cycle;
		
			% path1415
			\path[draw = black, line join = miter, line cap = butt, miter limit = 10.00, line width = 0.461pt] (244.7010, 582.9920) -- (247.5090, 582.9920) -- (247.5090, 590.9120) -- (244.7010, 590.9120) -- cycle;
		
			% path1417
			\path[fill = gray!50, even odd rule, line width = 0.480pt] (248.4450, 582.9920) -- (251.2530, 582.9920) -- (251.2530, 590.9120) -- (248.4450, 590.9120) -- cycle;
		
			% path1419
			\path[draw = black, line join = miter, line cap = butt, miter limit = 10.00, line width = 0.461pt] (248.4450, 582.9920) -- (251.2530, 582.9920) -- (251.2530, 590.9120) -- (248.4450, 590.9120) -- cycle;
		
			% path1421
			\path[fill = gray!50, even odd rule, line width = 0.480pt] (252.2610, 582.9920) -- (254.9970, 582.9920) -- (254.9970, 590.9120) -- (252.2610, 590.9120) -- cycle;
		
			% path1423
			\path[draw = black, line join = miter, line cap = butt, miter limit = 10.00, line width = 0.461pt] (252.2610, 582.9920) -- (254.9970, 582.9920) -- (254.9970, 590.9120) -- (252.2610, 590.9120) -- cycle;
		
			% path1465
			\path[draw = black, line join = miter, line cap = butt, miter limit = 10.00, line width = 0.749pt] (181.2330, 595.4840) -- (184.4790, 590.8682);
		
			% path1467
			\path[draw = black, line join = miter, line cap = butt, miter limit = 10.00, line width = 0.749pt] (183.8970, 595.4840) -- (191.9790, 590.8682);
		
			% path1469
			\path[draw = black, line join = miter, line cap = butt, miter limit = 10.00, line width = 0.749pt] (188.1870, 595.4840) -- (180.6570, 590.8682);
		
			% path1471
			\path[draw = black, line join = miter, line cap = butt, miter limit = 10.00, line width = 0.749pt] (190.8090, 595.4840) -- (207.2670, 590.8682);
		
			% path1473
			\path[draw = black, line join = miter, line cap = butt, miter limit = 10.00, line width = 0.749pt] (194.3490, 595.4840) -- (188.2170, 590.9096);
		
			% path1475
			\path[draw = black, line join = miter, line cap = butt, miter limit = 10.00, line width = 0.749pt] (200.2650, 595.4840) -- (195.7770, 590.8682);
		
			% path1477
			\path[draw = black, line join = miter, line cap = butt, miter limit = 10.00, line width = 0.749pt] (202.9050, 595.4840) -- (210.9870, 590.8682);
		
			% path1479
			\path[draw = black, line join = miter, line cap = butt, miter limit = 10.00, line width = 0.749pt] (217.6170, 595.4840) -- (199.7370, 590.8682);
		
			% path1481
			\path[draw = black, line join = miter, line cap = butt, miter limit = 10.00, line width = 0.749pt] (209.8170, 595.4840) -- (214.7790, 590.8682);
		
			% path1483
			\path[draw = black, line join = miter, line cap = butt, miter limit = 10.00, line width = 0.749pt] (213.3570, 595.4840) -- (207.2250, 590.9096);
		
			% path1485
			\path[draw = black, line join = miter, line cap = butt, miter limit = 10.00, line width = 0.749pt] (219.8970, 595.4840) -- (223.1430, 590.8682);
		
			% path1487
			\path[draw = black, line join = miter, line cap = butt, miter limit = 10.00, line width = 0.749pt] (222.5610, 595.4840) -- (230.6430, 590.8682);
		
			% path1489
			\path[draw = black, line join = miter, line cap = butt, miter limit = 10.00, line width = 0.749pt] (226.9230, 595.4840) -- (219.3930, 590.8682);
		
			% path1491
			\path[draw = black, line join = miter, line cap = butt, miter limit = 10.00, line width = 0.749pt] (229.4730, 595.4840) -- (234.4350, 590.8682);
		
			% path1493
			\path[draw = black, line join = miter, line cap = butt, miter limit = 10.00, line width = 0.749pt] (233.0010, 595.4840) -- (238.9530, 590.9096);
		
			% path1495
			\path[draw = black, line join = miter, line cap = butt, miter limit = 10.00, line width = 0.749pt] (239.4810, 595.4840) -- (242.7270, 590.8682);
		
			% path1497
			\path[draw = black, line join = miter, line cap = butt, miter limit = 10.00, line width = 0.749pt] (242.2170, 595.4840) -- (250.2990, 590.8682);
		
			% path1499
			\path[draw = black, line join = miter, line cap = butt, miter limit = 10.00, line width = 0.749pt] (246.5310, 595.4840) -- (234.4410, 590.8682);
		
			% path1501
			\path[draw = black, line join = miter, line cap = butt, miter limit = 10.00, line width = 0.749pt] (249.0570, 595.4840) -- (254.0190, 590.8682);
		
			% path1503
			\path[draw = black, line join = miter, line cap = butt, miter limit = 10.00, line width = 0.749pt] (252.6690, 595.4840) -- (246.5370, 590.9096);
		
			% path1509
			\path[draw = red, line join = miter, line cap = butt, miter limit = 10.00, line width = 1.094pt] (203.1210, 642.9320) .. controls (204.2610, 642.4460) and (205.3950, 641.9660) .. (205.4490, 641.1380) .. controls (205.4970, 640.3100) and (203.4570, 638.7200) .. (203.4270, 637.9640) .. controls (203.3970, 637.2020) and (205.2330, 637.4300) .. (205.2630, 636.5780) .. controls (205.2930, 635.7260) and (203.9250, 633.4940) .. (203.6130, 632.8520);
		
			\begin{scope}[cm = {{0.6, 0.0, 0.0, 0.6, (85.149, 560.6)}}]% g1513
			% path1511
			\path[draw = green, dash pattern = on 5.47pt off 1.82pt, line join = miter, line
			cap = butt, miter limit = 10.00, line width = 1.824pt] (167.9400, 137.2200) .. controls (169.8400, 136.4100) and (171.7300, 135.6100) .. (171.8200, 134.2300) .. controls (171.9000, 132.8500) and (168.5000, 130.2000) .. (168.4500, 128.9400) .. controls (168.4000, 127.6700) and (171.4600, 128.0500) .. (171.5100, 126.6300) .. controls (171.5600, 125.2100) and (169.2800, 121.4900) .. (168.7600, 120.4200);
		
			\end{scope}
			% path1515
			\path[draw = orange, line join = miter, line cap = butt, miter limit = 10.00, line width = 1.094pt] (244.4490, 643.0040) .. controls (245.6190, 642.5240) and (246.7890, 642.0440) .. (246.8430, 641.2220) .. controls (246.8970, 640.4000) and (244.7970, 638.8220) .. (244.7670, 638.0660) .. controls (244.7310, 637.3160) and (246.6210, 637.5440) .. (246.6570, 636.6980) .. controls (246.6870, 635.8520) and (245.2770, 633.6380) .. (244.9530, 632.9960);
		
			% path1517
			\path[draw = blue, line join = miter, line cap = butt, miter limit = 10.00, line width = 1.094pt] (223.9290, 643.0040) .. controls (225.0990, 642.5240) and (226.2690, 642.0440) .. (226.3230, 641.2220) .. controls (226.3770, 640.4000) and (224.2770, 638.8220) .. (224.2470, 638.0660) .. controls (224.2110, 637.3160) and (226.1010, 637.5440) .. (226.1370, 636.6980) .. controls (226.1670, 635.8520) and (224.7570, 633.6380) .. (224.4330, 632.9960);
		
			% path1519
			\path[fill = green, nonzero rule, line width = 0.480pt] (187.9650, 645.1940) -- (189.0630, 645.9080) -- (190.1790, 646.5980) -- (190.3590, 646.7060) -- (190.8270, 645.8960) -- (190.6530, 645.7940) -- (190.6650, 645.8000) -- (189.5610, 645.1160) -- (189.5670, 645.1220) -- (188.4690, 644.4140) -- cycle(191.1630, 647.1740) -- (191.3010, 647.2580) -- (192.4470, 647.8520) -- (193.6110, 648.3560) -- (193.7730, 648.4160) -- (194.0970, 647.5400) -- (193.9530, 647.4860) -- (193.9770, 647.4920) -- (192.8370, 647.0000) -- (192.8670, 647.0120) -- (191.7450, 646.4300) -- (191.7630, 646.4420) -- (191.6370, 646.3700) -- cycle(194.6850, 648.7220) -- (194.8110, 648.7640) -- (195.4290, 648.9200) -- (196.0470, 649.0400) -- (196.6710, 649.1180) -- (197.3130, 649.1540) -- (197.5890, 649.1540) -- (197.5830, 648.2180) -- (197.3190, 648.2180) -- (197.3550, 648.2180) -- (196.7430, 648.1820) -- (196.7790, 648.1880) -- (196.1790, 648.1100) -- (196.2090, 648.1160) -- (195.6150, 648.0020) -- (195.6450, 648.0080) -- (195.0570, 647.8580) -- (195.0810, 647.8700) -- (194.9610, 647.8280) -- cycle(198.5550, 649.1060) -- (198.6090, 649.1060) -- (199.2630, 649.0160) -- (199.9170, 648.8960) -- (200.5770, 648.7400) -- (201.2430, 648.5540) -- (201.4110, 648.5000) -- (201.1170, 647.6120) -- (200.9550, 647.6660) -- (200.9790, 647.6540) -- (200.3370, 647.8340) -- (200.3550, 647.8340) -- (199.7190, 647.9780) -- (199.7370, 647.9780) -- (199.1010, 648.0920) -- (199.1250, 648.0860) -- (198.4950, 648.1760) -- (198.5250, 648.1700) -- (198.4890, 648.1760) -- cycle(202.2990, 648.2000) -- (202.5690, 648.1100) -- (203.9070, 647.5820) -- (204.9390, 647.1260) -- (204.5610, 646.2680) -- (203.5350, 646.7180) -- (203.5530, 646.7120) -- (202.2330, 647.2340) -- (202.2570, 647.2280) -- (202.0050, 647.3120) -- cycle(205.8030, 646.7300) -- (206.5890, 646.3580) -- (207.0990, 646.1120) -- (206.6910, 645.2660) -- (206.1870, 645.5120) -- (206.1870, 645.5120) -- (205.4010, 645.8840) -- cycle(204.9570, 649.4060) -- (207.7290, 645.2840) -- (202.7790, 644.9000) .. controls (202.5210, 644.8760) and (202.2930, 645.0680) .. (202.2750, 645.3260) .. controls (202.2510, 645.5840) and (202.4430, 645.8120) .. (202.7010, 645.8300) -- (206.8590, 646.1540) -- (206.5050, 645.4280) -- (204.1770, 648.8840) .. controls (204.0330, 649.1000) and (204.0930, 649.3880) .. (204.3030, 649.5320) .. controls (204.5190, 649.6760) and (204.8130, 649.6220) .. (204.9570, 649.4060) -- cycle;
		
			% path1521
			\path[fill = green!50, even odd rule, line width = 0.480pt] (208.0050, 579.7184) -- (208.4790, 579.7184) -- (208.4790, 581.2022) -- (208.5690, 581.4740) -- (208.7070, 581.7134) -- (208.8390, 581.8898) -- (208.9410, 582.0260) -- (209.0730, 582.1526) -- (209.2350, 582.2564) -- (209.4210, 582.3650) -- (209.5830, 582.4376) -- (209.8110, 582.5192) -- (210.0330, 582.5690) -- (210.2910, 582.6140) -- (210.5670, 582.6278) -- (210.8070, 582.6140) -- (210.9930, 582.5960) -- (211.1550, 582.5552) -- (211.3350, 582.5054) -- (211.5090, 582.4466) -- (211.5690, 582.4238) -- (211.6290, 582.3926) -- (211.6710, 582.3698) -- (211.7250, 582.3428) -- (211.7370, 582.3378) -- (211.7730, 582.3108) -- (211.8150, 582.2838) -- (211.8570, 582.2472) -- (211.8990, 582.2154) -- (212.0370, 582.1068) -- (212.1210, 582.0258) -- (212.2350, 581.9172) -- (212.3250, 581.7810) -- (212.4030, 581.6634) -- (212.4870, 581.5278) -- (212.5590, 581.3784) -- (212.6250, 581.2020) -- (212.6730, 581.0574) -- (212.6790, 579.7182) -- (213.2370, 579.7182) -- (213.2370, 577.6908) -- (213.1890, 577.4196) -- (213.0990, 577.2024) -- (213.0090, 577.0572) -- (212.9310, 576.9126) -- (212.8290, 576.7680) -- (212.6970, 576.6234) -- (212.5590, 576.4920) -- (212.3790, 576.3876) -- (212.1810, 576.2976) -- (212.0010, 576.2064) -- (211.7670, 576.1344) -- (211.5810, 576.0978) -- (211.3050, 576.0576) -- (210.0210, 576.0616) -- (209.7270, 576.0976) -- (209.4630, 576.1612) -- (209.2110, 576.2338) -- (208.9530, 576.3334) -- (208.7730, 576.4462) -- (208.6530, 576.5278) -- (208.5390, 576.6496) -- (208.4370, 576.7450) -- (208.3410, 576.8446) -- (208.2390, 576.9664) -- (208.1610, 577.0936) -- (208.0590, 577.2742) -- (208.0230, 577.4644) -- (208.0050, 577.6366) -- cycle(209.4630, 579.7224) -- (209.4630, 581.0484) -- (209.4870, 581.1432) -- (209.5170, 581.2248) -- (209.5650, 581.3064) -- (209.6070, 581.3922) -- (209.6550, 581.4552) -- (209.7570, 581.5638) -- (209.8890, 581.6364) -- (210.0270, 581.7180) -- (210.1770, 581.7720) -- (210.3330, 581.8128) -- (210.5370, 581.8356) -- (210.6870, 581.8266) -- (210.8970, 581.7858) -- (211.0290, 581.7540) -- (211.1490, 581.7132) -- (211.2150, 581.6634) -- (211.3170, 581.6094) -- (211.3590, 581.5686) -- (211.4010, 581.5278) -- (211.4610, 581.4828) -- (211.4910, 581.4330) -- (211.5390, 581.3874) -- (211.5870, 581.3244) -- (211.6110, 581.2566) -- (211.6410, 581.1840) -- (211.6710, 581.1072) -- (211.6890, 581.0484) -- (211.6890, 579.7224) -- cycle;
		
			% path1523
			\path[fill = green!50, even odd rule, line width = 0.480pt] (210.8250, 577.4420) -- (210.9630, 576.7724) -- (210.3330, 576.7724) -- (210.4830, 577.4378) -- (210.4170, 577.4828) -- (210.3630, 577.5596) -- (210.3510, 577.6550) -- (210.3690, 577.7270) -- (210.4170, 577.7768) -- (210.4530, 577.8176) -- (210.5070, 577.8446) -- (210.5730, 577.8722) -- (210.6570, 577.8762) -- (210.7290, 577.8630) -- (210.8070, 577.8444) -- (210.8730, 577.7946) -- (210.9090, 577.7544) -- (210.9450, 577.7088) -- (210.9510, 577.6548) -- (210.9270, 577.5594) -- (210.8970, 577.5186) -- (210.8610, 577.4826) -- (210.8250, 577.4418);
		
			% path1525
			\path[draw = black, line join = miter, line cap = butt, miter limit = 10.00, line width = 0.461pt] (208.0050, 579.7184) -- (208.4790, 579.7184) -- (208.4790, 581.2022) -- (208.5690, 581.4740) -- (208.7070, 581.7134) -- (208.8390, 581.8898) -- (208.9410, 582.0260) -- (209.0730, 582.1526) -- (209.2350, 582.2564) -- (209.4210, 582.3650) -- (209.5830, 582.4376) -- (209.8110, 582.5192) -- (210.0330, 582.5690) -- (210.2910, 582.6140) -- (210.5670, 582.6278) -- (210.8070, 582.6140) -- (210.9930, 582.5960) -- (211.1550, 582.5552) -- (211.3350, 582.5054) -- (211.5090, 582.4466) -- (211.5690, 582.4238) -- (211.6290, 582.3926) -- (211.6710, 582.3698) -- (211.7250, 582.3428) -- (211.7370, 582.3378) -- (211.7730, 582.3108) -- (211.8150, 582.2838) -- (211.8570, 582.2472) -- (211.8990, 582.2154) -- (212.0370, 582.1068) -- (212.1210, 582.0258) -- (212.2350, 581.9172) -- (212.3250, 581.7810) -- (212.4030, 581.6634) -- (212.4870, 581.5278) -- (212.5590, 581.3784) -- (212.6250, 581.2020) -- (212.6730, 581.0574) -- (212.6790, 579.7182) -- (213.2370, 579.7182) -- (213.2370, 577.6908) -- (213.1890, 577.4196) -- (213.0990, 577.2024) -- (213.0090, 577.0572) -- (212.9310, 576.9126) -- (212.8290, 576.7680) -- (212.6970, 576.6234) -- (212.5590, 576.4920) -- (212.3790, 576.3876) -- (212.1810, 576.2976) -- (212.0010, 576.2064) -- (211.7670, 576.1344) -- (211.5810, 576.0978) -- (211.3050, 576.0576) -- (210.0210, 576.0616) -- (209.7270, 576.0976) -- (209.4630, 576.1612) -- (209.2110, 576.2338) -- (208.9530, 576.3334) -- (208.7730, 576.4462) -- (208.6530, 576.5278) -- (208.5390, 576.6496) -- (208.4370, 576.7450) -- (208.3410, 576.8446) -- (208.2390, 576.9664) -- (208.1610, 577.0936) -- (208.0590, 577.2742) -- (208.0230, 577.4644) -- (208.0050, 577.6366) -- cycle(209.4630, 579.7224) -- (209.4630, 581.0484) -- (209.4870, 581.1432) -- (209.5170, 581.2248) -- (209.5650, 581.3064) -- (209.6070, 581.3922) -- (209.6550, 581.4552) -- (209.7570, 581.5638) -- (209.8890, 581.6364) -- (210.0270, 581.7180) -- (210.1770, 581.7720) -- (210.3330, 581.8128) -- (210.5370, 581.8356) -- (210.6870, 581.8266) -- (210.8970, 581.7858) -- (211.0290, 581.7540) -- (211.1490, 581.7132) -- (211.2150, 581.6634) -- (211.3170, 581.6094) -- (211.3590, 581.5686) -- (211.4010, 581.5278) -- (211.4610, 581.4828) -- (211.4910, 581.4330) -- (211.5390, 581.3874) -- (211.5870, 581.3244) -- (211.6110, 581.2566) -- (211.6410, 581.1840) -- (211.6710, 581.1072) -- (211.6890, 581.0484) -- (211.6890, 579.7224) -- cycle;
		
			% path1527
			\path[draw = black, line join = miter, line cap = butt, miter limit = 10.00, line width = 0.461pt] (208.0050, 579.7184) -- (213.2370, 579.7184) -- cycle;
		
			% path1529
			\path[draw = black, line join = miter, line cap = butt, miter limit = 10.00, line width = 0.461pt] (210.8250, 577.4420) -- (210.9630, 576.7724) -- (210.3330, 576.7724) -- (210.4830, 577.4378) -- (210.4170, 577.4828) -- (210.3630, 577.5596) -- (210.3510, 577.6550) -- (210.3690, 577.7270) -- (210.4170, 577.7768) -- (210.4530, 577.8176) -- (210.5070, 577.8446) -- (210.5730, 577.8722) -- (210.6570, 577.8762) -- (210.7290, 577.8630) -- (210.8070, 577.8444) -- (210.8730, 577.7946) -- (210.9090, 577.7544) -- (210.9450, 577.7088) -- (210.9510, 577.6548) -- (210.9270, 577.5594) -- (210.8970, 577.5186) -- (210.8610, 577.4826) -- (210.8250, 577.4418);
		
			% path1531
			\path[draw = green, line join = miter, line cap = butt, miter limit = 10.00, line width = 1.094pt] (208.5210, 642.8600) .. controls (209.6610, 642.3800) and (210.7950, 641.9000) .. (210.8490, 641.0780) .. controls (210.8970, 640.2560) and (208.8570, 638.6780) .. (208.8270, 637.9220) .. controls (208.7970, 637.1720) and (210.6330, 637.4000) .. (210.6630, 636.5540) .. controls (210.6930, 635.7080) and (209.3250, 633.4940) .. (209.0130, 632.8520);
		
			% path1593
			\path[fill = gray!50, even odd rule, line width = 0.480pt] (183.5730, 620.4320) -- (190.8450, 620.4320) -- (190.8450, 627.7040) -- (183.5730, 627.7040) -- cycle;
		
			% path1595
			\path[draw = black, line join = miter, line cap = butt, miter limit = 10.00, line width = 0.461pt] (183.5730, 620.4320) -- (190.8450, 620.4320) -- (190.8450, 627.7040) -- (183.5730, 627.7040) -- cycle;
		
			\begin{scope}[cm = {{0.6, 0.0, 0.0, 0.6, (85.149, 560.6)}}]% g1603
				\begin{scope}% g1601
					% text1599
					\path[cm = {{1.0, 0.0, 0.0, -1.0, (166.37, 101.04)}}, fill = black, nonzero rule] (0.0000, 0.0000) node[above right, font = \scriptsize, inner sep = 0em] (text1599) {C};
				\end{scope}
			\end{scope}
			% path1605
			\path[fill = gray!50, even odd rule, line width = 0.480pt] (203.3010, 620.5040) -- (210.5730, 620.5040) -- (210.5730, 627.7760) -- (203.3010, 627.7760) -- cycle;
		
			% path1607
			\path[draw = black, line join = miter, line cap = butt, miter limit = 10.00, line width = 0.461pt] (203.3010, 620.5040) -- (210.5730, 620.5040) -- (210.5730, 627.7760) -- (203.3010, 627.7760) -- cycle;
		
			\begin{scope}[cm = {{0.6, 0.0, 0.0, 0.6, (85.149, 560.6)}}]% g1615
				\begin{scope}% g1613
					% text1611
					\path[cm = {{1.0, 0.0, 0.0, -1.0, (199.25, 101.18)}}, fill = black, nonzero rule] (0.0000, 0.0000) node[above right, font = \scriptsize, inner sep = 0em] (text1611) {C};
				\end{scope}
			\end{scope}
			% path1617
			\path[fill = gray!50, even odd rule, line width = 0.480pt] (221.8770, 620.2880) -- (229.1490, 620.2880) -- (229.1490, 627.5600) -- (221.8770, 627.5600) -- cycle;
		
			% path1619
			\path[draw = black, line join = miter, line cap = butt, miter limit = 10.00, line width = 0.461pt] (221.8770, 620.2880) -- (229.1490, 620.2880) -- (229.1490, 627.5600) -- (221.8770, 627.5600) -- cycle;
		
			\begin{scope}[cm = {{0.6, 0.0, 0.0, 0.6, (85.149, 560.6)}}]% g1627
				\begin{scope}% g1625
					% text1623
					\path[cm = {{1.0, 0.0, 0.0, -1.0, (230.26, 100.78)}}, fill = black, nonzero rule] (0.0000, 0.0000) node[above right, font = \scriptsize, inner sep = 0em] (text1623) {C};
				\end{scope}
			\end{scope}
			% path1629
			\path[fill = gray!50, even odd rule, line width = 0.480pt] (242.3250, 620.2160) -- (249.5970, 620.2160) -- (249.5970, 627.4160) -- (242.3250, 627.4160) -- cycle;
		
			% path1631
			\path[draw = black, line join = miter, line cap = butt, miter limit = 10.00, line width = 0.461pt] (242.3250, 620.2160) -- (249.5970, 620.2160) -- (249.5970, 627.4160) -- (242.3250, 627.4160) -- cycle;
		
			\begin{scope}[cm = {{0.6, 0.0, 0.0, 0.6, (85.149, 560.6)}}]% g1639
				\begin{scope}% g1637
					% text1635
					\path[cm = {{1.0, 0.0, 0.0, -1.0, (264.29, 100.58)}}, fill = black, nonzero rule] (0.0000, 0.0000) node[above right, font = \scriptsize, inner sep = 0em] (text1635) {C};
				\end{scope}
			\end{scope}
			% path1641
			\path[fill = gray!50, even odd rule, line width = 0.480pt] (178.8210, 595.5920) -- (216.4830, 618.3440) -- (255.3570, 595.5920) -- cycle;
		
			% path1643
			\path[draw = black, line join = miter, line cap = butt, miter limit = 10.00, line width = 0.461pt] (178.8210, 595.5920) -- (216.4830, 618.3440) -- (255.3570, 595.5920) -- cycle;
		
			\begin{scope}[cm = {{0.6, 0.0, 0.0, 0.6, (85.149, 560.6)}}]% g1647
				% path1645
				\path[draw = black, dash pattern = on 2.30pt off 0.77pt, line join = miter, line cap = butt, miter limit = 10.00, line width = 0.768pt] (154.9200, 90.8600) .. controls (154.9200, 93.8310) and (157.3300, 96.2400) .. (160.3000, 96.2400) -- (181.8200, 96.2400) .. controls (184.7900, 96.2400) and (187.2000, 93.8310) .. 	(187.2000, 90.8600) -- (187.2000, 29.7400) .. controls (187.2000, 26.7690) and (184.7900, 24.3600) .. (181.8200, 24.3600) -- (160.3000, 24.3600) .. controls (157.3300, 24.3600) and (154.9200, 26.7690) .. (154.9200, 29.7400) -- cycle;
			\end{scope}
			\begin{scope}[cm = {{0.6, 0.0, 0.0, 0.6, (85.149, 560.6)}}]% g1651
				% path1649
				\path[draw = black, dash pattern = on 2.30pt off 0.77pt, line join = miter, line cap = butt, miter limit = 10.00, line width = 0.768pt] (187.2000, 90.8200) .. controls (187.2000, 93.7470) and (189.5700, 96.1200) .. (192.5000, 96.1200) -- (213.7000, 96.1200) .. controls (216.6300, 96.1200) and (219.0000, 93.7470) .. 	(219.0000, 90.8200) -- (219.0000, 29.6600) .. controls (219.0000, 26.7330) and (216.6300, 24.3600) .. (213.7000, 24.3600) -- (192.5000, 24.3600) .. controls (189.5700, 24.3600) and (187.2000, 26.7330) .. (187.2000, 29.6600) -- cycle;
			\end{scope}
			\begin{scope}[cm = {{0.6, 0.0, 0.0, 0.6, (85.149, 560.6)}}]% g1655
				% path1653
				\path[draw = black, dash pattern = on 2.30pt off 0.77pt, line join = miter, line cap = butt, miter limit = 10.00, line width = 0.768pt] (218.7600, 90.6600) .. controls (218.7600, 93.6750) and (221.2000, 96.1200) .. (224.2200, 96.1200) -- (246.0600, 96.1200) .. controls (249.0800, 96.1200) and (251.5200, 93.6750) .. 	(251.5200, 90.6600) -- (251.5200, 29.7000) .. controls (251.5200, 26.6850) and (249.0800, 24.2400) .. (246.0600, 24.2400) -- (224.2200, 24.2400) .. controls (221.2000, 24.2400) and (218.7600, 26.6850) .. (218.7600, 29.7000) -- cycle;
			\end{scope}
			\begin{scope}[cm = {{0.6, 0.0, 0.0, 0.6, (85.149, 560.6)}}]% g1659
				% path1657
				\path[draw = black, dash pattern = on 2.30pt off 0.77pt, line join = miter, line cap = butt, miter limit = 10.00, line width = 0.768pt] (251.6400, 90.5400) .. controls (251.6400, 93.5550) and (254.0800, 96.0000) .. (257.1000, 96.0000) -- (278.9400, 96.0000) .. controls (281.9600, 96.0000) and (284.4000, 93.5550) .. 	(284.4000, 90.5400) -- (284.4000, 29.7000) .. controls (284.4000, 26.6850) and (281.9600, 24.2400) .. (278.9400, 24.2400) -- (257.1000, 24.2400) .. controls (254.0800, 24.2400) and (251.6400, 26.6850) .. (251.6400, 29.7000) -- cycle;
			\end{scope}
		\end{scope}
	\end{tikzpicture}%
}


\newcommand{\pseDiagram}[1][1.0]{%
	\begin{tikzpicture}[y = 0.80pt,
	                             x = 0.80pt,
	                             yscale = -1.000000,
	                             xscale = 1.000000,
	                             inner sep = 0pt,
	                             outer sep = 0pt,
	                             opacity = #1,
	                             baseline = (current bounding box.center)]
		\begin{scope}[cm = {{1.33333, 0.0, 0.0, -1.33333, (0.0, 1056.0)}}]% g823
			% path1073
			\path[fill = gray!50, even odd rule, line width = 0.480pt] (180.1170, 677.3840) -- (182.9250, 677.3840) -- (182.9250, 685.3040) -- (180.1170, 685.3040) -- cycle;
		
			% path1075
			\path[draw = black, line join = miter, line cap = butt, miter limit = 10.00, line width = 0.461pt] (180.1170, 677.3840) -- (182.9250, 677.3840) -- (182.9250, 685.3040) -- (180.1170, 685.3040) -- cycle;
		
			% path1077
			\path[fill = gray!50, even odd rule, line width = 0.480pt] (183.8610, 677.3840) -- (186.6690, 677.3840) -- (186.6690, 685.3040) -- (183.8610, 685.3040) -- cycle;
		
			% path1079
			\path[draw = black, line join = miter, line cap = butt, miter limit = 10.00, line width = 0.461pt] (183.8610, 677.3840) -- (186.6690, 677.3840) -- (186.6690, 685.3040) -- (183.8610, 685.3040) -- cycle;
		
			% path1081
			\path[fill = gray!50, even odd rule, line width = 0.480pt] (187.6770, 677.3840) -- (190.4130, 677.3840) -- (190.4130, 685.3040) -- (187.6770, 685.3040) -- cycle;
		
			% path1083
			\path[draw = black, line join = miter, line cap = butt, miter limit = 10.00, line width = 0.461pt] (187.6770, 677.3840) -- (190.4130, 677.3840) -- (190.4130, 685.3040) -- (187.6770, 685.3040) -- cycle;
		
			% path1085
			\path[fill = gray!50, even odd rule, line width = 0.480pt] (191.4210, 677.3840) -- (194.2290, 677.3840) -- (194.2290, 685.3040) -- (191.4210, 685.3040) -- cycle;
		
			% path1087
			\path[draw = black, line join = miter, line cap = butt, miter limit = 10.00, line width = 0.461pt] (191.4210, 677.3840) -- (194.2290, 677.3840) -- (194.2290, 685.3040) -- (191.4210, 685.3040) -- cycle;
		
			% path1089
			\path[fill = gray!50, even odd rule, line width = 0.480pt] (195.1650, 677.3840) -- (197.9730, 677.3840) -- (197.9730, 685.3040) -- (195.1650, 685.3040) -- cycle;
		
			% path1091
			\path[draw = black, line join = miter, line cap = butt, miter limit = 10.00, line width = 0.461pt] (195.1650, 677.3840) -- (197.9730, 677.3840) -- (197.9730, 685.3040) -- (195.1650, 685.3040) -- cycle;
		
			% path1093
			\path[fill = gray!50, even odd rule, line width = 0.480pt] (180.1170, 689.9120) -- (189.0450, 703.1600) -- (197.9730, 689.9120) -- cycle;
		
			% path1095
			\path[draw = black, line join = miter, line cap = butt, miter limit = 10.00, line width = 0.461pt] (180.1170, 689.9120) -- (189.0450, 703.1600) -- (197.9730, 689.9120) -- cycle;
		
			% path1097
			\path[draw = black, line join = miter, line cap = butt, miter limit = 10.00, line width = 0.749pt] (182.0970, 689.8760) -- (185.3430, 685.2620);
		
			% path1099
			\path[draw = black, line join = miter, line cap = butt, miter limit = 10.00, line width = 0.749pt] (184.7610, 689.8760) -- (192.8430, 685.2620);
		
			% path1101
			\path[draw = black, line join = miter, line cap = butt, miter limit = 10.00, line width = 0.749pt] (189.1230, 689.8760) -- (181.5930, 685.2620);
		
			% path1103
			\path[draw = black, line join = miter, line cap = butt, miter limit = 10.00, line width = 0.749pt] (191.6730, 689.8760) -- (196.6350, 685.2620);
		
			% path1105
			\path[draw = black, line join = miter, line cap = butt, miter limit = 10.00, line width = 0.749pt] (195.2130, 689.8760) -- (189.0810, 685.3040);
		
			% path1107
			\path[fill = gray!50, even odd rule, line width = 0.480pt] (199.1250, 677.3840) -- (201.9330, 677.3840) -- (201.9330, 685.3040) -- (199.1250, 685.3040) -- cycle;
		
			% path1109
			\path[draw = black, line join = miter, line cap = butt, miter limit = 10.00, line width = 0.461pt] (199.1250, 677.3840) -- (201.9330, 677.3840) -- (201.9330, 685.3040) -- (199.1250, 685.3040) -- cycle;
		
			% path1111
			\path[fill = gray!50, even odd rule, line width = 0.480pt] (202.9410, 677.3840) -- (205.6770, 677.3840) -- (205.6770, 685.3040) -- (202.9410, 685.3040) -- cycle;
		
			% path1113
			\path[draw = black, line join = miter, line cap = butt, miter limit = 10.00, line width = 0.461pt] (202.9410, 677.3840) -- (205.6770, 677.3840) -- (205.6770, 685.3040) -- (202.9410, 685.3040) -- cycle;
		
			% path1115
			\path[fill = gray!50, even odd rule, line width = 0.480pt] (206.6850, 677.3840) -- (209.4930, 677.3840) -- (209.4930, 685.3040) -- (206.6850, 685.3040) -- cycle;
		
			% path1117
			\path[draw = black, line join = miter, line cap = butt, miter limit = 10.00, line width = 0.461pt] (206.6850, 677.3840) -- (209.4930, 677.3840) -- (209.4930, 685.3040) -- (206.6850, 685.3040) -- cycle;
		
			% path1119
			\path[fill = gray!50, even odd rule, line width = 0.480pt] (210.4290, 677.3840) -- (213.2370, 677.3840) -- (213.2370, 685.3040) -- (210.4290, 685.3040) -- cycle;
		
			% path1121
			\path[draw = black, line join = miter, line cap = butt, miter limit = 10.00, line width = 0.461pt] (210.4290, 677.3840) -- (213.2370, 677.3840) -- (213.2370, 685.3040) -- (210.4290, 685.3040) -- cycle;
		
			% path1123
			\path[fill = gray!50, even odd rule, line width = 0.480pt] (214.2450, 677.3840) -- (216.9810, 677.3840) -- (216.9810, 685.3040) -- (214.2450, 685.3040) -- cycle;
		
			% path1125
			\path[draw = black, line join = miter, line cap = butt, miter limit = 10.00, line width = 0.461pt] (214.2450, 677.3840) -- (216.9810, 677.3840) -- (216.9810, 685.3040) -- (214.2450, 685.3040) -- cycle;
		
			% path1127
			\path[fill = gray!50, even odd rule, line width = 0.480pt] (199.1250, 689.9120) -- (208.0530, 703.1600) -- (216.9810, 689.9120) -- cycle;
		
			% path1129
			\path[draw = black, line join = miter, line cap = butt, miter limit = 10.00, line width = 0.461pt] (199.1250, 689.9120) -- (208.0530, 703.1600) -- (216.9810, 689.9120) -- cycle;
		
			% path1131
			\path[draw = black, line join = miter, line cap = butt, miter limit = 10.00, line width = 0.749pt] (201.1050, 689.8760) -- (204.3510, 685.2620);
		
			% path1133
			\path[draw = black, line join = miter, line cap = butt, miter limit = 10.00, line width = 0.749pt] (203.7690, 689.8760) -- (211.8510, 685.2620);
		
			% path1135
			\path[draw = black, line join = miter, line cap = butt, miter limit = 10.00, line width = 0.749pt] (208.1310, 689.8760) -- (200.6010, 685.2620);
		
			% path1137
			\path[draw = black, line join = miter, line cap = butt, miter limit = 10.00, line width = 0.749pt] (210.6810, 689.8760) -- (215.6430, 685.2620);
		
			% path1139
			\path[draw = black, line join = miter, line cap = butt, miter limit = 10.00, line width = 0.749pt] (214.2210, 689.8760) -- (208.0890, 685.3040);
		
			% path1141
			\path[fill = gray!50, even odd rule, line width = 0.480pt] (218.7810, 677.3840) -- (221.5890, 677.3840) -- (221.5890, 685.3040) -- (218.7810, 685.3040) -- cycle;
		
			% path1143
			\path[draw = black, line join = miter, line cap = butt, miter limit = 10.00, line width = 0.461pt] (218.7810, 677.3840) -- (221.5890, 677.3840) -- (221.5890, 685.3040) -- (218.7810, 685.3040) -- cycle;
		
			% path1145
			\path[fill = gray!50, even odd rule, line width = 0.480pt] (222.5970, 677.3840) -- (225.3330, 677.3840) -- (225.3330, 685.3040) -- (222.5970, 685.3040) -- cycle;
		
			% path1147
			\path[draw = black, line join = miter, line cap = butt, miter limit = 10.00, line width = 0.461pt] (222.5970, 677.3840) -- (225.3330, 677.3840) -- (225.3330, 685.3040) -- (222.5970, 685.3040) -- cycle;
		
			% path1149
			\path[fill = gray!50, even odd rule, line width = 0.480pt] (226.3410, 677.3840) -- (229.1490, 677.3840) -- (229.1490, 685.3040) -- (226.3410, 685.3040) -- cycle;
		
			% path1151
			\path[draw = black, line join = miter, line cap = butt, miter limit = 10.00, line width = 0.461pt] (226.3410, 677.3840) -- (229.1490, 677.3840) -- (229.1490, 685.3040) -- (226.3410, 685.3040) -- cycle;
		
			% path1153
			\path[fill = gray!50, even odd rule, line width = 0.480pt] (230.0850, 677.3840) -- (232.8930, 677.3840) -- (232.8930, 685.3040) -- (230.0850, 685.3040) -- cycle;
		
			% path1155
			\path[draw = black, line join = miter, line cap = butt, miter limit = 10.00, line width = 0.461pt] (230.0850, 677.3840) -- (232.8930, 677.3840) -- (232.8930, 685.3040) -- (230.0850, 685.3040) -- cycle;
		
			% path1157
			\path[fill = gray!50, even odd rule, line width = 0.480pt] (233.8290, 677.3840) -- (236.6370, 677.3840) -- (236.6370, 685.3040) -- (233.8290, 685.3040) -- cycle;
		
			% path1159
			\path[draw = black, line join = miter, line cap = butt, miter limit = 10.00, line width = 0.461pt] (233.8290, 677.3840) -- (236.6370, 677.3840) -- (236.6370, 685.3040) -- (233.8290, 685.3040) -- cycle;
		
			% path1161
			\path[fill = gray!50, even odd rule, line width = 0.480pt] (218.7810, 689.9120) -- (227.7090, 703.1600) -- (236.6370, 689.9120) -- cycle;
		
			% path1163
			\path[draw = black, line join = miter, line cap = butt, miter limit = 10.00, line width = 0.461pt] (218.7810, 689.9120) -- (227.7090, 703.1600) -- (236.6370, 689.9120) -- cycle;
		
			% path1165
			\path[draw = black, line join = miter, line cap = butt, miter limit = 10.00, line width = 0.749pt] (220.7610, 689.8760) -- (224.0070, 685.2620);
		
			% path1167
			\path[draw = black, line join = miter, line cap = butt, miter limit = 10.00, line width = 0.749pt] (223.4250, 689.8760) -- (231.5070, 685.2620);
		
			% path1169
			\path[draw = black, line join = miter, line cap = butt, miter limit = 10.00, line width = 0.749pt] (227.7870, 689.8760) -- (220.2570, 685.2620);
		
			% path1171
			\path[draw = black, line join = miter, line cap = butt, miter limit = 10.00, line width = 0.749pt] (230.3370, 689.8760) -- (235.2990, 685.2620);
		
			% path1173
			\path[draw = black, line join = miter, line cap = butt, miter limit = 10.00, line width = 0.749pt] (233.8770, 689.8760) -- (227.7450, 685.3040);
		
			% path1175
			\path[fill = gray!50, even odd rule, line width = 0.480pt] (238.4370, 677.3840) -- (241.1730, 677.3840) -- (241.1730, 685.3040) -- (238.4370, 685.3040) -- cycle;
		
			% path1177
			\path[draw = black, line join = miter, line cap = butt, miter limit = 10.00, line width = 0.461pt] (238.4370, 677.3840) -- (241.1730, 677.3840) -- (241.1730, 685.3040) -- (238.4370, 685.3040) -- cycle;
		
			% path1179
			\path[fill = gray!50, even odd rule, line width = 0.480pt] (242.1810, 677.3840) -- (244.9890, 677.3840) -- (244.9890, 685.3040) -- (242.1810, 685.3040) -- cycle;
		
			% path1181
			\path[draw = black, line join = miter, line cap = butt, miter limit = 10.00, line width = 0.461pt] (242.1810, 677.3840) -- (244.9890, 677.3840) -- (244.9890, 685.3040) -- (242.1810, 685.3040) -- cycle;
		
			% path1183
			\path[fill = gray!50, even odd rule, line width = 0.480pt] (245.9250, 677.3840) -- (248.7330, 677.3840) -- (248.7330, 685.3040) -- (245.9250, 685.3040) -- cycle;
		
			% path1185
			\path[draw = black, line join = miter, line cap = butt, miter limit = 10.00, line width = 0.461pt] (245.9250, 677.3840) -- (248.7330, 677.3840) -- (248.7330, 685.3040) -- (245.9250, 685.3040) -- cycle;
		
			% path1187
			\path[fill = gray!50, even odd rule, line width = 0.480pt] (249.7410, 677.3840) -- (252.4770, 677.3840) -- (252.4770, 685.3040) -- (249.7410, 685.3040) -- cycle;
		
			% path1189
			\path[draw = black, line join = miter, line cap = butt, miter limit = 10.00, line width = 0.461pt] (249.7410, 677.3840) -- (252.4770, 677.3840) -- (252.4770, 685.3040) -- (249.7410, 685.3040) -- cycle;
		
			% path1191
			\path[fill = gray!50, even odd rule, line width = 0.480pt] (253.4850, 677.3840) -- (256.2930, 677.3840) -- (256.2930, 685.3040) -- (253.4850, 685.3040) -- cycle;
		
			% path1193
			\path[draw = black, line join = miter, line cap = butt, miter limit = 10.00, line width = 0.461pt] (253.4850, 677.3840) -- (256.2930, 677.3840) -- (256.2930, 685.3040) -- (253.4850, 685.3040) -- cycle;
		
			% path1195
			\path[fill = gray!50, even odd rule, line width = 0.480pt] (238.4370, 689.9120) -- (247.3650, 703.1600) -- (256.2930, 689.9120) -- cycle;
		
			% path1197
			\path[draw = black, line join = miter, line cap = butt, miter limit = 10.00, line width = 0.461pt] (238.4370, 689.9120) -- (247.3650, 703.1600) -- (256.2930, 689.9120) -- cycle;
		
			% path1199
			\path[draw = black, line join = miter, line cap = butt, miter limit = 10.00, line width = 0.749pt] (240.3450, 689.8760) -- (243.5910, 685.2620);
		
			% path1201
			\path[draw = black, line join = miter, line cap = butt, miter limit = 10.00, line width = 0.749pt] (243.0810, 689.8760) -- (251.1630, 685.2620);
		
			% path1203
			\path[draw = black, line join = miter, line cap = butt, miter limit = 10.00, line width = 0.749pt] (247.3710, 689.8760) -- (239.8410, 685.2620);
		
			% path1205
			\path[draw = black, line join = miter, line cap = butt, miter limit = 10.00, line width = 0.749pt] (249.9210, 689.8760) -- (254.8830, 685.2620);
		
			% path1207
			\path[draw = black, line join = miter, line cap = butt, miter limit = 10.00, line width = 0.749pt] (253.5330, 689.8760) -- (247.4010, 685.3040);
		
			% path1543
			\path[fill = orange!50, even odd rule, line width = 0.480pt] (244.0770, 707.8760) -- (244.5570, 707.8760) -- (244.5570, 709.3640) -- (244.6470, 709.6340) -- (244.7850, 709.8740) -- (244.9230, 710.0480) -- (245.0310, 710.1860) -- (245.1630, 710.3120) -- (245.3190, 710.4140) -- (245.5110, 710.5280) -- (245.6790, 710.6000) -- (245.9070, 710.6780) -- (246.1290, 710.7260) -- (246.3930, 710.7740) -- (246.6750, 710.7860) -- (246.9150, 710.7740) -- (247.1070, 710.7560) -- (247.2690, 710.7140) -- (247.4490, 710.6660) -- (247.6290, 710.6060) -- (247.6890, 710.5820) -- (247.7490, 710.5520) -- (247.7970, 710.5280) -- (247.8450, 710.5040) -- (247.8570, 710.4980) -- (247.8990, 710.4680) -- (247.9410, 710.4440) -- (247.9770, 710.4080) -- (248.0250, 710.3780) -- (248.1630, 710.2700) -- (248.2470, 710.1860) -- (248.3670, 710.0780) -- (248.4570, 709.9400) -- (248.5350, 709.8260) -- (248.6250, 709.6880) -- (248.6910, 709.5380) -- (248.7630, 709.3640) -- (248.8050, 709.2200) -- (248.8110, 707.8760) -- (249.3810, 707.8760) -- (249.3810, 705.8540) -- (249.3330, 705.5780) -- (249.2430, 705.3620) -- (249.1530, 705.2180) -- (249.0690, 705.0740) -- (248.9670, 704.9300) -- (248.8350, 704.7860) -- (248.6910, 704.6540) -- (248.5110, 704.5460) -- (248.3130, 704.4560) -- (248.1270, 704.3660) -- (247.8870, 704.2940) -- (247.7070, 704.2580) -- (247.4250, 704.2160) -- (246.1170, 704.2220) -- (245.8230, 704.2580) -- (245.5590, 704.3240) -- (245.3010, 704.3960) -- (245.0430, 704.4920) -- (244.8570, 704.6060) -- (244.7370, 704.6900) -- (244.6170, 704.8100) -- (244.5150, 704.9060) -- (244.4190, 705.0020) -- (244.3170, 705.1280) -- (244.2390, 705.2540) -- (244.1310, 705.4340) -- (244.0950, 705.6260) -- (244.0770, 705.7940) -- cycle(245.5590, 707.8820) -- (245.5590, 709.2080) -- (245.5830, 709.3040) -- (245.6070, 709.3820) -- (245.6610, 709.4660) -- (245.7030, 709.5500) -- (245.7510, 709.6160) -- (245.8530, 709.7240) -- (245.9850, 709.7960) -- (246.1230, 709.8800) -- (246.2790, 709.9340) -- (246.4350, 709.9760) -- (246.6450, 709.9940) -- (246.7950, 709.9880) -- (247.0110, 709.9460) -- (247.1430, 709.9160) -- (247.2630, 709.8740) -- (247.3290, 709.8260) -- (247.4370, 709.7720) -- (247.4730, 709.7300) -- (247.5210, 709.6880) -- (247.5810, 709.6400) -- (247.6110, 709.5920) -- (247.6590, 709.5500) -- (247.7130, 709.4840) -- (247.7310, 709.4180) -- (247.7610, 709.3460) -- (247.7970, 709.2680) -- (247.8150, 709.2080) -- (247.8150, 707.8820) -- cycle;
		
			% path1545
			\path[fill = orange!50, even odd rule, line width = 0.480pt] (246.9330, 705.6020) -- (247.0770, 704.9300) -- (246.4350, 704.9300) -- (246.5910, 705.5960) -- (246.5190, 705.6440) -- (246.4710, 705.7220) -- (246.4590, 705.8120) -- (246.4770, 705.8900) -- (246.5190, 705.9380) -- (246.5550, 705.9800) -- (246.6150, 706.0040) -- (246.6810, 706.0340) -- (246.7710, 706.0340) -- (246.8370, 706.0220) -- (246.9150, 706.0040) -- (246.9870, 705.9560) -- (247.0230, 705.9140) -- (247.0530, 705.8720) -- (247.0590, 705.8120) -- (247.0410, 705.7220) -- (247.0110, 705.6800) -- (246.9750, 705.6440) -- (246.9330, 705.6020);
		
			% path1547
			\path[draw = black, line join = miter, line cap = butt, miter limit = 10.00, line width = 0.461pt] (244.0770, 707.8760) -- (244.5570, 707.8760) -- (244.5570, 709.3640) -- (244.6470, 709.6340) -- (244.7850, 709.8740) -- (244.9230, 710.0480) -- (245.0310, 710.1860) -- (245.1630, 710.3120) -- (245.3190, 710.4140) -- (245.5110, 710.5280) -- (245.6790, 710.6000) -- (245.9070, 710.6780) -- (246.1290, 710.7260) -- (246.3930, 710.7740) -- (246.6750, 710.7860) -- (246.9150, 710.7740) -- (247.1070, 710.7560) -- (247.2690, 710.7140) -- (247.4490, 710.6660) -- (247.6290, 710.6060) -- (247.6890, 710.5820) -- (247.7490, 710.5520) -- (247.7970, 710.5280) -- (247.8450, 710.5040) -- (247.8570, 710.4980) -- (247.8990, 710.4680) -- (247.9410, 710.4440) -- (247.9770, 710.4080) -- (248.0250, 710.3780) -- (248.1630, 710.2700) -- (248.2470, 710.1860) -- (248.3670, 710.0780) -- (248.4570, 709.9400) -- (248.5350, 709.8260) -- (248.6250, 709.6880) -- (248.6910, 709.5380) -- (248.7630, 709.3640) -- (248.8050, 709.2200) -- (248.8110, 707.8760) -- (249.3810, 707.8760) -- (249.3810, 705.8540) -- (249.3330, 705.5780) -- (249.2430, 705.3620) -- (249.1530, 705.2180) -- (249.0690, 705.0740) -- (248.9670, 704.9300) -- (248.8350, 704.7860) -- (248.6910, 704.6540) -- (248.5110, 704.5460) -- (248.3130, 704.4560) -- (248.1270, 704.3660) -- (247.8870, 704.2940) -- (247.7070, 704.2580) -- (247.4250, 704.2160) -- (246.1170, 704.2220) -- (245.8230, 704.2580) -- (245.5590, 704.3240) -- (245.3010, 704.3960) -- (245.0430, 704.4920) -- (244.8570, 704.6060) -- (244.7370, 704.6900) -- (244.6170, 704.8100) -- (244.5150, 704.9060) -- (244.4190, 705.0020) -- (244.3170, 705.1280) -- (244.2390, 705.2540) -- (244.1310, 705.4340) -- (244.0950, 705.6260) -- (244.0770, 705.7940) -- cycle(245.5590, 707.8820) -- (245.5590, 709.2080) -- (245.5830, 709.3040) -- (245.6070, 709.3820) -- (245.6610, 709.4660) -- (245.7030, 709.5500) -- (245.7510, 709.6160) -- (245.8530, 709.7240) -- (245.9850, 709.7960) -- (246.1230, 709.8800) -- (246.2790, 709.9340) -- (246.4350, 709.9760) -- (246.6450, 709.9940) -- (246.7950, 709.9880) -- (247.0110, 709.9460) -- (247.1430, 709.9160) -- (247.2630, 709.8740) -- (247.3290, 709.8260) -- (247.4370, 709.7720) -- (247.4730, 709.7300) -- (247.5210, 709.6880) -- (247.5810, 709.6400) -- (247.6110, 709.5920) -- (247.6590, 709.5500) -- (247.7130, 709.4840) -- (247.7310, 709.4180) -- (247.7610, 709.3460) -- (247.7970, 709.2680) -- (247.8150, 709.2080) -- (247.8150, 707.8820) -- cycle;
		
			% path1549
			\path[draw = black, line join = miter, line cap = butt, miter limit = 10.00, line width = 0.461pt] (244.0770, 707.8760) -- (249.3810, 707.8760) -- cycle;
		
			% path1551
			\path[draw = black, line join = miter, line cap = butt, miter limit = 10.00, line width = 0.461pt] (246.9330, 705.6020) -- (247.0770, 704.9300) -- (246.4350, 704.9300) -- (246.5910, 705.5960) -- (246.5190, 705.6440) -- (246.4710, 705.7220) -- (246.4590, 705.8120) -- (246.4770, 705.8900) -- (246.5190, 705.9380) -- (246.5550, 705.9800) -- (246.6150, 706.0040) -- (246.6810, 706.0340) -- (246.7710, 706.0340) -- (246.8370, 706.0220) -- (246.9150, 706.0040) -- (246.9870, 705.9560) -- (247.0230, 705.9140) -- (247.0530, 705.8720) -- (247.0590, 705.8120) -- (247.0410, 705.7220) -- (247.0110, 705.6800) -- (246.9750, 705.6440) -- (246.9330, 705.6020);
		
			% path1553
			\path[fill = blue!50, even odd rule, line width = 0.480pt] (224.2770, 707.9180) -- (224.7570, 707.9180) -- (224.7570, 709.4180) -- (224.8470, 709.6940) -- (224.9850, 709.9340) -- (225.1230, 710.1140) -- (225.2310, 710.2520) -- (225.3630, 710.3780) -- (225.5190, 710.4860) -- (225.7110, 710.5940) -- (225.8790, 710.6660) -- (226.1070, 710.7500) -- (226.3290, 710.7980) -- (226.5930, 710.8460) -- (226.8750, 710.8580) -- (227.1150, 710.8460) -- (227.3070, 710.8280) -- (227.4690, 710.7860) -- (227.6490, 710.7380) -- (227.8290, 710.6780) -- (227.8890, 710.6540) -- (227.9490, 710.6240) -- (227.9970, 710.6000) -- (228.0450, 710.5700) -- (228.0570, 710.5640) -- (228.0990, 710.5400) -- (228.1410, 710.5100) -- (228.1770, 710.4740) -- (228.2250, 710.4440) -- (228.3630, 710.3360) -- (228.4470, 710.2520) -- (228.5670, 710.1440) -- (228.6570, 710.0060) -- (228.7350, 709.8860) -- (228.8250, 709.7480) -- (228.8910, 709.5980) -- (228.9630, 709.4180) -- (229.0050, 709.2740) -- (229.0110, 707.9180) -- (229.5810, 707.9180) -- (229.5810, 705.8660) -- (229.5330, 705.5960) -- (229.4430, 705.3740) -- (229.3530, 705.2300) -- (229.2690, 705.0800) -- (229.1670, 704.9360) -- (229.0350, 704.7920) -- (228.8910, 704.6540) -- (228.7110, 704.5520) -- (228.5130, 704.4620) -- (228.3270, 704.3660) -- (228.0870, 704.2940) -- (227.9070, 704.2580) -- (227.6250, 704.2160) -- (226.3170, 704.2220) -- (226.0230, 704.2580) -- (225.7590, 704.3240) -- (225.5010, 704.3960) -- (225.2430, 704.4980) -- (225.0570, 704.6120) -- (224.9370, 704.6900) -- (224.8170, 704.8160) -- (224.7150, 704.9120) -- (224.6190, 705.0140) -- (224.5170, 705.1340) -- (224.4390, 705.2660) -- (224.3310, 705.4460) -- (224.2950, 705.6380) -- (224.2770, 705.8120) -- cycle(225.7590, 707.9240) -- (225.7590, 709.2620) -- (225.7830, 709.3580) -- (225.8070, 709.4420) -- (225.8610, 709.5260) -- (225.9030, 709.6100) -- (225.9510, 709.6760) -- (226.0530, 709.7840) -- (226.1850, 709.8560) -- (226.3230, 709.9400) -- (226.4790, 709.9940) -- (226.6350, 710.0360) -- (226.8450, 710.0600) -- (226.9950, 710.0480) -- (227.2110, 710.0060) -- (227.3430, 709.9760) -- (227.4630, 709.9340) -- (227.5290, 709.8860) -- (227.6370, 709.8320) -- (227.6730, 709.7900) -- (227.7210, 709.7480) -- (227.7810, 709.7000) -- (227.8110, 709.6520) -- (227.8590, 709.6040) -- (227.9130, 709.5440) -- (227.9310, 709.4720) -- (227.9610, 709.4000) -- (227.9970, 709.3220) -- (228.0150, 709.2620) -- (228.0150, 707.9240) -- cycle;
		
			% path1555
			\path[fill = blue!50, even odd rule, line width = 0.480pt] (227.1330, 705.6200) -- (227.2770, 704.9420) -- (226.6350, 704.9420) -- (226.7910, 705.6140) -- (226.7190, 705.6560) -- (226.6710, 705.7340) -- (226.6590, 705.8300) -- (226.6770, 705.9080) -- (226.7190, 705.9560) -- (226.7550, 705.9980) -- (226.8150, 706.0220) -- (226.8810, 706.0520) -- (226.9710, 706.0580) -- (227.0370, 706.0400) -- (227.1150, 706.0220) -- (227.1870, 705.9740) -- (227.2230, 705.9320) -- (227.2530, 705.8900) -- (227.2590, 705.8300) -- (227.2410, 705.7340) -- (227.2110, 705.6920) -- (227.1750, 705.6560) -- (227.1330, 705.6200);
		
			% path1557
			\path[draw = black, line join = miter, line cap = butt, miter limit = 10.00, line width = 0.461pt] (224.2770, 707.9180) -- (224.7570, 707.9180) -- (224.7570, 709.4180) -- (224.8470, 709.6940) -- (224.9850, 709.9340) -- (225.1230, 710.1140) -- (225.2310, 710.2520) -- (225.3630, 710.3780) -- (225.5190, 710.4860) -- (225.7110, 710.5940) -- (225.8790, 710.6660) -- (226.1070, 710.7500) -- (226.3290, 710.7980) -- (226.5930, 710.8460) -- (226.8750, 710.8580) -- (227.1150, 710.8460) -- (227.3070, 710.8280) -- (227.4690, 710.7860) -- (227.6490, 710.7380) -- (227.8290, 710.6780) -- (227.8890, 710.6540) -- (227.9490, 710.6240) -- (227.9970, 710.6000) -- (228.0450, 710.5700) -- (228.0570, 710.5640) -- (228.0990, 710.5400) -- (228.1410, 710.5100) -- (228.1770, 710.4740) -- (228.2250, 710.4440) -- (228.3630, 710.3360) -- (228.4470, 710.2520) -- (228.5670, 710.1440) -- (228.6570, 710.0060) -- (228.7350, 709.8860) -- (228.8250, 709.7480) -- (228.8910, 709.5980) -- (228.9630, 709.4180) -- (229.0050, 709.2740) -- (229.0110, 707.9180) -- (229.5810, 707.9180) -- (229.5810, 705.8660) -- (229.5330, 705.5960) -- (229.4430, 705.3740) -- (229.3530, 705.2300) -- (229.2690, 705.0800) -- (229.1670, 704.9360) -- (229.0350, 704.7920) -- (228.8910, 704.6540) -- (228.7110, 704.5520) -- (228.5130, 704.4620) -- (228.3270, 704.3660) -- (228.0870, 704.2940) -- (227.9070, 704.2580) -- (227.6250, 704.2160) -- (226.3170, 704.2220) -- (226.0230, 704.2580) -- (225.7590, 704.3240) -- (225.5010, 704.3960) -- (225.2430, 704.4980) -- (225.0570, 704.6120) -- (224.9370, 704.6900) -- (224.8170, 704.8160) -- (224.7150, 704.9120) -- (224.6190, 705.0140) -- (224.5170, 705.1340) -- (224.4390, 705.2660) -- (224.3310, 705.4460) -- (224.2950, 705.6380) -- (224.2770, 705.8120) -- cycle(225.7590, 707.9240) -- (225.7590, 709.2620) -- (225.7830, 709.3580) -- (225.8070, 709.4420) -- (225.8610, 709.5260) -- (225.9030, 709.6100) -- (225.9510, 709.6760) -- (226.0530, 709.7840) -- (226.1850, 709.8560) -- (226.3230, 709.9400) -- (226.4790, 709.9940) -- (226.6350, 710.0360) -- (226.8450, 710.0600) -- (226.9950, 710.0480) -- (227.2110, 710.0060) -- (227.3430, 709.9760) -- (227.4630, 709.9340) -- (227.5290, 709.8860) -- (227.6370, 709.8320) -- (227.6730, 709.7900) -- (227.7210, 709.7480) -- (227.7810, 709.7000) -- (227.8110, 709.6520) -- (227.8590, 709.6040) -- (227.9130, 709.5440) -- (227.9310, 709.4720) -- (227.9610, 709.4000) -- (227.9970, 709.3220) -- (228.0150, 709.2620) -- (228.0150, 707.9240) -- cycle;
		
			% path1559
			\path[draw = black, line join = miter, line cap = butt, miter limit = 10.00, line width = 0.461pt] (224.2770, 707.9180) -- (229.5810, 707.9180) -- cycle;
		
			% path1561
			\path[draw = black, line join = miter, line cap = butt, miter limit = 10.00, line width = 0.461pt] (227.1330, 705.6200) -- (227.2770, 704.9420) -- (226.6350, 704.9420) -- (226.7910, 705.6140) -- (226.7190, 705.6560) -- (226.6710, 705.7340) -- (226.6590, 705.8300) -- (226.6770, 705.9080) -- (226.7190, 705.9560) -- (226.7550, 705.9980) -- (226.8150, 706.0220) -- (226.8810, 706.0520) -- (226.9710, 706.0580) -- (227.0370, 706.0400) -- (227.1150, 706.0220) -- (227.1870, 705.9740) -- (227.2230, 705.9320) -- (227.2530, 705.8900) -- (227.2590, 705.8300) -- (227.2410, 705.7340) -- (227.2110, 705.6920) -- (227.1750, 705.6560) -- (227.1330, 705.6200);
		
			% path1563
			\path[fill = red!50, even odd rule, line width = 0.480pt] (205.7010, 707.9480) -- (206.1750, 707.9480) -- (206.1750, 709.4360) -- (206.2650, 709.7060) -- (206.4030, 709.9460) -- (206.5350, 710.1200) -- (206.6370, 710.2580) -- (206.7690, 710.3840) -- (206.9310, 710.4860) -- (207.1170, 710.6000) -- (207.2790, 710.6720) -- (207.5070, 710.7500) -- (207.7290, 710.7980) -- (207.9870, 710.8460) -- (208.2630, 710.8580) -- (208.5030, 710.8460) -- (208.6890, 710.8280) -- (208.8510, 710.7860) -- (209.0310, 710.7380) -- (209.2050, 710.6780) -- (209.2650, 710.6540) -- (209.3250, 710.6240) -- (209.3670, 710.6000) -- (209.4210, 710.5760) -- (209.4330, 710.5700) -- (209.4690, 710.5400) -- (209.5110, 710.5160) -- (209.5530, 710.4800) -- (209.5950, 710.4500) -- (209.7330, 710.3420) -- (209.8170, 710.2580) -- (209.9310, 710.1500) -- (210.0210, 710.0120) -- (210.0990, 709.8980) -- (210.1830, 709.7600) -- (210.2550, 709.6100) -- (210.3210, 709.4360) -- (210.3690, 709.2920) -- (210.3750, 707.9480) -- (210.9330, 707.9480) -- (210.9330, 705.9260) -- (210.8850, 705.6500) -- (210.7950, 705.4340) -- (210.7050, 705.2900) -- (210.6270, 705.1460) -- (210.5250, 705.0020) -- (210.3930, 704.8580) -- (210.2550, 704.7260) -- (210.0750, 704.6180) -- (209.8770, 704.5280) -- (209.6970, 704.4380) -- (209.4630, 704.3660) -- (209.2770, 704.3300) -- (209.0010, 704.2880) -- (207.7170, 704.2940) -- (207.4230, 704.3300) -- (207.1590, 704.3960) -- (206.9070, 704.4680) -- (206.6490, 704.5640) -- (206.4690, 704.6780) -- (206.3490, 704.7620) -- (206.2350, 704.8820) -- (206.1330, 704.9780) -- (206.0370, 705.0740) -- (205.9350, 705.2000) -- (205.8570, 705.3260) -- (205.7550, 705.5060) -- (205.7190, 705.6980) -- (205.7010, 705.8660) -- cycle(207.1590, 707.9540) -- (207.1590, 709.2800) -- (207.1830, 709.3760) -- (207.2130, 709.4540) -- (207.2610, 709.5380) -- (207.3030, 709.6220) -- (207.3510, 709.6880) -- (207.4530, 709.7960) -- (207.5850, 709.8680) -- (207.7230, 709.9520) -- (207.8730, 710.0060) -- (208.0290, 710.0480) -- (208.2330, 710.0660) -- (208.3830, 710.0600) -- (208.5930, 710.0180) -- (208.7250, 709.9880) -- (208.8450, 709.9460) -- (208.9110, 709.8980) -- (209.0130, 709.8440) -- (209.0550, 709.8020) -- (209.0970, 709.7600) -- (209.1570, 709.7120) -- (209.1870, 709.6640) -- (209.2350, 709.6220) -- (209.2830, 709.5560) -- (209.3070, 709.4900) -- (209.3370, 709.4180) -- (209.3670, 709.3400) -- (209.3850, 709.2800) -- (209.3850, 707.9540) -- cycle;
		
			% path1565
			\path[fill = red!50, even odd rule, line width = 0.480pt] (208.5210, 705.6740) -- (208.6590, 705.0020) -- (208.0290, 705.0020) -- (208.1790, 705.6680) -- (208.1130, 705.7160) -- (208.0590, 705.7940) -- (208.0470, 705.8840) -- (208.0650, 705.9620) -- (208.1130, 706.0100) -- (208.1490, 706.0520) -- (208.2030, 706.0760) -- (208.2690, 706.1060) -- (208.3530, 706.1060) -- (208.4250, 706.0940) -- (208.5030, 706.0760) -- (208.5690, 706.0280) -- (208.6050, 705.9860) -- (208.6410, 705.9440) -- (208.6470, 705.8840) -- (208.6230, 705.7940) -- (208.5930, 705.7520) -- (208.5570, 705.7160) -- (208.5210, 705.6740);
		
			% path1567
			\path[draw = black, line join = miter, line cap = butt, miter limit = 10.00, line width = 0.461pt] (205.7010, 707.9480) -- (206.1750, 707.9480) -- (206.1750, 709.4360) -- (206.2650, 709.7060) -- (206.4030, 709.9460) -- (206.5350, 710.1200) -- (206.6370, 710.2580) -- (206.7690, 710.3840) -- (206.9310, 710.4860) -- (207.1170, 710.6000) -- (207.2790, 710.6720) -- (207.5070, 710.7500) -- (207.7290, 710.7980) -- (207.9870, 710.8460) -- (208.2630, 710.8580) -- (208.5030, 710.8460) -- (208.6890, 710.8280) -- (208.8510, 710.7860) -- (209.0310, 710.7380) -- (209.2050, 710.6780) -- (209.2650, 710.6540) -- (209.3250, 710.6240) -- (209.3670, 710.6000) -- (209.4210, 710.5760) -- (209.4330, 710.5700) -- (209.4690, 710.5400) -- (209.5110, 710.5160) -- (209.5530, 710.4800) -- (209.5950, 710.4500) -- (209.7330, 710.3420) -- (209.8170, 710.2580) -- (209.9310, 710.1500) -- (210.0210, 710.0120) -- (210.0990, 709.8980) -- (210.1830, 709.7600) -- (210.2550, 709.6100) -- (210.3210, 709.4360) -- (210.3690, 709.2920) -- (210.3750, 707.9480) -- (210.9330, 707.9480) -- (210.9330, 705.9260) -- (210.8850, 705.6500) -- (210.7950, 705.4340) -- (210.7050, 705.2900) -- (210.6270, 705.1460) -- (210.5250, 705.0020) -- (210.3930, 704.8580) -- (210.2550, 704.7260) -- (210.0750, 704.6180) -- (209.8770, 704.5280) -- (209.6970, 704.4380) -- (209.4630, 704.3660) -- (209.2770, 704.3300) -- (209.0010, 704.2880) -- (207.7170, 704.2940) -- (207.4230, 704.3300) -- (207.1590, 704.3960) -- (206.9070, 704.4680) -- (206.6490, 704.5640) -- (206.4690, 704.6780) -- (206.3490, 704.7620) -- (206.2350, 704.8820) -- (206.1330, 704.9780) -- (206.0370, 705.0740) -- (205.9350, 705.2000) -- (205.8570, 705.3260) -- (205.7550, 705.5060) -- (205.7190, 705.6980) -- (205.7010, 705.8660) -- cycle(207.1590, 707.9540) -- (207.1590, 709.2800) -- (207.1830, 709.3760) -- (207.2130, 709.4540) -- (207.2610, 709.5380) -- (207.3030, 709.6220) -- (207.3510, 709.6880) -- (207.4530, 709.7960) -- (207.5850, 709.8680) -- (207.7230, 709.9520) -- (207.8730, 710.0060) -- (208.0290, 710.0480) -- (208.2330, 710.0660) -- (208.3830, 710.0600) -- (208.5930, 710.0180) -- (208.7250, 709.9880) -- (208.8450, 709.9460) -- (208.9110, 709.8980) -- (209.0130, 709.8440) -- (209.0550, 709.8020) -- (209.0970, 709.7600) -- (209.1570, 709.7120) -- (209.1870, 709.6640) -- (209.2350, 709.6220) -- (209.2830, 709.5560) -- (209.3070, 709.4900) -- (209.3370, 709.4180) -- (209.3670, 709.3400) -- (209.3850, 709.2800) -- (209.3850, 707.9540) -- cycle;
		
			% path1569
			\path[draw = black, line join = miter, line cap = butt, miter limit = 10.00, line width = 0.461pt] (205.7010, 707.9480) -- (210.9330, 707.9480) -- cycle;
		
			% path1571
			\path[draw = black, line join = miter, line cap = butt, miter limit = 10.00, line width = 0.461pt] (208.5210, 705.6740) -- (208.6590, 705.0020) -- (208.0290, 705.0020) -- (208.1790, 705.6680) -- (208.1130, 705.7160) -- (208.0590, 705.7940) -- (208.0470, 705.8840) -- (208.0650, 705.9620) -- (208.1130, 706.0100) -- (208.1490, 706.0520) -- (208.2030, 706.0760) -- (208.2690, 706.1060) -- (208.3530, 706.1060) -- (208.4250, 706.0940) -- (208.5030, 706.0760) -- (208.5690, 706.0280) -- (208.6050, 705.9860) -- (208.6410, 705.9440) -- (208.6470, 705.8840) -- (208.6230, 705.7940) -- (208.5930, 705.7520) -- (208.5570, 705.7160) -- (208.5210, 705.6740);
		
			% path1573
			\path[fill = green!50, even odd rule, line width = 0.480pt] (186.4770, 707.9180) -- (186.9570, 707.9180) -- (186.9570, 709.4180) -- (187.0470, 709.6940) -- (187.1850, 709.9340) -- (187.3230, 710.1140) -- (187.4310, 710.2520) -- (187.5630, 710.3780) -- (187.7190, 710.4860) -- (187.9110, 710.5940) -- (188.0790, 710.6660) -- (188.3070, 710.7500) -- (188.5290, 710.7980) -- (188.7930, 710.8460) -- (189.0750, 710.8580) -- (189.3150, 710.8460) -- (189.5070, 710.8280) -- (189.6690, 710.7860) -- (189.8490, 710.7380) -- (190.0290, 710.6780) -- (190.0890, 710.6540) -- (190.1490, 710.6240) -- (190.1970, 710.6000) -- (190.2450, 710.5700) -- (190.2570, 710.5640) -- (190.2990, 710.5400) -- (190.3410, 710.5100) -- (190.3770, 710.4740) -- (190.4250, 710.4440) -- (190.5630, 710.3360) -- (190.6470, 710.2520) -- (190.7670, 710.1440) -- (190.8570, 710.0060) -- (190.9350, 709.8860) -- (191.0250, 709.7480) -- (191.0910, 709.5980) -- (191.1630, 709.4180) -- (191.2050, 709.2740) -- (191.2110, 707.9180) -- (191.7810, 707.9180) -- (191.7810, 705.8660) -- (191.7330, 705.5960) -- (191.6430, 705.3740) -- (191.5530, 705.2300) -- (191.4690, 705.0800) -- (191.3670, 704.9360) -- (191.2350, 704.7920) -- (191.0910, 704.6540) -- (190.9110, 704.5520) -- (190.7130, 704.4620) -- (190.5270, 704.3660) -- (190.2870, 704.2940) -- (190.1070, 704.2580) -- (189.8250, 704.2160) -- (188.5170, 704.2220) -- (188.2230, 704.2580) -- (187.9590, 704.3240) -- (187.7010, 704.3960) -- (187.4430, 704.4980) -- (187.2570, 704.6120) -- (187.1370, 704.6900) -- (187.0170, 704.8160) -- (186.9150, 704.9120) -- (186.8190, 705.0140) -- (186.7170, 705.1340) -- (186.6390, 705.2660) -- (186.5310, 705.4460) -- (186.4950, 705.6380) -- (186.4770, 705.8120) -- cycle(187.9590, 707.9240) -- (187.9590, 709.2620) -- (187.9830, 709.3580) -- (188.0070, 709.4420) -- (188.0610, 709.5260) -- (188.1030, 709.6100) -- (188.1510, 709.6760) -- (188.2530, 709.7840) -- (188.3850, 709.8560) -- (188.5230, 709.9400) -- (188.6790, 709.9940) -- (188.8350, 710.0360) -- (189.0450, 710.0600) -- (189.1950, 710.0480) -- (189.4110, 710.0060) -- (189.5430, 709.9760) -- (189.6630, 709.9340) -- (189.7290, 709.8860) -- (189.8370, 709.8320) -- (189.8730, 709.7900) -- (189.9210, 709.7480) -- (189.9810, 709.7000) -- (190.0110, 709.6520) -- (190.0590, 709.6040) -- (190.1130, 709.5440) -- (190.1310, 709.4720) -- (190.1610, 709.4000) -- (190.1970, 709.3220) -- (190.2150, 709.2620) -- (190.2150, 707.9240) -- cycle;
		
			% path1575
			\path[fill = green!50, even odd rule, line width = 0.480pt] (189.3330, 705.6200) -- (189.4770, 704.9420) -- (188.8350, 704.9420) -- (188.9910, 705.6140) -- (188.9190, 705.6560) -- (188.8710, 705.7340) -- (188.8590, 705.8300) -- (188.8770, 705.9080) -- (188.9190, 705.9560) -- (188.9550, 705.9980) -- (189.0150, 706.0220) -- (189.0810, 706.0520) -- (189.1710, 706.0580) -- (189.2370, 706.0400) -- (189.3150, 706.0220) -- (189.3870, 705.9740) -- (189.4230, 705.9320) -- (189.4530, 705.8900) -- (189.4590, 705.8300) -- (189.4410, 705.7340) -- (189.4110, 705.6920) -- (189.3750, 705.6560) -- (189.3330, 705.6200);
		
			% path1577
			\path[draw = black, line join = miter, line cap = butt, miter limit = 10.00, line width = 0.461pt] (186.4770, 707.9180) -- (186.9570, 707.9180) -- (186.9570, 709.4180) -- (187.0470, 709.6940) -- (187.1850, 709.9340) -- (187.3230, 710.1140) -- (187.4310, 710.2520) -- (187.5630, 710.3780) -- (187.7190, 710.4860) -- (187.9110, 710.5940) -- (188.0790, 710.6660) -- (188.3070, 710.7500) -- (188.5290, 710.7980) -- (188.7930, 710.8460) -- (189.0750, 710.8580) -- (189.3150, 710.8460) -- (189.5070, 710.8280) -- (189.6690, 710.7860) -- (189.8490, 710.7380) -- (190.0290, 710.6780) -- (190.0890, 710.6540) -- (190.1490, 710.6240) -- (190.1970, 710.6000) -- (190.2450, 710.5700) -- (190.2570, 710.5640) -- (190.2990, 710.5400) -- (190.3410, 710.5100) -- (190.3770, 710.4740) -- (190.4250, 710.4440) -- (190.5630, 710.3360) -- (190.6470, 710.2520) -- (190.7670, 710.1440) -- (190.8570, 710.0060) -- (190.9350, 709.8860) -- (191.0250, 709.7480) -- (191.0910, 709.5980) -- (191.1630, 709.4180) -- (191.2050, 709.2740) -- (191.2110, 707.9180) -- (191.7810, 707.9180) -- (191.7810, 705.8660) -- (191.7330, 705.5960) -- (191.6430, 705.3740) -- (191.5530, 705.2300) -- (191.4690, 705.0800) -- (191.3670, 704.9360) -- (191.2350, 704.7920) -- (191.0910, 704.6540) -- (190.9110, 704.5520) -- (190.7130, 704.4620) -- (190.5270, 704.3660) -- (190.2870, 704.2940) -- (190.1070, 704.2580) -- (189.8250, 704.2160) -- (188.5170, 704.2220) -- (188.2230, 704.2580) -- (187.9590, 704.3240) -- (187.7010, 704.3960) -- (187.4430, 704.4980) -- (187.2570, 704.6120) -- (187.1370, 704.6900) -- (187.0170, 704.8160) -- (186.9150, 704.9120) -- (186.8190, 705.0140) -- (186.7170, 705.1340) -- (186.6390, 705.2660) -- (186.5310, 705.4460) -- (186.4950, 705.6380) -- (186.4770, 705.8120) -- cycle(187.9590, 707.9240) -- (187.9590, 709.2620) -- (187.9830, 709.3580) -- (188.0070, 709.4420) -- (188.0610, 709.5260) -- (188.1030, 709.6100) -- (188.1510, 709.6760) -- (188.2530, 709.7840) -- (188.3850, 709.8560) -- (188.5230, 709.9400) -- (188.6790, 709.9940) -- (188.8350, 710.0360) -- (189.0450, 710.0600) -- (189.1950, 710.0480) -- (189.4110, 710.0060) -- (189.5430, 709.9760) -- (189.6630, 709.9340) -- (189.7290, 709.8860) -- (189.8370, 709.8320) -- (189.8730, 709.7900) -- (189.9210, 709.7480) -- (189.9810, 709.7000) -- (190.0110, 709.6520) -- (190.0590, 709.6040) -- (190.1130, 709.5440) -- (190.1310, 709.4720) -- (190.1610, 709.4000) -- (190.1970, 709.3220) -- (190.2150, 709.2620) -- (190.2150, 707.9240) -- cycle;
		
			% path1579
			\path[draw = black, line join = miter, line cap = butt, miter limit = 10.00, line width = 0.461pt] (186.4770, 707.9180) -- (191.7810, 707.9180) -- cycle;
		
			% path1581
			\path[draw = black, line join = miter, line cap = butt, miter limit = 10.00, line width = 0.461pt] (189.3330, 705.6200) -- (189.4770, 704.9420) -- (188.8350, 704.9420) -- (188.9910, 705.6140) -- (188.9190, 705.6560) -- (188.8710, 705.7340) -- (188.8590, 705.8300) -- (188.8770, 705.9080) -- (188.9190, 705.9560) -- (188.9550, 705.9980) -- (189.0150, 706.0220) -- (189.0810, 706.0520) -- (189.1710, 706.0580) -- (189.2370, 706.0400) -- (189.3150, 706.0220) -- (189.3870, 705.9740) -- (189.4230, 705.9320) -- (189.4530, 705.8900) -- (189.4590, 705.8300) -- (189.4410, 705.7340) -- (189.4110, 705.6920) -- (189.3750, 705.6560) -- (189.3330, 705.6200);
		
			% path1717
			\path[draw = red, line join = miter, line cap = butt, miter limit = 10.00, line width = 1.094pt] (207.2970, 737.3240) .. controls (208.4370, 736.8440) and (209.5710, 736.3640) .. (209.6250, 735.5420) .. controls (209.6730, 734.7200) and (207.6330, 733.1420) .. (207.6030, 732.3860) .. controls (207.5730, 731.6360) and (209.4090, 731.8640) .. (209.4390, 731.0180) .. controls (209.4690, 730.1720) and (208.1010, 727.9580) .. (207.7890, 727.3160);
		
			% path1719
			\path[draw = green, line join = miter, line cap = butt, miter limit = 10.00, line width = 1.094pt] (187.2090, 737.3240) .. controls (188.3790, 736.8440) and (189.5490, 736.3640) .. (189.6030, 735.5420) .. controls (189.6570, 734.7200) and (187.5570, 733.1420) .. (187.5270, 732.3860) .. controls (187.4910, 731.6360) and (189.3810, 731.8640) .. (189.4170, 731.0180) .. controls (189.4470, 730.1720) and (188.0370, 727.9580) .. (187.7130, 727.3160);
		
			% path1721
			\path[draw = orange, line join = miter, line cap = butt, miter limit = 10.00, line width = 1.094pt] (245.8170, 737.4680) .. controls (246.9510, 736.9880) and (248.0910, 736.5080) .. (248.1450, 735.6860) .. controls (248.1930, 734.8640) and (246.1530, 733.2860) .. (246.1230, 732.5300) .. controls (246.0930, 731.7800) and (247.9290, 732.0080) .. (247.9590, 731.1620) .. controls (247.9890, 730.3160) and (246.6210, 728.1020) .. (246.3090, 727.4600);
		
			% path1723
			\path[draw = blue, line join = miter, line cap = butt, miter limit = 10.00, line width = 1.094pt] (225.2970, 737.4680) .. controls (226.4370, 736.9880) and (227.5710, 736.5080) .. (227.6250, 735.6860) .. controls (227.6730, 734.8640) and (225.6330, 733.2860) .. (225.6030, 732.5300) .. controls (225.5730, 731.7800) and (227.4090, 732.0080) .. (227.4390, 731.1620) .. controls (227.4690, 730.3160) and (226.1010, 728.1020) .. (225.7890, 727.4600);
		
			% path1725
			\path[fill = gray!50, even odd rule, line width = 0.480pt] (184.9410, 714.8960) -- (192.2130, 714.8960) -- (192.2130, 722.1680) -- (184.9410, 722.1680) -- cycle;
		
			% path1727
			\path[draw = black, line join = miter, line cap = butt, miter limit = 10.00, line width = 0.461pt] (184.9410, 714.8960) -- (192.2130, 714.8960) -- (192.2130, 722.1680) -- (184.9410, 722.1680) -- cycle;
		
			\begin{scope}[cm = {{0.6, 0.0, 0.0, 0.6, (85.149, 560.6)}}]% g1735
				\begin{scope}% g1733
					% text1731
					\path[cm = {{1.0, 0.0, 0.0, -1.0, (168.62, 258.5)}}, fill = black, nonzero rule] (0.0000, 0.0000) node[above right, font = \scriptsize, inner sep = 0em] (text1731) {C};
				\end{scope}
			\end{scope}
			% path1737
			\path[fill = gray!50, even odd rule, line width = 0.480pt] (204.6690, 714.9680) -- (211.9410, 714.9680) -- (211.9410, 722.2400) -- (204.6690, 722.2400) -- cycle;
		
			% path1739
			\path[draw = black, line join = miter, line cap = butt, miter limit = 10.00, line width = 0.461pt] (204.6690, 714.9680) -- (211.9410, 714.9680) -- (211.9410, 722.2400) -- (204.6690, 722.2400) -- cycle;
		
			\begin{scope}[cm = {{0.6, 0.0, 0.0, 0.6, (85.149, 560.6)}}]% g1747
				\begin{scope}% g1745
					% text1743
					\path[cm = {{1.0, 0.0, 0.0, -1.0, (201.5, 258.65)}}, fill = black, nonzero rule] (0.0000, 0.0000) node[above right, font = \scriptsize, inner sep = 0em] (text1743) {C};
				\end{scope}
			\end{scope}
			% path1749
			\path[fill = gray!50, even odd rule, line width = 0.480pt] (223.2450, 714.7520) -- (230.5170, 714.7520) -- (230.5170, 722.0240) -- (223.2450, 722.0240) -- cycle;
		
			% path1751
			\path[draw = black, line join = miter, line cap = butt, miter limit = 10.00, line width = 0.461pt] (223.2450, 714.7520) -- (230.5170, 714.7520) -- (230.5170, 722.0240) -- (223.2450, 722.0240) -- cycle;
		
			\begin{scope}[cm = {{0.6, 0.0, 0.0, 0.6, (85.149, 560.6)}}]% g1759
				\begin{scope}% g1757
					% text1755
					\path[cm = {{1.0, 0.0, 0.0, -1.0, (232.51, 258.22)}}, fill = black, nonzero rule] (0.0000, 0.0000) node[above right, font = \scriptsize, inner sep = 0em] (text1755) {C};
				\end{scope}
			\end{scope}
			% path1761
			\path[fill = gray!50, even odd rule, line width = 0.480pt] (243.6930, 714.6080) -- (250.9650, 714.6080) -- (250.9650, 721.8800) -- (243.6930, 721.8800) -- cycle;
		
			% path1763
			\path[draw = black, line join = miter, line cap = butt, miter limit = 10.00, line width = 0.461pt] (243.6930, 714.6080) -- (250.9650, 714.6080) -- (250.9650, 721.8800) -- (243.6930, 721.8800) -- cycle;
		
			\begin{scope}[cm = {{0.6, 0.0, 0.0, 0.6, (85.149, 560.6)}}]% g1771
				\begin{scope}% g1769
					% text1767
					\path[cm = {{1.0, 0.0, 0.0, -1.0, (266.54, 258.05)}}, fill = black, nonzero rule] (0.0000, 0.0000) node[above right, font = \scriptsize, inner sep = 0em] (text1767) {C};
				\end{scope}
			\end{scope}
			% path1829
			\path[draw = black, line join = miter, line cap = butt, miter limit = 10.00, line width = 0.461pt] (179.4690, 709.0040) .. controls (179.4690, 710.7860) and (180.9150, 712.2320) .. (182.6970, 712.2320) -- (195.6090, 712.2320) .. controls (197.3910, 712.2320) and (198.8370, 710.7860) .. (198.8370, 709.0040) -- (198.8370, 678.0200) .. controls (198.8370, 676.2380) and (197.3910, 674.7920) .. 	(195.6090, 674.7920) -- (182.6970, 674.7920) .. controls (180.9150, 674.7920) and (179.4690, 676.2380) .. (179.4690, 678.0200) -- cycle;
		
			% path1831
			\path[draw = black, line join = miter, line cap = butt, miter limit = 10.00, line width = 0.461pt] (198.8370, 709.0520) .. controls (198.8370, 710.8100) and (200.2590, 712.2320) .. (202.0170, 712.2320) -- (214.7370, 712.2320) .. controls (216.4950, 712.2320) and (217.9170, 710.8100) .. (217.9170, 709.0520) -- (217.9170, 677.9720) .. controls (217.9170, 676.2140) and (216.4950, 674.7920) .. 	(214.7370, 674.7920) -- (202.0170, 674.7920) .. controls (200.2590, 674.7920) and (198.8370, 676.2140) .. (198.8370, 677.9720) -- cycle;
		
			% path1833
			\path[draw = black, line join = miter, line cap = butt, miter limit = 10.00, line width = 0.461pt] (217.7730, 708.9560) .. controls (217.7730, 710.7680) and (219.2370, 712.2320) .. (221.0490, 712.2320) -- (234.1530, 712.2320) .. controls (235.9650, 712.2320) and (237.4290, 710.7680) .. (237.4290, 708.9560) -- (237.4290, 677.9960) .. controls (237.4290, 676.1840) and (235.9650, 674.7200) .. 	(234.1530, 674.7200) -- (221.0490, 674.7200) .. controls (219.2370, 674.7200) and (217.7730, 676.1840) .. (217.7730, 677.9960) -- cycle;
		
			% path1835
			\path[draw = black, line join = miter, line cap = butt, miter limit = 10.00, line width = 0.461pt] (237.5010, 708.8720) .. controls (237.5010, 710.6900) and (238.9710, 712.1600) .. (240.7890, 712.1600) -- (253.9410, 712.1600) .. controls (255.7590, 712.1600) and (257.2290, 710.6900) .. (257.2290, 708.8720) -- (257.2290, 678.0080) .. controls (257.2290, 676.1900) and (255.7590, 674.7200) .. 	(253.9410, 674.7200) -- (240.7890, 674.7200) .. controls (238.9710, 674.7200) and (237.5010, 676.1900) .. (237.5010, 678.0080) -- cycle;
		\end{scope}
	\end{tikzpicture}%
}


\newcommand{\lock}[3]{%
	$\text{lock}_{\tikz{\node[transaction] (t#1) {$t_#1$};}}\left(#2, #3\right)$%
}
\newcommand{\cancelledLock}[3]{%
	$\text{lock}_{\tikz{\node[transaction] (t#1) {$t_#1$};\node[cancel] at (t#1) {};}}\left(#2, #3\right)$%
}
\newcommand{\unlock}[2]{%
	$\text{unlock}_{\tikz{\node[transaction] (t#1) {$t_#1$};}}\left(#2\right)$%
}
	
\newcommand{\dlDetectDiagram}[1][1.0]{%
	\tikzset{
		transaction/.style = {draw = beamer@blendedblue,
					       text = beamer@blendedblue,
					       thick,
					       shape = circle},
		cancel/.style = {draw = red,
					thick,
					shape = cross out,
					inner sep = 1em,
					anchor = center},
		dependency/.style = { -stealth ,
						 draw = beamer@blendedblue,
						 fill = beamer@blendedblue,
						 thick}
	}	

	\begin{tikzpicture}[opacity = #1,
	                             baseline = (current bounding box.center)]
		\node[transaction]					(t1)			{$t_1$};
		\node[transaction, below = of t1]		(t2)			{$t_2$};
		\node[transaction, below = of t2]		(t3)			{$t_3$};
		\node[cancel]	at (t3)							{};
		
		\draw[dependency]		(t1)		edge					(t2);
		\draw[dependency]		(t2)		edge					(t3);
		\draw[dependency]		(t3)		edge[bend right = 25]	(t1);
	\end{tikzpicture}%
}


\newcommand{\noWaitDiagram}[1][1.0]{%
	\tikzset{
		invocation/.style = {text = beamer@blendedblue},
		transaction/.style = {draw = beamer@blendedblue,
					       text = beamer@blendedblue,
					       thin,
					       inner sep = 0.1pt,
					       shape = circle,
					       scale = 0.9},
		cancel/.style = {draw = red,
					thin,
					shape = cross out,
					inner sep = .5em,
					anchor = center,
					scale = 0.9},
		axis/.style = { -> ,
				    draw = beamer@blendedblue,
				    fill = beamer@blendedblue,
				    text = beamer@blendedblue,
				    thick}
	}	

	\begin{tikzpicture}[opacity = #1,
	                             baseline = (current bounding box.center),
	                             node distance = .5em]
		\node[invocation]					(i1)			{\lock{1}{a}{S}};
		\node[invocation, below = of i1]			(i2)			{\cancelledLock{2}{a}{X}};
		\node[invocation, below = of i2]			(i3)			{\unlock{1}{a}};
		\node[invocation, below = of i3]			(i4)			{\lock{3}{b}{X}};
		\node[invocation, below = of i4]			(i5)			{\cancelledLock{4}{b}{S}};
		
		\draw[axis, xshift = 2em]	(i1.north east)	--	(i5.south east)	node[pos = 0.925, right]{$t$};
	\end{tikzpicture}%
}


\newcommand{\twoVNoWaitDiagram}[1][1.0]{%
	\tikzset{
		invocation/.style = {text = beamer@blendedblue},
		transaction/.style = {draw = beamer@blendedblue,
					       text = beamer@blendedblue,
					       thin,
					       inner sep = 0.1pt,
					       shape = circle,
					       scale = 0.9},
		cancel/.style = {draw = red,
					thin,
					shape = cross out,
					inner sep = .5em,
					anchor = center,
					scale = 0.9},
		axis/.style = { -> ,
				    draw = beamer@blendedblue,
				    fill = beamer@blendedblue,
				    text = beamer@blendedblue,
				    thick}
	}	

	\begin{tikzpicture}[opacity = #1,
	                             baseline = (current bounding box.center),
	                             node distance = .5em]
		\node[invocation]					(i1)			{\lock{1}{a}{X}};
		\node[invocation, below = of i1]			(i2)			{\lock{2}{a}{S}};
		\node[invocation, below = of i2]			(i3)			{\cancelledLock{3}{a}{X}};
		\node[invocation, below = of i3]			(i4)			{\unlock{2}{a}};
		\node[invocation, below = of i4]			(i5)			{\lock{1}{a}{C}};
		
		\draw[axis, xshift = 2em]	(i1.north east)	--	(i5.south east)	node[pos = 0.925, right]{$t$};
	\end{tikzpicture}%
}


\newcommand{\siloDiagram}[1][1.0]{%
	\tikzset{
		phase/.style = {draw = beamer@blendedblue,
				        text = beamer@blendedblue,
				        text width = 4em,
				        align = center,
				        thick,
				        shape = ellipse,
				        inner sep = 1pt},
		transision/.style = { -stealth ,
					     draw = beamer@blendedblue,
					     fill = beamer@blendedblue,
					     thick}
	}	

	\begin{tikzpicture}[opacity = #1,
	                             baseline = (current bounding box.center),
	                             node distance = 1em]
		\node[phase]					(p1)			{read phase};
		\node[phase, below = of p1]		(p2)			{validation phase};
		\node[phase, below = of p2]		(p3)			{write phase};
		
		\draw[transision]		(p1)		edge					(p2);
		\draw[transision]		(p2)		edge					(p3);
	\end{tikzpicture}%
}


\newlength\architecturesheight
\settoheight\architecturesheight{%
	\begin{tabular}{>{\raggedleft}m{.15\linewidth}cc>{\raggedright}m{.15\linewidth}}
		Shared Everything	&	\seDiagram		&	\doraDiagram	&	Data-Oriented Transaction Execution (DORA)	\\
		Delegation		&	\delegationDiagram	&	\pseDiagram	&	Partitioned Serial Execution (PSE)
	\end{tabular}%
}

%%%%%%%%%%%%%%%%%%%%%%%%%%%%%%%%%%%%%%%%%%%%%%%
%%%%%%%%%%%%%%%%%%%%%%%%%%%%%%%%%%%%%%%%%%%%%%%
%%%%%%%%%%%%%%%%%%%%%%%%%%%%%%%%%%%%%%%%%%%%%%%
% Defines a shape 'square' for tikz that behaves like a square :-):
%%%%%%%%%%%%%%%%%%%%%%%%%%%%%%%%%%%%%%%%%%%%%%% begindefinition
\makeatletter
% the contents of \squarecorner were mostly stolen from pgfmoduleshapes.code.tex
\def\squarecorner#1{
    % Calculate x
    %
    % First, is width < minimum width?
    \pgf@x=\the\wd\pgfnodeparttextbox%
    \pgfmathsetlength\pgf@xc{\pgfkeysvalueof{/pgf/inner xsep}}%
    \advance\pgf@x by 2\pgf@xc%
    \pgfmathsetlength\pgf@xb{\pgfkeysvalueof{/pgf/minimum width}}%
    \ifdim\pgf@x<\pgf@xb%
        % yes, too small. Enlarge...
        \pgf@x=\pgf@xb%
    \fi%
    % Calculate y
    %
    % First, is height+depth < minimum height?
    \pgf@y=\ht\pgfnodeparttextbox%
    \advance\pgf@y by\dp\pgfnodeparttextbox%
    \pgfmathsetlength\pgf@yc{\pgfkeysvalueof{/pgf/inner ysep}}%
    \advance\pgf@y by 2\pgf@yc%
    \pgfmathsetlength\pgf@yb{\pgfkeysvalueof{/pgf/minimum height}}%
    \ifdim\pgf@y<\pgf@yb%
        % yes, too small. Enlarge...
        \pgf@y=\pgf@yb%
    \fi%
    %
    % this \ifdim is the actual part that makes the node dimensions square.
    \ifdim\pgf@x<\pgf@y%
        \pgf@x=\pgf@y%
    \else
        \pgf@y=\pgf@x%
    \fi
    %
    % Now, calculate right border: .5\wd\pgfnodeparttextbox + .5 \pgf@x + #1outer sep
    \pgf@x=#1.5\pgf@x%
    \advance\pgf@x by.5\wd\pgfnodeparttextbox%
    \pgfmathsetlength\pgf@xa{\pgfkeysvalueof{/pgf/outer xsep}}%
    \advance\pgf@x by#1\pgf@xa%
    % Now, calculate upper border: .5\ht-.5\dp + .5 \pgf@y + #1outer sep
    \pgf@y=#1.5\pgf@y%
    \advance\pgf@y by-.5\dp\pgfnodeparttextbox%
    \advance\pgf@y by.5\ht\pgfnodeparttextbox%
    \pgfmathsetlength\pgf@ya{\pgfkeysvalueof{/pgf/outer ysep}}%
    \advance\pgf@y by#1\pgf@ya%
}
\makeatother

\pgfdeclareshape{simplesquare}{
    \savedanchor\northeast{\squarecorner{}}
    \savedanchor\southwest{\squarecorner{-}}

    \foreach \x in {east,west} \foreach \y in {north,mid,base,south} {
        \inheritanchor[from=rectangle]{\y\space\x}
    }
    \foreach \x in {east,west,north,mid,base,south,center,text} {
        \inheritanchor[from=rectangle]{\x}
    }
    \inheritanchorborder[from=rectangle]
    \inheritbackgroundpath[from=rectangle]
}
%%%%%%%%%%%%%%%%%%%%%%%%%%%%%%%%%%%%%%%%%%%%%%% enddefinition

%%%%%%%%%%%%%%%%%%%%%%%%%%%%%%%%%%%%%%%%%%%%%%%
%%%%%%%%%%%%%%%%%%%%%%%%%%%%%%%%%%%%%%%%%%%%%%%
%%%%%%%%%%%%%%%%%%%%%%%%%%%%%%%%%%%%%%%%%%%%%%%
% Adds 'west north west', 'east north east', 'east south east', 'north north west', 
% 'south south west', 'south south east' to the tikz shape 'rectangle':
%%%%%%%%%%%%%%%%%%%%%%%%%%%%%%%%%%%%%%%%%%%%%%% begindefinition
\makeatletter
\pgfdeclareshape{square}{
  \inheritsavedanchors[from=simplesquare]
  \inheritanchorborder[from=simplesquare]
  \foreach \a in {%
      center,mid,base,north,south,west,east,%
      north west,mid west,base west,south west,%
      north east,mid east,base east,south east%
    }{\inheritanchor[from=simplesquare]{\a}}
  \inheritbackgroundpath[from=simplesquare]
  \anchor{north 1/3}{
    \southwest\pgf@xa=\pgf@x
    \northeast\pgfmathsetlength\pgf@x{\pgf@xa-(\pgf@xa-\pgf@x)/3}
  }
  \anchor{north 2/3}{
    \southwest\pgf@xa=\pgf@x
    \northeast\pgfmathsetlength\pgf@x{\pgf@x-(\pgf@x-\pgf@xa)/3}
  }
  \anchor{south 1/3}{
    \northeast\pgf@xa=\pgf@x
    \southwest\pgfmathsetlength\pgf@x{\pgf@x-(\pgf@x-\pgf@xa)/3}
  }
  \anchor{south 2/3}{
    \northeast\pgf@xa=\pgf@x
    \southwest\pgfmathsetlength\pgf@x{\pgf@xa-(\pgf@xa-\pgf@x)/3}
  }
  \anchor{east 1/3}{
    \southwest\pgf@ya=\pgf@y
    \northeast\pgfmathsetlength\pgf@y{\pgf@ya-(\pgf@ya-\pgf@y)/3}
  }
  \anchor{east 2/3}{
    \southwest\pgf@ya=\pgf@y
    \northeast\pgfmathsetlength\pgf@y{\pgf@y-(\pgf@y-\pgf@ya)/3}
  }
  \anchor{west 1/3}{
    \northeast\pgf@ya=\pgf@y
    \southwest\pgfmathsetlength\pgf@y{\pgf@y-(\pgf@y-\pgf@ya)/3}
  }
  \anchor{west 2/3}{
    \northeast\pgf@ya=\pgf@y
    \southwest\pgfmathsetlength\pgf@y{\pgf@ya-(\pgf@ya-\pgf@y)/3}
  }
}
\makeatother
%%%%%%%%%%%%%%%%%%%%%%%%%%%%%%%%%%%%%%%%%%%%%%% enddefinition

%%%%%%%%%%%%%%%%%%%%%%%%%%%%%%%%%%%%%%%%%%%%%%%
%%%%%%%%%%%%%%%%%%%%%%%%%%%%%%%%%%%%%%%%%%%%%%%
%%%%%%%%%%%%%%%%%%%%%%%%%%%%%%%%%%%%%%%%%%%%%%%
% Redefines the bibliography entry for inproceedings:
%%%%%%%%%%%%%%%%%%%%%%%%%%%%%%%%%%%%%%%%%%%%%%% begindefinition
\DeclareBibliographyDriver{inproceedings}{%
  \usebibmacro{bibindex}%
  \usebibmacro{begentry}%
  \usebibmacro{author}%
  \setunit{\printdelim{nametitledelim}}\newblock
  \usebibmacro{title}%
  \newunit\newblock
  \usebibmacro{date}%
  \usebibmacro{finentry}}
%%%%%%%%%%%%%%%%%%%%%%%%%%%%%%%%%%%%%%%%%%%%%%% enddefinition

%%%%%%%%%%%%%%%%%%%%%%%%%%%%%%%%%%%%%%%%%%%%%%%
%%%%%%%%%%%%%%%%%%%%%%%%%%%%%%%%%%%%%%%%%%%%%%%
%%%%%%%%%%%%%%%%%%%%%%%%%%%%%%%%%%%%%%%%%%%%%%%
% Redefines the bibliography entry for article:
%%%%%%%%%%%%%%%%%%%%%%%%%%%%%%%%%%%%%%%%%%%%%%% begindefinition
\DeclareBibliographyDriver{article}{%
  \usebibmacro{bibindex}%
  \usebibmacro{begentry}%
  \usebibmacro{author}%
  \setunit{\printdelim{nametitledelim}}\newblock
  \usebibmacro{title}%
  \newunit\newblock
  \usebibmacro{date}%
  \usebibmacro{finentry}}
%%%%%%%%%%%%%%%%%%%%%%%%%%%%%%%%%%%%%%%%%%%%%%% enddefinition

%%%%%%%%%%%%%%%%%%%%%%%%%%%%%%%%%%%%%%%%%%%%%%%
%%%%%%%%%%%%%%%%%%%%%%%%%%%%%%%%%%%%%%%%%%%%%%%
%%%%%%%%%%%%%%%%%%%%%%%%%%%%%%%%%%%%%%%%%%%%%%%
% Redefines the bibliography entry for book:
%%%%%%%%%%%%%%%%%%%%%%%%%%%%%%%%%%%%%%%%%%%%%%% begindefinition
\DeclareBibliographyDriver{book}{%
  \usebibmacro{bibindex}%
  \usebibmacro{begentry}%
  \usebibmacro{author/editor+others/translator+others}%
  \setunit{\printdelim{nametitledelim}}\newblock
  \usebibmacro{maintitle+title}%
  \newunit\newblock
  \usebibmacro{date}%
  \newunit\newblock
  \newunit\newblock
  \iftoggle{bbx:isbn}
    {\printfield{isbn}}
    {}%
  \usebibmacro{finentry}}
%%%%%%%%%%%%%%%%%%%%%%%%%%%%%%%%%%%%%%%%%%%%%%% enddefinition

%%%%%%%%%%%%%%%%%%%%%%%%%%%%%%%%%%%%%%%%%%%%%%%
%%%%%%%%%%%%%%%%%%%%%%%%%%%%%%%%%%%%%%%%%%%%%%%
%%%%%%%%%%%%%%%%%%%%%%%%%%%%%%%%%%%%%%%%%%%%%%%
% Redefines the bibliography entry for misc:
%%%%%%%%%%%%%%%%%%%%%%%%%%%%%%%%%%%%%%%%%%%%%%% begindefinition
\DeclareBibliographyDriver{misc}{%
  \usebibmacro{bibindex}%
  \usebibmacro{begentry}%
  \usebibmacro{author/editor+others/translator+others}%
  \setunit{\printdelim{nametitledelim}}\newblock
  \usebibmacro{title}%
  \newunit\newblock
  \printfield{type}%
  \newunit\newblock
  \usebibmacro{date}%
  \newunit\newblock
  \usebibmacro{url}%
  \usebibmacro{finentry}}
%%%%%%%%%%%%%%%%%%%%%%%%%%%%%%%%%%%%%%%%%%%%%%% enddefinition

%%%%%%%%%%%%%%%%%%%%%%%%%%%%%%%%%%%%%%%%%%%%%%%
%%%%%%%%%%%%%%%%%%%%%%%%%%%%%%%%%%%%%%%%%%%%%%%
%%%%%%%%%%%%%%%%%%%%%%%%%%%%%%%%%%%%%%%%%%%%%%%
% Redefines the bibliography entry for online:
%%%%%%%%%%%%%%%%%%%%%%%%%%%%%%%%%%%%%%%%%%%%%%% begindefinition
\DeclareBibliographyDriver{online}{%
  \usebibmacro{bibindex}%
  \usebibmacro{begentry}%
  \usebibmacro{author/editor+others/translator+others}%
  \setunit{\printdelim{nametitledelim}}\newblock
  \usebibmacro{title}%
  \newunit\newblock
  \usebibmacro{date}%
  \newunit\newblock
  \usebibmacro{url+urldate}%
  \usebibmacro{finentry}}
%%%%%%%%%%%%%%%%%%%%%%%%%%%%%%%%%%%%%%%%%%%%%%% enddefinition

%%%%%%%%%%%%%%%%%%%%%%%%%%%%%%%%%%%%%%%%%%%%%%%
%%%%%%%%%%%%%%%%%%%%%%%%%%%%%%%%%%%%%%%%%%%%%%%
%%%%%%%%%%%%%%%%%%%%%%%%%%%%%%%%%%%%%%%%%%%%%%%
% Redefines the bibliography entry for proceedings:
%%%%%%%%%%%%%%%%%%%%%%%%%%%%%%%%%%%%%%%%%%%%%%% begindefinition
\DeclareBibliographyDriver{proceedings}{%
  \usebibmacro{bibindex}%
  \usebibmacro{begentry}%
  \usebibmacro{editor+others}%
  \setunit{\printdelim{nametitledelim}}\newblock
  \usebibmacro{maintitle+title}%
  \newunit\newblock
  \usebibmacro{date}%
  \usebibmacro{finentry}}
%%%%%%%%%%%%%%%%%%%%%%%%%%%%%%%%%%%%%%%%%%%%%%% enddefinition

%%%%%%%%%%%%%%%%%%%%%%%%%%%%%%%%%%%%%%%%%%%%%%%
%%%%%%%%%%%%%%%%%%%%%%%%%%%%%%%%%%%%%%%%%%%%%%%
%%%%%%%%%%%%%%%%%%%%%%%%%%%%%%%%%%%%%%%%%%%%%%%
% Redefines the bibliography entry for report:
%%%%%%%%%%%%%%%%%%%%%%%%%%%%%%%%%%%%%%%%%%%%%%% begindefinition
\DeclareBibliographyDriver{report}{%
  \usebibmacro{bibindex}%
  \usebibmacro{begentry}%
  \usebibmacro{author}%
  \usebibmacro{title}%
  \newunit\newblock
  \printfield{type}%
  \newunit\newblock
  \usebibmacro{date}%
  \usebibmacro{finentry}}
%%%%%%%%%%%%%%%%%%%%%%%%%%%%%%%%%%%%%%%%%%%%%%% enddefinition

%%%%%%%%%%%%%%%%%%%%%%%%%%%%%%%%%%%%%%%%%%%%%%%
%%%%%%%%%%%%%%%%%%%%%%%%%%%%%%%%%%%%%%%%%%%%%%%
%%%%%%%%%%%%%%%%%%%%%%%%%%%%%%%%%%%%%%%%%%%%%%%
% Redefines the bibliography entry for thesis:
%%%%%%%%%%%%%%%%%%%%%%%%%%%%%%%%%%%%%%%%%%%%%%% begindefinition
\DeclareBibliographyDriver{thesis}{%
  \usebibmacro{bibindex}%
  \usebibmacro{begentry}%
  \usebibmacro{author}%
  \setunit{\printdelim{nametitledelim}}\newblock
  \usebibmacro{title}%
  \newunit\newblock
  \printfield{type}%
  \newunit
  \usebibmacro{date}%
  \usebibmacro{finentry}}
%%%%%%%%%%%%%%%%%%%%%%%%%%%%%%%%%%%%%%%%%%%%%%% enddefinition

%%%%%%%%%%%%%%%%%%%%%%%%%%%%%%%%%%%%%%%%%%%%%%%
%%%%%%%%%%%%%%%%%%%%%%%%%%%%%%%%%%%%%%%%%%%%%%%
%%%%%%%%%%%%%%%%%%%%%%%%%%%%%%%%%%%%%%%%%%%%%%%
% Redefines the bibliography entry for unpublished:
%%%%%%%%%%%%%%%%%%%%%%%%%%%%%%%%%%%%%%%%%%%%%%% begindefinition
\DeclareBibliographyDriver{unpublished}{%
  \usebibmacro{bibindex}%
  \usebibmacro{begentry}%
  \usebibmacro{author}%
  \setunit{\printdelim{nametitledelim}}\newblock
  \usebibmacro{title}%
  \newunit\newblock
  \usebibmacro{date}%
  \newunit\newblock
  \iftoggle{bbx:url}
    {\usebibmacro{url+urldate}}
    {}%
  \usebibmacro{finentry}}
%%%%%%%%%%%%%%%%%%%%%%%%%%%%%%%%%%%%%%%%%%%%%%% enddefinition

%%%%%%%%%%%%%%%%%%%%%%%%%%%%%%%%%%%%%%%%%%%%%%%
%%%%%%%%%%%%%%%%%%%%%%%%%%%%%%%%%%%%%%%%%%%%%%%
%%%%%%%%%%%%%%%%%%%%%%%%%%%%%%%%%%%%%%%%%%%%%%%
% Redefines the bibliography field definitions for url and doi to reduce their font size:
%%%%%%%%%%%%%%%%%%%%%%%%%%%%%%%%%%%%%%%%%%%%%%% begindefinition
\DeclareFieldFormat{url}{\mkbibacro{URL}\addcolon\space\footnotesize\url{#1}}
\DeclareFieldFormat{doi}{%
  \mkbibacro{DOI}\addcolon\space\footnotesize
  \ifhyperref
    {\href{https://doi.org/#1}{\nolinkurl{#1}}}
    {\nolinkurl{#1}}}
%%%%%%%%%%%%%%%%%%%%%%%%%%%%%%%%%%%%%%%%%%%%%%% enddefinition

%%%%%%%%%%%%%%%%%%%%%%%%%%%%%%%%%%%%%%%%%%%%%%%
%%%%%%%%%%%%%%%%%%%%%%%%%%%%%%%%%%%%%%%%%%%%%%%
%%%%%%%%%%%%%%%%%%%%%%%%%%%%%%%%%%%%%%%%%%%%%%%
% Allows the positioning of nodes on a circle:
%%%%%%%%%%%%%%%%%%%%%%%%%%%%%%%%%%%%%%%%%%%%%%% begindefinition
\usepackage{tikz}
\usetikzlibrary{chains}
\tikzset{
  nodes around center/.style args={#1:#2:#3:#4}{%
    % #1 = Startwinkel,   #2 = Anzahl Knoten
    % #3 = Zentrums-Node, #4 = Abstand
    at={([shift={(#3)}] {{(\tikzchaincount-1)*360/(#2)+#1}}:{#4})}
  },
  nodes around center*/.style args={#1:#2:#3:#4}{% gleiche Optionen wie oben
    at={([shift={(#3.{(\tikzchaincount-1)*360/(#2)+#1})}] {{(\tikzchaincount-1)*360/(#2)+#1}}:{#4})},
    anchor={(\tikzchaincount-1)*360/(#2)+#1+180}
  }
}
%%%%%%%%%%%%%%%%%%%%%%%%%%%%%%%%%%%%%%%%%%%%%%% enddefinition

%%%%%%%%%%%%%%%%%%%%%%%%%%%%%%%%%%%%%%%%%%%%%%%
%%%%%%%%%%%%%%%%%%%%%%%%%%%%%%%%%%%%%%%%%%%%%%%
%%%%%%%%%%%%%%%%%%%%%%%%%%%%%%%%%%%%%%%%%%%%%%%
% Transforms biblioraphy items:
%%%%%%%%%%%%%%%%%%%%%%%%%%%%%%%%%%%%%%%%%%%%%%% begindefinition
\setbeamertemplate{bibliography item}{%
  \ifboolexpr{ test {\ifentrytype{book}} or test {\ifentrytype{mvbook}}
    or test {\ifentrytype{collection}} or test {\ifentrytype{mvcollection}}
    or test {\ifentrytype{reference}} or test {\ifentrytype{mvreference}} }
    {\setbeamertemplate{bibliography item}[book]}
    {\ifentrytype{online}
       {\setbeamertemplate{bibliography item}[online]}
       {\setbeamertemplate{bibliography item}[article]}}%
  \usebeamertemplate{bibliography item}}
%%%%%%%%%%%%%%%%%%%%%%%%%%%%%%%%%%%%%%%%%%%%%%% enddefinition


\newlength{\enhancedtextwidth}
\setlength{\enhancedtextwidth}{\dimexpr\textwidth+4em}
	
\newlength{\fiberwidth}
\settowidth{\fiberwidth}{Fiber 0 \textit{suspended}}
	
\lstset{basicstyle = \ttfamily}

% secIntro,dbmsRequirements,dbmsProperties,secArchi,subsecSE,se,seProCon,subsecDORA,dora,doraExample,doraProCon,doraProCon,subsecDelegation,delegation,delegationExample,delegationProCon,subsecPSE,pse,pseProCon,archiSummary,subsecDLDETECT,dlDetect,dlDetectExample,dlDetectProCon,subsecNOWAIT,noWait,noWaitExample,noWaitProCon,subsec2VNOWAIT,2vNoWait,2vNoWaitProtocol,2vNoWaitProCon,secEval,testSetUp,benchmarks,readThroughput,readThroughputMultiSite,updateThroughput,readYCSB,updateYCSB,updateYCSBTheta,mixedYCSB,discussion,conclusion
%\includeonlyframes{title,toc,end,secCC,subsecSILO,silo,siloProCon}

\begin{document}

\newlength{\lockwidth}
\settowidth{\lockwidth}{\lstinline{X} $\left(t_2\right)$}
\newlength{\datawidth}
\settowidth{\datawidth}{$x_1'''$}

%----------------------------------------------------------------------------------------
%	TITLE PAGE
%----------------------------------------------------------------------------------------

\title[Analyzing the Impact of System Architecture on the Scalability of OLTP Engines for High--Contention Workloads \textit{by R. Appuswamy et al.}]{Analyzing the Impact of System Architecture on the Scalability of OLTP Engines for High--Contention Workloads \newline\textit{by R. Appuswamy, A. Anadiotis, D. Porobic, M. Iman, A. Ailamaki}} % The short title appears at the bottom of every slide, the full title is only on the title page

\addauthor{Max Gilbert}{m\_gilbert13@cs.uni-kl.de}
\setauthors

\institute[Technische Universität Kaiserslautern] % Your institution as it will appear on the bottom of every slide, may be shorthand to save space
{
Lehrgebiet Informationssysteme \\ \medskip
Technische Universität Kaiserslautern % Your institution for the title page
}
\date{\specificdate{2018}{7}{16}} % Date, can be changed to a custom date

\tikzset{%
	invisible/.style = {opacity = 0},
	visible on/.style = {alt = {#1{}{invisible, prefix after command = {\pgfextra{\tikzset{every label/.style = {invisible}}}}}}},
	alt/.code args = {<#1>#2#3}{\alt<#1>{\pgfkeysalso{#2}}{\pgfkeysalso{#3}}},
	onslide/.code args={<#1>#2}{\only<#1>{\pgfkeysalso{#2}}},
}

\pgfplotsset{
	select row/.style = {
    	x filter/.code = {\ifnum\coordindex = #1\else\def\pgfmathresult{}\fi}
	}
}

{%
	\setbeamertemplate{background canvas}{%
		\vbox to \paperheight{%
			\vfil\hbox to \paperwidth{%
				\hfil\begin{adjustbox}{height = .825\paperheight}%
				\TULogo[.2]%
				\end{adjustbox}\hfil%
			}\vfil%
		}%
	}%
	 \begin{frame}%
		\titlepage%
	\end{frame}%
}

{%
	\setbeamertemplate{section in toc}[circle]
	\setbeamertemplate{subsection in toc}[square]
	 \begin{frame}
		\frametitle{Table of Contents}
%		\begin{multicols}{2}
			\scriptsize
			\tableofcontents%[pausesections = true]
%		\end{multicols}%
	\end{frame}%
}

% Main-Memory-DBMS
% Problems with their measurements

%----------------------------------------------------------------------------------------
%	PRESENTATION SLIDES
%----------------------------------------------------------------------------------------

	\section[Introduction]{Introduction} \label{sec_intro}

\begin{frame}
	\sectionpage
\end{frame}

\begin{frame}
	
    \frametitle{Requirements for a DBMS}
    
    \begin{itemize}
    	\item Reliability
              \begin{itemize}
                  \item ACID Transactions
                  \item high availability
                  \item etc.
              \end{itemize}
        \item Functionality
              \begin{itemize}
                  \item simple to use programming model
                  \item simple to use API
                  \item etc.
              \end{itemize}
    \end{itemize}
    
    \begin{quote}[1em]
       	\small Performance isn't everything, but without it, everything else is nothing.
    \end{quote}
    
    \begin{itemize}
    	\item Performance
        	  \begin{itemize}
        	          \item high transaction throughput
                  \item low latency
                  \item etc.
        	  \end{itemize}
    \end{itemize}
    
\end{frame}

\begin{frame}
	\frametitle{Some Implications of those Requirements}

	\begin{itemize}
		\item	work purely in-memory when the working set completely fits in main memory
		\item proper utilization of the computational resources is required
			\begin{itemize}
				\item available CPU time (usually not the bottleneck)
				\item available hardware contexts (simultaneous threads)
				\item Cache Oblivious Algorithms (e.g. partitioning Hash-JOINs)
				\item[$\rightarrow$] Interleaved transaction execution to exploit abundant thread-level parallelism without violating the ACID properties!
				\item[$\rightarrow$] Interleaved operation execution to exploit intra-transaction parallelism!
			\end{itemize}
		\item[$\rightarrow$] physical \& logical Synchronization
		\uncover<2->{\item[\textcolor{red}{\bm{$\rightarrow$}}] \textcolor{red}{\textbf{Limits concurrency for high-contention workloads!}}}
	\end{itemize}
\end{frame}

	
	\section[DB Architectures]{Database Architectures} \label{sec:architectures}

\begin{frame}
	\vbox to .6\textheight{%
		\parbox[c][.225\textheight][c]{\linewidth}{%
			\sectionpage%
		}
		\centering
		\resizebox{!}{.975\architecturesheight}{%
			\begin{tabular}{>{\raggedleft}m{.15\linewidth}cc>{\raggedright}m{.15\linewidth}}
				Shared Everything/ Non-Partitioned	&	\seDiagram		&	\doraDiagram	&	Data-Oriented Transaction Execution (DORA)	\\
				Delegation					&	\delegationDiagram	&	\pseDiagram	&	Partitioned Serial Execution (PSE)
			\end{tabular}%
		}%
	}
\end{frame}

\subsection[Shared Everything/\-Non-Partitioned]{Shared Everything/\-Non-Partitioned (SE/NP)}

\begin{frame}
	\vbox to .6\textheight{%
		\parbox[c][.225\textheight][c]{\linewidth}{%
			\subsectionpage%
		}
		\centering
		\only<1>{%
			\resizebox{!}{.975\architecturesheight}{%
				\begin{tabular}{>{\raggedleft}m{.15\linewidth}cc>{\raggedright}m{.15\linewidth}}
					Shared Everything/ Non-Partitioned	&	\seDiagram			&	\doraDiagram[0.5]	&	\textcolor{black!50}{Data-Oriented Transaction Execution (DORA)}	\\
					\textcolor{black!50}{Delegation}		&	\delegationDiagram[0.5]	&	\pseDiagram[0.5]	&	\textcolor{black!50}{Partitioned Serial Execution (PSE)}
				\end{tabular}%
			}%
		}%
		\only<2>{%
			\resizebox{!}{.91\architecturesheight}{%
				\seDiagram%
			}%
		}%
	}
\end{frame}

\begin{frame}
	\frametitle{Properties of SE/NP}
	
	\begin{itemize}
		\item	metadata (incl. locks) are not partitioned
		\item[$\rightarrow$]	physical synchronization (latches, atomics) required
		\item	data and indices are not partitioned
		\item[$\rightarrow$]	logical synchronization using a concurrency control protocol also required
		\item	transactions completely executed by one thread
		\item	thread-assignment depends only on load
	\end{itemize}
\end{frame}

\begin{frame}[fragile]
	\frametitle{Pros \& Cons of SE/NP}
	
	\begin{itemize}
		\visible<1->{\item[$+$]	no partitioning required (e.g. manual selection of a strategy)}
		\visible<1->{\item[$+$]	partitioning would be sensitive to the workload}
		\visible<1->{\item[$+$]	changed workloads would require repartitioning to benefit from partitioning}
		\visible<2->{\item[$-$]	each thread might access every record at arbitrary times}
			\begin{itemize}
				\visible<2->{\item[$-$]	each CPU cache may contain any part of the data \\ \bm{$\rightarrow$} cache pollution}
				\visible<2->{\item[$-$]	each CPU may access any part of the data \\ \bm{$\rightarrow$} data movement between NUMA regions}
				\visible<2->{\item[$-$]	each CPU may acquire any latch \\ \bm{$\rightarrow$} data movement between NUMA regions}
				\visible<2->{\item[$-$]	each CPU may atomically write to any semaphore \\ \bm{$\rightarrow$} hardware cache coherence overhead}
			\end{itemize}
	\end{itemize}
    
	\tikzset{%
		changes/.style = { -> , very thick, > = stealth},
		volatile/.style = {fill = red!50},
		nonvolatile/.style = {fill = green!50},
		changes/.append style = {color = red}
	}

	\begin{tikzpicture}[remember picture, overlay]
		\node[visible on = <3-13>, shape = rectangle, draw = black, fill = black!0] at (current page.center) {%
			\begin{adjustbox}{width = .85\enhancedtextwidth}%
				\begin{tikzpicture}
					\begin{axis}[xlabel = {TPC-C runtime $\left[\si{\second}\right]$},
							  xlabel near ticks,
							  xmin = 0,
							  xmax = 0.9,
							  xtick distance = {0.1},
							  scaled x ticks = false,
							  minor x tick num = 9,
							  ylabel = {TPC-C district},
							  ylabel near ticks,
							  ymin = 0,
							  ymax = 100,
							  ymode = normal,
							  scaled y ticks = false,
							  legend entries = {{Thread 0}, {Thread 1}, {Thread 2}, {Thread 3}, {Thread 4}, {Thread 5}, {Thread 6}, {Thread 7}, {Thread 8}, {Thread 9}},
							  legend style = {font = \footnotesize},
							  legend pos = outer north east,
							  width = \linewidth,
							  height = .95\textheight,
							  axis on top = true,
							  title = {Record Accesses of Conventional DB Threads (\cite{Pandis:2010})}]		
						\addplot[visible on = <4->, only marks, mark = x, draw = black] table {data/record_accesses_per_thread_conventional_0.csv};
						\addplot[visible on = <5->, only marks, mark = x, draw = cyan] table {data/record_accesses_per_thread_conventional_1.csv};
						\addplot[visible on = <6->, only marks, mark = x, draw = red] table {data/record_accesses_per_thread_conventional_2.csv};
						\addplot[visible on = <7->, only marks, mark = x, draw = green] table {data/record_accesses_per_thread_conventional_3.csv};
						\addplot[visible on = <8->, only marks, mark = x, draw = blue] table {data/record_accesses_per_thread_conventional_4.csv};
						\addplot[visible on = <9->, only marks, mark = x, draw = black!50] table {data/record_accesses_per_thread_conventional_5.csv};
						\addplot[visible on = <10->, only marks, mark = x, draw = cyan!50] table {data/record_accesses_per_thread_conventional_6.csv};
						\addplot[visible on = <11->, only marks, mark = x, draw = red!50] table {data/record_accesses_per_thread_conventional_7.csv};
						\addplot[visible on = <12->, only marks, mark = x, draw = green!50] table {data/record_accesses_per_thread_conventional_8.csv};
						\addplot[visible on = <13->, only marks, mark = x, draw = blue!50] table {data/record_accesses_per_thread_conventional_9.csv};
					\end{axis}
				\end{tikzpicture}%
			\end{adjustbox}%
    												   };
		\node[visible on = <15>, shape = rectangle, draw = black, fill = black!0] at (current page.center) {%
			\begin{adjustbox}{width = .85\enhancedtextwidth}
				\begin{tikzpicture}[]
							% Center points of the baseline of each pyramid layer (0: Peak of the pyramid; 1: Center of the baseline of the top layer; ...)
							\node[]			(0)															{};
							\node[]			(1)		[below = 1.2*1.5cm of 0.center, anchor = center]					{};
							\node[]			(2a)		[below = 1.2*(1/3)*1.25cm of 1.center, anchor = center]				{};
							\node[]			(2b)		[below = 1.2*(1/3)*1.25cm of 2a.center, anchor = center]				{};
							\node[]			(2c)		[below = 1.2*(1/3)*1.25cm of 2b.center, anchor = center]				{};
							\node[]			(3)		[below = 1.2*0.75cm of 2c.center, anchor = center]					{};
							\node[]			(4)		[below = 1.2*0.75cm of 3.center, anchor = center]					{};
							\node[]			(5)		[below = 1.2*0.75cm of 4.center, anchor = center]					{};
							\node[]			(6)		[below = 1.2*0.75cm of 5.center, anchor = center]					{};
	
							% Anchor points of the pyramid (n0: Left anchor point; n1: Right anchor point):
							\node[]			(10)		[left = (2/3)*1.5cm of 1.center, anchor = center]					{};
							\node[]			(11)		[right = (2/3)*1.5cm of 1.center, anchor = center]				{};
							\node[]			(2a0)		[left = (2/3)*(1.5cm+(1/3)*1.25cm) of 2a.center, anchor = center]	{};
							\node[]			(2a1)		[right = (2/3)*(1.5cm+(1/3)*1.25cm) of 2a.center, anchor = center]	{};
							\node[]			(2b0)		[left = (2/3)*(1.5cm+(2/3)*1.25cm) of 2b.center, anchor = center]	{};
							\node[]			(2b1)		[right = (2/3)*(1.5cm+(2/3)*1.25cm) of 2b.center, anchor = center]	{};
							\node[]			(2c0)		[left = (2/3)*(1.5cm+(3/3)*1.25cm) of 2c.center, anchor = center]	{};
							\node[]			(2c1)		[right = (2/3)*(1.5cm+(3/3)*1.25cm) of 2c.center, anchor = center]	{};
							\node[]			(30)		[left = (2/3)*3.5cm of 3.center, anchor = center]					{};
							\node[]			(31)		[right = (2/3)*3.5cm of 3.center, anchor = center]				{};
							\node[]			(40)		[left = (2/3)*4.25cm of 4.center, anchor = center]				{};
							\node[]			(41)		[right = (2/3)*4.25cm of 4.center, anchor = center]				{};
							\node[]			(50)		[left = (2/3)*5cm of 5.center, anchor = center]					{};
							\node[]			(51)		[right = (2/3)*5cm of 5.center, anchor = center]					{};
							\node[]			(60)		[left = (2/3)*5.75cm of 6.center, anchor = center]				{};
							\node[]			(61)		[right = (2/3)*5.75cm of 6.center, anchor = center]				{};
			
							% "Center" of the pyramid (shifted towards top) used for the designation of performance measures:
							\node[]			(cen)		[below = 2.5cm of 0.center]								{};
			
							% Designation of performance measures:
							\node[rotate = 61]	(cpu)		[left = 3.25cm of cen.center, anchor = center]				{\footnotesize \textit{\textbf{Capacity per CPU}}};
							\node[rotate = -61]	(acc)		[right = 4.25cm of cen.center, anchor = center]				{\footnotesize \textit{\textbf{Access Time}}};
			
							% Fills the pyramid to mark volatile and non-volatile layers:
							\fill[volatile]
								(0.center) -- (30.center) -- (31.center) -- cycle;
							\fill[nonvolatile]
								(30.center) -- (31.center) -- (61.center) -- (60.center) -- cycle;
			
							% Key the markings for volatile and non-volatile layers
							\node[draw, volatile]		(red)		[left = 4cm of 0.center, anchor = north]		{};
							\node[]					(legred)	[right = 0cm of red.east, anchor = west]		{\footnotesize Volatile};
							\node[draw, nonvolatile]		(green)	[below = 0.2cm of red.south, anchor = north]	{};
							\node[]					(leggreen)	[right = 0cm of green.east, anchor = west]		{\footnotesize Non-volatile};
	
							% Storage layer names:
							\node[]	(reg)		[above = 0cm of 1.center, anchor = south]		{\footnotesize Registers};
							\node[]	(l1)		[above = 0cm of 2a.center, anchor = south]	{\footnotesize L1 Cache};
							\node[]	(l2)		[above = 0cm of 2b.center, anchor = south]	{\footnotesize L2 Cache};
							\node[]	(l3)		[above = 0cm of 2c.center, anchor = south]	{\footnotesize L3 Cache};
							\node[]	(mem)	[above = 0.05cm of 3.center, anchor = south]	{\small Main Memory};
							\node[]	(ssd)		[above = 0.05cm of 4.center, anchor = south]	{\small Online Storage \scriptsize \textit{(Secondary S.) - SDD}};
							\node[]	(mas)	[above = 0.05cm of 5.center, anchor = south]	{\small Online Storage \scriptsize \textit{(Secondary S.) - HDD}};
							\node[]	(mem)	[above = 0.05cm of 6.center, anchor = south]	{\small Offline Storage \scriptsize \textit{(Tertiary Storage)}};
	
							% Anchor points for the brace used to mark the example data from Intel Xeon:
							\node[]	(0h)		[right = 2.75cm of 0]		{};
							\node[]	(31h)		[right = 2.75cm of 31]	{};
			
							% Draw the pyramid:
							\path[ - , thick]
								(0.center)			edge	[]	(10.center)
								(0.center)			edge[]	(11.center)
								(10.center)		edge[]	(11.center)
								(10.center)		edge	[]	(2a0.center)
								(11.center)		edge	[]	(2a1.center)
								(2a0.center)		edge	[]	(2a1.center)
								(2a0.center)		edge	[]	(2b0.center)
								(2a1.center)		edge[]	(2b1.center)
								(2b0.center)		edge[]	(2b1.center)
								(2b0.center)		edge[]	(2c0.center)
								(2b1.center)		edge[]	(2c1.center)
								(2c0.center)		edge[]	(2c1.center)
								(2c0.center)		edge[]	(30.center)
								(2c1.center)		edge[]	(31.center)
								(30.center)		edge[]	(31.center)
								(30.center)		edge[]	(40.center)
								(31.center)		edge	[]	(41.center)
								(40.center)		edge	[]	(41.center)
								(40.center)		edge[]	(50.center)
								(41.center)		edge	[]	(51.center)
								(50.center)		edge	[]	(51.center)
								(50.center)		edge	[]	(60.center)
								(51.center)		edge[]	(61.center)
								(60.center)		edge	[]	(61.center);
			
							% Draw the connections between the brace used to mark the example data from Intel Xeon and the appropriate pyramid layers:
							\path[dotted]
								(0.center)				edge		(0h.center)
								(31.center)			edge		(31h.center);
				
							% Add the capacities:
							\draw [decorate, decoration = {brace, mirror, amplitude = 3.75pt}]
								(0.center) -- (10.center) node[black, midway, xshift = -0.1cm, yshift = 0.1cm, anchor = east] {\SI{128}{\byte}};
							\draw [decorate, decoration = {brace, mirror, amplitude = 3.75pt}]
								(10.center) -- (2a0.center) node[black, midway, xshift = -0.1cm, yshift = 0.1cm, anchor = east] {\small \SI{256}{\kibi\byte}};
							\draw [decorate, decoration = {brace, mirror, amplitude = 3.75pt}]
								(2a0.center) -- (2b0.center) node[black, midway, xshift = -0.1cm, yshift = 0.1cm, anchor = east] {\small \SI{1}{\mebi\byte}};
							\draw [decorate, decoration = {brace, mirror, amplitude = 3.75pt}]
								(2b0.center) -- (2c0.center) node[black, midway, xshift = -0.1cm, yshift = 0.1cm, anchor = east] {\small \SI{8}{\mebi\byte}};
							\draw [decorate, decoration = {brace, mirror, amplitude = 3.75pt}]
								(2c0.center) -- (30.center) node[black, midway, xshift = -0.1cm, yshift = 0.1cm, anchor = east] {\SI{64}{\gibi\byte}};
							\draw [decorate, decoration = {brace, mirror, amplitude = 3.75pt}]
								(30.center) -- (40.center) node[black, midway, xshift = -0.1cm, yshift = 0.1cm, anchor = east] {\si{\tebi\byte}};
							\draw [decorate, decoration = {brace, mirror, amplitude = 3.75pt}]
								(40.center) -- (50.center) node[black, midway, xshift = -0.1cm, yshift = 0.1cm, anchor = east] {\si{\tebi\byte}--\si{\pebi\byte}};
							\draw [decorate, decoration = {brace, mirror, amplitude = 3.75pt}]
								(50.center) -- (60.center) node[black, midway, xshift = -0.1cm, yshift = 0.1cm, anchor = east] {\si{\pebi\byte}};
	
							% Add the access times:
							\draw [decorate, decoration = {brace, amplitude = 3.75pt}]
								(0.center) -- (11.center) node[black, midway, xshift = 0.1cm, yshift = 0.1cm, anchor = west] {\SI{1}{\cy} $\approx$ \SI{0.25}{\nano\second}};
							\draw [decorate, decoration = {brace, amplitude = 3.75pt}]
								(11.center) -- (2a1.center) node[black, midway, xshift = 0.1cm, yshift = 0.1cm, anchor = west] {\small \SI{4}{\cy} $\approx$ \SI{1}{\nano\second}};
							\draw [decorate, decoration = {brace, amplitude = 3.75pt}]
								(2a1.center) -- (2b1.center) node[black, midway, xshift = 0.1cm, yshift = 0.1cm, anchor = west] {\small \SI{12}{\cy} $\approx$ \SI{3}{\nano\second}};
							\draw [decorate, decoration = {brace, amplitude = 3.75pt}]
								(2b1.center) -- (2c1.center) node[black, midway, xshift = 0.1cm, yshift = 0.1cm, anchor = west] {\small \SI{42}{\cy} $\approx$ \SI{11}{\nano\second}};
							\draw [decorate, decoration = {brace, amplitude = 3.75pt}]
								(2c1.center) -- (31.center) node[black, midway, xshift = 0.1cm, yshift = 0.1cm, anchor = west] {\SI{51}{\nano\second}};
							\draw [decorate, decoration = {brace, amplitude = 3.75pt}]
								(31.center) -- (41.center) node[black, midway, xshift = 0.1cm, yshift = 0.1cm, anchor = west] {\SI{\approx 0.1}{\milli\second} \tiny \cite{samsung:ssdprimer}};
							\draw [decorate, decoration = {brace, amplitude = 3.75pt}]
								(41.center) -- (51.center) node[black, midway, xshift = 0.1cm, yshift = 0.1cm, anchor = west] {\numrange{5}{20}\si{\milli\second} \tiny \cite{EntHDD} \cite{HDD2016}};
							\draw [decorate, decoration = {brace, amplitude = 3.75pt}]
								(51.center) -- (61.center) node[black, midway, xshift = 0.1cm, yshift = 0.1cm, anchor = west] {$s - min$};
	
							% Mark the source of the performance data of the volatile layers:
							\draw [decorate, decoration = {brace, amplitude = 3.75pt}]
								(0h.center) -- (31h.center) node[black, midway, xshift = -0.625cm, yshift = 0.1cm, anchor = west] {\footnotesize \begin{tabular}{r}\textit{\textbf{Example System:}}\\\textit{Intel\textsuperscript{\tiny\textregistered} Xeon\textsuperscript{\tiny\textregistered}}\\\textit{E3-1280 v5}\\\textit{(Skylake)}\\\cite{7cpu:skylake}\end{tabular}};
				
							% Orientation of changes of performance data:
							\path[changes]
								([xshift = (-1/3)*4cm+0.25cm]60.center)			edge		node[above, rotate = 90]		{\footnotesize Price per Capacity}	([xshift = (-1/3)*14cm-0.25cm]0.center)
								([xshift = (1/3)*18cm+0.5cm]0.center)			edge		node[above, rotate = -90]		{\footnotesize Access Time}		([xshift = (1/3)*8cm]61.center);
				\end{tikzpicture}
			\end{adjustbox}
    												   };
		\node[visible on = <17>, shape = rectangle, draw = black, fill = black!0] at (current page.center) {%
			\begin{adjustbox}{width = .85\enhancedtextwidth}%
				\begin{tikzpicture}
					\begin{axis}[major x tick style = transparent,
							   ybar = 2*\pgflinewidth,
							   bar width = 14pt,
							   ymajorgrids = true,
							   ylabel = {Message Throughput $\left[\si{\kilo\messages\per\second}\right]$},
							   symbolic x coords = {{FIFO}, {POSIX Message Queues}, {Pipes}, {TCP Sockets}, {UNIX Sockets}},
							   xtick = data,
							   x tick label style = {text width = width("Message"),
											align = center},
							   scaled y ticks = false,
							   enlarge x limits = 0.1,
							   ymin = 0,
							   legend style = {at = {(.5, 1.09)},
										  anchor = north},
							   width = \linewidth,
							   height = .7\textheight,
							   title= {Throughput of Inter-Process Communication Mechanisms (\cite{Porobic:2012})}]
						\addplot[style = {blue, fill = blue, mark = none}]	coordinates {({FIFO}, 58.9078) ({POSIX Message Queues}, 61.5127) ({Pipes}, 59.832) ({TCP Sockets}, 21.2451) ({UNIX Sockets}, 66.8908)};
						\addplot[style = {red, fill = red, mark = none}]	coordinates {({FIFO}, 41.5969) ({POSIX Message Queues}, 42.6893) ({Pipes}, 40.0843) ({TCP Sockets}, 16.6391) ({UNIX Sockets}, 44.454)};

						\legend{{Local NUMA region}, {Remote NUMA region}}
					\end{axis}
				\end{tikzpicture}%
			\end{adjustbox}%
    												   };
	\end{tikzpicture}
\end{frame}

\subsection[DORA]{Data-Oriented Transaction Execution (DORA)}

\begin{frame}
	\vbox to .6\textheight{%
		\parbox[c][.225\textheight][c]{\linewidth}{%
			\subsectionpage%
		}
		\centering
		\only<1>{%
			\resizebox{!}{.975\architecturesheight}{%
				\begin{tabular}{>{\raggedleft}m{.15\linewidth}cc>{\raggedright}m{.15\linewidth}}
					\textcolor{black!50}{Shared Everything/ Non-Partitioned}	&	\seDiagram[0.5]		&	\doraDiagram		&	Data-Oriented Transaction Execution (DORA)			\\
					\textcolor{black!50}{Delegation}						&	\delegationDiagram[0.5]	&	\pseDiagram[0.5]	&	\textcolor{black!50}{Partitioned Serial Execution (PSE)}
				\end{tabular}%
			}%
		}%
		\only<2>{%
			\resizebox{!}{.91\architecturesheight}{%
				\doraDiagram%
			}%
		}%
	}
\end{frame}

\begin{frame}
	\frametitle{Properties of DORA}
	
	\begin{itemize}
		\item	metadata (incl. locks) are physically partitioned
		\item[$\rightarrow$]	no physical synchronization (latches, atomics) required
		\item	data and indices are logically partitioned
		\item[$\rightarrow$]	logical synchronization using a concurrency control protocol only locally required
		\item	threads are assigned to data
		\item	transactions migrate to threads owning the accessed data
	\end{itemize}
\end{frame}

\begin{frame}[fragile]
	\frametitle{Interactive Example}
	
	\tikzset{
		fiber/.style = {shape = rectangle},
		thread/.style = {shape = rectangle,
					anchor = north west},
		record/.style = {shape = rectangle,
					draw = black,
					anchor = west,
					scale = 2}
	}
	
	\begin{adjustbox}{width = \linewidth}
		\begin{tikzpicture}
			\node[fiber, visible on = <2-26>]							(f0)	{%
				\begin{tabular}{| r | l |}
					\multicolumn{2}{c}{\textbf{Fiber 0}}															\\	\hline
					\only<-3>{\textbf{Locks:}		&															\\	\hline}%
					\only<4-7>{\textbf{Locks:}		&	$\left(5, S\right)$											\\	\hline}%
					\only<8>{\textbf{Locks:}		&	$\left(5, S\right), \left(3, S\right)$								\\	\hline}%
					\only<9-14>{\textbf{Locks:}	&	$\left(5, S\right), \left(3, S\right), \left(2, X\right)$					\\	\hline}%
					\only<15>{\textbf{Locks:}		&	$\left(5, S\right), \left(3, S\right), \left(2, X\right),$					\\%
											&	$\left(0, S\right)$											\\	\hline}%
					\only<16>{\textbf{Locks:}		&	$\left(5, S\right), \left(3, S\right), \left(2, X\right),$					\\%
											&	$\left(0, S\right), \left(1, S\right)$								\\	\hline}%
					\only<17-20>{\textbf{Locks:}	&	$\left(5, S\right), \left(3, S\right), \left(2, X\right)$					\\	\hline}%
					\only<21-24>{\textbf{Locks:}	&	$\left(5, S\right)$											\\	\hline}%
					\only<25->{\textbf{Locks:}		&															\\	\hline}%
					\textbf{Code:}				&	\only<4->{\color{green}}\ttfamily BOT								\\
											&	\only<5->{\color{green}}\ttfamily \quad read(5);\visible<4>{ <-}			\\
											&	\only<9->{\color{green}}\ttfamily \quad read(3);\visible<8>{ <-}			\\
											&	\only<10->{\color{green}}\ttfamily \quad write(2, x);\visible<9>{ <-}		\\
											&	\only<16->{\color{green}}\ttfamily \quad read(1);\visible<15>{ <-}		\\
											&	\only<26->{\color{green}}\ttfamily \quad commit();\visible<16-25>{ <-}	\\
											&	\only<26->{\color{green}}\ttfamily EOT							\\	\hline
					\textbf{Locks:}				&															\\	\hline
				\end{tabular}%
																};
			\node[fiber, below = 17.5em of f0.north, visible on = <9-14>]	(f1)	{%
				\begin{tabular}{| r | l |}
					\multicolumn{2}{c}{\textbf{Fiber 1}}													\\	\hline
					\only<-10>{\textbf{Locks:}	&														\\	\hline}%
					\only<11>{\textbf{Locks:}	&	$\left(0, S\right)$										\\	\hline}%
					\only<12>{\textbf{Locks:}	&	$\left(0, S\right), \left(1, S\right)$							\\	\hline}%
					\only<13->{\textbf{Locks:}	&														\\	\hline}%
					\textbf{Code:}			&	\only<11->{\color{green}}\ttfamily BOT						\\
										&	\only<12->{\color{green}}\ttfamily \quad read(0);\visible<11>{ <-}	\\
										&	\only<13->{\color{green}}\ttfamily \quad read(1);\visible<12>{ <-}	\\
										&	\only<14->{\color{green}}\ttfamily \quad commit();\visible<13>{ <-}	\\
										&	\only<14->{\color{green}}\ttfamily EOT						\\	\hline
				\end{tabular}%
																};

			\node[above right = 8em and 4em of f0.north east, thread]	(t0)	{%
				\begin{tabular}{| r | p{\fiberwidth} |}
					\multicolumn{2}{c}{\textbf{Thread 0}}									\\	\hline
					\only<-8>{\textbf{Fibers:}		&	\textit{idle}							\\	\hline}%
					\only<9>{\textbf{Fibers:}		&	Fiber 1 \textit{created}				\\	\hline}%
					\only<10>{\textbf{Fibers:}		&	Fiber 1 \textit{waiting}				\\	\hline}%
					\only<11>{\textbf{Fibers:}		&	Fiber 1 \textit{running}				\\%
											&	Fiber 0 \textit{suspended}				\\	\hline}%
					\only<12>{\textbf{Fibers:}		&	Fiber 1 \textit{running}				\\%
											&	Fiber 0 \textit{waiting}				\\	\hline}%
					\only<13>{\textbf{Fibers:}		&	Fiber 1 \textit{committing}				\\%
											&	Fiber 0 \textit{waiting}				\\	\hline}%
					\only<14>{\textbf{Fibers:}		&	Fiber 1 \textit{terminated}				\\%
											&	Fiber 0 \textit{waiting}				\\	\hline}%
					\only<15-16>{\textbf{Fibers:}	&	Fiber 0 \textit{running}				\\	\hline}%
					\only<17>{\textbf{Fibers:}		&	Fiber 0 \textit{committing}				\\	\hline}%
					\only<18>{\textbf{Fibers:}		&	Fiber 0 \textit{suspended}				\\	\hline}%
					\only<19->{\textbf{Fibers:}		&	\textit{idle}							\\	\hline}%
					\only<-10>{\textbf{Locks:}		&									\\	\hline}%
					\only<11>{\textbf{Locks:}		&	$\left(0, S\right)_1$					\\	\hline}%
					\only<12>{\textbf{Locks:}		&	$\left(0, S\right)_1, \left(1, S\right)_1$	\\	\hline}%
					\only<13-14>{\textbf{Locks:}	&									\\	\hline}%
					\only<15>{\textbf{Locks:}		&	$\left(0, S\right)_0$					\\	\hline}%
					\only<16>{\textbf{Locks:}		&	$\left(0, S\right)_0, \left(1, S\right)_0$	\\	\hline}%
					\only<17->{\textbf{Locks:}		&									\\	\hline}%
					\textbf{Partition:}			&	$0-1, ...$							\\	\hline
					
				\end{tabular}%
										};
			\node[right = 1em of t0.north east, thread]	(t1)	{%
				\begin{tabular}{| r | p{\fiberwidth} |}
					\multicolumn{2}{c}{\textbf{Thread 1}}									\\	\hline
					\only<-5>{\textbf{Fibers:}		&	\textit{idle}							\\	\hline}%
					\only<6>{\textbf{Fibers:}		&	Fiber 0 \textit{suspended}				\\	\hline}%
					\only<7>{\textbf{Fibers:}		&	Fiber 0 \textit{waiting}				\\	\hline}%
					\only<8-9>{\textbf{Fibers:}		&	Fiber 0 \textit{running}				\\	\hline}%
					\only<10>{\textbf{Fibers:}		&	Fiber 0 \textit{suspended}				\\	\hline}%
					\only<11-18>{\textbf{Fibers:}	&	\textit{idle}							\\	\hline}%
					\only<19>{\textbf{Fibers:}		&	Fiber 0 \textit{suspended}				\\	\hline}%
					\only<20>{\textbf{Fibers:}		&	Fiber 0 \textit{waiting}				\\	\hline}%
					\only<21>{\textbf{Fibers:}		&	Fiber 0 \textit{committing}				\\	\hline}%
					\only<22>{\textbf{Fibers:}		&	Fiber 0 \textit{suspended}				\\	\hline}%
					\only<23->{\textbf{Fibers:}		&	\textit{idle}							\\	\hline}%
					\only<-7>{\textbf{Locks:}		&									\\	\hline}%
					\only<8>{\textbf{Locks:}		&	$\left(3, S\right)_0$					\\	\hline}%
					\only<9-20>{\textbf{Locks:}	&	$\left(3, S\right)_0, \left(2, X\right)_0$	\\	\hline}%
					\only<21->{\textbf{Locks:}		&									\\	\hline}%
					\textbf{Partition:}			&	$2-3, ...$							\\	\hline
					
				\end{tabular}%
										};
			\node[right = 1em of t1.north east, thread]	(t2)	{%
				\begin{tabular}{| r | p{\fiberwidth} |}
					\multicolumn{2}{c}{\textbf{Thread 2}}						\\	\hline
					\only<-1>{\textbf{Fibers:}		&	\textit{idle}				\\	\hline}%
					\only<2>{\textbf{Fibers:}		&	Fiber 0 \textit{created}	\\	\hline}%
					\only<3>{\textbf{Fibers:}		&	Fiber 0 \textit{waiting}	\\	\hline}%
					\only<4>{\textbf{Fibers:}		&	Fiber 0 \textit{running}	\\	\hline}%
					\only<5>{\textbf{Fibers:}		&	Fiber 0 \textit{suspended}	\\	\hline}%
					\only<6-22>{\textbf{Fibers:}	&	\textit{idle}				\\	\hline}%
					\only<23>{\textbf{Fibers:}		&	Fiber 0 \textit{suspended}	\\	\hline}%
					\only<24>{\textbf{Fibers:}		&	Fiber 0 \textit{waiting}	\\	\hline}%
					\only<25>{\textbf{Fibers:}		&	Fiber 0 \textit{committing}	\\	\hline}%
					\only<26>{\textbf{Fibers:}		&	Fiber 0 \textit{terminated}	\\	\hline}%
					\only<27->{\textbf{Fibers:}		&	\textit{idle}				\\	\hline}%
					\only<-3>{\textbf{Locks:}		&						\\	\hline}%
					\only<4-24>{\textbf{Locks:}	&	$\left(5, S\right)_0$		\\	\hline}%
					\only<25->{\textbf{Locks:}		&						\\	\hline}%
					\textbf{Partition:}			&	$4-5, ...$				\\	\hline
					
				\end{tabular}%
										};
			
			\node[right = 20em of f1, record]	(r0)		{0};
			\node[right = 0em of r0.east, record]	(r1)		{1};
			\node[right = 0em of r1.east, record]	(r2)		{2};
			\node[right = 0em of r2.east, record]	(r3)		{3};
			\node[right = 0em of r3.east, record]	(r4)		{4};
			\node[right = 0em of r4.east, record]	(r5)		{5};
			\node[below = 1em of r2.south east, scale = 2]	{...};
			
			\draw[ -> , visible on = <2>]		(f0)		edge[bend right = 15]	node[below]		{create \textbf{Fiber 0}}	(t2);
			\draw[ -> , visible on = <4>]		(t2)		edge[bend right = 15]	node[right]		{\ttfamily read}			(r5);
			\draw[ -> , visible on = <6>]		(t2)		edge[bend left = 45]		node[below]		{migrate \textbf{Fiber 0}}	(t1);
			\draw[ -> , visible on = <8>]		(t1)		edge[]				node[right]		{\ttfamily read}			(r3);
			\draw[ -> , visible on = <9>]		(t1)		edge[]				node[right]		{\ttfamily write(x)}		(r2);
			\draw[ -> , visible on = <9>]		(f1)		edge[bend right = 15]	node[below right]	{create \textbf{Fiber 1}}	(t0);
			\draw[ -> , visible on = <11>]		(t1)		edge[bend left = 45]		node[below]		{migrate \textbf{Fiber 0}}	(t0);
			\draw[ -> , visible on = <11>]		(t0)		edge[bend left = 15]		node[left]			{\ttfamily read}			(r0);
			\draw[ -> , visible on = <12>]		(t0)		edge[bend left = 15]		node[left]			{\ttfamily read}			(r1);
			\draw[ -> , visible on = <15>]		(t0)		edge[bend left = 15]		node[left]			{\ttfamily read}			(r0);
			\draw[ -> , visible on = <16>]		(t0)		edge[bend left = 15]		node[left]			{\ttfamily read}			(r1);
			\draw[ -> , visible on = <19>]		(t0)		edge[bend right = 45]	node[below]		{migrate \textbf{Fiber 0}}	(t1);
			\draw[ -> , visible on = <23>]		(t1)		edge[bend right = 45]	node[below]		{migrate \textbf{Fiber 0}}	(t2);
		\end{tikzpicture}%
	\end{adjustbox}
\end{frame}

\begin{frame}
	\frametitle{Pros of DORA}
	
	\begin{itemize}
		\item[$+$]	each thread accesses only the records of its partition
			\begin{itemize}
				\item[$+$]	each CPU cache may contain only data of its partition \\ \bm{$\rightarrow$} lower cache pollution
				\item[$+$]	each CPU may access only data of its partitions \\ \bm{$\rightarrow$} no data movement between NUMA regions (for single-CPU transactions)
				\item[$\rightarrow$]	No physical synchronization required!
			\end{itemize}
		\item[$+$]	logical partitioning allows fast repartitioning when the workload changes
		\item[$+$]	intra-transaction parallelism could be exploited for multi-site transactions
	\end{itemize}
	
	\begin{tikzpicture}[remember picture, overlay]
		\node[visible on = <2>, shape = rectangle, draw = black, fill = black!0] at (current page.center) {%
			\begin{adjustbox}{width = .85\enhancedtextwidth}%
				\begin{tikzpicture}
					\begin{axis}[xlabel = {TPC-C runtime $\left[\si{\second}\right]$},
							  xlabel near ticks,
							  xmin = 0,
							  xmax = 0.9,
							  xtick distance = {0.1},
							  scaled x ticks = false,
							  minor x tick num = 9,
							  ylabel = {TPC-C district},
							  ylabel near ticks,
							  ymin = 0,
							  ymax = 100,
							  ymode = normal,
							  scaled y ticks = false,
							  legend entries = {{Thread 0}, {Thread 1}, {Thread 2}, {Thread 3}, {Thread 4}, {Thread 5}, {Thread 6}, {Thread 7}, {Thread 8}, {Thread 9}},
							  legend style = {font = \footnotesize},
							  legend pos = outer north east,
							  width = \linewidth,
							  height = .95\textheight,
							  axis on top = true,
							  title = {Record Accesses of DORA DB Threads (\cite{Pandis:2010})}]		
						\addplot[only marks, mark = x, draw = black] table {data/record_accesses_per_thread_dora_0.csv};
						\addplot[only marks, mark = x, draw = cyan] table {data/record_accesses_per_thread_dora_1.csv};
						\addplot[only marks, mark = x, draw = red] table {data/record_accesses_per_thread_dora_2.csv};
						\addplot[only marks, mark = x, draw = green] table {data/record_accesses_per_thread_dora_3.csv};
						\addplot[only marks, mark = x, draw = blue] table {data/record_accesses_per_thread_dora_4.csv};
						\addplot[only marks, mark = x, draw = black!50] table {data/record_accesses_per_thread_dora_5.csv};
						\addplot[only marks, mark = x, draw = cyan!50] table {data/record_accesses_per_thread_dora_6.csv};
						\addplot[only marks, mark = x, draw = red!50] table {data/record_accesses_per_thread_dora_7.csv};
						\addplot[only marks, mark = x, draw = green!50] table {data/record_accesses_per_thread_dora_8.csv};
						\addplot[only marks, mark = x, draw = blue!50] table {data/record_accesses_per_thread_dora_9.csv};
					\end{axis}
				\end{tikzpicture}%
			\end{adjustbox}%
		};
	\end{tikzpicture}
\end{frame}

\begin{frame}
	\frametitle{Cons of DORA}
	
	\begin{itemize}
		\item[$-$]	partitioning required (e.g. manual selection of a partitioning strategy---\textit{called routing rule})
		\item[$-$]	partitioning is sensitive to the workload
		\item[$-$]	multi-site transactions require expensive fiber-migration (probably between NUMA regions)
		\item[$-$]	accessed partitions need to be calculated during query analysis for optimal performance \\ \bm{$\rightarrow$} slower accesses with secondary index
		\item[$-$]	primary index is shared \\ \bm{$\rightarrow$} centralized latching for inserts/\-deletes still required \\ \bm{$\rightarrow$} some contention on the shared latch
		\item[$-$]	centralized deadlock detection still required (for \lstinline{DL\_DETECT})
	\end{itemize}
\end{frame}

\subsection[Delegation]{Delegation}

\begin{frame}
	\vbox to .6\textheight{%
		\parbox[c][.225\textheight][c]{\linewidth}{%
			\subsectionpage%
		}
		\centering
		\only<1>{%
			\resizebox{!}{.975\architecturesheight}{%
				\begin{tabular}{>{\raggedleft}m{.15\linewidth}cc>{\raggedright}m{.15\linewidth}}
					\textcolor{black!50}{Shared Everything/ Non-Partitioned}	&	\seDiagram[0.5]	&	\doraDiagram[0.5]	&	\textcolor{black!50}{Data-Oriented Transaction Execution (DORA)}	\\
					Delegation									&	\delegationDiagram	&	\pseDiagram[0.5]	&	\textcolor{black!50}{Partitioned Serial Execution (PSE)}
				\end{tabular}%
			}%
		}%
		\only<2>{%
			\resizebox{!}{.91\architecturesheight}{%
				\delegationDiagram
			}%
		}%
	}
\end{frame}

\subsection[Partitioned Serial Execution]{Partitioned Serial Execution (PSE)}

\begin{frame}
	\vbox to .6\textheight{%
		\parbox[c][.225\textheight][c]{\linewidth}{%
			\subsectionpage%
		}
		\centering
		\only<1>{%
			\resizebox{!}{.975\architecturesheight}{%
				\begin{tabular}{>{\raggedleft}m{.15\linewidth}cc>{\raggedright}m{.15\linewidth}}
					\textcolor{black!50}{Shared Everything/ Non-Partitioned}	&	\seDiagram[0.5]		&	\doraDiagram[0.5]	&	\textcolor{black!50}{Data-Oriented Transaction Execution (DORA)}	\\
					\textcolor{black!50}{Delegation}						&	\delegationDiagram[0.5]	&	\pseDiagram		&	Partitioned Serial Execution (PSE)
				\end{tabular}%
			}%
		}%
		\only<2>{%
			\resizebox{!}{.91\architecturesheight}{%
				\pseDiagram
			}%
		}%
	}
\end{frame}

\subsection[Summary]{Summary}

\begin{frame}
	\frametitle{Summary}

	\centering
	\begin{tabular}{|>{\raggedright}m{\widthof{DORA}}|>{\raggedright}m{\widthof{Message}}|>{\raggedright}m{\widthof{thread-to-data}}|>{\raggedright}m{\widthof{\textbf{Synchro-}}}|>{\raggedright}m{\widthof{Transaction}}|}
																																																																												\hline
		\multirow{2}{*}{\centering\parbox[c]{\widthof{DORA}}{\centering\vspace{.75\baselineskip}\textbf{Ar\-chi\-tec\-ture}}}	&	\multicolumn{2}{c|}{\hspace{0pt}\visible<2->{\textbf{Process Management}}}											&	\multicolumn{2}{c|}{\parbox[c]{\widthof{\textbf{Transactional Storage}}}{\centering\vspace{.1\baselineskip}\hspace{0pt}\visible<4->{\textbf{Transactional Storage Management}}\vspace{.1\baselineskip}}} 	\\	\cline{2-5}
																								&	\hspace{0pt}\visible<2->{\centering\textbf{Pa\-ral\-lel\-ism}}	&	\hspace{0pt}\visible<3->{\centering\textbf{Thread Assignment}}	&	\hspace{0pt}\visible<4->{\centering\textbf{Logical Synchronization}}	&	\hspace{0pt}\visible<5->{\centering\textbf{Physical Synchronization}}													\\	\hline
		SE/NP																					&	\hspace{0pt}\visible<2->{Shared Memory}	 				& 	\hspace{0pt}\visible<3->{thread-to-txn}	 				&	\hspace{0pt}\visible<4->{CC Protocols}	 					& 	\hspace{0pt}\visible<5->{latch/\-atomics}																		\\	\hline
		PSE																						&	\hspace{0pt}\visible<2->{Shared Nothing}	 				& 	\hspace{0pt}\visible<3->{thread-to-txn}	 				& 	\hspace{0pt}\visible<4->{Partition Lock	}						& 	\hspace{0pt}\visible<5->{partition lock}																		\\	\hline
		De\-le\-ga\-tion																				&	\hspace{0pt}\visible<2->{Message Passing}	 			& 	\hspace{0pt}\visible<3->{thread-to-txn}	 				& 	\hspace{0pt}\visible<4->{CC Protocols}	 					& 	\hspace{0pt}\visible<5->{Message Passing}																	\\	\hline
		DORA																					&	\hspace{0pt}\visible<2->{Shared Memory}					& 	\hspace{0pt}\visible<3->{thread-to-data}					& 	\hspace{0pt}\visible<4->{CC Protocols}	 					& 	\hspace{0pt}\visible<5->{Transaction Migration}																	\\	\hline
	\end{tabular}
\end{frame}


	\section[CC Algorithms]{Concurrency Control Algorithms} \label{sec:cc}

\begin{frame}
	\vbox to .6\textheight{%
		\parbox[c][.225\textheight][c]{\linewidth}{%
			\sectionpage%
		}
		\centering
		\resizebox{\linewidth}{!}{%
			\begin{tabular}{c c c c}
				\dlDetectDiagram		&	\noWaitDiagram	&	\twoVNoWaitDiagram	&	\siloDiagram	\\
				\lstinline{DL\_DETECT}	&	\lstinline{NO\_WAIT}	&	\lstinline{2V\_NO\_WAIT}	&	\lstinline{SILO}	\\
				(2PL)				&	(No-Waiting-2PL)	&	(Two-Version-2PL)		&	(OCC)
			\end{tabular}%
		}%
	}
\end{frame}

\subsection[\lstinline{DL\_DETECT}]{\lstinline{DL\_DETECT} (2PL)}

\begin{frame}
	\vbox to .6\textheight{%
		\parbox[c][.225\textheight][c]{\linewidth}{%
			\subsectionpage%
		}
		\centering
		\resizebox{\linewidth}{!}{%
			\begin{tabular}{c c c c}
				\dlDetectDiagram		&	\noWaitDiagram[0.5]					&	\twoVNoWaitDiagram[0.5]					&	\siloDiagram[0.5]				\\
				\lstinline{DL\_DETECT}	&	\textcolor{black!50}{\lstinline{NO\_WAIT}}	&	\textcolor{black!50}{\lstinline{2V\_NO\_WAIT}}	&	\textcolor{black!50}{\lstinline{SILO}}	\\
				(2PL)				&	\textcolor{black!50}{(No-Waiting-2PL)}	&	\textcolor{black!50}{(Two-Version-2PL)}		&	\textcolor{black!50}{(OCC)}
			\end{tabular}%
		}%
	}
\end{frame}

\begin{frame}
	\frametitle{Properties of \lstinline{DL\_DETECT} (2PL)}
	
	\begin{itemize}
		\item	pessimistic concurrency control protocol
		\item	transactions lock database objects (databases, tables, records, key ranges, etc.) before reading (shared mode $S$) or updating (exclusive mode $X$) them	\cite{Mohan:1990}
		\item	$t_0$ tries to acquire lock held by $t_1$ in compatible mode \\ \bm{$\rightarrow$} $t_0$ can immediately acquire lock as well (starvation needs to be prevented)
		\item	$t_0$ tries to acquire lock held by $t_1$ in incompatible mode \\ \bm{$\rightarrow$} $t_0$ waits until $t_1$ releases lock
		\item	deadlock detection using a repeatedly generated and analyzed wait-for graph
	\end{itemize}
	\newcommand{\plus}{\textcolor{green!15}{\LARGE\raisebox{-.0875em}{\bm{$\bullet$}}}\textcolor{green}{\kern-0.9em\bm{$\oplus$}}}
	\newcommand{\minus}{\textcolor{red!25}{\LARGE\raisebox{-.0875em}{\bm{$\bullet$}}}\textcolor{red}{\kern-0.9em\bm{$\ominus$}}}
	
	\centering
	\begin{tabular}{| c | c | c |}
															\hline
		compatibility	&	shared mode	&	exclusive mode		\\	\hline
		shared mode	&	\plus			&	\minus			\\	\hline
		exclusive mode	&	\minus		&	\minus			\\	\hline
	\end{tabular}%
\end{frame}

\begin{frame}[fragile]
	\frametitle{Interactive Example}
	
	\tikzset{
		identifier/.style = {font = \bfseries},
		transactionName/.style = {},
		operation/.style = {draw = none,
					     shape = rectangle,
					     inner sep = 0pt,
					     outer sep = 0pt,
					     label = right:{\scriptsize #1},
					     anchor = center},
		record/.style = {draw = none,
					inner sep = 0pt,
					outer sep = 0pt,
					font = \scriptsize},
		transaction/.style = {draw = black,
					       shape = circle}					       
	}
	
	\resizebox{\linewidth}{!}{%
		\begin{tikzpicture}
			\node[identifier]													(txn)				{Transactions:};
			
			\node[transactionName, below = .25em of txn]							(t1)				{$t_1$};
			\node[transactionName, left = 2em of t1]								(t0)				{$t_0$};
			\node[transactionName, right = 2em of t1]								(t2)				{$t_2$};
			
			\node[visible on = <2->, below = .5em of t0, operation = {BOT}]				(op00)			{$-$};
			\node[visible on = <3->, below = 1em of op00.center, operation = {$r_0$}]		(op01)			{$-$};
			\draw[visible on = <3->,  - ]								(op00.center)		--		(op01.center);
			\node[visible on = <8->, below = 5em of op01.center, operation = {$w_1$}]	(op02)			{$-$};
			\draw[visible on = <8->,  - ]								(op01.center)		--		(op02.center);
			\node[visible on = <11->, below = 3em of op02.center, operation = {$c$}]		(op03)			{$-$};
			\draw[visible on = <11->,  - ]								(op02.center)		--		(op03.center);
			\node[visible on = <11->, below = 1em of op03.center, operation = {$EOT$}]	(op04)			{$-$};
			\draw[visible on = <11->, - ]								(op03.center)		--		(op04.center);
			
			\node[visible on = <4->, below = 2.5em of t1, operation = {BOT}]			(op10)			{$-$};
			\node[visible on = <5->, below = 1em of op10.center, operation = {$r_0$}]		(op11)			{$-$};
			\draw[visible on = <5->,  - ]								(op10.center)		--		(op11.center);
			\node[visible on = <9->, below = 4em of op11.center, operation = {$w_1$}]		(op12)			{$-$};
			\draw[visible on = <9->, - ]									(op11.center)		--		(op12.center);
			\node[visible on = <13->, below = 5em of op12.center, operation = {$w_2$}]	(op13)			{$-$};
			\draw[visible on = <13->, - ]								(op12.center)		--		(op13.center);
			\node[visible on = <15->, below = 2em of op13.center, operation = {$c$}]		(op14)			{$-$};
			\draw[visible on = <15->, - ]								(op13.center)		--		(op14.center);
			\node[visible on = <15->, below = 1em of op14.center, operation = {$EOT$}]	(op15)			{$-$};
			\draw[visible on = <15->, - ]								(op14.center)		--		(op15.center);
			
			\node[visible on = <6->, below = 4.5em of t2, operation = {BOT}]			(op20)			{$-$};
			\node[visible on = <7->, below = 1em of op20.center, operation = {$r_0$}]		(op21)			{$-$};
			\draw[visible on = <7->,  - ]								(op20.center)		--		(op21.center);
			\node[visible on = <10->, below = 3em of op21.center, operation = {$w_1$}]	(op22)			{$-$};
			\draw[visible on = <10->, - ]								(op21.center)		--		(op22.center);
			\node[visible on = <12->, below = 3em of op22.center, operation = {$w_2$}]	(op23)			{$-$};
			\draw[visible on = <12->, - ]								(op22.center)		--		(op23.center);
			\node[visible on = <14->, below = 2em of op23.center, operation = {\color{red}$a$}]	(op24)		{$-$};
			\draw[visible on = <14->, - ]								(op23.center)		--		(op24.center);
			\node[visible on = <16->, below = 3em of op24.center, operation = {$r_0$}]	(op25)			{$-$};
			\draw[visible on = <16->, - ]								(op24.center)		--		(op25.center);
			\node[visible on = <17->, below = 1em of op25.center, operation = {$w_1$}]	(op26)			{$-$};
			\draw[visible on = <17->, - ]								(op25.center)		--		(op26.center);
			\node[visible on = <18->, below = 1em of op26.center, operation = {$w_2$}]	(op27)			{$-$};
			\draw[visible on = <18->, - ]								(op26.center)		--		(op27.center);
			\node[visible on = <19->, below = 1em of op27.center, operation = {$c$}]		(op28)			{$-$};
			\draw[visible on = <19->, - ]								(op27.center)		--		(op28.center);
			\node[visible on = <19->, below = 1em of op28.center, operation = {$EOT$}]	(op29)			{$-$};
			\draw[visible on = <19->, - ]								(op28.center)		--		(op29.center);
			
			
			
			\node[identifier, right = 15em of txn]							(lock)			{Locks:};
			
			\node[below = .5em of lock, record]							(f1)				{%
				\begin{tabular}{| r | p{\lockwidth} |}
																				\hline
					\multicolumn{2}{| c |}{\textbf{Record 1}}							\\	\hline
					\only<-7>{Current Mode:		&	\lstinline{NL}					\\	\hline}%
					\only<8-10>{Current Mode:	&	\lstinline{X} $\left(t_0\right)$		\\	\hline}%
					\only<11-14>{Current Mode:	&	\lstinline{X} $\left(t_1\right)$		\\	\hline}%
					\only<15-16>{Current Mode:	&	\lstinline{NL}					\\	\hline}%
					\only<17-18>{Current Mode:	&	\lstinline{X} $\left(t_2\right)$		\\	\hline}%
					\only<19->{Current Mode:		&	\lstinline{NL}					\\	\hline}%
					\only<-8>{Waiters:			&								\\	\hline}%
					\only<9>{Waiters:			&	$\left(\text{\lstinline{X}}, t_1\right)$	\\	\hline}%
					\only<10>{Waiters:			&	$\left(\text{\lstinline{X}}, t_1\right)$	\\%
											&	$\left(\text{\lstinline{X}}, t_2\right)$	\\	\hline}%
					\only<11-13>{Waiters:		&	$\left(\text{\lstinline{X}}, t_2\right)$	\\	\hline}%
					\only<14->{Waiters:			&								\\	\hline}%
					Data:		&	\only<-8>{$x_1$}\only<9-11>{$x_1'$}\only<12-17>{$x_1''$}\only<18->{$x_1'''$}	\\	\hline
				\end{tabular}%
			};			
			\node[left = 0em of f1.north west, anchor = north east, record]		(f0)				{%
				\begin{tabular}{| r | p{\lockwidth} |}
																			\hline
					\multicolumn{2}{| c |}{\textbf{Record 0}}						\\	\hline
					\only<-2>{Current Mode:		&	\lstinline{NL}				\\	\hline}%
					\only<3-4>{Current Mode:		&	\lstinline{S} $\left(1\right)$		\\	\hline}%
					\only<5-6>{Current Mode:		&	\lstinline{S} $\left(2\right)$		\\	\hline}%
					\only<7-10>{Current Mode:	&	\lstinline{S} $\left(3\right)$		\\	\hline}%
					\only<11-13>{Current Mode:	&	\lstinline{S} $\left(2\right)$		\\	\hline}%
					\only<14>{Current Mode:		&	\lstinline{S} $\left(1\right)$		\\	\hline}%
					\only<15>{Current Mode:		&	\lstinline{NL}				\\	\hline}%
					\only<16-18>{Current Mode:	&	\lstinline{S} $\left(1\right)$		\\	\hline}%
					\only<19->{Current Mode:		&	\lstinline{NL}				\\	\hline}%
					Waiters:			&									\\	\hline
					Data:			&	$x_0$							\\	\hline
				\end{tabular}%
			};
			\node[right = 0em of f1.north east, anchor = north west, record]		(f2)				{%
				\begin{tabular}{| r | p{\lockwidth} |}
																				\hline
					\multicolumn{2}{| c |}{\textbf{Record 2}}							\\	\hline
					\only<-11>{Current Mode:		&	\lstinline{NL}					\\	\hline}%
					\only<12-13>{Current Mode:	&	\lstinline{X} $\left(t_2\right)$		\\	\hline}%
					\only<14>{Current Mode:		&	\lstinline{X} $\left(t_1\right)$		\\	\hline}%
					\only<15-17>{Current Mode:	&	\lstinline{NL}					\\	\hline}%
					\only<18>{Current Mode:		&	\lstinline{X} $\left(t_2\right)$		\\	\hline}%
					\only<19->{Current Mode:		&	\lstinline{NL}					\\	\hline}%
					\only<-12>{Waiters:			&								\\	\hline}%
					\only<13>{Waiters:			&	$\left(\text{\lstinline{X}}, t_1\right)$	\\	\hline}%
					\only<14->{Waiters:			&								\\	\hline}%
					Data:		&	\only<-12>{$x_2$}\only<13>{$x_2'$}\only<14>{$x_2$}\only<15-18>{$x_2''$}\only<19->{$x_2'$}	\\	\hline
				\end{tabular}%
			};
			\node[draw = none, right = 0em of f2.north east, anchor = north west](fmore)			{$...$};
			
			\node[identifier, below = 10em of lock]						(wfg)				{Wait-for Graph:};
			
			\node[visible on = <4-14>, transaction, below = 1em of wfg]		(wt1)				{$t_1$};
			\node[visible on = <2-10>, transaction, left = 3em of wt1]			(wt0)				{$t_0$};
			\node[visible on = <6-18>, transaction, right = 3em of wt1]			(wt2)				{$t_2$};
			
			\draw[visible on = <9-10>, -> ]		(wt1)			edge						(wt0);
			\draw[visible on = <10>, -> ]		(wt2)			edge[bend left = 25]			(wt0);
			\draw[visible on = <11-12>, -> ]		(wt2)			edge[]					(wt1);
			\draw[visible on = <13>, -> ]		(wt2)			edge[bend left = 25]		node[below, yshift = -6pt, text = red]	{Cycle $\rightarrow$ Deadlock $\rightarrow$ Rollback a blocked Transaction}	(wt1);
			\draw[visible on = <13>, -> ]		(wt1)			edge[bend left = 25]			(wt2);
			
		\end{tikzpicture}
	}
\end{frame}

\begin{frame}
	\frametitle{Pros \& Cons of \lstinline{DL\_DETECT} (2PL)}
	
	\begin{itemize}
		\visible<1->{\item[$+$]	aborts only after deadlocks}
		\visible<2->{\item[$-$]	deadlocks are possible}
		\visible<2->{\item[$-$]	locks prevent concurrency too often (e.g. blind writes)}
		\visible<2->{\item[$-$]	calculation and analysis of wait-for graph expensive \\ \bm{$\rightarrow$} done offline \bm{$\rightarrow$} transactions deadlocked for a while}
		\visible<2->{\item[$-$]	aborts happen \\ \bm{$\rightarrow$} work done before needs to be repeated}
		\visible<2->{\item[$-$]	queue of waiters requires latching \\ \bm{$\rightarrow$} limits scalability}
		\visible<2->{\item[$-$]	even writes need to acquire latches and wait}
	\end{itemize}
\end{frame}

\subsection[\lstinline{NO\_WAIT}]{\lstinline{NO\_WAIT} (No-Waiting-2PL)}

\begin{frame}
	\vbox to .6\textheight{%
		\parbox[c][.225\textheight][c]{\linewidth}{%
			\subsectionpage%
		}
		\centering
		\resizebox{\linewidth}{!}{%
			\begin{tabular}{c c c c}
				\dlDetectDiagram[0.5]					&	\noWaitDiagram	&	\twoVNoWaitDiagram[0.5]					&	\siloDiagram[0.5]				\\
				\textcolor{black!50}{\lstinline{DL\_DETECT}}	&	\lstinline{NO\_WAIT}	&	\textcolor{black!50}{\lstinline{2V\_NO\_WAIT}}	&	\textcolor{black!50}{\lstinline{SILO}}	\\
				\textcolor{black!50}{(2PL)}					&	(No-Waiting-2PL)	&	\textcolor{black!50}{(Two-Version-2PL)}		&	\textcolor{black!50}{(OCC)}
			\end{tabular}%
		}%
	}
\end{frame}

\begin{frame}
	\frametitle{Properties of \lstinline{NO\_WAIT} (No-Waiting-2PL)}
	
	\begin{itemize}
		\item	pessimistic concurrency control protocol
		\item	transactions lock database objects (databases, tables, records, key ranges, etc.) before reading (shared mode $S$) or updating (exclusive mode $X$) them	\cite{Mohan:1990}
		\item	$t_0$ tries to acquire lock held by $t_1$ in compatible mode \\ \bm{$\rightarrow$} $t_0$ can immediately acquire lock as well (starvation needs to be prevented)
		\item	$t_0$ tries to acquire lock held by $t_1$ in incompatible mode \\ \bm{$\rightarrow$} $t_0$ aborts
	\end{itemize}
	\newcommand{\plus}{\textcolor{green!15}{\LARGE\raisebox{-.0875em}{\bm{$\bullet$}}}\textcolor{green}{\kern-0.9em\bm{$\oplus$}}}
	\newcommand{\minus}{\textcolor{red!25}{\LARGE\raisebox{-.0875em}{\bm{$\bullet$}}}\textcolor{red}{\kern-0.9em\bm{$\ominus$}}}
	
	\centering
	\begin{tabular}{| c | c | c |}
															\hline
		compatibility	&	shared mode	&	exclusive mode		\\	\hline
		shared mode	&	\plus			&	\minus			\\	\hline
		exclusive mode	&	\minus		&	\minus			\\	\hline
	\end{tabular}%
\end{frame}

\begin{frame}[fragile]
	\frametitle{Interactive Example}
	
	\tikzset{
		identifier/.style = {font = \bfseries},
		transactionName/.style = {},
		operation/.style = {draw = none,
					     shape = rectangle,
					     inner sep = 0pt,
					     outer sep = 0pt,
					     label = right:{\scriptsize #1},
					     anchor = center},
		record/.style = {draw = none,
					inner sep = 0pt,
					outer sep = 0pt,
					font = \scriptsize},
		transaction/.style = {draw = black,
					       shape = circle}					       
	}
	
	\centering
	\resizebox{.975\linewidth}{!}{%
		\begin{tikzpicture}
			\node[identifier]													(txn)				{Transactions:};
			
			\node[transactionName, below = .25em of txn]							(t1)				{$t_1$};
			\node[transactionName, left = 2em of t1]								(t0)				{$t_0$};
			\node[transactionName, right = 2em of t1]								(t2)				{$t_2$};
			
			\node[visible on = <2->, below = .5em of t0, operation = {BOT}]				(op00)			{$-$};
			\node[visible on = <3->, below = 1em of op00.center, operation = {$r_0$}]		(op01)			{$-$};
			\draw[visible on = <3->,  - ]								(op00.center)		--		(op01.center);
			\node[visible on = <8->, below = 5em of op01.center, operation = {$w_1$}]	(op02)			{$-$};
			\draw[visible on = <8->,  - ]								(op01.center)		--		(op02.center);
			\node[visible on = <11->, below = 5em of op02.center, operation = {$c$}]		(op03)			{$-$};
			\draw[visible on = <11->,  - ]								(op02.center)		--		(op03.center);
			\node[visible on = <11->, below = 1em of op03.center, operation = {$EOT$}]	(op04)			{$-$};
			\draw[visible on = <11->, - ]								(op03.center)		--		(op04.center);
			
			\node[visible on = <4->, below = 2.5em of t1, operation = {BOT}]			(op10)			{$-$};
			\node[visible on = <5->, below = 1em of op10.center, operation = {$r_0$}]		(op11)			{$-$};
			\draw[visible on = <5->,  - ]								(op10.center)		--		(op11.center);
			\node[visible on = <9->, below = 4em of op11.center, operation = {$w_1$}]		(op12)			{$-$};
			\draw[visible on = <9->, - ]									(op11.center)		--		(op12.center);
			\node[visible on = <9->, below = 1em of op12.center, operation = {\color{red}$a$}]	(op13)		{$-$};
			\draw[visible on = <9->, - ]									(op12.center)		--		(op13.center);
			\node[visible on = <13->, below = 6em of op13.center, operation = {$r_0$}]	(op14)			{$-$};
			\draw[visible on = <13->, - ]								(op13.center)		--		(op14.center);
			\node[visible on = <17->, below = 5em of op14.center, operation = {$w_1$}]	(op15)			{$-$};
			\draw[visible on = <17->, - ]								(op14.center)		--		(op15.center);
			\node[visible on = <18->, below = 1em of op15.center, operation = {$w_2$}]	(op16)			{$-$};
			\draw[visible on = <18->, - ]								(op15.center)		--		(op16.center);
			\node[visible on = <19->, below = 1em of op16.center, operation = {$c$}]		(op17)			{$-$};
			\draw[visible on = <19->, - ]								(op16.center)		--		(op17.center);
			\node[visible on = <19->, below = 1em of op17.center, operation = {$EOT$}]	(op18)			{$-$};
			\draw[visible on = <19->, - ]								(op17.center)		--		(op18.center);
			
			\node[visible on = <6->, below = 4.5em of t2, operation = {BOT}]			(op20)			{$-$};
			\node[visible on = <7->, below = 1em of op20.center, operation = {$r_0$}]		(op21)			{$-$};
			\draw[visible on = <7->,  - ]								(op20.center)		--		(op21.center);
			\node[visible on = <10->, below = 4em of op21.center, operation = {$w_1$}]	(op22)			{$-$};
			\draw[visible on = <10->, - ]								(op21.center)		--		(op22.center);
			\node[visible on = <10->, below = 1em of op22.center, operation = {\color{red}$a$}]	(op23)		{$-$};
			\draw[visible on = <10->, - ]								(op22.center)		--		(op23.center);
			\node[visible on = <12->, below = 3em of op23.center, operation = {$r_0$}]		(op24)			{$-$};
			\draw[visible on = <12->, - ]								(op23.center)		--		(op24.center);
			\node[visible on = <14->, below = 2em of op24.center, operation = {$w_1$}]	(op25)			{$-$};
			\draw[visible on = <14->, - ]								(op24.center)		--		(op25.center);
			\node[visible on = <15->, below = 1em of op25.center, operation = {$w_2$}]	(op26)			{$-$};
			\draw[visible on = <15->, - ]								(op25.center)		--		(op26.center);
			\node[visible on = <16->, below = 1em of op26.center, operation = {$c$}]		(op27)			{$-$};
			\draw[visible on = <16->, - ]								(op26.center)		--		(op27.center);
			\node[visible on = <16->, below = 1em of op27.center, operation = {$EOT$}]	(op28)			{$-$};
			\draw[visible on = <16->, - ]								(op27.center)		--		(op28.center);
			
			
			
			\node[identifier, right = 15em of txn]							(lock)			{Locks:};
			
			\node[below = .5em of lock, record]							(f1)				{%
				\begin{tabular}{| r | p{\datawidth} |}
																				\hline
					\multicolumn{2}{| c |}{\textbf{Record 1}}							\\	\hline
					\only<-7>{Current Mode:		&	\lstinline{0}					\\	\hline}%
					\only<8-10>{Current Mode:	&	\lstinline{1}					\\	\hline}%
					\only<11-13>{Current Mode:	&	\lstinline{0}					\\	\hline}%
					\only<14-15>{Current Mode:	&	\lstinline{1}					\\	\hline}%
					\only<16>{Current Mode:		&	\lstinline{0}					\\	\hline}%
					\only<17-18>{Current Mode:	&	\lstinline{1}					\\	\hline}%
					\only<19->{Current Mode:		&	\lstinline{0}					\\	\hline}%
					Data:		&	\only<-8>{$x_1$}\only<9-14>{$x_1'$}\only<15-17>{$x_1''$}\only<18->{$x_1'''$}	\\	\hline
				\end{tabular}%
			};			
			\node[left = 0em of f1.north west, anchor = north east, record]		(f0)				{%
				\begin{tabular}{| r | p{\datawidth} |}
																			\hline
					\multicolumn{2}{| c |}{\textbf{Record 0}}						\\	\hline
					\only<-2>{Current Mode:		&	\lstinline{0}				\\	\hline}%
					\only<3-4>{Current Mode:		&	\lstinline{2}				\\	\hline}%
					\only<5-6>{Current Mode:		&	\lstinline{4}				\\	\hline}%
					\only<7-8>{Current Mode:		&	\lstinline{6}				\\	\hline}%
					\only<9>{Current Mode:		&	\lstinline{4}				\\	\hline}%
					\only<10>{Current Mode:		&	\lstinline{2}				\\	\hline}%
					\only<11>{Current Mode:		&	\lstinline{0}				\\	\hline}%
					\only<12>{Current Mode:		&	\lstinline{2}				\\	\hline}%
					\only<13-15>{Current Mode:	&	\lstinline{4}				\\	\hline}%
					\only<16-18>{Current Mode:	&	\lstinline{2}				\\	\hline}%
					\only<19->{Current Mode:		&	\lstinline{0}				\\	\hline}%
					Data:			&	$x_0$							\\	\hline
				\end{tabular}%
			};
			\node[right = 0em of f1.north east, anchor = north west, record]		(f2)				{%
				\begin{tabular}{| r | p{\datawidth} |}
																				\hline
					\multicolumn{2}{| c |}{\textbf{Record 2}}							\\	\hline
					\only<-14>{Current Mode:		&	\lstinline{0}					\\	\hline}%
					\only<15>{Current Mode:		&	\lstinline{1}					\\	\hline}%
					\only<16-17>{Current Mode:	&	\lstinline{0}					\\	\hline}%
					\only<18>{Current Mode:		&	\lstinline{1}					\\	\hline}%
					\only<19->{Current Mode:		&	\lstinline{0}					\\	\hline}%
					Data:		&	\only<-15>{$x_2$}\only<16-18>{$x_2'$}\only<19->{$x_2''$}	\\	\hline
				\end{tabular}%
			};
			\node[draw = none, right = 0em of f2.north east, anchor = north west](fmore)			{$...$};
		\end{tikzpicture}
	}
\end{frame}

\begin{frame}
	\frametitle{Pros \& Cons of \lstinline{NO\_WAIT} (No-Waiting-2PL)}
	
	\begin{itemize}
		\visible<1->{\item[$+$]	deadlocks are impossible}
		\visible<1->{\item[$+$]	locks can be implemented using a semaphore and atomics \\ \bm{$\rightarrow$} scales better than latches}
		\visible<1->{\item[$+$]	no need to expensively calculate and analysis a wait-for graph}
		\visible<2->{\item[$-$]	many lock conflicts for update-intensive high-contention workloads \\ \bm{$\rightarrow$} many aborts \bm{$\rightarrow$} work done before needs to be repeated}
		\visible<2->{\item[$-$]	locks prevent concurrency too often (e.g. blind writes)}
	\end{itemize}
\end{frame}

\subsection[\lstinline{2V\_NO\_WAIT}]{\lstinline{2V\_NO\_WAIT} (Two-Version-2PL)}

\begin{frame}
	\vbox to .6\textheight{%
		\parbox[c][.225\textheight][c]{\linewidth}{%
			\subsectionpage%
		}
		\centering
		\resizebox{\linewidth}{!}{%
			\begin{tabular}{c c c c}
				\dlDetectDiagram[0.5]					&	\noWaitDiagram[0.5]					&	\twoVNoWaitDiagram	&	\siloDiagram[0.5]				\\
				\textcolor{black!50}{\lstinline{DL\_DETECT}}	&	\textcolor{black!50}{\lstinline{NO\_WAIT}}	&	\lstinline{2V\_NO\_WAIT}	&	\textcolor{black!50}{\lstinline{SILO}}	\\
				\textcolor{black!50}{(2PL)}					&	\textcolor{black!50}{(No-Waiting-2PL)}	&	(Two-Version-2PL)		&	\textcolor{black!50}{(OCC)}
			\end{tabular}%
		}%
	}
\end{frame}

\begin{frame}
	\frametitle{Properties of \lstinline{2V\_NO\_WAIT} (Two-Version-2PL) I}
	
	\begin{itemize}
		\item	multiversion pessimistic concurrency control protocol
		\item	3 phases: read, certify, write/commit
		\item	transactions lock records before reading (shared mode $S$), privately updating (exclusive mode $X$) or certifying/globally updating (certify mode $C$) them
		\item	updates happen first on a private copy \\ \bm{$\rightarrow$} committed copy can still be read by other transactions
		\item	before committing writes (replace original record with private copy) the absence of relevant conflicts needs to be certified (certification step)
	\end{itemize}
\end{frame}

\begin{frame}
	\frametitle{Properties of \lstinline{2V\_NO\_WAIT} (Two-Version-2PL) II}
	
	\begin{itemize}
		\item	$t_0$ tries to acquire lock held by $t_1$ in compatible mode \\ \bm{$\rightarrow$} $t_0$ can immediately acquire lock as well (starvation needs to be prevented)
		\item	$t_0$ tries to acquire lock held by $t_1$ in incompatible mode \\ \bm{$\rightarrow$} $t_0$ aborts
	\end{itemize}
	\newcommand{\plus}{\textcolor{green!15}{\LARGE\raisebox{-.0875em}{\bm{$\bullet$}}}\textcolor{green}{\kern-0.9em\bm{$\oplus$}}}
	\newcommand{\minus}{\textcolor{red!25}{\LARGE\raisebox{-.0875em}{\bm{$\bullet$}}}\textcolor{red}{\kern-0.9em\bm{$\ominus$}}}
	
	\centering
	\begin{tabular}{| c | c | c | c |}
																		\hline
		compatibility	&	shared mode	&	exclusive mode	&	certify mode	\\	\hline
		shared mode	&	\plus			&	\plus			&	\minus		\\	\hline
		exclusive mode	&	\plus			&	\minus		&	\minus		\\	\hline
		certify mode	&	\minus		&	\minus		&	\minus		\\	\hline
	\end{tabular}
\end{frame}

\begin{frame}
	\frametitle{Protocol I}
	
	\begin{itemize}
		\item[\textbf{read $r_i$}]	\hfill
							\begin{itemize}
								\item[$C$ acquired]	$r_i$ gets certified or committed by another transaction
															\begin{itemize}
																\item	\textcolor{red}{abort} this transaction
															\end{itemize}
								\item[$C$ not acquired]	other threads might read $r_i$ or privately update $r_i$
															\begin{itemize}
																\item	acquire $r_i$'s lock in $S$ mode
																\item	read global (committed) value of $r_i$
															\end{itemize}
							\end{itemize}
		\item[\textbf{update $r_i$}]	\hfill
							\begin{itemize}
								\item[$W$ acquired]	$r_i$ gets (privately or globally) updated by another transaction \\ \bm{$\rightarrow$} there are already two versions
															\begin{itemize}
																\item	\textcolor{red}{abort} this transaction
															\end{itemize}
								\item[$W$ not acquired]	other threads at most read $r_i$
															\begin{itemize}
																\item	acquire $r_i$'s lock in $X$ mode
																\item	create local copy of global (committed) value of $r_i$
																\item	update local copy of $r_i$
															\end{itemize}
							\end{itemize}
	\end{itemize}
\end{frame}

\begin{frame}
	\frametitle{Protocol II}
	
	\begin{itemize}
		\item[\textbf{certify $r_i$}]	\textit{(only if $r_i$ was updated)}
							\begin{itemize}
								\item[$S$ acquired]	$r_i$'s global (committed) value gets read by other transaction \\ \bm{$\rightarrow$} globally updating $r_i$ would cause non-repeatable reads
															\begin{itemize}
																\item	\textcolor{red}{abort} this transaction
															\end{itemize}
								\item[$S$ not acquired]	other threads at most read $r_i$
															\begin{itemize}
																\item	acquire $r_i$'s lock in $C$ mode
															\end{itemize}
							\end{itemize}
		\item[\textbf{commit}]	\hfill
							\begin{itemize}
								\item	\textit{(only if $r_i$ was updated)} replace global $r_i$ with updated local version
								\item	release the locks held on $r_i$
							\end{itemize}
	\end{itemize}
\end{frame}

\begin{frame}
	\frametitle{Pros \& Cons \lstinline{2V\_NO\_WAIT} (Two-Version-2PL)}
	
	\begin{itemize}
		\visible<1->{\item[$+$]	deadlocks are impossible}
		\visible<1->{\item[$+$]	transactions can read while a transaction updates a record}
		\visible<1->{\item[$+$]	locks can be implemented using a semaphore and a flag updated with atomic instructions \\ \bm{$\rightarrow$} scales better than latches}
		\visible<1->{\item[$+$]	no need to expensively calculate and analysis a wait-for graph}
		\visible<2->{\item[$-$]	update-intensive high-contention workloads result in many lock conflicts \\ \bm{$\rightarrow$} many aborts \bm{$\rightarrow$} work done before needs to be repeated}
		\visible<2->{\item[$-$]	locks still prevent concurrency too often (e.g. blind writes)}
		\visible<2->{\item[$-$]	additional steps for updates required:}
							\begin{itemize}
								\visible<2->{\item	create private copy of updated record (expensive but scalable)}
								\visible<2->{\item	certify update (cheap)}
							\end{itemize}
	\end{itemize}
\end{frame}

\subsection[\lstinline{SILO}]{\lstinline{SILO} (OCC)}

\begin{frame}
	\vbox to .6\textheight{%
		\parbox[c][.225\textheight][c]{\linewidth}{%
			\subsectionpage%
		}
		\centering
		\resizebox{\linewidth}{!}{%
			\begin{tabular}{c c c c}
				\dlDetectDiagram[0.5]					&	\noWaitDiagram[0.5]					&	\twoVNoWaitDiagram[0.5]							&	\siloDiagram			\\
				\textcolor{black!50}{\lstinline{DL\_DETECT}}	&	\textcolor{black!50}{\lstinline{NO\_WAIT}}	&	\textcolor{black!50}{\lstinline{2V\_NO\_WAIT}}	&	\lstinline{SILO}	\\
				\textcolor{black!50}{(2PL)}					&	\textcolor{black!50}{(No-Waiting-2PL)}	&	\textcolor{black!50}{(Two-Version-2PL)}				&	(OCC)
			\end{tabular}%
		}%
	}
\end{frame}

\begin{frame}
	\frametitle{Properties of \lstinline{SILO} (OCC) I}
	
	\begin{itemize}
		\item	optimistic concurrency control protocol
		\item	3 phases: read, validate, write/commit
		\item	each record contains the transaction ID (global ordered number based on epochs) from the last update
		\item	transactions perform reads and local writes during the read phase without acquiring locks
	\end{itemize}
\end{frame}

\begin{frame}
	\frametitle{Properties of \lstinline{SILO} (OCC) II}
	
	\begin{itemize}
		\item	read and write sets (records read and written by the transaction) are recorded locally
		\item	read and write sets used to validate absence of relevant conflicts
		\item	commit requires three phases: locking of updated records, verification of read set (based on TIDs), execute global writes
		\item	deletes invalidate records using updates \\ \bm{$\rightarrow$} garbage collection required
		\item	records for inserts created before validation to provide locks
	\end{itemize}
\end{frame}

\begin{frame}
	\frametitle{Pros \& Cons \lstinline{SILO} (OCC)}
	
	\begin{itemize}
		\visible<1->{\item[$+$]	deadlocks are impossible (locks acquired only in last phase \bm{$\rightarrow$} global order can be used)}
		\visible<1->{\item[$+$]	transactions can concurrently read and write}
		\visible<1->{\item[$+$]	only actual conflicts cause aborts (optimism)}
		\visible<2->{\item[$-$]	update-intensive high-contention workloads result in many invalid reads \\ \bm{$\rightarrow$} many aborts \bm{$\rightarrow$} work done before needs to be repeated}
		\visible<2->{\item[$-$]	globally sorted transaction IDs need to be generated (epochs make that cheap)}
		\visible<2->{\item[$-$]	additional steps for updating transactions required:}
							\begin{itemize}
								\visible<2->{\item	write and read sets locally (expensive but scalable)}
								\visible<2->{\item	validate reads}
							\end{itemize}
	\end{itemize}
\end{frame}


	\section[Performance Evaluation]{Performance Evaluation} \label{sec:evaluation}

 \begin{frame}
	\sectionpage
	
	\vspace{3em}
	
	\newcommand{\plus}{\textcolor{green!15}{\LARGE\raisebox{-.0875em}{\bm{$\bullet$}}}\textcolor{green}{\kern-0.9em\bm{$\oplus$}}}
	\newcommand{\minus}{\textcolor{red!25}{\LARGE\raisebox{-.0875em}{\bm{$\bullet$}}}\textcolor{red}{\kern-0.9em\bm{$\ominus$}}}
	\newcommand{\neutral}{\textcolor{black!25}{\LARGE\raisebox{-.0875em}{\bm{$\bullet$}}}\textcolor{black}{\kern-0.9em\bm{$\odot$}}}
	
	\centering
	\begin{tabular}{| c | c | c | c | c |}
																								\cline{2-5}
		\multicolumn{1}{c|}{}		&	SE/NP	&	DORA	&	Delegation	&	PSE					\\	\hline
		\lstinline{DL\_DETECT}	&	\plus		&	\plus		&	\plus			&	\multirow{4}{*}{\neutral}	\\	\cline{1-4}
		\lstinline{NO\_WAIT}		&	\plus		&	\plus		&	\plus			&						\\	\cline{1-4}
		\lstinline{2V\_NO\_WAIT}	&	\plus		&	\plus		&	\plus			&						\\	\cline{1-4}
		\lstinline{SILO}			&	\plus		&	\minus	&	\plus			&						\\	\hline
	\end{tabular}
\end{frame}

\tikzset{
	commonMark/.style = {scale = 1.5},
	legendArchitecture/.style = {mark = x},
	legendCC/.style = {color = black!80, mark options = {fill = black!40, commonMark}},
	se/.style = {color = red, mark options = {fill = red!50, commonMark}},
	dora/.style = {color = blue, mark options = {fill = blue!50, commonMark}},
	delegation/.style = {color = orange, mark options = {fill = orange!50, commonMark}},
	pse/.style = {color = ForestGreen, mark options = {fill = ForestGreen!50, commonMark}, mark = pentagon*},
	dlDetect/.style = {mark = square*},
	noWait/.style = {mark = *},
	2vNoWait/.style = {mark = oplus*},
	silo/.style = {mark = triangle*},
	abortRate/.style = {dashed, every mark/.append style = {solid, scale = 0.75}}
}

%\makeatletter%
%\ifcsundef{r@label:se}{%
	\begin{tikzpicture}[outer sep = 0pt, inner sep = 0pt]
		\pgfplotsset{
			legend image code/.code = {
				\draw[mark repeat = 2, mark phase = 2]
				plot coordinates {
					(0cm, 0cm)
					(0.15cm, 0cm)       %% default is (0.3cm,0cm)
					(0.3cm, 0cm)         %% default is (0.6cm,0cm)
				};%
			}
		}
		\begin{axis}[hide axis,
				   xmin = 10,
				   xmax = 50,
				   ymin = 0,
				   ymax = 0.4]
	 		\addlegendimage{se, legendArchitecture}
	 		\label{label:se}
	 		\addlegendimage{dlDetect, legendCC}
	 		\label{label:dlDetect}
	 		\addlegendimage{dora, legendArchitecture}
	 		\label{label:dora}
	 		\addlegendimage{noWait, legendCC}
	 		\label{label:noWait}
	 		\addlegendimage{delegation, legendArchitecture}
	 		\label{label:delegation}
	 		\addlegendimage{2vNoWait, legendCC}
	 		\label{label:2vNoWait}
	 		\addlegendimage{pse}
	 		\label{label:pse}
	 		\addlegendimage{silo, legendCC}
	 		\label{label:silo}
	 		\addlegendimage{abortRate, legendArchitecture}
	 		\label{label:abort}
	    \end{axis}
	\end{tikzpicture}%
%}{}%
%\makeatother%

 \begin{frame}
	\frametitle{Evaluation Set-Up}
	
	\begin{itemize}
		\visible<2->{\item	4x Intel Xeon E7-8890 v3 NUMA machine (72 cores @ \SI{2.5}{\giga\hertz})}
		\visible<2->{\item	\SI{32}{\kilo\byte} L1I cache and \SI{32}{\kilo\byte} L1D cache per core}
		\visible<2->{\item	\SI{256}{\kilo\byte} L2 cache per core}
		\visible<2->{\item	\SI{45}{\mega\byte} L3 cache per CPU}
		\visible<3->{\item	\SI{512}{\giga\byte} DDR4 RAM}
		\visible<4->{\item	hyperThreading not used}
		\visible<4->{\item	threads pinned to physical cores}
		\visible<4->{\item	sockets filled sequentially with threads}
	\end{itemize}
\end{frame}

 \begin{frame}
	\frametitle{Benchmarks}
	
	\begin{block}{Microbenchmark}
		\vspace{-.5em}
		\begin{itemize}
			\item	\SI{13}{\giga\byte} database
			\item	Hot Set: $16$ records \emph{distributed to 16 partitions}
			\item	Cold Set: $\num{100000000} - 16$ records
			\item	Txn: 2 accesses to Hot Set \& 8 accesses to \emph{(thread-local)} Cold Set
		\end{itemize}
	\end{block}
	\vspace{-.25em}	
	\begin{block}{Yahoo! Cloud Serving Benchmark (YCSB)}
		\vspace{-.5em}
		\begin{itemize}
			\item	\SI{20}{\giga\byte} database
			\item	\num{20000000} records
			\item	Txn: reads/updates 16 records following zipfian distribution according to parameter $\Theta$
		\end{itemize}
	\end{block}
\end{frame}

\subsection[Read-Only Workload]{Read-Only Workload} \label{subsec:read}

 \begin{frame}
	\frametitle{Read-Only Microbenchmark}

	\hspace{-2em}%
	\begin{columns}[T]
		\begin{column}{.667\enhancedtextwidth}
			\centering
 			\resizebox{\linewidth}{!}{%
 				\begin{tikzpicture}
 					\begin{axis}[xlabel = {Number of Threads},
 							   xlabel near ticks,
 							   xmin = 0,
 							   xmax = 72,
 							   xmode = normal,
 							   scaled x ticks = false,
 							   ylabel = {Transaction Throughput $\left[\si{\mega\transactions\per\second}\right]$},
 							   ylabel near ticks,
 							   ymin = 0,
 							   ymax = 20,
 							   ymode = normal,
 							   scaled y ticks = false,
 							   grid = major,
 							   scale only axis,
 							   legend pos = north west,
 							   legend columns = 2,
 							   legend style = {font = \small},
 							   width = \linewidth,
 							   height = .8\textheight]
 						\addlegendimage{se, legendArchitecture}
 						\addlegendentry{SE/NP}
 						\addlegendimage{dlDetect, legendCC}
 						\addlegendentry{\lstinline{DL_DETECT}}
 						\addlegendimage{dora, legendArchitecture}
 						\addlegendentry{DORA}
 						\addlegendimage{noWait, legendCC}
 						\addlegendentry{\lstinline{NO_WAIT}}
 						\addlegendimage{delegation, legendArchitecture}
 						\addlegendentry{Delegation}
 						\addlegendimage{2vNoWait, legendCC}
 						\addlegendentry{\lstinline{2V_NO_WAIT}}
 						\addlegendimage{pse}
 						\addlegendentry{PSE}
 						\addlegendimage{silo, legendCC}
 						\addlegendentry{\lstinline{SILO}}
 		
 						\addplot[se, dlDetect, visible on = <2->]		table[x = Threads, y = SE]				{./data/read_only_dl_detect_2a.csv};
 						\addplot[delegation, dlDetect, visible on = <2->]	table[x = Threads, y = Delegation]		{./data/read_only_dl_detect_2a.csv};
 						\addplot[dora, dlDetect, visible on = <2->]		table[x = Threads, y = DORA]			{./data/read_only_dl_detect_2a.csv};
 						\addplot[se, noWait, visible on = <5->]		table[x = Threads, y = SE]				{./data/read_only_no_wait_2b.csv};
 						\addplot[delegation, noWait, visible on = <8->]	table[x = Threads, y = Delegation]		{./data/read_only_no_wait_2b.csv};
 						\addplot[dora, noWait, visible on = <8->]		table[x = Threads, y = DORA]			{./data/read_only_no_wait_2b.csv};
 						\addplot[se, 2vNoWait, visible on = <10->]		table[x = Threads, y = 2V_NO_WAIT]	{./data/read_only_best_architecture_2c.csv};
 						\addplot[se, silo, visible on = <12->]			table[x = Threads, y = SILO]			{./data/read_only_best_architecture_2c.csv};
 						\addplot[pse, visible on = <14->]			table[x = Threads, y = PSE]			{./data/read_only_best_architecture_2c.csv};
 					\end{axis}
 				\end{tikzpicture}%
 			}%
		\end{column}
		\begin{column}{.333\enhancedtextwidth}
			\footnotesize
			\begin{block}{Observations}
				\only<3->{\begin{itemize}
					\only<3-8>{\item	\ref{label:dora}/\ref{label:delegation} suffer from remote data access overhead}
					\only<4-10>{\item	\ref{label:se} suffers from latch contention on locks}
					\only<6-12>{\item	atomics of \ref{label:noWait} outperform latches of \ref{label:dlDetect}}
					\only<7-14>{\item	scaling of \ref{label:noWait} limited by hardware cache coherence mechanism}
					\only<9->{\item	\ref{label:dora}/\ref{label:delegation} suffer more from remote data accesses than \ref{label:se} suffers from cache coherence}
					\only<11->{\item	\ref{label:2vNoWait} and \ref{label:noWait} perform identical for read-only}
					\only<13->{\item	\ref{label:silo} behaves identical for \ref{label:se} and \ref{label:delegation} for read-only}
					\only<15->{\item	coarse-grained partition locking of \ref{label:pse} doesn't scale due to multi-site workload}
				\end{itemize}}
			\end{block}
		\end{column}
	\end{columns}
\end{frame}

 \begin{frame}
	\frametitle{Multi-Site Read-Only Microbenchmark}

	\hspace{-2em}%
	\begin{columns}[T]
		\begin{column}{.667\enhancedtextwidth}
			\centering
 			\resizebox{\linewidth}{!}{%
 				\begin{tikzpicture}
 					\begin{axis}[xlabel = {Number of Remote Records},
 							   xlabel near ticks,
 							   xmin = 0,
 							   xmax = 10,
 							   xmode = normal,
 							   scaled x ticks = false,
 							   ylabel = {Transaction Throughput $\left[\si{\mega\transactions\per\second}\right]$},
 							   ylabel near ticks,
 							   ymin = 0,
 							   ymax = 20,
 							   ymode = normal,
 							   scaled y ticks = false,
 							   grid = major,
 							   scale only axis,
 							   legend pos = north east,
 							   legend columns = 2,
 							   legend style = {font = \small},
 							   width = \linewidth,
 							   height = .8\textheight]
 						\addlegendimage{se, legendArchitecture}
 						\addlegendentry{SE/NP}
 						\addlegendimage{dlDetect, legendCC}
 						\addlegendentry{\lstinline{DL_DETECT}}
 						\addlegendimage{dora, legendArchitecture}
 						\addlegendentry{DORA}
 						\addlegendimage{noWait, legendCC}
 						\addlegendentry{\lstinline{NO_WAIT}}
 						\addlegendimage{delegation, legendArchitecture}
 						\addlegendentry{Delegation}
 						\addlegendimage{2vNoWait, legendCC}
 						\addlegendentry{\lstinline{2V_NO_WAIT}}
 						\addlegendimage{pse}
 						\addlegendentry{PSE}
 						\addlegendimage{silo, legendCC}
 						\addlegendentry{\lstinline{SILO}}
 		
 						\addplot[se, noWait, visible on = <2->]		table[x = RemoteRecords, y = SE]		{./data/read_only_multi_site_transactions_2d.csv};
 						\addplot[delegation, noWait, visible on = <4->]	table[x = RemoteRecords, y = Delegation]	{./data/read_only_multi_site_transactions_2d.csv};
 						\addplot[dora, noWait, visible on = <4->]		table[x = RemoteRecords, y = DORA]	{./data/read_only_multi_site_transactions_2d.csv};
 						\addplot[pse, visible on = <7->]				table[x = RemoteRecords, y = PSE]		{./data/read_only_multi_site_transactions_2d.csv};
 					\end{axis}
 				\end{tikzpicture}%
 			}%
		\end{column}
		\begin{column}{.333\enhancedtextwidth}
			\footnotesize
			\begin{block}{Observations}
				\only<3->{\begin{itemize}
					\only<3-7>{\item	\ref{label:se} doesn't know remote records}
					\only<5-8>{\item	\ref{label:dora}/\ref{label:delegation} outperform \ref{label:se} for $0$ remote records due to lower cache coherence activity}
					\only<6->{\item	\ref{label:dora}/\ref{label:delegation} suffer from remote data access overhead for $>0$ remote records}
					\only<8->{\item	coarse-grained partition locking of \ref{label:pse} imposes nearly no overhead for suitable partitioning}
					\only<9->{\item	coarse-grained partition locking of \ref{label:pse} limits the concurrency drastically for $>0$ remote records}
				\end{itemize}}
			\end{block}
		\end{column}
	\end{columns}
\end{frame}

\subsection[Update-Only Workload]{Update-Only Workload} \label{subsec:update}

 \begin{frame}
	\frametitle{Update-Only Microbenchmark}

	\hspace{-2em}%
	\begin{columns}[T]
		\begin{column}{.667\enhancedtextwidth}
			\centering
 			\resizebox{\linewidth}{!}{%
 				\begin{tikzpicture}
 					\pgfplotsset{set layers}
 					\begin{axis}[xlabel = {Number of Threads},
 							   xlabel near ticks,
 							   xmin = 0,
 							   xmax = 72,
 							   xmode = normal,
 							   scaled x ticks = false,
 							   ylabel = {Transaction Throughput $\left[\si{\mega\transactions\per\second}\right]$},
 							   ylabel near ticks,
 							   ymin = 0,
 							   ymax = 1,
 							   ymode = normal,
 							   scaled y ticks = false,
 							   axis y line* = left,
 							   grid = major,
 							   scale only axis,
 							   width = \linewidth,
 							   height = .9\textheight]
 						\addlegendimage{se, legendArchitecture}
 						\label{plot:se}
 						\addlegendimage{dlDetect, legendCC}
 						\label{plot:dlDetect}
 						\addlegendimage{dora, legendArchitecture}
 						\label{plot:dora}
 						\addlegendimage{noWait, legendCC}
 						\label{plot:noWait}
 						\addlegendimage{delegation, legendArchitecture}
 						\label{plot:delegation}
 						\addlegendimage{2vNoWait, legendCC}
 						\label{plot:2vNoWait}
 						\addlegendimage{pse}
 						\label{plot:pse}
 						\addlegendimage{silo, legendCC}
 						\label{plot:silo}
 						\addlegendimage{abortRate, legendArchitecture}
 						\label{plot:abort}
 		
 						\addplot[se, dlDetect, visible on = <4->]		table[x = Threads, y = SE]				{./data/update_dl_detect_3a.csv};
 						\addplot[delegation, dlDetect, visible on = <4->]	table[x = Threads, y = Delegation]		{./data/update_dl_detect_3a.csv};
 						\addplot[dora, dlDetect, visible on = <4->]		table[x = Threads, y = DORA]			{./data/update_dl_detect_3a.csv};
 						\addplot[se, noWait, visible on = <12->]		table[x = Threads, y = SE]				{./data/update_no_wait_4a.csv};
 						\addplot[delegation, noWait, visible on = <12->]	table[x = Threads, y = Delegation]		{./data/update_no_wait_4a.csv};
 						\addplot[dora, noWait, visible on = <12->]		table[x = Threads, y = DORA]			{./data/update_no_wait_4a.csv};
 						\addplot[se, 2vNoWait, visible on = <15->]		table[x = Threads, y = 2V_NO_WAIT]	{./data/update_best_architecture_5a.csv};
 						\addplot[se, silo, visible on = <19->]			table[x = Threads, y = SE]				{./data/update_silo_4b.csv};
 						\addplot[delegation, silo, visible on = <19->]	table[x = Threads, y = Delegation]		{./data/update_silo_4b.csv};
 						\addplot[pse, visible on = <21->]			table[x = Threads, y = PSE]			{./data/update_best_architecture_5a.csv};
 					\end{axis}
 					
 					\begin{axis}[xlabel = {Number of Threads},
 							   xlabel near ticks,
 							   xmin = 0,
 							   xmax = 72,
 							   xmode = normal,
 							   scaled x ticks = false,
 							   ylabel = {Abort Rate},
 							   ylabel near ticks,
 							   ymin = 0,
							   ymax = 1.5,
 							   ymode = normal,
 							   scaled y ticks = false,
							   ytick distance = 0.5,
 							   axis y line* = right,
 							   scale only axis,
 							   onslide = <8->{/pgfplots/ticks = {none}},
							   width = \linewidth,
 							   height = .9\textheight,
							   visible on = <-7>]
 						\addplot[se, dlDetect, abortRate, visible on = <2-7>]			table[x = Threads, y = SE]				{./data/update_dl_detect_abort_rate_3b.csv};
 						\addplot[delegation, dlDetect, abortRate, visible on = <2-7>]	table[x = Threads, y = Delegation]		{./data/update_dl_detect_abort_rate_3b.csv};
 						\addplot[dora, dlDetect, abortRate, visible on = <2-7>]		table[x = Threads, y = DORA]			{./data/update_dl_detect_abort_rate_3b.csv};
 					\end{axis}
 					
 					\begin{axis}[xlabel = {Number of Threads},
 							   xlabel near ticks,
 							   xmin = 0,
 							   xmax = 72,
 							   xmode = normal,
 							   scaled x ticks = false,
 							   ylabel = {Abort Rate},
 							   ylabel near ticks,
 							   ymin = 0,
							   ymax = 20,
 							   ymode = normal,
 							   scaled y ticks = false,
							   yticklabel = {\axisdefaultticklabel\hphantom{$.$}},
							   extra y tick label = {\hphantom{0.5}},
 							   axis y line* = right,
 							   scale only axis,
 							   onslide = <-7>{/pgfplots/ticks = {none}},
 							   width = \linewidth,
 							   height = .9\textheight,
							   visible on = <8->]
 						\addplot[se, dlDetect, abortRate, visible on = <8->]			table[x = Threads, y = SE]				{./data/update_dl_detect_abort_rate_3b.csv};
 						\addplot[delegation, dlDetect, abortRate, visible on = <8->]	table[x = Threads, y = Delegation]		{./data/update_dl_detect_abort_rate_3b.csv};
 						\addplot[dora, dlDetect, abortRate, visible on = <8->]			table[x = Threads, y = DORA]			{./data/update_dl_detect_abort_rate_3b.csv};
 						\addplot[se, noWait, abortRate, visible on = <9->]			table[x = Threads, y = SE]				{./data/update_no_wait_abort_rate_4c.csv};
 						\addplot[delegation, noWait, abortRate, visible on = <9->]		table[x = Threads, y = Delegation]		{./data/update_no_wait_abort_rate_4c.csv};
 						\addplot[dora, noWait, abortRate, visible on = <9->]			table[x = Threads, y = DORA]			{./data/update_no_wait_abort_rate_4c.csv};
 						\addplot[se, 2vNoWait, abortRate, visible on = <15->]		table[x = Threads, y = 2V_NO_WAIT]	{./data/update_best_architecture_abort_rate_5b.csv};
 						\addplot[se, silo, abortRate, visible on = <17->]				table[x = Threads, y = SILO]			{./data/update_best_architecture_abort_rate_5b.csv};
 					\end{axis}
 				\end{tikzpicture}%
 			}%
		\end{column}
		\begin{column}{.333\enhancedtextwidth}
			\vspace{-2em}
  			\resizebox{\linewidth}{!}{%
				\begin{tikzpicture}
					\node[draw, inner ysep = .35em, inner xsep = .25em]	{%
 						\begin{tabular}{@{}c@{}}
 							\begin{tikzpicture}[inner sep = .1em]
 								\matrix[matrix of nodes,
 								            anchor = center,
 								            ampersand replacement = \&] {
 									\ref{plot:se}		\&	SE/NP		\&	\ref{plot:dlDetect}	\&	\lstinline{DL_DETECT}	\\
 									\ref{plot:dora}		\&	DORA		\&	\ref{plot:noWait}	\&	\lstinline{NO_WAIT}		\\
	 								\ref{plot:delegation}	\&	Delegation	\&	\ref{plot:2vNoWait}	\&	\lstinline{2V_NO_WAIT}	\\
 									\ref{plot:pse}		\&	PSE			\&	\ref{plot:silo}		\&	\lstinline{SILO}			\\
	 							};
 							\end{tikzpicture}	\\
	 						\begin{tikzpicture}[inner sep = .1em]
 								\matrix[matrix of nodes,
 								            anchor = center,
 								            ampersand replacement = \&] {
 									\ref{plot:abort}	\&	Abort Rate	\\
 								};
 							\end{tikzpicture}
 						\end{tabular}%
 					};
 				\end{tikzpicture}%
 			}%
			
			\footnotesize
			\begin{block}{Observations}
				\only<3->{\begin{itemize}
					\only<3-6>{\item	abort rate scales for \ref{label:dlDetect} due to higher contention \bm{$\rightarrow$} deadlocks}
					\only<5-9>{\item	$\left[\si{\mega\transactions\per\second}\right]$ suffers from aborts and lock thrashing}
					\only<6-10>{\item	\ref{label:dora}/\ref{label:delegation} suffer more from remote data access overhead}
					\only<7-10>{\item	latch contention isn't the bottleneck \bm{$\rightarrow$} \ref{label:se} can outperform \ref{label:dora}/\ref{label:delegation}}
					\only<10-13>{\item	lock thrashing doesn't cause many aborts for \ref{label:noWait} with \ref{label:se} for few threads}
					\only<11-15>{\item	lock thrashing caused by long commit latencies caused by overloaded (hot) partitions causes many aborts for \ref{label:dora}/\ref{label:delegation}}
					\only<13-19>{\item	the aborts are the major bottleneck for \ref{label:noWait}}
					\only<14-19>{\item	latching overhead and deadlocks $\rightarrow$ \ref{label:noWait} outperforms \ref{label:dlDetect} for \ref{label:se}}
					\only<16-21>{\item	for update-only \ref{label:noWait} and \ref{label:2vNoWait} behave identical}
					\only<18-22>{\item	\ref{label:silo} causes less aborts than \ref{label:dlDetect} due its optimism}
					\only<20-22>{\item	long commit latencies of \ref{label:delegation} cause high update contention and therefore many aborts (low $\left[\si{\mega\transactions\per\second}\right]$) for \ref{label:silo}}
					\only<22->{\item	coarse-grained partition locking of \ref{label:pse} is identical for read and update}
					\only<23->{\item	\ref{label:pse} scales according to the number of hot records (each transaction locks 2 of 16 (hot) partitions)}
				\end{itemize}}
			\end{block}
		\end{column}
	\end{columns}
\end{frame}

\subsection[Read-Only YCSB Workload]{Read-Only YCSB Workload} \label{subsec:readYCSB}

 \begin{frame}
	\frametitle{Read-Only YCSB ($\Theta = 0.8$)}

	\hspace{-2em}%
	\begin{columns}[T]
		\begin{column}{.667\enhancedtextwidth}
			\centering
 			\resizebox{\linewidth}{!}{%
 				\begin{tikzpicture}
 					\begin{axis}[xlabel = {Number of Threads},
 							   xlabel near ticks,
 							   xmin = 0,
 							   xmax = 72,
 							   xmode = normal,
 							   scaled x ticks = false,
 							   ylabel = {Transaction Throughput $\left[\si{\mega\transactions\per\second}\right]$},
 							   ylabel near ticks,
 							   ymin = 0,
 							   ymax = 5,
 							   ymode = normal,
 							   scaled y ticks = false,
							   yticklabel = {\axisdefaultticklabel\hphantom{$.0$}},
 							   grid = major,
 							   scale only axis,
 							   legend pos = north west,
 							   legend columns = 2,
 							   width = \linewidth,
 							   height = .8\textheight]
 						\addlegendimage{se, legendArchitecture}
 						\addlegendentry{SE/NP}
 						\addlegendimage{dlDetect, legendCC}
 						\addlegendentry{\lstinline{DL_DETECT}}
 						\addlegendimage{dora, legendArchitecture}
 						\addlegendentry{DORA}
 						\addlegendimage{noWait, legendCC}
 						\addlegendentry{\lstinline{NO_WAIT}}
 						\addlegendimage{delegation, legendArchitecture}
 						\addlegendentry{Delegation}
 						\addlegendimage{2vNoWait, legendCC}
 						\addlegendentry{\lstinline{2V_NO_WAIT}}
 						\addlegendimage{pse}
 						\addlegendentry{PSE}
 						\addlegendimage{silo, legendCC}
 						\addlegendentry{\lstinline{SILO}}
 		
 						\addplot[se, dlDetect, visible on = <2->]		table[x = Threads, y = DL_DETECT]		{./data/read_only_ycsb_6a.csv};
 						\addplot[delegation, dlDetect, visible on = <2->]	table[x = Threads, y = DELEGATION]	{./data/read_only_ycsb_6a.csv};
 						\addplot[se, noWait, visible on = <5->]		table[x = Threads, y = NO_WAIT]		{./data/read_only_ycsb_6a.csv};
 						\addplot[se, 2vNoWait, visible on = <5->]		table[x = Threads, y = 2V_NO_WAIT]	{./data/read_only_ycsb_6a.csv};
 						\addplot[se, silo, visible on = <8->]			table[x = Threads, y = SILO]			{./data/read_only_ycsb_6a.csv};
 						\addplot[pse, visible on = <10->]			table[x = Threads , y = PSE]			{./data/read_only_ycsb_6a.csv};
 					\end{axis}
 				\end{tikzpicture}%
 			}%
		\end{column}
		\begin{column}{.333\enhancedtextwidth}
			\footnotesize
			\begin{block}{Observations}
				\only<3->{\begin{itemize}
					\only<3-8>{\item	\ref{label:se} scales well with \ref{label:dlDetect} until the latch contention becomes a bottleneck}
					\only<4-10>{\item	\ref{label:delegation} (and \ref{label:dora}) doesn't scale well due to partition-unfriendly zipfian access distribution}
					\only<6->{\item	atomics of \ref{label:noWait} scale better than latches of \ref{label:dlDetect}}
					\only<7->{\item	\ref{label:2vNoWait} and \ref{label:noWait} perform identical for read-only}
					\only<9->{\item	\ref{label:silo} lags behind \ref{label:2vNoWait} due to the overhead of copying read (large) records for validation}
					\only<11->{\item	coarse-grained partition locking of \ref{label:pse} is identical for read and update}
				\end{itemize}}
			\end{block}
		\end{column}
	\end{columns}
\end{frame}

\subsection[Update-Only YCSB Workload]{Update-Only YCSB Workload} \label{subsec:updateYCSB}

 \begin{frame}
	\frametitle{Update-Only YCSB ($\Theta = 0.8$)}

	\hspace{-2em}%
	\begin{columns}[T]
		\begin{column}{.667\enhancedtextwidth}
			\centering
 			\resizebox{\linewidth}{!}{%
 				\begin{tikzpicture}
 					\begin{axis}[xlabel = {Number of Threads},
 							   xlabel near ticks,
 							   xmin = 0,
 							   xmax = 72,
 							   xmode = normal,
 							   scaled x ticks = false,
 							   ylabel = {Transaction Throughput $\left[\si{\mega\transactions\per\second}\right]$},
 							   ylabel near ticks,
 							   ymin = 0,
 							   ymax = 1.5,
 							   ymode = normal,
 							   scaled y ticks = false,
 							   grid = major,
 							   scale only axis,
 							   legend pos = north west,
 							   legend columns = 2,
 							   width = \linewidth,
 							   height = .8\textheight]
 						\addlegendimage{se, legendArchitecture}
 						\addlegendentry{SE/NP}
 						\addlegendimage{dlDetect, legendCC}
 						\addlegendentry{\lstinline{DL_DETECT}}
 						\addlegendimage{dora, legendArchitecture}
 						\addlegendentry{DORA}
 						\addlegendimage{noWait, legendCC}
 						\addlegendentry{\lstinline{NO_WAIT}}
 						\addlegendimage{delegation, legendArchitecture}
 						\addlegendentry{Delegation}
 						\addlegendimage{2vNoWait, legendCC}
 						\addlegendentry{\lstinline{2V_NO_WAIT}}
 						\addlegendimage{pse}
 						\addlegendentry{PSE}
 						\addlegendimage{silo, legendCC}
 						\addlegendentry{\lstinline{SILO}}
 		
 						\addplot[se, dlDetect, visible on = <2->]		table[x = Threads, y = DL_DETECT]		{./data/update_only_ycsb_6b.csv};
 						\addplot[se, noWait, visible on = <4->]		table[x = Threads, y = NO_WAIT]		{./data/update_only_ycsb_6b.csv};
 						\addplot[se, 2vNoWait, visible on = <4->]		table[x = Threads, y = 2V_NO_WAIT]	{./data/update_only_ycsb_6b.csv};
 						\addplot[se, silo, visible on = <7->]			table[x = Threads, y = SILO]			{./data/update_only_ycsb_6b.csv};
 						\addplot[pse, visible on = <9->]				table[x = Threads , y = PSE]			{./data/update_only_ycsb_6b.csv};
 					\end{axis}
 				\end{tikzpicture}%
 			}%
		\end{column}
		\begin{column}{.333\enhancedtextwidth}
			\footnotesize
			\begin{block}{Observations}
				\only<3->{\begin{itemize}
					\only<3-9>{\item	\ref{label:dlDetect} suffers from deadlocks for many threads}
					\only<5->{\item	lock thrashing (aborts for \ref{label:noWait}) isn't a bottleneck due to lower contention}
					\only<6->{\item	\ref{label:2vNoWait} and \ref{label:noWait} perform identical for update-only}
					\only<8->{\item	\ref{label:silo} causes less aborts than \ref{label:noWait} due its optimism $\rightarrow$ higher $\left[\si{\mega\transactions\per\second}\right]$}
					\only<10->{\item	\ref{label:pse} (and \ref{label:delegation}/\ref{label:dora}) doesn't scale well due to partition-unfriendly zipfian access distribution}
				\end{itemize}}
			\end{block}
		\end{column}
	\end{columns}
\end{frame}

 \begin{frame}
	\frametitle{Update-Only YCSB (72 Threads)}

	\hspace{-2em}%
	\begin{columns}[T]
		\begin{column}{.667\enhancedtextwidth}
			\centering
 			\resizebox{\linewidth}{!}{%
 				\begin{tikzpicture}
 					\begin{axis}[xlabel = {$\Theta$},
 							   xlabel near ticks,
 							   xmin = 0,
 							   xmax = 1,
 							   xmode = normal,
 							   scaled x ticks = false,
 							   ylabel = {Transaction Throughput $\left[\si{\mega\transactions\per\second}\right]$},
 							   ylabel near ticks,
 							   ymin = 0,
 							   ymax = 2.75,
 							   ymode = normal,
 							   scaled y ticks = false,
 							   grid = major,
 							   scale only axis,
 							   legend pos = south west,
 							   legend columns = 2,
 							   legend style = {yshift = 1em},
 		 					   width = \linewidth,
 		 					   height = .8\textheight]
 						\addlegendimage{se, legendArchitecture}
 						\addlegendentry{SE/NP}
 						\addlegendimage{dlDetect, legendCC}
 						\addlegendentry{\lstinline{DL_DETECT}}
 						\addlegendimage{dora, legendArchitecture}
 						\addlegendentry{DORA}
 						\addlegendimage{noWait, legendCC}
 						\addlegendentry{\lstinline{NO_WAIT}}
 						\addlegendimage{delegation, legendArchitecture}
 						\addlegendentry{Delegation}
 						\addlegendimage{2vNoWait, legendCC}
 						\addlegendentry{\lstinline{2V_NO_WAIT}}
 						\addlegendimage{pse}
 						\addlegendentry{PSE}
 						\addlegendimage{silo, legendCC}
 						\addlegendentry{\lstinline{SILO}}
 		
 						\addplot[se, dlDetect, visible on = <2->]		table[x = Theta, y = DL_DETECT]	{./data/update_only_ycsb_contention_6c.csv};
 						\addplot[se, noWait, visible on = <2->]		table[x = Theta, y = NO_WAIT]		{./data/update_only_ycsb_contention_6c.csv};
 						\addplot[se, 2vNoWait, visible on = <2->]		table[x = Theta, y = 2V_NO_WAIT]	{./data/update_only_ycsb_contention_6c.csv};
 						\addplot[se, silo, visible on = <2->]			table[x = Theta, y = SILO]			{./data/update_only_ycsb_contention_6c.csv};
 						\addplot[pse, visible on = <7->]				table[x = Theta , y = PSE]			{./data/update_only_ycsb_contention_6c.csv};
 					\end{axis}
 				\end{tikzpicture}%
 			}%
		\end{column}
		\begin{column}{.333\enhancedtextwidth}
			\footnotesize
			\begin{block}{Observations}
				\only<3->{\begin{itemize}
					\only<3-7>{\item	for $\Theta\leq0.4$ the contention is very low $\bm{\rightarrow}$ high concurrency possible}
					\only<4->{\item	copying records imposes an overhead to \ref{label:2vNoWait}/\ref{label:silo}}
					\only<5->{\item	atomics of \ref{label:noWait} scale better than latches of \ref{label:dlDetect}}
					\only<6->{\item	\ref{label:silo} causes less aborts than \ref{label:noWait} due its optimism $\rightarrow$ higher $\left[\si{\mega\transactions\per\second}\right]$}
					\only<8->{\item	\ref{label:pse} doesn't scale well due to partition-unfriendly zipfian access distribution}
				\end{itemize}}
			\end{block}
		\end{column}
	\end{columns}
\end{frame}

\subsection[Mixed YCSB Workload]{Mixed YCSB Workload} \label{subsec:mixedYCSB}

 \begin{frame}
	\frametitle{Mixed YCSB ($\Theta = 0.8$, 72 Threads)}

	\hspace{-2em}%
	\begin{columns}[T]
		\begin{column}{.667\enhancedtextwidth}
			\centering
 			\resizebox{\linewidth}{!}{%
 				\begin{tikzpicture}
 					\begin{axis}[xlabel = {Number of Writers},
 							   xlabel near ticks,
 							   xmin = 0,
 							   xmax = 72,
 							   xmode = normal,
 							   scaled x ticks = false,
 							   ylabel = {Transaction Throughput $\left[\si{\mega\transactions\per\second}\right]$},
 							   ylabel near ticks,
 							   ymin = 0,
 							   ymax = 5,
 							   ymode = normal,
 							   scaled y ticks = false,
							   yticklabel = {\axisdefaultticklabel\hphantom{$.0$}},
 							   grid = major,
 							   scale only axis,
 							   legend pos = north east,
 							   legend columns = 2,
 		 					   width = \linewidth,
 		 					   height = .8\textheight]
 						\addlegendimage{se, legendArchitecture}
 						\addlegendentry{SE/NP}
 						\addlegendimage{dlDetect, legendCC}
 						\addlegendentry{\lstinline{DL_DETECT}}
 						\addlegendimage{dora, legendArchitecture}
 						\addlegendentry{DORA}
 						\addlegendimage{noWait, legendCC}
 						\addlegendentry{\lstinline{NO_WAIT}}
 						\addlegendimage{delegation, legendArchitecture}
 						\addlegendentry{Delegation}
 						\addlegendimage{2vNoWait, legendCC}
 						\addlegendentry{\lstinline{2V_NO_WAIT}}
 						\addlegendimage{pse}
 						\addlegendentry{PSE}
 						\addlegendimage{silo, legendCC}
 						\addlegendentry{\lstinline{SILO}}
 		
 						\addplot[se, dlDetect, visible on = <2->]		table[x = Writers, y = DL_DETECT]	{./data/mixed_ycsb_writers_6d.csv};
 						\addplot[se, noWait, visible on = <5->]		table[x = Writers, y = NO_WAIT]	{./data/mixed_ycsb_writers_6d.csv};
 						\addplot[se, 2vNoWait, visible on = <5->]		table[x = Writers, y = 2V_NO_WAIT]	{./data/mixed_ycsb_writers_6d.csv};
 						\addplot[se, silo, visible on = <8->]			table[x = Writers, y = SILO]		{./data/mixed_ycsb_writers_6d.csv};
 						\addplot[pse, visible on = <11->]				table[x = Writers , y = PSE]		{./data/mixed_ycsb_writers_6d.csv};
 					\end{axis}
 				\end{tikzpicture}%
 			}%
		\end{column}
		\begin{column}{.333\enhancedtextwidth}
			\footnotesize
			\begin{block}{Observations}
				\only<3->{\begin{itemize}
					\only<3-8>{\item	\ref{label:dlDetect} suffers from latch contention for 72 reading threads}
					\only<4-9>{\item	\ref{label:dlDetect} suffers from deadlocks for writing threads}
					\only<6-11>{\item	atomics of \ref{label:noWait} scale better than latches of \ref{label:dlDetect}}
					\only<7-11>{\item	multi-versioning of \ref{label:2vNoWait} improves concurrency for mixed workloads}
					\only<9->{\item	\ref{label:silo} lags behind \ref{label:2vNoWait} due to the overhead of copying read (large) records for validation}
					\only<10->{\item	\ref{label:silo} causes less aborts than \ref{label:2vNoWait} due its optimism for many writers}
					\only<12->{\item	\ref{label:pse} (and \ref{label:delegation}/\ref{label:dora}) doesn't scale well due to partition-unfriendly zipfian access distribution}
				\end{itemize}}
			\end{block}
		\end{column}
	\end{columns}
\end{frame}

\subsection[Conclusion]{Conclusion} \label{subsec:conclusion}

 \begin{frame}
	\frametitle{Conclusion I}

	\begin{itemize}
		\visible<2->{\item	optimistic CC scales better than pessimistic CC for most workloads}
		\visible<3->{\item	optimistic CC suffers from large record sizes}
		\visible<4->{\item	atomic operations scale better than latches}
		\visible<5->{\item	partitioning make latches scalable}
		\visible<6->{\item	2PL doesn't scale for mixed workloads}
		\visible<7->{\item	partitioning DB architectures perform bad under partition-unfriendly workloads}
		\visible<8->{\item	partitioning DB architectures perform bad under multi-sited transactions}
	\end{itemize}
\end{frame}

 \begin{frame}
	\frametitle{Conclusion II}

	\begin{itemize}
		\visible<2->{\item	the transaction throughput decreases by an order of magnitude for update-only instead of read-only workloads (PSE is insensitive to writes) \bm{$\rightarrow$} PSE scales best for update-intensive workloads}
		\visible<3->{\item	PSE doesn't scale for read-intensive high contention workloads wth small hot sets}
		\visible<4->{\item[$\rightarrow$]	None of the architectures or CC protocols outperform the others for any workload!}
		\visible<5->{\item[$\rightarrow$]	Every architecture and CC protocol performs very bad for some specific workload!}
	\end{itemize}
\end{frame}

\subsection[Discussion]{Discussion} \label{subsec:discussion}

 \begin{frame}
	\frametitle{Discussion of the Performance Evaluation}

	\begin{itemize}
		\visible<2->{\item	read-only and update-only workload aren't appropriate to evaluate concurrency control algorithms}
		\visible<3->{\item	partition-unfriendly workloads aren't appropriate to evaluate database architectures that use partitioning}
		\visible<4->{\item	neither the microbenchmark nor YCSB are OLTP benchmarks}
		\visible<5->{\item[$\rightarrow$]	The authors didn't properly analyze the combination of database architecture and concurrency control algorithm for OLTP workloads!}
	\end{itemize}
\end{frame}

	
\subsection*{}
 \begin{frame}[t, allowframebreaks]
    \frametitle{References}
    \printbibliography
\end{frame}

%------------------------------------------------
\section*{End}
%------------------------------------------------

 \begin{frame}
\Huge{\centerline{Any Questions?}}
\end{frame}

%----------------------------------------------------------------------------------------

\end{document} 